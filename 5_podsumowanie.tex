\chapter*{Zakończenie}
\addcontentsline{toc}{chapter}{Zakończenie}
\addtocontents{toc}{\protect\vspace{20pt}}

W~niniejszej pracy przedstawiono trzy logiczne aspekty oraz związane z~nimi interpretacje teologii apofatycznej.

\begin{enumerate}
\item  W~pierwszej części pracy na tle teologii milczenia, czyli apofatycznej doktryny podkreślającej transcendencję Boga za pomocą tezy o~boskiej niewysławialności, ukazano semantyczny (oraz ogólno-logiczny) aspekt teologii negatywnej. W~szczególności:
\begin{enumerate}
\item Przedstawiono tę interpretację teologii apofatycznej i~wskazano Pseudo-Dionizego Areopagitę\index[names]{Pseudo-Dionizy Areopagita} jako jej modelowego reprezentanta.
\item Zidentyfikowano i~ujawniono ciążący na tej doktrynie paradoks (Niewyrażalnego) o~charakterystyce zbliżonej do znanych paradoksów semantycznych (na przykład paradoksu kłamcy).
\item Uprzedzono o~niejednoznaczności terminu ,,Bóg'' w~sensie jego kategorii językowej.
\item Przedstawiono i~poddano dyskusji szereg prób obrony teologii apofatycznej przed sprzecznościami (rozpatrzono propozycje Johna N. Jonesa\index[names]{Jones, John J.}, Johna Hicka\index[names]{Hick, John}, Petera Küglera\index[names]{Kügler, Peter}, Jerome'a I. Gellmana\index[names]{Gellman, Jerome I.}, Jonathana D. Jacobsa\index[names]{Jacobs, Jonathan D.}, Józefa Marii Bocheńskiego\index[names]{Bocheński, Józef Maria} oraz Pawła Rojka\index[names]{Rojek, Paweł}).
\item Przeprowadzono analizy, z~których wniosek każe uznać wszystkie dyskutowane próby zachowania spójności tej doktryny za niesatysfakcjonujące.
\end{enumerate}
\item W~drugiej części pracy przedstawiono teologię negatywną w~aspekcie epistemicznym, korzystając w~tym celu z~tej interpretacji apofatyzmu, która za centralną tezę uznaje nie tyle niewysławialność, co niepoznawalność Boga, tzn. z~teologicznego sceptycyzmu. W~szczególności:
\begin{enumerate}
\item Zaprezentowano tę interpretację teologii apofatycznej, ujawniono jej pokrewieństwo z~teologią milczenia oraz przedstawiono jej modelowego reprezentanta -- Mojżesza Majmonidesa\index[names]{Majmonides, Mojżesz}.
\item Podjęto zagadnienie niekomunikowalności języka religijnego i~wskazano związki tego problemu z~tezą o~niezdeterminowaniu przekładu Willarda Van Ormana Quine'a\index[names]{Quine, Willard V.O.}.
\item Wymieniono możliwe postawy względem tezy o~niewysławialności i~tezy o~niepojmowalności oraz ich inferencyjnych powiązań.
\item Ujawniono, że za rozważaniami o~zależnościach między tezą o~niewysławialności i~tezą o~niepojmowalności kryje się bardziej ogólny problem z~zakresu filozofii i~nauk kognitywnych -- tzw. spór o~relatywizm językowy. Zidentyfikowano pozycje teoretyczne (oraz wskazano niewielki zestaw badań eksperymentalnych) wyznaczające stanowiska w~tym sporze.
\item Sformułowano paradoks Niepoznawalnego o~tej samej samozwrotnej charakterystyce, z~jaką mieliśmy do czynienia w~przypadku paradoksu Niewyrażalnego.
\item Argumentowano za wyborem zjawiska (posiadania) wiedzy jako reprezentacji procesów mentalnych, o~których mówi się w~kontekście teologicznego sceptycyzmu, co pozwoliło na użycie epistemicznej logiki modalnej w~celu rekonstrukcji tezy o~niepojmowalności i~teologicznego sceptycyzmu.
\item Pokazano, że wykładnia teologicznego sceptycyzmu wyrażona w~języku epistemicznej logiki modalnej przyjmuje postać znanych paradoksów epistemicznych: problemu Moore'a\index[names]{Moore, George E.}, paradoksu poznawalności Churcha\index[names]{Church, Alonzo}-Fitcha\index[names]{Fitch, Frederic B.} oraz paradoksu wiedzy (znawcy).
\end{enumerate}
\item W~końcu, w~trzeciej części pracy, na tle teologii negatywnej rozumianej jako mistycyzm o~pochodzeniu neoplatońskim zaprezentowano także aspekt teorio-porządkowy tej doktryny. W~szczególności:
\begin{enumerate}
\item Ujawniono neoplatońskie źródła teologii apofatycznej.
\item Pokazano podobieństwa i~różnice między neoplatońską ontologią a~teologią apofatyczną, a~na modelowego przedstawiciela tej pierwszej wybrano Damascjusza\index[names]{Damascjusz}.
\item Przedstawiono i~poddano dyskusji próbę formalnej rekonstrukcji neoplatońskich rozważań o~Jedni (a zarazem apofatycznych rozważań o~Bogu) w~obrębie algebry zbioru potęgowego autorstwa Uwe Meixnera\index[names]{Meixner, Uwe}.
\item Pokazano teologiczne, filozoficzne i~logiczne problemy tej rekonstrukcji wliczając warunki, pod jakimi można w~jej ramach wywieść sprzeczność w~stylu antynomii Russella\index[names]{Russell, Bertrand}.
\end{enumerate}
\end{enumerate}
%Gdyby wynik niniejszego opracowania przedstawić w~klasycznej, szkolnej postaci, należy
Można uznać, że praca ta stanowi argumentację za tezą, według której paradoksalna samozwrotność stanowi inherentną i~niezbywalną własność teologii apofatycznej, przynajmniej w~rozumieniu wyznaczonym przez powyżej wskazane interpretacje tej doktryny -- teologię milczenia i~teologiczny sceptycyzm. Zatem ostateczny wynik tej pracy jest negatywny -- z~jednej strony paradoksalny charakter został ujawniony we wszystkich aspektach teologii apofatycznej, z~drugiej pokazano, że analizowanym tu autorom opracowań mających na celu obronę spójności apofatyzmu nie udało się podać takiego modelu którejkolwiek interpretacji teologii apofatycznej, który byłby jednocześnie logicznie spójny i~teologicznie akceptowalny.

Przypuszczam jednak, że niejeden teolog apofatyczny przyjąłby ten wynik z~ulgą i~zadowoleniem uznając, że jest on… apofatyczny -- wielu przedstawicieli tej doktryny twierdzi, że na końcu \textit{viae negativae}, gdy odrzucimy już wszystkie przymioty Boga i~każdą mowę o~Bogu, ostatecznym krokiem jest odrzucenie samej apofatycznej doktryny.
W~tym kontekście wskazuje się czasem podobieństwo teologii negatywnej do rozważań Ludwiga Wittgensteina\index[names]{Wittgenstein, Ludwig} z~\textit{Traktatu logiczno-filozoficznego}\footnote{Zob. B. Brożek, \textit{Marzenie Leibniza. Rzecz o~języku religii}, Copernicus Center Press, Kraków 2016; S.R. Lebens, \textit{Negative Theology as Illuminating and/or Therapeutic Falsehood}, [w:] \textit{Negative Theology as Jewish Modernity}, red. M. Fagenblat, Indiana University Press, Bloomington 2017, ss.~85-108.}.
Wittgenstein\index[names]{Wittgenstein, Ludwig} za pomocą listy tez przedstawia w~nim m.in. swoją ,,obrazkową'' teorię znaczenia, według której ,,istotą języka -- ogólną formą zdania -- jest bycie obrazem tego, jak się rzeczy mają''\footnote{H.-J. Glock, \textit{Słownik Wittgensteinowski}, tłum. M. Hernik, M. Szczubiałka, Wydawnictwo Spacja, Warszawa 2001, s.~347.}. Wypowiedzi językowe można uznać za sensowne wtedy, gdy są obrazami świata. Teoria ta załamuje się jednak w~momencie, gdy przestajemy mówić o~faktach. Zatem za bezsensowne należy uznać także tezy zawarte w~\textit{Traktacie}.

\begin{quote}
Tezy moje wnoszą jasność przez to, że kto mnie rozumie, rozpozna je w~końcu jako niedorzeczne; gdy przez nie -- po nich -- wyjdzie ponad nie. (Musi niejako odrzucić drabinę, uprzednio po niej się wspiąwszy). Musi te tezy przezwyciężyć, wtedy świat przedstawi mu się właściwie\footnote{L. Wittgenstein, \textit{Tractatus logico-philosophicus}, tłum. B. Wolniewicz, Wydawnictwo Naukowe PWN, Warszawa 2022, 6.53.}.
\end{quote}
Jak zauważa Bartosz Brożek\index[names]{Brożek, Bartosz},

\begin{quote}
poręczne jest rozróżnienie na \textit{sagen} i~\textit{zeigen}, mówienie i~pokazywanie. Mówić można o~faktach, natomiast to, co faktem nie jest, można co najwyżej pokazać. Do obu celów służy język -- ale w~drugim przypadku wyłącznie jako Wittgensteinowska\index[names]{Wittgenstein, Ludwig} drabina, po której wspinamy się tylko po to, by na końcu ją odrzucić. Łatwo dostrzec, że tę strategię Wittgensteina\index[names]{Wittgenstein, Ludwig} da się dostosować do teologii negatywnej\footnote{B. Brożek, \textit{Marzenie Leibniza}…, dz. cyt., rozdz.~4.}.
\end{quote}

W~tym sensie można mówić o~pewnym filozoficznym sposobie zaakceptowania, a~zarazem ,,przeskoczenia'' sprzeczności zawartych w~tej doktrynie. Wyrażając apofatyczne sądy, według których ,,Bóg nie jest ani duszą, ani intelektem''\footnote{Pseudo-Dionizy Areopagita, \textit{Teologia mistyczna}, V, [w:] tenże, \textit{Pisma teologiczne}, tłum. M. Dzielska, Wydawnictwo Znak, Kraków 1997.}, albo że ,,Nie istnieje ani słowo, ani imię, ani wiedza o~Nim''\footnote{Tamże.}, nie wyrażamy żadnych faktów. Zdania te może niczego nie mówią, ale mogą coś pokazywać.

Trudno nie zgodzić się z~tezą, że paradoksy pełnią istotną rolę w~rozważaniach z~zakresu logiki i~filozofii języka. Często stanowią ,,filozoficzną'' motywację pewnych rozwiązań wypracowywanych w~obrębie tych dyscyplin. Na przykład paradoks kłamcy pełni taką rolę dla logicznych teorii prawdy, a~paradoks Russella\index[names]{Russell, Bertrand} dla ustaleń z~zakresu teorii mnogości. Można je także uznać za probierz, czy też ,,papierek lakmusowy'' wszelkich konstrukcji tworzonych w~ramach tego typu teorii. Dodatkowo, paradoks kłamcy z~pewnością można uważać także za flagowy przykład cyrkularności myślenia, używany daleko poza obszarem badań logicznych. W~tym sensie stosowany jest także w~argumentacji w~ramach kontrprzykładu. Wskazanie pewnego paradoksalnego typu samoodniesienia w~wywodach adwersarza oznacza w~zasadzie sprowadzenie ich do absurdu i~odrzucenie zawartych tam tez. W~mojej opinii można zakładać, że analizowana w~tej pracy teologia apofatyczna ma szansę stać się w~tym sensie teologicznym odpowiednikiem paradoksu kłamcy -- konstrukcji, o~którą rozbijają się najtęższe umysły i~która w~podobny sposób prowadzi do płodnych filozoficznych, logicznych i~teologicznych rozważań.

Można zatem wskazać drugą tezę niniejszego opracowania. Rozważania tu przedstawione pozwalają sądzić, że -- mimo paradoksalnego charakteru -- teologia apofatyczna może być badana za pomocą narzędzi logicznych (w tym narzędzi logiki formalnej), a~wynik tego badania może być owocny i~ciekawy z~logicznego, filozoficznego i~teologicznego punktu widzenia.

W~niniejszej pracy przedstawiłem wybrane logiczne aspekty teologii negatywnej. Bynajmniej nie oznacza to, że temat logicznych i~filozoficznych zagadnień związanych z~apofatyzmem został wyczerpany. Żywię nadzieję, że uważny czytelnik odnajdzie w~lekturze tej pracy kolejne problemy warte podjęcia i~opracowania, a~być może i~inspirację do dalszych analitycznych badań w~tym zakresie. By nie być gołosłownym, poniżej przedstawiam przykładowy zbiór kilku takich zagadnień.

\begin{enumerate}[label = \arabic*), itemindent=6mm, labelwidth=4mm, labelsep=2mm, itemsep=1em, leftmargin=0mm]
\item Spór o~kategorię językową terminu ,,Bóg''\\
Zasadniczo problem ten nie wydaje się bezpośrednio związany z~niepoznawalnością i~niepojmowalnością Boga. Warto jednak zauważyć, że istnieją stanowiska, w~obrębie których za wyborem kategorii językowej terminu ,,Bóg'' argumenty formułuje się głównie w~kontekście teologii negatywnej. Do protagonistów takich stanowisk należą Walter Terence Stace\index[names]{Stace, Terence S.} oraz Janet Martin Soskice\index[names]{Soskice, Janet M.}. Sądzą oni, że pewne apofatyczne rozważania każą uznać termin ,,Bóg'' za nazwę własną, której nie można przypisać żadnej własności. Przeciwko tym podejściom występują William P. Alston\index[names]{Alston, William P.} oraz Michael Durrant\index[names]{Durrant, Michael} uważając, że jeśli używamy terminu ,,Bóg'', musimy być w~stanie podać przynajmniej jakąś jego minimalną deskrypcję -- w~przeciwnym razie nie moglibyśmy wiedzieć, o~czym mówimy. Jak zauważa Adam Olszewski\index[names]{Olszewski, Adam}, kwestia kategorii językowej terminu ,,Bóg'' domaga się bardziej szczegółowego opracowania i~wiąże się z~różnymi teoriami odniesienia rozwijanymi w~obrębie filozofii języka\footnote{Zob. rozdz.~\ref{sil-kt-jez}.}.

\item Klasyfikacja i~badania nad naturą (różnych rodzajów) negacji\\
W~obrębie rozważań zawartych w~niniejszej pracy do analiz teologii apofatycznej często wykorzystuje się pewne nieklasyczne pojęcia negacji. Na przykład przy okazji dyskusji nad uniwersalną i~egzystencjalną zasadą teologii negatywnej Petera Küglera\index[names]{Kügler, Peter} dyskutowany jest model z~negacją intuicjonistyczną\footnote{Zob. rozdz.~\ref{sil-kugler}.}. Podczas analiz pojęcia zakresu przedmiotowego oddziaływania predykatów wprowadzonego przez Jerome'a I. Gellmana\index[names]{Gellman, Jerome I.}\footnote{Zob. rozdz.~\ref{sil-gell}.} pojawia się zagadnienie podziału na negację wykluczającą oraz negację wyboru\footnote{Zob. F. Sommers, \textit{Predicability}, [w:] \textit{Philosophy in America}, red. M. Black, Routledge, London 2002, ss.~262-281 oraz R.H. Thomason, \textit{A~semantic theory of sortal incorrectness}, ,,Journal of Philosophical Logic'', vol. 1 (1972), ss.~209-258.}. W~logice Aleksandra Zinowjewa\index[names]{Zinowjew, Aleksander}\footnote{Zob. rozdz.~\ref{roj-agnostycna}.} mówimy o~wewnętrznej, przypredykatowej negacji \textit{de re} oraz zewnętrznej, przyzdaniowej negacji \textit{de dicto}\footnotetext{ Zob. A. Zinowjew, \textit{Logika nauki}, tłum. Z. Simbierowicz, Wydawnictwo Naukowe PWN, Warszawa 1976, ss.~120-124. Por. I. Sedlár, K. Ebela, \textit{Term Negation in First-Order Logic}, ,,Logique et Analyse'', vol. 247 (2019), ss.~265-284.}.
Paweł Rojek\index[names]{Rojek, Paweł}, w~swojej oryginalnej interpretacji teologii negatywnej wprowadza pojęcie ,,pozytywnej'' negacji\footnote{Zob. rozdz.~\ref{roj-pozytywna}.} i~próbuje zarówno umotywować je przykładami pochodzącymi z~języka naturalnego i~historii filozofii, jak i~sformalizować w~postaci zestawu aksjomatów. W~końcu, Zbigniew Król\index[names]{Król, Zbigniew} analizując apofatyczne tezy w~platońskim dialogu \textit{Parmenides} wyróżnia trzy rodzaje negacji: klasyczną, globalną i~lokalną\footnote{Zob. rozdz.~\ref{neopl}.}. Zarówno pogłębione badania w~tym zakresie, jak i~nawet sama typologia i~podanie \textit{fundamenta divisionis} różnych pojęć negacji używanych w~kontekście teologii negatywnej (i nie tylko)\footnote{W~pewnym zakresie tego typu rozważania przedstawiono np. w~L.R. Horn, \textit{A~Natural History of Negation}, CSLI Publications, Stanford 2001. Monografia ta nie zawiera jednak klasyfikacji negacji w~sensie, o~którym tu piszę. Dobrym punktem wyjścia takich badań może być praca J.L. Speranza, L.R. Horn, \textit{A~brief history of negation}, ,,Journal of Applied Logic'', vol.~8 (2010), nr~3, ss.~277-301.} byłyby interesujące i~potencjalnie owocne dla rozważań z~zakresu logiki, filozofii czy teologii\footnote{Do prac, które w~obrębie wymienionych tutaj dyscyplin twórczo powołują się na różne pojęcia negacji, zaliczyć można np. W. Stróżewski, \textit{Z~historii problematyki negacji, cz. 1: Ontologiczna problematyka negacji w~,,De quatuor oppositis}'', ,,Studia Mediewistyczne'', vol. 8 (1967), ss.~183-246, czy też A. Olszewski, \textit{Negation in the language of theology -- some issues}, ,,Zagadnienia Filozoficzne w~Nauce'', nr 65 (2018), ss.~87-107.}.

\item Problem pozytywnej predykacji\\
Na dopuszczeniu lub zablokowaniu możliwości orzekania o~Bogu pozytywnych własności Paweł Rojek\index[names]{Rojek, Paweł} opiera podział między teologią pozytywną i~negatywną. Problem w~tym, że nigdy nie podaje definicji tego typu własności. Jednakże nietrudno zauważyć, że wyraźną inspiracją dla jego rozważań są prace Jerzego Perzanowskiego\index[names]{Perzanowski, Jerzy}, który na potrzeby formalizacji tzw. dowodów ontologicznych adaptuje formalną teorię pozytywności Kurta Gödla\index[names]{Gödel, Kurt} (i Dany Scotta\index[names]{Scott, Dana})\footnote{Zob. rozdz.~\ref{roj-dyskusja}.}. Niestety, w~jej obrębie za pozytywne można uznać wyłącznie predykaty ,,jest sobie tożsamy'', ,,jest Bogiem'' oraz wszystkie, które można z~nich wyprowadzić w~skonstruowanym do tego celu rachunku. Wydaje się, że nie jest to odpowiedni zbiór, by sensownie mówić o~rozróżnieniu doktryn apofatycznej i~katafatycznej teologii. W~tym celu przydałaby się ostra definicja lub kryterium oceny pozytywności własności. Tymczasem na beznadziejne rokowania zadania utworzenia takiej definicji wskazuje większość filozofów, którzy badają teologię negatywną w~duchu analitycznym. Walter Terence Stace\index[names]{Stace, Terence S.}\footnote{Por. W.T. Stace, \textit{Mysticism and Philosophy}, Macmillan, London 1961, ss.~288-290.} zauważa, że nie ma wyraźnego rozróżnienia między predykatami pozytywnymi i~negatywnymi: ,,ciężki'' wydaje się być predykatem pozytywnym, podczas gdy ,,nielekki'' powinien być uznany za predykat negatywny, mimo iż ich znaczenia są tożsame. Nie ma powodu, dla którego powinniśmy traktować je jako dwa różne rodzaje predykatów. Alvin Plantinga\index[names]{Plantinga, Alvin}\footnote{A. Plantinga, \textit{Warranted Christian Belief}, Oxford University Press, New York 2000, ss.~51-53.} przedstawia cały zestaw argumentów za bezzasadnością takiego rozróżnienia. Uważa on, że -- choć można próbować podawać jakieś metody czy kryteria wyłaniania predykatów pozytywnych (lub negatywnych) -- mimo wszystko nie ma między nimi żadnych różnic metafizycznych, co powinno być ostatecznym argumentem dla teologa apofatycznego\footnote{Zob. także S. Gäb, \textit{Languages of ineffability: the rediscovery of apophaticism in contemporary analytic philosophy of religion}, [w:] \textit{Negative Knowledge}, red. S. Hüsch i~in., Narr, Tübingen 2020, ss.~192-193.}. Prawdopodobnie pewne koncyliacyjne stanowisko w~tej kwestii można wypracować nawiązując do ustaleń z~zakresu \textit{Transparent Intensional Logic} (TIL)\footnote{Zob. B. Jespersen, M. Carrara, M. Duží, \textit{Iterated privation and positive predication}, ,,Journal of Applied Logic'', vol. 25 (2017), ss.~S48-S71.}.

\item Orzekanie o~Bogu jako błąd przesunięcia kategorialnego rozumiany w~terminach metajęzykowej negacji\\
Pojęcie błędu przesunięcia kategorialnego może zaoferować analizie filozoficznej pewną ramę interpretacyjną dla teologii apofatycznej. W~myśl takiego rozumienia uważa się, że gdy teolog apofatyczny wypowiada zdanie ,,Bóg nie jest ani duszą, ani intelektem'', nie ma na myśli prostych negacji, lecz stwierdzenie, że kategoria duszy i~intelektu nie przystaje do Boga. Inaczej mówiąc, stwierdzenie ,,Bóg jest duszą'' wiąże się z~błędem przypisania Bogu niewłaściwej kategorii ontologicznej. Podobne błędy popełnia się mówiąc na przykład, że ,,Liczba dwa jest zielona'' lub ,,Wiedza opiera się na empiryzmie''. Z~semantycznego punktu widzenia taką koncepcję można łatwo strywializować uznając, że wszystkie sądy zawierające błąd przesunięcia kategorialnego są zwyczajnie fałszywe\footnote{Zob. choćby rozdz.~\ref{sil-slabatn}.}. Jednakże w~pewien sposób może uratować ją podejście Michaela Scotta\index[names]{Scott, Michael} i~Gabriela Cintrona\index[names]{Cintron, Gabriel}\footnote{Zob. M. Scott, G. Citron, \textit{What is apophaticism? Ways of talking about an ineffable God}, ,,European Journal for Philosophy of Religion'', vol. 8 (2016), nr 4, ss.~23-49. Zob. także S. Gäb, \textit{Languages of ineffability}…, dz. cyt., ss.~191-206.} tłumaczące tego rodzaju błędy w~terminach metajęzykowej negacji\footnote{Zob. L.R. Horn, \textit{A~Natural History of Negation}, dz. cyt., rozdz.~6.} lub zaprzeczania, czy też niezgody na uznanie danego sądu. Analiza zdań teologii negatywnej w~kontekście zagadnienia metajęzykowej negacji ma szansę przynieść ciekawe rozstrzygnięcia.
\end{enumerate}

Oczywiście, powyższa lista logicznych problemów uwikłanych w~teologiczną doktrynę apofatyzmu nie jest zamknięta. Warto także zauważyć, że rozważania przedstawione w~tej pracy angażują także szereg interesujących zagadnień o~charakterze filozoficznym\footnote{Jak na przykład kwestię istnienia niepoznawalnych prawd i~jej konsekwencje dla sporu realizm -- antyrealizm w~obrębie różnych teorii filozoficznych; zagadnienie istnienia obiektów niewyrażalnych w~danym języku wobec teorii ,,zobowiązań ontologicznych'' Willarda Van Ormana Quine'a\index[names]{Quine, Willard V.O.}; teorię niekomunikowalności Boga wobec niezdeterminowania przekładu; epistemologiczne zagadnienie wiedzy osobistej i~,,o~osobie'' oraz związane z~powyższymi emocjonalistyczne ujęcia dyskursu religijnego itp.} oraz problemy z~zakresu nauk kognitywnych\footnote{Jak na przykład kwestię ludzkich zdolności poznawczych i~możliwości ekspresyjnych ludzkiego języka; zależności i~związki między językiem a~poznaniem -- spór o~relatywizm językowy; zależności między różnorodnymi procesami umysłowymi, na przykład myśleniem, pojmowaniem, rozumowaniem, nabywaniem i~posiadaniem przekonań i~wiedzy; kwestię normatywności formalnych teorii myślenia i~rozumowania i~ich przystawanie do rzeczywistych przypadków działania umysłu itp.}.

W~końcu należy przyznać, że -- o~ile poprawnie zidentyfikowałem nękający teologię apofatyczną problem samozwrotności -- można oczekiwać, że obiecujące wyniki przyniosą analizy wykorzystujące uznane logiczne sposoby radzenia sobie z~tego typu antynomiami -- takie jak na przykład:
\begin{enumerate}[label = \arabic*), itemindent=6mm, labelwidth=4mm, labelsep=2mm, itemsep=1em, leftmargin=0mm]
\item Teoria prawdy Alfreda Tarskiego\index[names]{Tarski, Alfred}\\
Analizę teologii apofatycznej w~stylu semantycznej teorii prawdy przedstawiono w~niniejszej pracy w~wersjach zaproponowanych przez Józefa Marię Bocheńskiego\index[names]{Bocheński, Józef Maria} i~Bartosza Brożka\index[names]{Brożek, Bartosz}\footnote{Zob. rozdz.~\ref{sil-boch}.}. Teoria Bocheńskiego\index[names]{Bocheński, Józef Maria} budowana jest już nie wokół pojęcia prawdziwości sądów, lecz (nie)wysławialności terminów, co czyni niejasną kwestię cząstkowej definicji niewysławialności i~odpowiednika T-równoważności dla takiej teorii. W~alternatywnej, zaproponowanej przez Brożka\index[names]{Brożek, Bartosz} analizie teologii apofatycznej w~stylu Tarskiego\index[names]{Tarski, Alfred}, mówimy o~bezsensowności sądów i~godzimy się na obecność sprzeczności na każdym z~poziomów języka. Nie oznacza to jednak, że jakaś adaptacja teorii Tarskiego\index[names]{Tarski, Alfred}, pozbawiona powyżej wspomnianych wad, nie mogłaby skutecznie poradzić sobie z~samozwrotnością tej doktryny.

\item Teoria prawdy Saula Kripkego\index[names]{Kripke, Saul}\\
Nieco inne podejście do definiowania prawdy i~rozwiązywania paradoksów semantycznych przedstawił Saul Kripke\index[names]{Kripke, Saul}\footnote{S. Kripke, \textit{Outline of a~Theory of Truth}, ,,The Journal of Philosophy'', vol. 72 (1975), nr 19, ss.~690-716. Zob. także R.L. Martin, P.W. Woodruff, \textit{On representing ‘true-in-L' in L}, ,,Philosophia'', vol. 5 (1975), nr 3, ss.~213-217.}. Odrzucił on ideę rozwarstwiania języka, w~którym o~prawdziwości jakiegoś zdania możemy mówić tylko na wyższych poziomach języka. Zamiast tego rozwarstwienie wprowadził do samego pojęcia prawdy. W~jego koncepcji pojęcie to definiowane jest iteracyjnie (co zdaniem Kripkego\index[names]{Kripke, Saul} ma odzwierciedlać proces uczenia się używania słowa ,,prawda''), aż do momentu, w~którym nie dodajemy już nic więcej do jego interpretacji (tzw. stałego punktu). Zatem w~koncepcji Kripkego\index[names]{Kripke, Saul} można budować języki zawierające własny predykat ,,jest prawdziwy'', ale logika tych języków nie może pozostawać dwuwartościowa. W~jej obrębie paradoksalne zdania są sensowne, ale nie posiadają ani wartości prawdy, ani fałszu (w żadnym punkcie stałym danego języka). Analiza teologii negatywnej przy wykorzystaniu tego podejścia wydaje się naturalną alternatywą dla przedstawionych powyżej rozważań wykorzystujących rozwiązanie Tarskiego\index[names]{Tarski, Alfred}.

\item Logiki parakonsystentne i~dialeteizm\\
Na początku naszych rozważań ustaliliśmy, że w~racjonalnym dyskursie zasadniczo unika się antynomii, ponieważ systemy zawierające parę zdań, z~których jedno jest negacją drugiego, ulegają przepełnieniu -- można w~ich obrębie dowieść dowolne zdanie. Jest to prawdą dla logiki klasycznej i~przeważającej większości logik nieklasycznych. Istnieją jednak takie rachunki, które nie ,,wybuchają'' napotykając sprzeczności. Z~reguły rachunki te w~jakimś sensie zachowują prawo niesprzeczności ograniczając zasięg działania prawa przepełnienia. Zwyczajowo nazywa się je logikami parakonsystentnymi. Teologia apofatyczna wydaje się naturalnym obszarem stosowania parakonsystentnych analiz. Co zaskakujące, takie propozycje raczej nie pojawiają się w~literaturze przedmiotu\footnote{Mimo iż pewne dialeteiczne wątki pojawiają się w~ramach apofatycznej doktryny i~jej rozmaitych interpretacji. Najlepszym tego przykładem jest koncepcja Boga jako \textit{coincindentia oppositorum} Mikołaja z~Kuzy\index[names]{Mikołaj z~Kuzy}. Należy jednak przyznać, że propozycje, o~których tu mówię, można spotkać w~analizie innych paradoksalnych sytuacji obecnych w~teologii. Zob. np. D. Rybarkiewicz, \textit{Dialeteizm punktem spotkania Boga religii z~Bogiem filozofów? Uwag kilka: więcej pytań niż odpowiedzi}, [w:] \textit{Analiza, racjonalność, filozofia religii. Księga jubileuszowa dedykowana Profesorowi Ryszardowi Kleszczowi}, Wydawnictwo Uniwersytetu Łódzkiego, Łódź 2020, ss.~375-385. Zob. także ogólną uwagę w: J. Dadaczyński, \textit{What kind of logic does contemporary theology need}?, [w:] \textit{Logic in Theology}, red. B. Brożek, A. Olszewski, M. Hohol, Copernicus Center Press, Kraków 2013, s.~50; oraz J. Dadaczyński, \textit{Kilka uwag o logice teologii}, ,,Zagadnienia Filozoficzne w~Nauce'', nr 57 (2014), ss.~33-58.}.

\item Podejścia intensjonalne\\
Duże nadzieje na ciekawe wyniki w~obszarze analiz paradoksalnego charakteru teologii apofatycznej dają podejścia intensjonalne, jak na przykład propozycja rozwiązania antynomii kłamcy autorstwa Piotra Łukowskiego\index[names]{Łukowski, Piotr}\footnote{Zob. P. Łukowski, \textit{Paradoksy}, Wydawnictwo Uniwersytetu Łódzkiego, Łódź 2006, ss.~206-214. Zob. także tenże, \textit{An approach to the liar paradox}, ,,RIMS Kokyuroku'', vol. 1010 (1997), ss.~68-80.}. Łukowski\index[names]{Łukowski, Piotr} rozszerza klasyczny rachunek zdań wprowadzając do jego alfabetu dodatkowy (intensjonalny, binarny) spójnik mający wskazywać istnienie treściowego związku zachodzącego między połączonymi w~ten sposób zdaniami (w tym sensie można uważać, że jest to rozwiązanie intensjonalne). Jeśli spójnik ten oznaczymy średnikiem, wyrażenie $p:p$ przeczytamy jako ,,Zdanie $p$~mówi, że $p$''. Formalna ekspozycja intuicji stojących za tym, co \textit{mówią} zdania każe oczekiwać, że zastosowanie jej do paradoksu, w~którym istotnym składnikiem jest pojęcie wyrażalności czy wysławialności, przyniesie interesujące wyniki.
\end{enumerate}

Mam nadzieję, że powyższe rozważania na temat logicznych aspektów teologii apofatycznej, a~także ujawnienie paradoksalnego charakteru tej doktryny oraz pokazanie, że może ona stać się przedmiotem owocnych badań logicznych, zachęcą do ich prowadzenia.


