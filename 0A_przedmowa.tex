\chapter*{Przedmowa}
\addcontentsline{toc}{chapter}{Przedmowa} 


Umieszczenie formalnych roztrząsań z~zakresu logiki czy matematyki na przeciwległym biegunie względem zagadnień o~charakterze teologicznym zdaje się być w~zgodzie z~rozpowszechnionym przekonaniem. W~świetle podobnych opinii %całkiem
zabawnym może wydać się fakt, że obu tym dyscyplinom przypisywano -- przynajmniej historycznie -- miano ,,królowych'' nauk. Tymczasem historia nauki odnotowuje przypadki, w~których dyscypliny te były związane zarówno unią personalną, jak i~-- jeśli trzymać się tych ,,monarszych'' przenośni -- realną. Trudno jednak oprzeć się wrażeniu, że ze współczesnego punktu widzenia przypadki te należy podzielić z~grubsza na anegdotyczne, epizodyczne lub daleko odległe w~czasie (co, rzecz jasna, do pewnego stopnia tłumaczyć musi zakreślony wyżej \textit{status quo}). Zasadniczo współczesny matematyk nie widzi w~teologii potencjalnego obszaru zastosowań swoich wyników a~współczesny teolog narzędzia formalne uzna raczej za bezużyteczne dla swoich dociekań. Niniejsze opracowanie stoi w~poprzek tym trendom stosując metody właściwe filozofii analitycznej, w~tym metody logiki formalnej, w~celu analizy teologii apofatycznej -- doktryny mówiącej, że o~Bogu nic powiedzieć nie można.

O~ile można mówić o~prawdach częściowych, powiedzieć, że niniejsza książka jest owocem wieloletnich badań, byłoby tylko częściową prawdą. Część zawartych w~niej wyników została zreferowana w~2014 roku na konferencji ,,Is God Beyond Logic?'' w~Louvain-la-Neuve
%, w 2015 roku podczas wykładu gościnnego w Instytucie Filozofii im. Edyty Stein w Granadzie
oraz w~2017 roku w~Krakowie na konferencji ,,Negation'' dedykowanej profesorowi Pavlowi Maternie\index[names]{Materna, Pavel}, a~także przedstawiona w~poświęconym zagadnieniu negacji numerze specjalnym czasopisma ,,Studies in Logic, Grammar and Rhetoric''. W~końcu, wyniki te zebrane w~formie rozprawy zostały złożone w~2022 roku w~ramach przewodu doktorskiego prowadzonego na Wydziale Filozoficznym Uniwersytetu Jana Pawła II w~Krakowie.

Na obecny kształt książki wpływ miało grono osób, którym winny jestem podziękowania. Chciałbym publicznie wyrazić swoją wdzięczność Annie Brożek\index[names]{Brożek, Anna}, Bartoszowi Brożkowi\index[names]{Brożek, Bartosz}, nieodżałowanej pamięci Davidowi Charlesowi McCarty'emu\index[names]{McCarty, David C.}, Adamowi Olszewskiemu\index[names]{Olszewski, Adam}, Wojciechowi Załuskiemu\index[names]{Załuski, Wojciech} oraz dwóm anonimowym recenzentom, których książka uzyskała w~postępowaniu konkursowym ,,Monografie FNP''. Dziękuję także Zarządowi i~Radzie Wydawniczej Fundacji na rzecz Nauki Polskiej za ich decyzję i~rekomendację, wskutek których niniejsza praca została przyjęta do wydania w~ramach tej serii.

Przynajmniej część badań zawartych w~niniejszej monografii nie mogłaby zostać przeprowadzona bez możliwości skorzystania ze stypendium doktorskiego przyznanego mi w~ramach grantu ,,The Limits of Scientific Explanation'' (ID 20237) udzielonego Centrum Kopernika Badań Interdyscyplinarnych przez Fundację Johna Templetona. Bez specyficznego środowiska intelektualnego, które w~poprzednim dziesięcioleciu otaczało Centrum i~jego założyciela -- Michała Hellera\index[names]{Heller, Michał}, praca nie miałby szans powstać w~ogóle.

\begin{flushright}
Piotr Urbańczyk\\
maj 2023
\end{flushright}