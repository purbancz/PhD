
%\part{Aspekt logiczny \textit{sensu largo}}

\chapter{Teologia apofatyczna jako teologia milczenia}\label{sil-general}
%\section{Wprowadzenie}


%\chapter{Wprowadzenie}


Jedna z~najczęściej spotykanych interpretacji teologii negatywnej zwraca szczególną uwagę na boską transcendencję. Interpretacja ta ma swoje źródła w~tym, że dla wielu teologów negatywnych zaprzeczenia, w~takim samym stopniu jak potwierdzenia, nie mogą stanowić odpowiednich środków do opisu Boga. Transcendentny Bóg jest do tego stopnia ,,ponad'' niedoskonałymi pojęciami ludzkiego języka, że w~zasadzie żadnego z~nich nie powinniśmy mu przypisywać. Zatem -- według takiej interpretacji -- teologia negatywna głosi przede wszystkim, że Bóg jest co do zasady niepojmowalny i~niewyrażalny. To podejście ma daleko idące konsekwencje, ponieważ -- skoro nie możemy przypisać Bogu żadnej własności -- powinniśmy zaprzestać mówienia o~nim i~zamilczeć.

%***

Warto odnotować, że podkreślanie transcendencji Boga jest popularną strategią wśród teologów i~filozofów religii nie zawsze utożsamianych wprost z~nurtem apofatycznym. Trudno nie dostrzec takiego podejścia w~pismach wpływowych dwudziestowiecznych przedstawicieli tych dziedzin, takich jak Rudolf Otto\index[names]{Otto, Rudolf}, Karl Barth\index[names]{Barth, Karl} czy Karl Rahner\index[names]{Rahner, Karl}.

Pierwszy z~nich najbardziej znany jest z~analizy doświadczenia, które -- w~jego opinii -- leży u~podstaw jakiejkolwiek religii. To, co jest doświadczane podczas przeżycia religijnego, Otto\index[names]{Otto, Rudolf} określa terminem \textit{numinosum}. Doświadczeniu numinotycznemu towarzyszą dwa komplementarne uczucia: \textit{misterium tremendum} -- uczucie przerażenia, grozy i~lęku, lecz także mocy i~majestatu oraz \textit{misterium fascinans}\- -- uczucie fascynacji i~zachwytu. Jednakże kluczowym dla niniejszych rozważań jest to, że stanowią one \textit{misterium} -- zawierają element tajemnicy. Dla Otto\index[names]{Otto, Rudolf} rzeczywistość sakralna jest czymś całkowicie innym (niem. \textit{ganz Andere}, ang. \textit{wholly Other}) zarówno w~stosunku do świata naturalnego, jak i~człowieka. Radykalna odmienność tej rzeczywistości sprawia, że jest ona niewysłowiona, niedyskursywna i~irracjonalna\footnote{Por. R.A. Rappaport, \textit{Ritual and Religion in the Making of Humanity}, Cambridge University Press, Cambridge 1999, s.~377.}. Można jej doświadczyć, ale nie da się jej wyrazić słowami. Wykracza ona poza możliwości poznawcze i~zdolności językowe człowieka. Mimo tego, że według Otto\index[names]{Otto, Rudolf} doświadczenie numinotyczne jest prawdziwym spotkaniem ze świętością, nie może być ono przetłumaczone na mowę. Wiedza o~bóstwie jest ponad pojęciowym rozumieniem i~opisem, nie jest możliwe, by ująć ją w~pojęciowe kategorie ludzkiego języka\footnote{Na temat relacji pomiędzy doświadczeniem a~językiem religijnym w~teorii Rudolfa Otto\index[names]{Otto, Rudolf} pojawiło się wiele dyskusji. Zob. np. L. Schlamm, \textit{Numinous Experience and Religious Language}, ,,Religious Studies'', vol. 28 (1992), nr 4, ss.~533-551 oraz L.P. Barnes, \textit{Rudolf Otto and the Limits of Religious Description}, ,,Religious Studies'', vol. 30 (1994), nr 2, ss.~219-230.}. Z~tego powodu właściwą reakcją na doświadczenie \textit{numinosum} pozostaje milczenie\footnote{Zob. R.~Otto, \textit{The Idea of the Holy. An Inquiry into the Non-Rational Factor in the Idea of the Divine and Its Relation to the Rational}, tłum. J.W. Harvey, Oxford University Press, London 1923, ss.~216-220.}.

%***

Przyjęcie, że zasadnicza teza teologii negatywnej głosi, że Bóg przekracza wszystko, co możemy o~nim powiedzieć, jest popularną interpretacją pism takich autorów, jak Grzegorz z~Nyssy\index[names]{Grzegorz z~Nyssy, \textit{św}.}, Pseudo-Dionizy Areopagita\index[names]{Pseudo-Dionizy Areopagita}, Augustyn\index[names]{Augustyn@Augustyn, \textit{św}.}, Tomasz z~Akwinu\index[names]{Tomasz z~Akwinu@Tomasz z~Akwinu, \textit{św}.} czy Mistrz Eckhart\index[names]{Eckhart@Mistrz Eckhart}. Wydaje się jednak, że to w~pismach Dionizego\index[names]{Pseudo-Dionizy Areopagita} można odnaleźć najbardziej dosadny wyraz takiego sposobu myślenia o~Bogu.


\section{Pseudo-Dionizyjskie źródła teologii milczenia}\label{sil-dionizy}

Tożsamość Pseudo-Dionizego Areopagity\index[names]{Pseudo-Dionizy Areopagita} nie jest do końca znana. Współcześnie najczęściej przyjmuje się, że był on syryjskim mnichem żyjącym na przełomie V~oraz VI wieku, wywodzącym się ze szkoły neoplatońskiej. On sam przedstawia siebie jako św.~Dionizego, członka ateńskiej rady sądowniczej -- Areopagu, który jako jeden z~nielicznych Ateńczyków nawrócił się pod wpływem przemówienia św. Pawła\footnote{Zob. K. Corrigan, M.L. Harrington, \textit{Pseudo-Dionysius the Areopagite}, [w:] \textit{The Stanford Encyclopedia of Philosophy}, wyd. zima 2019, red. E.N. Zalta, {\textless}\url{https://plato.stanford.edu/archives/spr2015/entries/pseudo-dionysius-areopagite/}{\textgreater} oraz T. Stępień, \textit{Przedmowa}, [w:] Pesudo-Dionizy Areopagita, \textit{Pisma teologiczne}, tom I, Wydawnictwo Znak, Kraków 1997, s.~9. Nawrócenie Dionizego\index[names]{Pseudo-Dionizy Areopagita} na rynku ateńskim jest wydarzeniem biblijnym, opisanym w~\textit{Dziejach Apostolskich}: Dz~17,~32nn.}. W~swoich pismach niejednokrotnie nazywa św. Pawła\index[names]{Paweł@Paweł z~Tarsu, \textit{św}.} swoim nauczycielem\footnote{Trudno nie dostrzec racji, dla których ,,ojciec teologii negatywnej'' przybrał tożsamość ucznia św.~Pawła. Pewne apofatyczne wątki można odnaleźć już w~sformułowaniach biblijnego autora. Paweł w~wielu swoich listach stosuje taką apofatyczną terminologię, jak:
%$\text{\textgreek{>a}}\text{\textgreek{'o}}\rho \alpha \tau o\varsigma $
\textgreek{>a'oratos}
-- niewidzialny (Rz 1, 20; Kol 1, 15; 1~Tm~1,~17; Hbr 11, 27);
%$\text{\textgreek{>'a}}\rho \rho \eta \tau o\varsigma $
\textgreek{>'arrhtos}
-- niewyrażalny, niewysłowiony (2~Kor~12,~4);
%$\text{\textgreek{>a}}\nu \varepsilon \kappa \delta \iota \text{\textgreek{'h}}\gamma \eta \tau o\varsigma $
\textgreek{>anekdi'hghtos}
-- niewysłowiony, nieopisywalny (2 Kor 9, 15);
%$\text{\textgreek{>a}}\pi \rho \text{\textgreek{'o}}\sigma \iota \tau o\varsigma $
\textgreek{>apr'ositos}
-- niedostępny (1~Tm~6,~16) itp. Por. G. Rocca, \textit{Speaking the Incomprehensible God. Thomas Aquinas on the Interplay of Positive and Negative Theology}, The Catholic University of America Press, Washington 2004, s.~8. Warto dodać, że motywem, od którego Paweł rozpoczął swoje kazanie na Areopagu, był ateński ołtarz poświęcony \textit{Nieznanemu} Bogu (Dz 17, 23).},
a~niektóre spośród listów adresuje do jego towarzyszy, Tymoteusza\index[names]{Tymoteusz@Tymoteusz Apostoł, \textit{św}.} i~Tytusa\index[names]{Tytus@Tytus z~Krety, \textit{św}.}, czy nawet Jana Apostoła\index[names]{Jan Apostoł@Jan Apostoł, \textit{św}.}\footnote{Zob. choćby Pseudo-Dionizy Areopagita, \textit{Listy}, [w:] tenże, \textit{Pisma teologiczne}, tłum. M. Dzielska, Wydawnictwo Znak, Kraków 1997, s.~197-207.}. Taką tożsamość chrześcijańskiego autora zaczęto podważać dopiero na przełomie XV i~XVI wieku. Fakt, że Pseudo-Dionizy\index[names]{Pseudo-Dionizy Areopagita} przez nieomal dziesięć wieków cieszył się niezachwianym autorytetem ucznia św. Pawła\index[names]{Paweł@Paweł z~Tarsu, \textit{św}.} sprawił, że to pisma zebrane w~\textit{Corpus Dionysiacum} wywarły największy wpływ na kształtowanie się późniejszej tradycji apofatycznej -- nie tylko w~późnej patrystyce i~średniowieczu, lecz także w~renesansie i~czasach współczesnych. To właśnie Dionizego\index[names]{Pseudo-Dionizy Areopagita} nazywa się ,,ojcem teologii negatywnej'' -- mimo iż myślenie apofatyczne było obecne w~tradycji chrześcijańskiej niemalże od samego początku\footnote{Zob. P. Sikora, \textit{Logos niepojęty}, Wydawnictwo Universitas, Kraków 2010, s.~58. O~teologii apofatycznej przed Dionizym\index[names]{Pseudo-Dionizy Areopagita} przeczytać można: tamże, Rozdziały I-II; C.M. Stang, \textit{Negative Theology from Gregory of Nyssa to Dionysius the Areopagite}, [w:] \textit{The Wiley-Blackwell Companion to Christian Mysticism}, red. J.A. Lamm, Wiley-Blackwell, Malden 2013, ss.~161-176; G. Rocca, \textit{Speaking the Incomprehensible God}\ldots, dz. cyt., rozdział I.}.


\subsection*{Teologia krytyczna}

Według Johna N. Jonesa\index[names]{Jones, John J.}\footnote{J.N. Jones, \textit{Sculpting God: The Logic of Dionysian Negative Theology}, ,,Harvard Theological Review'', vol. 89 (1996), ss.~355–371.}, teologia dionizyjska jest w~dużej mierze teologią krytyczną. Polemizuje ona z~błędnym sposobem mówienia o~Bogu -- takim, który traktuje go jak inne byty, czyli rzeczy lub pojęcia. W~\textit{Teologii mistycznej} Dionizy\index[names]{Pseudo-Dionizy Areopagita} wspomina o~dwóch typach nieporozumień:

\begin{quote}
Mówię tu o~tych, którzy grzęznąc w~bytach nie są zdolni wyobrazić sobie czegoś, co rzeczywiście nadsubstancjalnie istnieje ponad bytami, I~twierdzą, że w~wiedzy, która jest w~nich, płynie znajomość Tego, który wybrał ,,ciemność za swoje schronienie''. Skoro nawet dla tego typu ludzi dostęp do świętych wtajemniczeń nie jest możliwy, to cóż dopiero można powiedzieć o~jeszcze większych profanach, którzy najpośledniejsze spośród bytów poczytują za przekraczającą wszystko najwznioślejszą przyczynę i~zaprzeczają jej wyższości nad ich bezbożnymi idolami o~różnorodnych kształtach\footnote{Pseudo-Dionizy Areopagita, \textit{Teologia mistyczna}, I, 2, [w:] tenże, \textit{Pisma teologiczne}, tłum. M. Dzielska, Wydawnictwo Znak, Kraków 1997, ss.~163-164. O~ile nie podano inaczej, wszystkie poniższe cytaty z~Pseudo-Dionizego Areopagity\index[names]{Pseudo-Dionizy Areopagita} pochodzą z~niniejszego, dwutomowego wydania.}.
\end{quote}

Według Areopagity\index[names]{Pseudo-Dionizy Areopagita}, bałwochwalcy mylą Boga z~przedmiotami, zaś inni ,,profani'' -- prawdopodobnie ma tu na myśli środkowych platoników -- z~pojęciami. W~innym tekście próbuje przedstawić, jak ci ostatni mogliby krytykować wykorzystywanie materialnych obrazów do przestawienia Boga, preferując raczej utożsamianie Boga z~pojęciem lub pojęciami:

\begin{quote}
ktoś [\ldots] mógłby dowodzić, że święci autorzy, chcąc uformować cieleśnie te czyste bezcielesności, powinni je wymodelować i~ukazać pod stosownymi dla nich kształtami im pokrewnymi, na ile to możliwe wzorując się na substancjach najbardziej przez nas cenionych. [\ldots] Tego rodzaju ujęcia lepiej by przecież służyły anagogicznej drodze naszego intelektu i~nie ściągałyby nadprzyrodzonych objawień w~dół, do poziomu absurdalnych niepodobieństw. Tymczasem to postępowanie zdaje się, w~sposób niedopuszczalny, ubliżać boskim mocom i~równocześnie wypacza nasz intelekt, wpędzając go w~pułapkę bezbożnych alegorii\footnote{Pseudo-Dionizy Areopagita, \textit{Hierarchia niebiańska}, II, 2, , [w:] tenże  \textit{Pisma teologiczne II}, tłum. M.~Dzielska, Wydawnictwo Znak, Kraków 1999.}.
\end{quote}

Dionizy\index[names]{Pseudo-Dionizy Areopagita} zgadza się z~teologicznym stanowiskiem, wedle którego materialne obrazy nie mogą przedstawiać boskiej istoty. Jednakże odrzuca on także takie rozwiązanie, wedle którego lepszym sposobem przedstawiania Boga są pojęcia. Zarówno przedmioty, jak i~pojęcia nie stanowią odpowiednich reprezentacji Boga z~tego samego powodu -- ponieważ jest on ponad wszelkim bytem. Wnioskiem, jaki wypływa z~tego obrazu, jest fakt, że język, który służy do opisu bytów, nie może być wykorzystywany do opisu Boga. Skoro Bóg nie należy do kategorii bytów, nie można o~nim mówić w~taki sposób, w~jaki mówi się o~czymkolwiek innym. Zdaniem Jonesa\index[names]{Jones, John J.} konsekwencją negatywnego języka teologii Dionizego\index[names]{Pseudo-Dionizy Areopagita} jest niemożliwość powiedzenia o~Bogu czegokolwiek\footnote{Por. niżej -- rozdz.~\ref{sil-jones}.}.


\subsection*{Apofatyzm kompletny}

Podobną interpretację dzieł Dionizego\index[names]{Pseudo-Dionizy Areopagita} przedstawił jeden z~jego najbardziej znanych komentatorów -- Paul Rorem\index[names]{Rorem, Paul}. Teologię negatywną Areopagity\index[names]{Pseudo-Dionizy Areopagita} -- w~przeciwieństwie do tej, którą można odnaleźć u~Grzegorza z~Nyssy\index[names]{Grzegorz z~Nyssy, \textit{św}.}, Maksyma Wyznawcy\index[names]{Maksym Wyznawca, \textit{św}.} czy Bonawentury\index[names]{Bonawentura, \textit{św}.} -- Rorem\index[names]{Rorem, Paul} nazywa apofatyzmem kompletnym\footnote{P. Rorem, \textit{Negative Theologies and the Cross}, ,,Harvard Theological Review'', vol. 101 (2008), ss.~451-464.}. Twierdzi on, że ostatnie dwa rozdziały najbardziej ,,negatywnego'' dzieła Areopagity\index[names]{Pseudo-Dionizy Areopagita} -- \textit{Teologii mistycznej} -- tłumaczą, na czym polega ,,anagogiczna droga przez negację'' i~należy je odczytywać łącznie. Pierwszy z~nich głosi, że najwyższa przyczyna wszystkich rzeczy postrzegalnych sama nie jest postrzegalna, ten drugi natomiast, że najwyższa przyczyna wszystkich pojęć sama nie ma charakteru pojęciowego\footnote{P. Rorem, \textit{Pseudo-Dionysius. A~Commentary on the Texts and an Introduction to Their Influence}, Oxford University Press, New York -- Oxford 1993, ss.~205-213.}.

W~interpretacji Rorema\index[names]{Rorem, Paul} wspomniane dzieło Dionizego\index[names]{Pseudo-Dionizy Areopagita} ma przede wszystkim wartość duchową, a~jego podstawowym celem jest przedstawianie sposobów służących do zjednoczenia z~Bogiem. Droga ku temu prowadzi najpierw poprzez zanegowanie wszystkich rzeczy postrzegalnych, zwłaszcza wszystkich symboli, które mają wskazywać na najwyższa przyczynę. Dzięki temu wstępujemy na poziom pojęć, które są przez te symbole reprezentowane. Kolejny krok, opisany w~rozdziale piątym \textit{Teologii mistycznej}, polega na zanegowaniu także i~tych pojęć:

\begin{quote}
Wznosząc się coraz wyżej, mówimy, że Bóg nie jest duszą, intelektem, wyobrażeniem, mniemaniem, rozumem i~rozumieniem, słowem i~pojmowaniem; [\ldots] nie jest liczbą, porządkiem, wielkością, małością, równością, nierównością, podobieństwem, niepodobieństwem; nie stoi, nie porusza się, nie odpoczywa, nie posiada mocy i~nie jest ani mocą, ani światłością, nie żyje i~nie jest życiem; nie jest substancją, wiecznością i~czasem; [\ldots] nie jest królem ani mądrością, ani jednią ani jednością, ani Boskością, ani dobrocią, ani duchem (o ile znamy ducha), ani synostwem, ani ojcostwem [\ldots]\footnote{Pseudo-Dionizy Areopagita, \textit{Teologia mistyczna}, V.}.
\end{quote}

Jak zauważa Rorem\index[names]{Rorem, Paul}, wiele z~pojęć, które pojawiają się w~niniejszym fragmencie, służyło Dionizemu\index[names]{Pseudo-Dionizy Areopagita} do określenia Boga w~tych traktatach, w~których pozostawał na poziomie teologii pozytywnej\footnote{Por. np. tenże, \textit{Imiona Boskie}: VI-X.}. Co więcej, zanegowane są tutaj także imiona osób Trójcy Świętej. One także należą do kategorii niedoskonałych pojęć ludzkiego umysłu. Ich znaczenie jest skończone, a~więc ostatecznie nie można ich przypisać nieskończonej naturze Boga. Na tym jednak droga do zjednoczenia z~Bogiem się nie kończy. Cały ten proces osiąga swój szczyt, by w~końcu go przekroczyć w~ostatnich zdaniach najczęściej komentowanego rozdziału \textit{Teologii mistycznej}:

\begin{quote}
[\ldots] nie istnieje ani słowo, ani imię, ani wiedza o~Nim; nie jest ani ciemnością, ani światłością, ani błędem, ani prawdą; nie można o~Nim niczego zaprzeczać ani nic pewnego orzekać, bo twierdząc o~Nim lub zaprzeczając rzeczy niższego rzędu, nic o~Nim nie stwierdzamy, ani nie zaprzeczamy. Ta najdoskonalsza przyczyna wszystkiego jest bowiem ponad wszelkim potwierdzeniem i~ponad wszelkim zaprzeczeniem: wyższa nad wszystko, całkowicie niezależna od wszystkiego i~przenosząca wszystko\footnote{Tenże, \textit{Teologia mistyczna}: V.}.
\end{quote}

Na końcu tej drogi zaprzeczamy nawet samym zaprzeczeniom. Ponieważ negacja także należy do pojęć ludzkiego języka, również przy jej pomocy nie da się uchwycić nieskończonego, transcendentnego Boga. Proces wznoszenia się ku Bogu kończy się w~ciemności niewiedzy. Twierdzenia, a~następnie zaprzeczenia, są jedynie środkami do spotkania Boga, ale ostateczne zjednoczenie z~nim odbywa się nie tylko ponad wszelkim twierdzeniem, ale i~ponad wszelkim zaprzeczeniem. Na tym etapie język nie odgrywa już żadnej roli. Swoje rozważania na ten temat Rorem\index[names]{Rorem, Paul} kończy w~następujący sposób:

\begin{quote}
Według ostatnich słów traktatu ,,Bóg jest ponad wszelkim zaprzeczeniem''. Negacja zostaje zanegowana, a~zamroczony umysł ludzki popada w~milczenie. Traktat, \textit{corpus}, jego autor, a~także niniejszy komentarz nie mają nic więcej do powiedzenia. Pozostaje wyłącznie milczenie\footnote{P. Rorem, \textit{Pseudo-Dionysius. A~Commentary on the Texts}\ldots, dz. cyt., s.~213.}.
\end{quote}


\section{Teologia milczenia -- źródła niewysławialności i~kwestia nazewnictwa}\label{sil-int-nazw}

Powyższe paragrafy pokazują, że kluczowa teza omawianej w~tym rozdziale interpretacji teologii negatywnej -- ilustrowanej najchętniej dziełami Pseudo-Dionizego Areopagity\index[names]{Pseudo-Dionizy Areopagita} -- głosi, że ludzki język jest bezsilny wobec zadania opisu i~wyrażenia transcendentnego Boga. Można próbować wskazać dwie (niewykluczające się) przyczyny takiego stanu rzeczy. Przede wszystkim, może być on spowodowany samymi ograniczeniami ludzkiego języka i~niedostatecznymi zdolnościami poznawczymi człowieka, które czynią go niezdolnym do opisania Boga w~należyty sposób. Stanowisko to jest zdecydowanie mniej popularne. Najpoważniejszym autorem, który reprezentuje taki pogląd, jest John Hick\index[names]{Hick, John}. Twierdzi on, że:

\begin{quote}
boska transkategorialność\footnote{\textit{Transcategriality} -- jest to (dość niezgrabne) określenie, które Hick\index[names]{Hick, John} w~późniejszych pracach stosował zamiennie ze słowem ,,niewysławialność'' (\textit{ineffability}).} nie pociąga za sobą wniosku, że Bóstwo nie posiada żadnej natury, lecz jedynie taki, który mówi, że ta natura nie może zostać ujęta w~ludzkich myślach i~języku, ponieważ niewysławialność odnosi się do zdolności poznawczych poznającego\footnote{J. Hick, \textit{Ineffability}, ,,Religious Studies'', vol. 36 (2000), ss.~41-42.}.
\end{quote}

Z~drugiej strony, bezsilność ludzkiego języka w~staraniach o~podanie opisu Boga może być ugruntowana w~samej naturze Boga i~jego transcendencji. Oznaczałoby to, że Bóg jest niewysławialny ze swojej natury, jest to Jego istotna, ,,wewnętrzna'' własność. Wydaje się, że właśnie takie stanowisko jest zdecydowanie częściej reprezentowane zarówno wśród samych myślicieli apofatycznych, jak i~badaczy zajmujących się tym rodzajem teologii\footnote{Por. T. Dzidek, \textit{Teologia apofatyczna. Uznana bezradność rozumu}, [w:] tenże, \textit{Granice rozumu w~teologicznym poznaniu Boga}, Wydawnictwo M, Kraków 2001, ss.~275-314.}. Peter Kügler\index[names]{Kügler, Peter} nawiązując bezpośrednio do pracy Hicka\index[names]{Hick, John} wyraża przekonanie, że:

\begin{quote}
Z~pewnością niewysławialność Boga jest związana z~poznawczymi ograniczeniami ludzkiego umysłu, ale to \textit{natura} Boga jest taka, że ludzki język nie może jej uchwycić\footnote{P. Kügler, \textit{The meaning of mystical ‘darkness'}, ,,Religious Studies'', vol. 41 (2005), s.~101. Podobne stanowisko zajmuje Christopher Insole\index[names]{Insole, Christopher} -- por. C.J. Insole, \textit{Why John Hick cannot, and should not, stay out of the jam pot}, ,,Religious Studies'', vol. 36 (2000), ss.~28-30, Jonathan Jacobs\index[names]{Jacobs, Jonathan D.} -- por. J.D. Jacobs, \textit{The Ineffable, Inconceivable, and Incomprehensible God: Fundamentality and Apophatic Theology}, [w:] \textit{Oxford Studies in Philosophy of Religion VI}, red. R. Audi i~in., Oxford University Press, New York 2015, s.~165 i~wielu innych -- por. dalsze części niniejszego rozdziału.}.
\end{quote}

W~podobnym duchu wypowiadają się inni autorzy. Na przykład Jonathan D.~Jacobs\index[names]{Jacobs, Jonathan D.} zakłada, że:

\begin{quote}
Teologia apofatyczna nie polega na twierdzeniu, że Bóg jedynie jest trudny do opisania, że z~ogromnym wysiłkiem moglibyśmy go sobie wyobrazić, albo że istnieją tylko pewne prawdy o~Bogu, których nie jesteśmy w~stanie pojąć. To nie jest zwykły chwyt retoryczny. [\ldots] Bóg jest istotowo niewyrażalny\footnote{J.D. Jacobs, \textit{The Ineffable, Inconceivable, and Incomprehensible God}\ldots, dz. cyt., s.~159.}.
\end{quote}
Z~kolei Józef Maria Bocheński\index[names]{Bocheński, Józef Maria} pisze o~,,absolutnej'' naturze boskiej niewyrażalności:

\begin{quote}
Jedną z cech charakterystycznych wszystkich tych teorii [\ldots] jest okoliczność, iż ograniczenia nałożone na znaczenie w~związku z~wykorzystaniem ,,tajemnicy'' i~,,tajemniczości'' są traktowane poniekąd absolutnie, co znaczy, że przypisuje się te ograniczenia samej naturze przedmiotu religii sądząc, iż żaden człowiek nie jest w~stanie ich pokonać [\ldots]\footnote{J.M. Bocheński, \textit{Logika religii}, tłum. S. Magala, Wydawnictwo Naukowe PWN, Warszawa 1993, s.~415.}.
\end{quote}

Ponieważ w~ramach tej interpretacji teologii negatywnej największy nacisk kładzie się na to, że Bóg jest zasadniczo, istotowo i~substancjalnie niewysławialny, nieopisywalny i~niewyrażalny, a~ludzki język nie jest w~stanie w~żaden sposób powiedzieć o~nim czegokolwiek, często nazywa się ją \textit{teorią Niewyrażalnego}, \textit{teorią Niewysłowionego}\footnote{Taką nazwę zaproponował Józef Maria Bocheński\index[names]{Bocheński, Józef Maria} -- zob. tamże, s.~352. Odróżniał on jednak teorię Niewysłowionego od teologii negatywnej \textit{tout court} -- por. tamże, ss.~416-418. Polski tłumacz pracy Bocheńskiego\index[names]{Bocheński, Józef Maria} zostawił ten apofatyczny przymiotnik w~wersji dokonanej, używając sformułowania ,,to, co niewysłowione''. W~oryginale brak możliwości wypowiedzenia czegokolwiek o~Bogu jest mocniej zaznaczony -- Bocheński\index[names]{Bocheński, Józef Maria} pisze o~,,the Unspeakable'' (zapisując ten przymiotnik wielką literą), a~nie o~,,unspoken''. Por. np. J.M. Bocheński, \textit{The Logic of Religion}, New York University Press, New York 1965, ss.~31-36.} lub nawet \textit{teologią milczenia}\footnote{Autorem tego określenia jest George Englebretsen\index[names]{Englebretsen, George} -- zob. G. Englebretsen, \textit{The Logic of Negative Theology}, ,,New Scholasticism'', vol. 47 (1973), s.~232. W~niniejszej pracy tych i~podobnych im nazw będę używał zamiennie.}.


\section{Paradoksalny charakter teologii milczenia}\label{sil-int-par}

Teologia apofatyczna jest często oskarżana o~sprzeczność. Najczęściej wytykanym problemem tej doktryny jest ciążący na niej pewien rodzaj paradoksu samoodniesienia. Nietrudno postawić taki zarzut także teorii Niewysłowionego -- skoro głosi ona, że o~Bogu nie można nic powiedzieć, tym samym sama mówi coś Bogu, a~zatem jest niespójna i~należy ją odrzucić. Michael Durrant\index[names]{Durrant, Michael} paradoks teologii milczenia ujmuje w~następujący sposób:

\begin{quote}
w~tej teorii, mówiąc, że natura Boga jest zasadniczo niewyrażalna, opisujemy właśnie naturę Boga -- jest mianowicie zasadniczo niewyrażalna. Innymi słowy, ci, którzy bronią tego stanowiska, nie mogą tego robić, nie przecząc sobie\footnote{M. Durrant, \textit{The Meaning of ‘God'– I}, [w:] \textit{Religion and Philosophy}, red. M. Warner, Cambridge University Press, Cambridge 1992, s.~74. Cytat w~j. polskim za P. Rojek, \textit{Logika teologii negatywnej}, ,,Pressje'', nr 29 (2012), ss.~222-223.}.

\end{quote}
Podobnie argumentuje John Hick\index[names]{Hick, John}, który uważa, że nie ma sensu

\begin{quote}
mówić o~$X$, że żadne nasze pojęcie się do niego nie stosuje. Jest bowiem w~oczywisty sposób niemożliwe odnosić się do czegoś, co nie posiada nawet własności bycia możliwym przedmiotem odniesienia\footnote{J. Hick, \textit{An Interpretation of Religion. Human Responses to the Transcendent}, Yale University Press, New Haven -- London 1989, s.~239. Cytuję za P. Sikora, \textit{Logos Niepojęty}, Wydawnictwo Universitas, Kraków 2010, s.~118.}.

\end{quote}
Dodaje on także, że określenie

\begin{quote}
,,taki, że nasze pojęcia się do niego nie stosują'' nie może, jeśli chcemy uniknąć paradoksu, odnosić się do własności, którą opisuje\footnote{Tamże.}.
\end{quote}

To właśnie z~paradoksalnym charakterem teorii Niewysłowionego najczęściej mierzą się ci badacze, którzy próbują rozważać jej logiczno-językową strukturę. Warto więc wyjaśnić, co będziemy rozumieć przez sprzeczność, paradoks i~jakie są ich logiczne konsekwencje.

%\begin{defin}[sprzeczność]
%,,Sprzecznością'' lub ,,antynomią'' będę tu nazywał parę zdań, z~których jedno jest negacją drugiego.
%\end{defin}
%\begin{defin}[paradoks]
%Terminem ,,paradoks'' tradycyjnie zwykło się określać twierdzenie, które prowadzi do zaskakujących lub sprzecznych wniosków. Sprzeczność tak rozumianych paradoksów nie musi stanowić antynomii w~powyższym sensie -- może być sprzecznością pozorną, sprzecznością z~tzw. zdrowym rozsądkiem, z~dobrze uzasadnionymi przekonaniami i~wynikającymi z~nich oczekiwaniami\footnote{Zob. P. Łukowski, \textit{Paradoksy}, Wydawnictwo Uniwersytetu Łódzkiego, Łódź 2006, s.~7.} czy z~,,powszechną opinią''\footnote{Zob. A. Cantini. R. Bruni, \textit{Paradoxes and Contemporary Logic}, [w:] \textit{The Stanford Encyclopedia of Philosophy}, wyd. jesień 2021, red. E.N. Zalta, {\textless}\url{https://plato.stanford.edu/archives/fall2021/entries/paradoxes-contemporary-logic/}{\textgreater}.}. W~niniejszej pracy termin ,,paradoks'', o~ile nie zostanie wskazane inaczej, zasadniczo będzie używany na określenie sprzeczności nietrywialnej. Mówiąc krótko, w~paradoksie będziemy mieć do czynienia z~sytuacją, w~której w~obrębie danej teorii, rachunku lub sytemu zarówno pewne zdanie, jak i~zdanie z~nim sprzeczne, wydają się być jednakowo dowiedzione lub przynajmniej w~jednakowy sposób ugruntowane czy uprawnione do utrzymywania.
%\end{defin}
,,Sprzecznością'' lub ,,antynomią'' będę tu nazywał parę zdań, z~których jedno jest negacją drugiego. Terminem ,,paradoks'' tradycyjnie zwykło się określać twierdzenie, które prowadzi do zaskakujących lub sprzecznych wniosków. Sprzeczność tak rozumianych paradoksów nie musi stanowić antynomii w~powyższym sensie -- może być sprzecznością pozorną, sprzecznością z~tzw. zdrowym rozsądkiem, z~dobrze uzasadnionymi przekonaniami i~wynikającymi z~nich oczekiwaniami\footnote{Zob. P. Łukowski, \textit{Paradoksy}, Wydawnictwo Uniwersytetu Łódzkiego, Łódź 2006, s.~7.} czy z~,,powszechną opinią''\footnote{Zob. A. Cantini. R. Bruni, \textit{Paradoxes and Contemporary Logic}, [w:] \textit{The Stanford Encyclopedia of Philosophy}, wyd. jesień 2021, red. E.N. Zalta, {\textless}\url{https://plato.stanford.edu/archives/fall2021/entries/paradoxes-contemporary-logic/}{\textgreater}.}. W~niniejszej pracy termin ,,paradoks'', o~ile nie zostanie wskazane inaczej, zasadniczo będzie używany na określenie sprzeczności nietrywialnej. Mówiąc krótko, w~paradoksie będziemy mieć do czynienia z~sytuacją, w~której w~obrębie danej teorii, rachunku lub sytemu zarówno pewne zdanie, jak i~zdanie z~nim sprzeczne, wydają się być jednakowo dowiedzione lub przynajmniej w~jednakowy sposób ugruntowane czy uprawnione do utrzymywania.
%\enlargethispage{-5\baselineskip}

W~historii myśli paradoksy niejednokrotnie zmuszały do intelektualnych zmagań. W~obliczu nietrywialnych sprzeczności należało odnaleźć błąd ukryty w~dowodzie, dokonać rewizji założeń lub zrekonstruować cały system. Znanym przykładem z~zakresu logiki będzie tutaj paradoks kłamcy, który inspiruje logików do dziś, czy też antynomie Russella\index[names]{Russell, Bertrand}, które doprowadziły do filozoficznych badań nad podstawami matematyki. Fizyka również zna taki inspirujący wpływ ,,paradoksalnych'' eksperymentów myślowych prowadzących do zaskakujących lub sprzecznych wniosków, jak ma to miejsce na przykład w~przypadku tzw. paradoksu bliźniąt czy paradoksu kota Schrödingera\index[names]{Schrödinger, Erwin}\footnote{Fizyczne ,,paradoksy'' często nie zawierają logicznych sprzeczności, ale bez wątpienia można je uznawać za paradoksy w~tym szerszym, ogólnym sensie.}.

Paradoksy samoodniesienia (zwane także paradoksami samoodnoszenia\footnote{Por. Z.~Tworak, \textit{Kłamstwo kłamcy i~zbiór zbiorów. O~problemie antynomii}, ser. \textit{Filozofia i logika}, nr~89, Wydawnictwo Naukowe UAM, Poznań 2004, rozdz. 4.}, samozwrotności\footnote{Por. np. J. Woleński, \textit{Samozwrotność i~odrzucanie}, ,,Filozofia Nauki'', vol. 1 (1993), nr 1, ss.~89-102.}, cyrkularności\footnote{Por, P. Łukowski, dz. cyt, ss.~178-250.} lub, rzadziej, autoreferencji\footnote{Por. np. R. Poczobut, \textit{Paradoksy w~wyjaśnianiu świadomości}, ,,Ethos. Kwartalnik Instytutu Jana Pawła II KUL'', vol. 26 (2013), nr 1(101), ss.~62-80.}) to najczęstsza grupa paradoksów badanych narzędziami logicznymi. Do grupy tej należą najchętniej rozważane tzw. paradoksy semantyczne\footnote{Paradoksy semantyczne czasem nazywane są także ,,syntaktycznymi'' lub, rzadziej, ,,epistemicznymi''. Por. P. Łukowski, dz. cyt., s.~185. W~niniejszej pracy termin ten rozumiem w standardowy, ugruntowany w logice sposób wyznaczony m.in. klasyczną pracą Ramseya\index[names]{Ramsey, Frank P.}, jako paradoksy ,,językowe'' -- takie, w których istotną rolę odgrywają pojęcia oznaczania i odnoszenia się. Zob. F.P. Ramsey, \textit{The Foundations of Mathematics}, ,,Proceedings of the London Mathematical Society'', vol. s2-25 (1926), nr 1, ss. 338-384.} (na przykład paradoks kłamcy), paradoksy teoriomnogościowe\footnote{Ramsey\index[names]{Ramsey, Frank P.} nazywał takie paradoksy ,,logicznymi''. Zob. tamże.} (na przykład paradoks Russella\index[names]{Russell, Bertrand}) czy paradoksy epistemiczne\footnote{Związane z pojęciem wiedzy. Zob. R. Sorensen, \textit{Epistemic Paradoxes}, [w:] \textit{The Stanford Encyclopedia of Philosophy}, wyd. wiosna 2022, red. E.N. Zalta, <\url{https://plato.stanford.edu/archives/spr2022/entries/epistemic-paradoxes/}>.} (na przykład paradoks znawcy). Choć te trzy grupy paradoksów tworzone są w~obrębie odmiennych rachunków, a~ich konsekwencje istotne są dla innych dyscyplin -- odpowiednio dla teorii prawdy, podstaw matematyki i~epistemologii -- mają one wspólną strukturę i~często bada się je przy użyciu podobnych narzędzi logicznych\footnote{Zob. G. Priest, \textit{The Structure of the Paradoxes of Self-Reference}, ,,Mind'', vol. 103 (1994), nr 409, ss.~25-34. Zob. także T. Bolander, \textit{Self-Reference}, [w:] \textit{The Stanford Encyclopedia of Philosophy}, wyd. jesień 2017, red. E.N. Zalta, {\textless}\url{https://plato.stanford.edu/archives/fall2017/entries/self-reference/}{\textgreater}.}. By dostrzec tę strukturę prześledźmy kilka przykładów samozwrotnych paradoksów semantycznych.

%\bigskip
%\noindent
\subsubsection[Paradoks Grellinga-Nelsona]{Paradoks Grellinga-Nelsona\footnote{Zob. K. Grelling, \textit{The Logical Paradoxes}, ,,Mind'', vol. 45 (1936), nr 180, ss.~481-486.}}\index[names]{Grelling, Kurt}\index[names]{Nelson, Leonard}


%\begin{quotation}
Przymiotnik nazwiemy autologicznym, jeśli posiada wyrażoną przez siebie własność -- na przykład ,,polski'', ,,pięciozgłoskowy'', ,,sześciosylabowy'' itp.

Przymiotnik nazwiemy heterologicznym, jeśli nie posiada wyrażanej przez siebie własności -- na przykład ,,chiński'', ,,jednosylabowy'', ,,złożony'' itp.
%\end{quotation}

Czy przymiotnik ,,heterologiczny'' jest autologiczny czy heterologiczny? Jeśli przyjmiemy, że jest on przymiotnikiem autologicznym, to ma własność, którą wyraża, a~więc jest heterologiczny. Jeśli założymy, że jest on przymiotnikiem heterologicznym, to nie posiada on własności, którą wyraża, a~wyraża własność heterologiczności, a~zatem jest autologiczny. W~konsekwencji ,,heterologiczny'' jest przymiotnikiem autologicznym wtedy i~tylko wtedy, gdy jest przymiotnikiem heterologicznym.


%\bigskip
%\noindent
\subsubsection[Paradoks liczb Richardowskich]{Paradoks liczb Richardowskich\footnote{Zob. J. Richard, \textit{The principles of mathematics and the problem of sets}, [w:] \textit{From Frege to Gödel. A~source book in mathematical logic 1879–1931}, red. J. van Heijenoort, Harvard University Press, Cambridge 1967, ss.~142-144. W~oryginalnym sformułowaniu paradoksu Richard\index[names]{Richard, Jules} mówi o~liczbach rzeczywistych i~wykorzystuje metodę przekątniową, ale nie ma to większego znaczenia dla zrozumienia idei zamozwrotności ilustrowanej przedstawianymi tu paradoksami.}}\index[names]{Richard, Jules}


Rozważmy skończone ciągi słów języka naturalnego definiujące arytmetyczne własności liczb naturalnych, na przykład ,,liczba naturalna posiadająca dokładnie dwa dzielniki całkowite'', ,,liczba pierwsza taka, że liczba większa od niej o~2 też jest liczbą pierwszą'', ,,liczba podzielna przez 7'' itp. i~uporządkujmy te definicje w~sposób leksykograficzny przyporządkowując każdej z~nich liczbę naturalną. Załóżmy, że na miejscu $n$-tym znalazła się definicja liczby Richardowskiej\index[names]{Richard, Jules}: ,,liczba naturalna $k$, która nie posiada własności wyrażonej $k$-tą definicją''.

%\begin{quotation}
\smallskip
\bgroup
\def\arraystretch{1.3}%
\noindent
\begin{tabular*}{.9\linewidth}{@{\extracolsep{\fill}}m{0.2\linewidth}m{0.7\linewidth}@{}}
	\centering 1 &  liczba naturalna posiadająca dokładnie dwa dzielniki całkowite\\
	\centering 2 &  liczba pierwsza taka, że liczba większa od niej o~2 też jest liczbą pierwszą\\
	\centering 3 &  liczba naturalna podzielna przez 7 co najmniej siedmiokrotnie\\
	\centering 4 &  liczba naturalna równa sumie swoich dzielników\\
	\centering $\vdots$ & $\vdots$ \\
	\centering n &  liczba naturalna $k$, która nie posiada własności wyrażonej $k$-tą definicją\\
	\centering $\vdots$ & $\vdots$ \\
\end{tabular*}
\egroup
\smallskip
%\end{quotation}

Czy $n$ jest liczbą Richardowską\index[names]{Richard, Jules}? Jeśli odpowiemy twierdząco, to $n$ nie posiada własności wyrażonej $n$-tą definicją, a~zatem nie jest liczbą Richardowską\index[names]{Richard, Jules}. Jeśli odpowiemy przecząco, to $n$ posiada własność wyrażoną $n$-tą definicją, a~jest to definicja liczby Richardowskiej\index[names]{Richard, Jules}. Zatem $n$ jest liczbą Richardowską\index[names]{Richard, Jules} wtedy i~tylko wtedy, gdy $n$ nie jest liczbą Richardowską\index[names]{Richard, Jules}.


%\bigskip
%\noindent
\subsubsection[Paradoks kłamcy]{Paradoks kłamcy\footnote{Paradoks kłamcy jest niekwestionowanym ,,celebrytą'' wśród paradoksów, któremu poświęcono niejedną monografię, pracę zbiorową czy artykuł. Przegląd podejść do rozwiązania tego paradoksu można odnaleźć w~J.C. Beall, M. Glanzberg, D. Ripley, \textit{Liar Paradox}, [w:] \textit{The Stanford Encyclopedia of Philosophy}, wyd. jesień 2020, red. E.N. Zalta, {\textless}\url{https://plato.stanford.edu/archives/fall2020/entries/liar-paradox/}{\textgreater}. Dobry wgląd z~punktu widzenia teorii prawdy dają także: B. Brożek, \textit{Rola paradoksu kłamcy w~konstrukcji logicznych teorii prawdy}, ,,Zagadnienia Filozoficzne w~Nauce'', nr 30 (2002), ss.~48-88 oraz J. Pruś, \textit{Semantyczna teoria prawdy a~antynomie semantyczne}, ,,Rocznik Filozoficzny Ignatianum'', vol 27 (2021), nr 1, ss.~341-363.}}
%\nopagebreak[2]

Chyba najbardziej eleganckim sformułowaniem tego paradoksu jest stwierdzenie \textit{Hoc est falsum}:
\begin{flalign}
& \text{Zdanie $l$ jest fałszywe.} &&\tag{$l$}\label{sil-klamca}
\end{flalign}


%(l) Zdanie (l) jest fałszywe.\label{sil-klamca}

Jaka jest wartość logiczna zdania \ref{sil-klamca}? Załóżmy wpierw, że \ref{sil-klamca} jest prawdziwe. Zatem jest tak, jak \ref{sil-klamca} głosi, a~mówi ono o~sobie, że jest fałszywe. A~więc \ref{sil-klamca} jest fałszywe. Jeśli natomiast założymy, że \ref{sil-klamca} jest fałszywe, to nie jest tak, jak \ref{sil-klamca} głosi, a~mówi ono o~sobie, że jest fałszywe. Skoro tak nie jest, to \ref{sil-klamca} jest prawdziwe. W~konsekwencji \ref{sil-klamca} jest prawdziwe wtedy i~tylko wtedy, gdy jest fałszywe.


%\bigskip
%\noindent
\subsubsection{Paradoks Niewyrażalnego}

Możemy teraz sformatować paradoks teorii Niewysławianego w~sposób przedstawiony w~powyższych ilustracjach:

%\begin{quotation}
W~myśl teologii milczenia jedynym sposobem wyrażenia Boga jest stwierdzenie, że jest on niewyrażalny.
%\end{quotation}

Czy zatem -- zgodnie z~tą interpretacją teologii negatywnej -- Bóg jest, czy nie jest niewyrażalny? Jeśli przyjmiemy, że nie jest on niewyrażalny, to nie ma innego sposobu, by go wyrazić, a~zatem jest niewyrażalny. Jeśli założymy, że jest on niewyrażalny, to wyraziliśmy, że jest niewyrażalny, a~zatem nie jest niewyrażalny. W~konsekwencji Bóg nie jest niewyrażalny wtedy i~tylko wtedy, gdy jest niewyrażalny.

Oczywiście, powyższy paradoks\footnote{Różne aspekty paradoksu niewyrażalności, także poza kontekstem religijnym, przedstawione zostały w: J. Shaw, \textit{Truth, Paradox, and Ineffable Propositions}, ,,Philosophy and Phenomenological Research'', vol. 86 (2013), nr 1, ss.~64-104.} będzie wciąż generował sprzeczność, gdy użyjemy innych przymiotników z~apofatycznego słownika, które negatywni teologowie chętnie przypisują Bogu, jak na przykład ,,nieopisywalny'' czy ,,niewysławialny''.

Warto w~tym miejscu zatrzymać się na chwilę i~wyjaśnić, dlaczego racjonalny dyskurs nie powinien dopuszczać sprzeczności. Co jest złego w~sprzecznościach, że muszą zostać usunięte? Odpowiedź na to pytanie niekoniecznie musi być oczywista. Wydaje się, że najważniejszym argumentem przeciw dopuszczaniu sprzeczności~-- przynajmniej z~punktu widzenia logiki klasycznej i~większości innych użytecznych rachunków logicznych -- jest fakt, że w~systemach, w~których pojawia się para zdań sprzecznych, można dowieść cokolwiek, nawet kompletny nonsens\footnote{Dobry przegląd argumentów za unikaniem sprzeczności -- \textit{notabene} wraz z~próbą ich odrzucenia -- można odnaleźć w: G. Priest, \textit{What so bad about contradictions}?, ,,The Journal of Philosophy'', vol. 95 (1998), nr 8, ss.~410-426.}. Mówiąc bardziej ścisłym językiem, systemy takie ulegają przepełnieniu. Wnioskowanie prowadzące do niedorzeczności sformalizowane jest w~postaci prawa nazywanego \textit{ex contradictione quodlibet} (w literaturze anglosaskiej: \textit{law of explosion}, w~literaturze polskiej: prawo przepełnienia, prawo Dunsa Szkota\index[names]{Duns Szkot, Jan}):
\begin{flalign*}
&  A \to (\neg A \to B). & 
\end{flalign*}

Przedstawiony powyżej związany bezpośrednio z~samozwrotnością paradoks Niewyrażalnego można nazwać ,,wewnętrznym'' paradoksem niniejszej interpretacji teologii apofatycznej. Wynika on bowiem wprost faktu, że w~teologii milczenia Boga próbuje się opisać jako nieopisywalnego. Jednakże -- w~oczywisty sposób -- teoria ta uwikłana jest jeszcze w~inne rodzaje paradoksów. Niektórzy badacze nazywają je paradoksami ,,pośrednimi'' lub ,,zewnętrznymi'' teorii Niewysłowionego. Mogą one przybrać charakter twierdzeniowy lub nietwierdzeniowy.


%\bigskip
%\noindent
\subsubsection{Zewnętrzny twierdzeniowy paradoks teologii milczenia (paradoks niekonsekwentnego apofatyzmu)}

Teologia negatywna powstawała i~była rozwijana w~obrębie wszystkich wielkich tradycji religijnych\footnote{Por. T.D. Knepper, L.E. Kalmanson (red.), \textit{Ineffability: An Exercise in Comparative Philosophy of Religion}, ser. \textit{Comparative Philosophy of Religion}, vol. 1, Springer, Cham 2017.}, a~każda z~tych tradycji posiada swoje święte księgi, \textit{credo} lub po prostu zbiór tez czy dogmatów, które przyjmuje się w~obrębie danego dyskursu religijnego, np. ,,Bóg jest jeden w~trzech osobach'', ,,Allah jest jedynym Bogiem, a~Mahomet jest jego prorokiem'', ,,Reinkarnacja to cykl życia i~ponownych narodzin kierowany karmą'' itp. Inaczej mówiąc, każdy dyskurs religijny zawiera jakiś niepusty zbiór zdań, które mają przypisywać Bogu przedmiotowo-językowe własności. Teologia apofatyczna nigdy nie rozwijała się w~oderwaniu od teologii katafatycznej, lecz raczej w~obecności i~w głębokim związku ze swoją pozytywną odpowiedniczką. Zasadniczo wydaje się, że myśliciele apofatyczni byli głęboko wierzącymi ludźmi (a przynajmniej nie ma większych powodów, by w~to wątpić\footnote{W~literaturze pojawiają się analizy teologii milczenia w~odniesieniu do ateizmu, lecz raczej w~kontekście obrony przed takim zarzutem. Zob. np. R. Pouivet, \textit{Bocheński on divine ineffability}, ,,Studies in East European Thought'', vol. 65 (2013), nr 1-2, ss.~50-51 lub J.D. Jacobs, \textit{The Ineffable, Inconceivable, and Incomprehensible God...}, dz. cyt., ss.~168-171.}) i~starali się zachowywać prawomyślność wiary -- zachowywali i~wyznawali wszystkie tezy i~dogmaty swoich doktryn religijnych. Nawet więcej, utrzymując lub rozwijając apofatyczne stanowiska, wchodzili w~dyskusje, wyrażali silne przekonania i~wydawali sądy dotyczące katafatycznych ustaleń, często tych najważniejszych -- w~zakresie prawomyślności doktryny. Dobrym przykładem są tutaj zwłaszcza chrześcijańscy przedstawiciele teologii milczenia. Ci sami Ojcowie Kościoła, którzy nalegali, by Boga uznawać za niewyrażalnego, niepojmowalnego i~ponad umysłem, potrafili wchodzić w~zażarte spory o~bardzo precyzyjne sformułowania dogmatyczne, jak np. rozróżnianie między
%$\text{\textgreek{<o}}\mu oo\text{\textgreek{'u}}\sigma \iota o\nu $
\textgreek{<omoo'usion}
i~\textgreek{<omoio'usion}
%$\text{\textgreek{<o}}\mu o\iota o\text{\textgreek{'u}}\sigma \iota o\nu $
czy w~spór o~\textit{Filioque}.

,,Zewnętrzny'' paradoks teologii milczenia w~swojej twierdzeniowej postaci polega więc na utrzymywaniu, że Bóg jest nieopisywalny, niewyrażalny i~niepojmowalny przy jednoczesnym twierdzeniu, że jest \textit{jakiś}, na przykład wszechmogący, jeden w~trzech osobach, jest stworzycielem świata itp.

Mimo iż paradoks ten ani nie zawiera, ani bezpośrednio nie generuje sprzeczności, przy odpowiedniej interpretacji może pociągać za sobą parę zdań sprzecznych.


%\bigskip
%\noindent
\subsubsection{Zewnętrzny nietwierdzeniowy (performatywny) paradoks teologii milczenia}

Dyskurs religijny często zawiera wypowiedzi, które są performatywami: modlitwą, wyrażeniami pochwały, czci itd. W~takich aktach mowy wierni często 1)~zwracają się do przedmiotu tych wypowiedzi oraz 2)~przypisują przedmiotowi tych wypowiedzi pewną wartość. Wielu badaczy\footnote{Wśród nich np. J.M. Bocheński, \textit{Logika religii}, dz. cyt., ss.~355-356; S. Gäb, \textit{Languages of ineffability: the rediscovery of apophaticism in contemporary analytic philosophy of religion}, [w:] \textit{Negative Knowledge}, red. S. Hüsch i~in., Narr, Tübingen 2020, ss.~191-206; Por. także dyskusję na temat kategorii językowej terminu ,,Bóg'' poniżej -- rozdz.~\ref{sil-kt-jez}.} twierdzi, że zakłada się tu pewne przedmiotowo-językowe własności, co stoi w~sprzeczności z~główną ideą teologii milczenia. Po pierwsze, trudno zwracać się do czegoś, o~czym wiemy jedynie, że nic o~tym nie można powiedzieć\footnote{Zob. dyskusję w~sekcji \ref{sil-kt-jez} poniżej.}. Po drugie, trudno czemuś takiemu przypisywać jakąkolwiek wartość -- w~zasadzie nie wiedząc, czemu jakaś wartość ma być przypisywana. Jak formułuje to Bocheński\index[names]{Bocheński, Józef Maria}:

\begin{quote}
Niemożliwe byłoby zapewne oddawanie czci, tzn. przypisywanie wartości czemuś, o~czym zakładamy wyłącznie, iż nie da się o~tym nic powiedzieć. Takie coś byłoby dla wypowiadającego się całkowicie pozbawione własności przedmiotowo-językowych. Mógłby to być np. szatan. Nie byłoby absolutnie żadnego powodu, by go wielbić, chwalić itd. Musimy przeto odrzucić teorię tego, co niewysłowione\footnote{J.M. Bocheński, \textit{Logika religii}, dz. cyt., s.~356.}.
\end{quote}
,,Zewnętrzny'' nietwierdzeniowy paradoks teologii milczenia wynika zatem z~tego, że niemożliwe wydaje się zwracanie się w~aktach mowy będących np. wyrażeniami czci do czegoś, o~czym nic nie można powiedzieć.

Oczywiście, bez odpowiedniej, zaawansowanej filozoficznie interpretacji i~formalnej rekonstrukcji powyższy paradoks nie będzie prowadził do pary zdań sprzecznych. Można zatem uznać, że jest to paradoks w~ogólnym, szerszym sensie tego słowa. Niektórzy badacze\footnote{Np. P. Rojek, \textit{Logika teologii negatywnej}, dz. cyt., s.~225.} mówią w~jego kontekście o~niezgodności teologii apofatycznej z~praktyką religijną. Z~tego powodu paradoks ten można także nazwać paradoksem performatywnym, praktycznym lub prakseologicznym.

W~poniższych rozdziałach przedstawię rozmaite próby obrony teologii milczenia i~zachowania jej spójności. W~większości przypadków analizowani przeze mnie autorzy mierzą się z~wynikającym z~samoodniesienia ,,wewnętrznym'' paradoksem tej teorii. W~niektórych pracach pojawiają się jednak także bezpośrednie odwołania  do paradoksów ,,zewnętrznych'' -- w~pewnych przypadkach w~celu obrony apofatyzmu i~przed tym rodzajem paradoksu\footnote{Zob. rozdz.~\ref{sil-jac}.}, w~innych po to, by argumentować przeciw teologii milczenia, mimo przekonania o zachowaniu jej spójności w~zakresie paradoksu Niewyrażalnego\footnote{Zob. rozdz.~\ref{sil-boch}.}.


\section{Uwaga o~kategorii językowej terminu ,,Bóg''}\label{sil-kt-jez}

Do poniższych rozdziałów należy dodać jeszcze jedną uwagę. Używany w~języku naturalnym termin ,,Bóg'' jest dwuznaczny w~sensie jego kategorii językowej. W~analitycznej filozofii Boga nie został jeszcze rozstrzygnięty spór, czy należy rozumieć go jako nazwę własną, czy może stanowi on deskrypcję określoną. Jeśli ktoś jest skłonny uważać, że termin ,,Bóg'' jest nazwą, w~języku formalnym będzie przedstawiał go za pomocą stałej indywiduowej (na przykład $g$). W~przeciwnym wypadku termin ten zostanie wyrażony przy pomocy predykatu $G(x)$ rozumianego jako ,,$x$ jest Bogiem'' (lub -- w~zależności od religijnego kontekstu dyskursu -- ,,$x$ jest bóstwem'', ,,$x$ jest boskie'', ,,$x$ jest absolutem'', ,,$x$ jest ostateczną rzeczywistością'' itp.). W~ostateczności $G(x)$ możemy zdefiniować poprzez iloczyn wszystkich predykatów przypisywanych Bogu przez dane wyznanie wiary (tu oznaczony jako $\phi(x)$):
%$$G(x) \equiv_{\text{def}} \exists x (\phi(x) \land \forall y (\phi(y) \to x = y))\footnote{Por. J.M. Bocheński, \textit{Logika religii}, dz. cyt., §21.1. Oczywiście, taki iloczyn mógłby zostać zastąpiony kwantyfikacją po predykatach w~logice drugiego rzędu.}.$$
\begin{flalign*}
&G(x) \equiv_{\text{def}} \exists x (\phi(x) \land \forall y (\phi(y) \to x = y))\footnotemark. &
\end{flalign*}\footnotetext{Por. J.M. Bocheński, \textit{Logika religii}, dz. cyt., §21.1. Oczywiście, taki iloczyn mógłby zostać zastąpiony kwantyfikacją po predykatach w~logice drugiego rzędu.}
\indent Oba te rozumienia terminu ,,Bóg'' muszą, siłą rzeczy, prowadzić do nieco odmiennych zapisów zdań dyskursu religijnego. Przykładowo, przypisywanie Bogu jakiejś własności będzie różnić się w~obu tych rozstrzygnięciach. Załóżmy, że chcemy stwierdzić, że Bóg jest wszechmocny, a~predykat $W(x)$ oznacza ,,$x$ jest wszechmocny''. Jeśli uznajemy, że termin ,,Bóg'' należy do kategorii nazw, zdanie to zapiszemy po prostu jako
%$$W(g).$$
\begin{flalign*}
		& W(g). &
\end{flalign*}
Jeśli natomiast przypiszemy termin ,,Bóg'' do kategorii deskrypcji określonych, zdanie to przyjmie postać
%$$
%\forall x (G(x) \to W(x)) \text{ lub}$$
%$$\neg \exists x (G(x) \land \neg W(x).$$
\begin{flalign*}
		& \forall x (G(x) \to W(x)) \text{ lub} & \\
		& \neg \exists x (G(x) \land \neg W(x)). &
\end{flalign*}


Z~pozoru wskazana dwuznaczność może wydawać się trywialnym sporem o~notację i~logicznym ,,dzieleniem włosa na czworo''. W~rzeczywistości jednak rzadko zdarza się, by ustalania o~charakterze formalnym miały tak daleko idące konsekwencje filozoficzne, jak ma to miejsce w~przypadku rozstrzygnięcia niniejszej kwestii. Przyjęcie, że jakiś termin należy do kategorii nazw własnych, w~konsekwencji pociąga za sobą przyjęcie istnienia denotowanego przez nazwę obiektu. W~tym wypadku wiązałoby się to z~założeniem istnienia Boga. Drugie z~przedstawionych rozwiązań pozbawione jest już tak silnych zobowiązań o~charakterze ontologicznym\footnote{Co nie oznacza, że pozbawione jest jakichkolwiek wymogów. Jednym z~warunków, jaki powinno ono spełniać, jest warunek niesprzeczności.}. I~choć kwestia ta jest drugorzędna z~punktu widzenia przedmiotu niniejszej pracy -- w~teologii apofatycznej istnienie Boga nie jest podważane, zasadniczo należy zwracać na nią szczególną uwagę przy formalizowaniu dyskursu teologicznego, zwłaszcza w~rozważaniach dotyczących tzw. dowodów za istnieniem Boga. Tak czy owak, jak zauważa Adam Olszewski\index[names]{Olszewski, Adam}, kwesta kategorii językowej terminu ,,Bóg'' ,,nie została [\ldots] ostatecznie rozstrzygnięta, a~nawet wyczerpująco rozważona. Zresztą podejście do kwestii nazw i~deskrypcji posiada wiele różnych wersji i~każda z~nich ma swoje \textit{pro} i~\textit{contra}''\footnote{A. Olszewski, \textit{Pewna krytyka teologii naturalnej}, ,,Analecta Cracoviensia'', vol. 46 (2014), s.~214.}.

Skrajne stanowisko w~tej kwestii zajęli Walter Terence Stace\index[names]{Stace, Terence S.} oraz Janet Martin Soskice\index[names]{Soskice, Janet M.}. Co ciekawe, oboje przeprowadzają swoją argumentację w~kontekście teologii negatywnej i~w związku z~teologią milczenia. Stace\index[names]{Stace, Terence S.} przekonuje, że termin ,,Bóg'' należy rozumieć jako nazwę własną. Zwraca on uwagę na te fragmenty \textit{Kazań} Mistrza Eckharta\index[names]{Eckhart@Mistrz Eckhart}, w~których nazywa on Boga bezimiennym, ponieważ ,,wszystkie nazwy, jakie mu nadaje dusza, pochodzą od jej umysłu''\footnote{Mistrz Eckhart, \textit{Kazanie 80}, [w:] tenże, \textit{Kazania}, tłum. i~oprac. W. Szymona, W~drodze, Poznań 1986, s.~436.} . Stace\index[names]{Stace, Terence S.} argumentuje, że w~celu uniknięcia sprzeczności termin ,,Bóg'' należy w~wypowiedziach Eckharta\index[names]{Eckhart@Mistrz Eckhart} traktować jak nazwę własną, natomiast wszystkie imiona i~nazwy, których przypisywania Bogu Eckhart\index[names]{Eckhart@Mistrz Eckhart} odmawia, odnoszą się do predykatów języka:

\begin{quote}
Każdy logik wie, że dowolna nazwa, dowolne słowo w~dowolnym języku, z~wyjątkiem nazw własnych, oznacza pojęcie lub powszechnik [\ldots]. Ani ,,Bóg'', ani ,,Nirwana'' nie oznaczają pojęć. Oba te terminy są nazwami własnymi. Nie ma sprzeczności, gdy Eckhart\index[names]{Eckhart@Mistrz Eckhart} używa nazwy ,,Bóg'', a~jednocześnie uznaje Go za bezimiennego, ponieważ -- mimo, iż ma On nazwę własną -- nie ma dla Niego nazwy w~sensie słowa oznaczającego pojęcie\footnote{W.T. Stace, \textit{Time and Eternity: An Essay in the Philosophy of Religion}, Princeton University Press, Princeton 1952, s.~24 -- cyt. za W.P. Alston, \textit{Ineffability}, ,,The Philosophical Review'', vol 65, nr 4 (1956), s.~511; cudzysłowy moje. Warto zwrócić uwagę na podobieństwo takiej egzegezy Eckharta\index[names]{Eckhart@Mistrz Eckhart} do \ref{sil-kug-ent} -- zob.~rozdz.~\ref{sil-kugler}. W~kontekście teologii negatywnej problem ten rozważany jest także w: E.Z. Benor, \textit{Meaning and reference in Maimonides' negative theology}, ,,Harvard Theological Review'', vol. 88 (1995), nr 3, ss.~339-360.}.
\end{quote}

W~podobnym tonie wypowiada się Soskice\index[names]{Soskice, Janet M.}\footnote{J.M. Soskice, \textit{Metaphor and Religious Language}, Clarendon Press, Oxford 1985, s.~24.} twierdząc, że Boga możemy tylko nazywać, a~nigdy opisywać. Ona także twierdzi, że kategorią językową terminu ,,Bóg'' jest w~rzeczywistości wyłącznie nazwa własna. Nazwa ta posiada swój desygnat, możemy więc ,,wskazywać na Boga'', ale nie jesteśmy w~stanie podać jego opisu i~przypisać mu żadnej własności.

Takie podejście szybko znalazło krytykę w~pracach m.in. Williama P.~Alstona\index[names]{Alston, William P.}\footnote{Zob. W.P. Alston, \textit{Ineffability}, dz. cyt., ss.~506-522.} oraz Michaela Durranta\index[names]{Durrant, Michael}\footnote{Zob. M. Durrant, \textit{The Meaning of 'God' -- I},
dz. cyt.,
%[w:] Religion and Philosophy, \textit{Royal Institute of Philosophy Supplement}, vol. 31, red. M. Warner, Cambridge University Press, Cambridge 1992,
ss.~71-84. Dyskusji nad słusznością obu sposobów reprezentowania terminu ,,Bóg'' poświęcona jest jego książka: M. Durrant, \textit{The Logical Status of ‘God' and the Function of Theological Sentences}, Macmillan, Edinburgh 1973.}. Zwracają oni uwagę na możliwość weryfikacji poprawności danego odniesienia. W~obrębie takiego stanowiska nie ma żadnego sposobu sprawdzenia, czy dwie różne osoby mówiące o~Bogu mówią o~jednym i~tym samym. Boga nie da się wskazać przez ostensję, ani w~żaden inny sposób. Z~tego powodu, jeśli używamy terminu ,,Bóg'', musimy być w~stanie podać przynajmniej jego minimalną deskrypcję -- w~przeciwnym razie nie wiedzielibyśmy nawet, o~czym mówimy. Durrant\index[names]{Durrant, Michael}, odnosząc się wprost do pomysłu Soskice\index[names]{Soskice, Janet M.}, pyta:

\begin{quote}
Jeśli jednak nie można w~żaden sposób opisać Boga [\ldots], a~jedynie ,,wskazywać na Niego'' poprzez użycie nazwy własnej ,,Bóg'', to jak można (a) twierdzić w~sposób zrozumiały, że się wskazuje na \textit{Niego} -- aby tak twierdzić w~sposób zrozumiały, Bóg musi być już pomyślany jako osoba lub jako byt analogiczny do osoby; (b) twierdzić, że się w~ogóle na cokolwiek ,,wskazuje''? Mogę twierdzić, że wskazuję na coś -- jeśli już używać tego wyrażenia -- tylko wtedy, gdy mogę zaoferować przynajmniej \textit{jakiś} opis tego, na co wskazuję -- w~przeciwnym razie jak mogę twierdzić, że moje wskazywanie było lub jest \textit{skuteczne}? A~jeśli nie potrafię powiedzieć, co stanowiłoby sukces lub porażkę mojego wskazania, to jak mogę w~ogóle mówić o~,,wskazywaniu''\footnote{M. Durrant, \textit{The Meaning of 'God'}\ldots, dz. cyt., s.~73.}?
\end{quote}

W~kontekście powyższego sporu Bocheński\index[names]{Bocheński, Józef Maria} rozpatruje dwie przeciwstawne teorie dotyczące sytuacji poznawczej użytkowników dyskursu religijnego. Według pierwszej z~nich, wierny może bezpośrednio spotkać Boga np. w~akcie oddawania czci. Według drugiej teorii wierny nie ma możliwości bezpośredniego kontaktu z~Bogiem -- żyje on ,,wiarą'', w~,,mroku wiary'', a~Bóg znany mu jest tylko dzięki niektórym predykatom obecnym w~pismach lub \textit{credo} danej religii. Zgodnie z~pierwszą teorią, dla wiernych termin ,,Bóg'' byłby nazwą. W~myśl drugiej teorii byłaby to dla nich deskrypcja. Bocheński\index[names]{Bocheński, Józef Maria} konkluduje, że to ta druga teoria jest bardziej trafnym opisem współczesnego dyskursu religijnego:

\begin{quote}
Mimo braku poważniejszych badań empirycznych w~tym zakresie, wydaje się jednak, że większość wiernych, jakich znamy dzisiaj, nie ma żadnego rzeczywistego doświadczenia Boga w~ogóle. Modlą się i~oddają Mu cześć takiemu, jakim Go znają, a~nic w~ich wypowiedziach nie wskazuje na to, by w~akcie modlitwy czy innych czynnościach religijnych dowiadywali się czegoś więcej o~Bogu niż ze swego \textit{credo}. Ale \textit{credo} zawsze opisuje Boga i~ze swej natury nie może przekazywać wiedzy o~Nim opartej na osobistej znajomości. Zakładając, że tak jest, mamy prawo stwierdzić, co następuje: termin ,,Bóg'', którym posługuje się dzisiaj większość wiernych, jest deskrypcją\footnote{J.M. Bocheński, \textit{Logika religii}, dz. cyt., s.~381.}.
\end{quote}
Przywoływany powyżej Olszewski\index[names]{Olszewski, Adam} uważa, że raczej należy

\begin{quote}
rozumieć ten termin jako deskrypcję określoną (bądź nawet nieokreśloną), gdyż traktowanie go jako nazwy własnej nie pozwala jasno wyjaśnić kwestii z~fundamentalną wieloznacznością, czy też jej wielodenotacyjnością. Natomiast można pojmować termin ,,Bóg'' jako ,,metanazwę'', czyli nazwę deskrypcji określonych, które to deskrypcje denotują Boga, zaś ,,metanazwa'' należy do metajęzyka. [\ldots] Sprawa ta jest ciekawa sama w~sobie i~wymagałaby pogłębionego studium\footnote{A. Olszewski, \textit{Pewna krytyka}\ldots, dz. cyt., ss.~214-215.}.
\end{quote}

Niniejsza praca nie jest miejscem na takie pogłębione studium. Oczywiście, w~poniższych rozdziałach dokonuję formalnej rekonstrukcji pewnych zdań teologii negatywnej, a~zdania te często zawierają naturalno-językowy termin ,,Bóg''. W~tych rekonstrukcjach termin ten interpretowany jest w~taki sposób, by jak najlepiej oddać sens formalizowanej teorii. Jak się okaże, użycie predykatu w~celu oddania słowa ,,Bóg'' jest czasem niemożliwe, nieintuicyjne lub prowadzi wprost do sprzeczności – zwłaszcza w~tych interpretacjach teologii milczenia, które odmawiają przypisywania Bogu jakiegokolwiek predykatu. Z~drugiej strony, w~innych interpretacjach, na przykład w~interpretacji Bocheńskiego\index[names]{Bocheński, Józef Maria}, twierdzi się wprost, że ,,Bóg'' jest deskrypcją. Oczywiście, wiele z~analizowanych stanowisk pozostawia dowolność co do wyboru kategorii językowej tego problematycznego terminu. W~takich sytuacjach albo przedstawię obie alternatywy zapisu analizowanych zdań, albo podam argumenty za wyborem danego sposobu rekonstrukcji w~języku formalnym.


\chapter{Negacja jako ,,zaprzeczenie wszystkich bytów''}\label{sil-jones}

Deklarację podjęcia próby uchronienia teologii apofatycznej przed zarzutami o~logiczną sprzeczność składa między innymi John N. Jones\index[names]{Jones, John J.} w~cytowanym wyżej artykule \textit{Sculpting God: The Logic of Dionysian Negative Theology}\footnote{J.N. Jones, \textit{Sculpting God: The Logic of Dionysian Negative Theology}, ,,Harvard Theological Review'', nr 89 (1996), ss.~355–371.}. Oprócz 1) stawienia czoła sprzecznościom uwikłanym w~doktrynę Pseudo-Dionizego Areopagity\index[names]{Pseudo-Dionizy Areopagita}, Jones\index[names]{Jones, John J.} analizuje fragmenty różnych jego prac po to, by 2) właściwie zinterpretować niejasne wyrażenia zawarte w~\textit{Teologii mistycznej} i~3) odsłonić logiczną strukturę jego apofatycznej wykładni. Sugeruje on, że głównym celem Dionizego\index[names]{Pseudo-Dionizy Areopagita} było zaprzeczenie, że Bóg należy do kategorii bytów. Jones\index[names]{Jones, John J.} uważa, że jego odczytanie doktryny Dionizego\index[names]{Pseudo-Dionizy Areopagita} pozwala w~efekcie utrzymywać sąd o~tym, że -- wbrew powszechnemu mniemaniu -- teologia negatywna jest teorią logicznie spójną.

Fakt, że Bóg przekracza wszelki byt, nadaje strukturę językowi dyskursu teologicznego, bowiem w~takim wypadku nadawanie Bogu jakichkolwiek przymiotów przysługujących bytom jest z~gruntu błędne. Jak wskazuje Jones\index[names]{Jones, John J.}, w~języku naturalnym, gdy ktoś powie ,,$x$ jest białe'', odbiorca tej wiadomości zrozumie także, że $x$~nie jest czerwone. Przypisywanie jakiemuś obiektowi danych własności jest jednocześnie zaprzeczeniem, że posiada on pewne inne własności. Podobnie, w~drugą stronę, gdy ktoś powie, że ,,$x$ nie jest czerwone'', odbiorca może zakładać, że $x$~posiada inne własności -- jest białe, przeźroczyste lub niewidzialne (lecz np. słyszalne). Odbiorca takiej informacji ma prawo zakładać, że istnieje pewna charakterystyka tego obiektu, można mu przypisać pewne własności, mimo że nie będzie wiedział, jakie własności mu rzeczywiście przysługują. Każdy przedmiot posiada jakąś -- taką a~nie inną -- charakterystykę.

Zwykle, gdy mówimy o~rzeczach, twierdzenia i~przeczenia sprzeciwiają się sobie. Według Dionizego\index[names]{Pseudo-Dionizy Areopagita} nie dzieje się tak w~przypadku Boga. Bóg nie jest jednym z~bytów, zatem język służący do opisu bytów nie jest dla Niego właściwy. W~\textit{Imionach Boskich} Dionizy\index[names]{Pseudo-Dionizy Areopagita} przekonuje, że:

\begin{quote}
nie jest tak, że On jest tym, a~nie tamtym, że istnieje w~jakiś jeden sposób, a~w~inny nie, lecz jest wszystkim jako przyczyna wszystkiego, współposiadając i~przedposiadając w~sobie wszelki początek i~kres wszystkich bytów, i~jest ponad wszystkim, istniejąc ponad bytem, wcześniej niż wszystko, co jest. Dlatego też wszystko naraz można o~Nim twierdzić, choć On nie jest żadną rzeczą z~tego wszystkiego: jest wszechkształtny i~wszechpiękny, i~bezkształtny, i~pozbawiony piękna\footnote{Pseudo-Dionizy Areopagita, \textit{Imiona Boskie}, V, 8.}.
\end{quote}

Według interpretacji Jonesa\index[names]{Jones, John J.}, w~języku teologicznym Areopagity\index[names]{Pseudo-Dionizy Areopagita} twierdzenia i~zaprzeczenia należą do odmiennych grup, tworząc odmienne sposoby mówienia o~Bogu. Ponieważ funkcjonują one w~odmienny sposób, nie należy ich ze sobą mieszać. Te pierwsze przedstawiają Boga jako przyczynę wszystkiego, te drugie wyrażają jego transcendencję. Oba sposoby mówienia można stosować naraz zarówno do opisu Boga, jak i~opisu przedmiotów, jednakże w~ten sposób nie zdołamy wyrazić unikalności Boga -- tego, że jest czymś odrębnym od wszystkich bytów.


\section{Twierdzenia}

W~celu uniknięcia stosowania języka, który nie odzwierciedla wyjątkowości Boga, można próbować każde twierdzenie o~Bogu interpretować w~taki sposób, by nie wynikało z~niego żadne zaprzeczenie. Na przykład, można zestawić kilka twierdzeń, które w~języku naturalnym nie mogą służyć do opisania żadnego z~bytów. Jak przekonuje Jones\index[names]{Jones, John J.}, ten sposób mówienia charakteryzują różnorodne imiona nadawane Bogu w~\textit{Imionach Boskich}, takie jak ,,moc sama w~sobie'' czy ,,prawda''. Skoro połączenie tych określeń w~sposób oczywisty nie może odnosić się do żadnego z~bytów, nadaje się ono do wyróżnienia Boga spośród bytów. Dionizy nazywa to teologią pozytywną. Według Jonesa\index[names]{Jones, John J.}, w~orzeczeniach tego typu twierdzeń -- na przykład w~twierdzeniu ,,Bóg jest prawdą'' -- słowo ,,jest'' występuje w~sensie metaforycznym. Bóg jednocześnie posiada, jak i~nie posiada przypisywanych mu w~orzeczniku własności, w~zależności od kontekstu, którym to zdanie jest użyte. Jak wskazuje Jones\index[names]{Jones, John J.}, ten podwójny sens -- tożsamość i~odmienność -- wynika z~roli, jaką twierdzenia odgrywają w~wyrażaniu boskiej przyczynowości. Dla Dionizego\index[names]{Pseudo-Dionizy Areopagita}, tak samo jak dla greckich neoplatoników, przyczyna jest jednocześnie immanentna, jak i~odrębna względem swojego skutku. Z~tego powodu twierdził on, że twierdzenia zawarte w~teologii pozytywnej są metaforycznym sposobem wyrażania odmienności Boga od wszelkich bytów.


\section{Zaprzeczenia ,,jednostkowe'' oraz ,,zaprzeczenie wszystkich bytów''}

Według Jonesa\index[names]{Jones, John J.}, z~,,logicznego'' punktu widzenia w~doktrynie dionizyjskiej zaprzeczenia są bardziej wymagające od twierdzeń. Gdy ktoś stwierdzi, że Bóg jest mocą i~prawdą, unika w~ten sposób pomylenia go z~przedmiotami i~pojęciami, ponieważ żadne z~pojęć i~przedmiotów nie jest jednocześnie mocą i~prawdą. Gdy jednak ktoś oznajmi, że Bóg nie jest ani mocą, ani prawdą, tak naprawdę nie wykluczy w~ten sposób wiele: Bóg wciąż może być ,,lwem'' albo ,,pijakiem'' lub też wieloma innymi rzeczami\footnote{Przykłady te pochodzą od Dionizego\index[names]{Pseudo-Dionizy Areopagita}. Zob. tenże, \textit{Imiona Boskie}, III, 1.}. Zdaniem Jonesa\index[names]{Jones, John J.}, strategią, jaką przyjął Dionizy\index[names]{Pseudo-Dionizy Areopagita}, by odróżnić Boga od bytów, jest używanie wzajemnie sprzecznych zaprzeczeń o~Bogu -- takich, które nie mogą być jednocześnie prawdziwe, gdy orzekamy je o~jakimkolwiek bycie. Z~tego powodu w~ostatnim rozdziale \textit{Teologii mistycznej} Areopagita pisze, że Bóg ani nie jest żywy, ani nie pozostaje bez życia itp. To właśnie taki nietypowy sposób mówienia nadaje teologii Dionizego\index[names]{Pseudo-Dionizy Areopagita} jej paradoksalny charakter i~w~sposób oczywisty łamie prawo wyłączonego środka.

Dionizy\index[names]{Pseudo-Dionizy Areopagita} często powtarza, że Bóg jest zaprzeczeniem wszelkich bytów. Dlaczego zatem pod koniec \textit{Teologii mistycznej} stwierdza, że jest on także ponad wszelkim zaprzeczeniem\footnote{Tenże, \textit{Teologia mistyczna}, V.}? W~przeciwieństwie do wielu autorów, którzy chętnie widzieliby w~tym kolejną sprzeczność apofatycznej doktryny Dionizego\index[names]{Pseudo-Dionizy Areopagita}, Jones\index[names]{Jones, John J.} próbuje odpowiedzieć na to pytanie wyraźnie odróżniając \textit{zaprzeczenia jednostkowe} od \textit{zaprzeczenia wszystkich bytów}\footnote{J.N. Jones, \textit{Sculpting God}\ldots, dz. cyt., ss.~360-363.}. Uważa on, że to rozróżnienie jest kluczowe dla zrozumienia teologii Dionizego\index[names]{Pseudo-Dionizy Areopagita}. Według jego interpretacji w~teologii dionizyjskiej Bóg przekracza wszystko, co można wyrazić jakimkolwiek jednostkowym stwierdzeniem lub zaprzeczeniem, jest ponad każdym z~jednostkowych zaprzeczeń. Natomiast transcendencja Boga polega właśnie na tym, że jest on zaprzeczeniem wszelkich bytów. Zaprzeczenie wszelkich bytów Jones\index[names]{Jones, John J.} nazywa \textit{negacją} i~-- w~przeciwieństwie do twierdzeń i~zaprzeczeń jednostkowych -- uznaje ją za uprzywilejowany sposób mówienia o~Bogu, powołując się na fragment \textit{Hierarchii niebiańskiej}:

\begin{quote}
[\ldots] ale równocześnie w~tych samych Pismach wielbi się w~sposób pozaświatowy Boską Zwierzchność w~objawiających Ją, zupełnie niepodobnych do Niej, przedstawieniach. Opisuje się ją jako Niewidzialną i~Nieskończoną, i~Niepojętą, i~jeszcze innymi imionami, które nie oddają tego, czym Ona jest, a~raczej to, czym Ona nie jest. Dlatego wydaje mi się, że ten sposób -- mówienia o~Niej przez negację -- bardziej odpowiada jej dostojności, bo przecież, zgodnie z~pouczeniami naszej tajemnicy i~świętej tradycji, zwykliśmy twierdzić, że nie istnieje Ona podobnie jak inne byty i~że zupełnie nic nie wiemy o~Jej, nie dającej się pojąć intelektem i~wyrazić żadnym słowem, bezgranicznej nadsubstancjalności\footnote{Pseudo-Dionizy Areopagita, \textit{Hierarchia niebiańska}, II, 3.}.
\end{quote}

Dla Jonesa\index[names]{Jones, John J.} przedstawiona w~ostatnim zdaniu niniejszego fragmentu ,,negacja'' jest wręcz ,,tautologią'' -- Bóg jest niepojmowalny dlatego, że nie możemy pojąć Jego niepojmowalności. Własność przypisywana tutaj Bogu, jak i~uzasadnienie takiego przypisywania są tożsame. Wszystkie zaprzeczenia jednostkowe są ,,logicznymi operacjami'', które wynikają z~tego typu negacji. Na tym polega różnica między tymi dwoma sposobami mówienia o~Bogu -- w~przeciwieństwie do zaprzeczeń jednostkowych, ,,negacja odnosi się do (nie)możliwości poznania i~powiedzenia czegokolwiek o~Bogu. Jest to, jeśli można tak powiedzieć, reguła drugiego rzędu posługiwania się nazwami pierwszego rzędu''\footnote{J.N. Jones, \textit{Sculpting God}\ldots, dz. cyt., s.~368. Cytat w~j. polskim za P. Rojek, \textit{Logika teologii negatywnej}, ,,Pressje'', nr 29 (2012), s.~222.}.


\section{Dyskusja}

Wydaje się, że z~deklarowanych na początku swojej pracy celów Jonesowi\index[names]{Jones, John J.} udało się jedynie dokonać reinterpretacji teologii negatywnej Pseudo-Dionizego Areopagity\index[names]{Pseudo-Dionizy Areopagita}. Według niej teologia dionizyjska polega na radykalnym oddzieleniu Boga od kategorii bytów. Sposób, w~jaki Jones\index[names]{Jones, John J.} odczytuje doktrynę Dionizego\index[names]{Pseudo-Dionizy Areopagita} można streścić w~następującym sformułowaniu: ,,Bóg nie jest bytem, a~zatem nie może zostać poznany i~wyrażony w~taki sposób, w~jaki poznaje i~mówi się o~bytach''\footnote{Tamże, s.~369.}. Jego argumentację za takim odczytywaniem Dionizego\index[names]{Pseudo-Dionizy Areopagita} uznaję za przekonującą i~dostatecznie podpartą źródłami.

W~mojej opinii Jones\index[names]{Jones, John J.} nie był jednak wystarczająco skuteczny w~kwestii próby uchronienia teologii apofatycznej przed zarzutami o~sprzeczność. Można jeszcze bronić tezy, że w~interpretacji Jonesa\index[names]{Jones, John J.} nie ma sprzeczności w~tych sformułowaniach Dionizego\index[names]{Pseudo-Dionizy Areopagita}, w~których twierdzi, że Bóg jest jednocześnie zaprzeczeniem wszelkich bytów oraz jest ponad wszelkim zaprzeczeniem. Spójność jest tutaj zachowana zarówno dzięki odpowiedniemu rozumieniu słowa ,,wszelkie'', jak i~na oddzieleniu dwóch pojęć: zaprzeczeń jednostkowych oraz zaprzeczeń wszelkich bytów (negacji). Jednakże za niewystarczające uważam argumenty, które mają służyć do wykazania spójności tego, co Jones\index[names]{Jones, John J.} nazywa zaprzeczeniami jednostkowymi. Ostatni rozdział \textit{Teologii mistycznej} przepełniony jest takimi sformułowaniami Dionizego\index[names]{Pseudo-Dionizy Areopagita}, w~których twierdzi, że Bóg nie jest ani podobieństwem, ani niepodobieństwem, nie znajduje się w~ruchu, ani w~bezruchu itp. Według Jonesa\index[names]{Jones, John J.} łącznik ,,jest'' występujący w~tego typu sformułowaniach nie został użyty w~sensie literalnym, lecz metaforycznym, a~zatem każdy, kto chce oskarżyć Dionizego\index[names]{Pseudo-Dionizy Areopagita} o~sprzeczność, powinien wyjaśnić metaforyczny sens tego wyrażenia i~dopiero wtedy wykazać sprzeczność\footnote{Tamże, s.~365.}. Pomijając fakt, że takie przerzucenie odpowiedzialności dowodu (mimo deklaracji zawartych we wstępie) jest cokolwiek nieuczciwe, trudno wskazać na tyle słabe i~,,metaforyczne'' rozumienie słowa ,,jest'', by po przypisaniu dwóch przeciwnych własności temu samemu obiektowi nie produkowało ono sprzeczności. Można jeszcze próbować obronić stanowisko Jonesa\index[names]{Jones, John J.} powołując się na fakt, że w~jego interpretacji język zaprzeczeń jednostkowych jest podporządkowanym sposobem mówienia o~Bogu, który może zostać właściwe określony dopiero w~języku ,,negacji''. Ale to właśnie owe negacje produkują najbardziej typowe dla teologii negatywnej paradoksy. Skoro negacja jest właściwym sposobem mówienia o~Bogu, jak może polegać na niemożliwości powiedzenia o~nim czegokolwiek? Nieprzekonujące są też tłumaczenia Jonesa ,,tautologiczności'' negacji Dionizego\index[names]{Pseudo-Dionizy Areopagita} -- skoro nie pojmujemy niepojmowalności Boga, skąd wiemy, że jest niepojmowalny? W~końcu, skoro negacja dotyczy niemożliwości powiedzenia czegokolwiek o~Bogu, co nas uprawnia do tego, by nazywać Go niepojmowalnym? Tego typu pytania można mnożyć dla każdej ,,negatywnej'' własności przypisanej Bogu w~\textit{Corpus Dionysiacum}. Zresztą Jones\index[names]{Jones, John J.} sam przyznaje, że wielu myślicieli jego odczytanie teologii negatywnej uzna za twierdzenie czegoś o~Bogu -- pewien rodzaj wiedzy i~mówienia o~nim\footnote{Tamże, s.~369.}, wskazując tym samym paradoksalny charakter takiej interpretacji.

Jednakże, z~punktu widzenia niniejszej pracy, artykuł Jonesa\index[names]{Jones, John J.} najbardziej traci na tym, że -- mimo początkowych deklaracji -- nie dokonał on logicznej rekonstrukcji apofatycznej wykładni Areopagity\index[names]{Pseudo-Dionizy Areopagita}. Niestety, w~swoich rozważaniach Jones\index[names]{Jones, John J.} nawet nie próbuje korzystać ze środków logiki formalnej. Co prawda, w~tekście można odnaleźć nieliczne pojęcia właściwe dla rozważań tego typu -- takie jak ,,tautologia'', ,,negacja'' czy ,,reguła''. Nietrudno jednak zauważyć, że są one używane w~sensie
oderwanym
%często bardzo dalekim
od tego znanego z~roztrząsań logicznych. Żaden z~tych terminów nie został jednak należycie zdefiniowany. Ktoś, kto chciałby odsłonić logiczną strukturę teologii apofatycznej bazując na tej interpretacji, mógłby ostatecznie skorzystać z~podziału na trzy sposoby mówienia o~Bogu, który Jones\index[names]{Jones, John J.} wyróżnił w~pismach Dionizego\index[names]{Pseudo-Dionizy Areopagita} (i nazwał twierdzeniami, zaprzeczeniami jednostkowymi oraz zaprzeczeniem wszelkich bytów, czyli ,,negacją''). Problem jednak w~tym, że wywód Jonesa\index[names]{Jones, John J.} jest tylko nieco bardziej klarowny od pism Dionizego\index[names]{Pseudo-Dionizy Areopagita}, które są przedmiotem jego interpretacji, i~taka praca wymagałaby wpierw odpowiedniej interpretacji jego interpretacji (\textit{sic}!).


\chapter{Słaba teoria Niewysłowionego}\label{sil-slabatn}

Próbę obrony teorii Niewysłowionego przed zarzutami o~sprzeczność podjął także jeden z~najbardziej wpływowych współczesnych myślicieli odwołujących się do nurtu apofatycznego -- John Hick\index[names]{Hick, John}\footnote{Por. P. Sikora, \textit{Logos Niepojęty}, Wydawnictwo Universitas, Kraków 2010, s.~118.}. Hick\index[names]{Hick, John} co prawda nawiązuje wprost do doktryny Pseudo-Dionizego Areopagity\index[names]{Pseudo-Dionizy Areopagita}\footnote{J. Hick, \textit{Ineffability}, ,,Religious Studies'', vol. 36 (2000), ss.~37-40.}, ale jego punktem wyjścia bynajmniej nie jest interpretacja dzieł średniowiecznego filozofa, lecz studium religii jako takiej\footnote{Zob. J. Hick, \textit{An Interpretation of Religion. Human Responses to the Transcendent}, Yale University Press, New Haven -- Londyn 1989.}. Celem, jaki przyświeca jego badaniom, jest uzasadnienie fenomenu religijnego pluralizmu\footnote{Zob. tamże, rozdział 14.}.

Przyjmując taki punkt wyjścia, Hick\index[names]{Hick, John} na określenie rzeczywistości transcendentnej nie używa terminu ,,Bóg'', czy ,,bóstwo'', lecz ,,Rzeczywiste'' (\textit{the Real})\footnote{Polskie tłumaczenie terminologii Hicka\index[names]{Hick, John} podaję za P. Sikora, \textit{Logos Niepojęty}, dz. cyt.}. W~swojej teorii religijnego pluralizmu posługuje się on kantowskim rozróżnieniem na Rzeczywiste samo w~sobie, \textit{an sich} (\textit{noumenon}) oraz Rzeczywiste, jakie jest doświadczane i~pojmowane w~rozmaitych religiach (\textit{phenomenon}). Boska rzeczywistość doświadczana przez liczne wspólnoty religijne może zostać ujęta w~pojęcia i~kategorie ludzkiego języka. Może być na przykład osobowym Bogiem, stwórcą świata lub bezosobowym absolutem, jaki można spotkać w~niektórych tradycjach religijnych. Natomiast Rzeczywiste samo w~sobie jest bezwzględnie niewysławialne.

\begin{quote}
Przez ,,niewysławialne'' rozumiem [\ldots] posiadanie takiej natury, która jest poza zasięgiem siatki ludzkich pojęć. Dlatego o~Rzeczywistym samym w~sobie nie można w~sposób uprawniony twierdzić, że jest osobowe lub bezosobowe, celowe lub niecelowe, dobre lub złe, substancją lub procesem, czy nawet że jest jedno lub jest ich wiele. Jednakże zaprzeczając na przykład temu, że Rzeczywiste jest osobowe, nie twierdzi się tym samym, że jest bezosobowe, lecz raczej, że taka polaryzacja pojęciowa czy dualizm nie ma do niego zastosowania\footnote{J. Hick, \textit{A~Christian Theology of Religions: The Rainbow of Faiths}, Westminster John Knox Press, Louisville 1995, ss.~27-28.}.
\end{quote}

Jednakże, jak wykazałem powyżej\footnote{Por. rozdz.~\ref{sil-int-par}.}, Hick\index[names]{Hick, John} zdaje sobie sprawę z~paradoksu, do jakiego prowadzi takie podejście. Ma świadomość, że próba opisu czegoś, co z~zasady jest niewysławialne lub inaczej: odnoszenie się do czegoś, co z~zasady nie może być przedmiotem odniesienia, jest pozbawione sensu. Takie podejście Hick\index[names]{Hick, John} nazywa silną teorią Niewysłowionego (\textit{strong ineffability}). W~zamian za nią proponuje jej słabszą wersję.


\section{Własności substancjalne i~formalne}

Kluczowym elementem słabej teorii Niewysłowionego Hicka\index[names]{Hick, John} jest rozróżnienie pomiędzy dwoma rodzajami własności: 1) własnościami \textit{substancjalnymi} (treściowymi)\footnote{W~\textit{A Christian Theology of Religions}\ldots Hick\index[names]{Hick, John} nazywa te własności wewnętrznymi (\textit{intrinsic}).\label{przypis-hick-wew}} oraz 2) ,,generowanymi logicznie'' własnościami czysto \textit{formalnymi}. Słaba teoria Niewysłowionego, którą -- zdaniem Hicka\index[names]{Hick, John} -- reprezentowali teologowie apofatyczni wszystkich większych religii, polega na stwierdzeniu, że Bogu (Rzeczywistemu) nie można przypisać żadnej własności substancjalnej. Dopuszcza ona jednak możliwość przypisywania mu własności formalnych. Ponieważ ,,żaden konkretny opis, który ma zastosowanie w~dziedzinie ludzkiego doświadczenia nie może zostać zastosowany w~sposób dosłowny do jego niedoświadczalnej podstawy'', o~Rzeczywistym samym w~sobie możemy jedynie formułować ,,pewne czysto formalne wypowiedzi''\footnote{J. Hick, \textit{An Interpretation of Religion}\ldots, dz. cyt., s.~246.}.
%WNT W~przeciwieństwie do własności formalnych, Bogu (Rzeczywistemu) nie można przypisać własności substancjalnych.
\begin{flalign*}
		& \parbox[t]{.87\linewidth}{ 
		Bogu (Rzeczywistemu) nie można przypisać własności substancjalnych,
		(w~przeciwieństwie do własności formalnych, które mu przysługują).} &\tag{WNT}\label{sil-hick-wnt}
	\end{flalign*}

Niestety, Hick\index[names]{Hick, John} nie podaje żadnej definicji, która pozwalałaby ostro odróżnić jedne własności od drugich, a~czytelnika pozostawia wyłącznie z~krótką listą obu rodzajów takich predykatów:

\begin{itemize}
\item własności substancjalne: ,,jest dobry'', ,,jest potężny'', ,,posiada wiedzę''.
\item własności formalne: ,,jest przedmiotem odniesienia jakiegoś terminu'', ,,jest taki, że nasze substancjalne pojęcia się do niego nie stosują''\footnote{Tamże, s.~239. Por. P. Sikora, \textit{Logos Niepojęty}, dz. cyt., s.~119-120.}.
\end{itemize}
Takie podejście nie tylko otworzyło drogę dla różnych interpretacji jego teorii, lecz także bezpośrednio naraziło ją na rozmaitą krytykę\footnote{Według Hicka\index[names]{Hick, John} w~ciągu piętnastu lat od opublikowania \textit{An Interpretation of Religion}\ldots powstało ponad sto trzydzieści artykułów i~około stu książek zawierających głosy krytyczne na temat jego pracy -- por. J. Hick, \textit{Introduction to the Second Edition}, [w:] tenże, \textit{An Interpretation of Religion}\ldots, dz. cyt., s.~xvii.}.

William Rowe\index[names]{Rowe, William}\footnote{W.L. Rowe, \textit{Religious pluralism}, ,,Religious Studies'', vol. 35 (1999), ss.~139-150.} -- tak jak większość głosów krytycznych -- podkreśla brak rzetelnego zdefiniowania tak ważnych dla tej koncepcji pojęć. Bazując na pracy Hicka\index[names]{Hick, John} próbuje on podać swoje własne definicje. Według niego ,,własność formalna Rzeczywistego jest pewną \textit{abstrakcyjną} charakterystyką posiadaną przez Rzeczywiste, która jest warunkiem możliwości naszego odnoszenia się do niego lub postulowania go jako tego, co spotykane za pośrednictwem osobowych bóstw lub nieosobowych absolutów wielkich religii świata''. Natomiast własność substancjalna Rzeczywistego to ,,istotna (istotowa) własność należąca do jego natury''\footnote{Tamże, s.~145. Pierwszy cytat za: P. Sikora, \textit{Logos Niepojęty}, dz. cyt., s.~120. Warto zauważyć, że w~niektórych swoich pracach Hick\index[names]{Hick, John} własności substancjalne nazywa własnościami wewnętrznymi -- zob. przypis \ref{przypis-hick-wew}.}.

Podobny zarzut przeciwko słabej teorii Niewysłowionego Hicka\index[names]{Hick, John} wysuwa Christopher Insole\index[names]{Insole, Christopher}. On także uważa, że pojęcia własności formalnych i~substancjalnych nie zostały dostatecznie dobrze określone. Odrzuca on jednak koncepcję Rowe'a\index[names]{Rowe, William} twierdząc, że terminy użyte w~jego definicjach -- \textit{abstrakcyjny} oraz \textit{istotowy} -- nie są wcale mniej tajemnicze, niż terminy \textit{formalny i~substancjalny} obecne w~oryginalnej teorii Hicka\index[names]{Hick, John}. W~zamian za nie proponuje swoje definicje. Według niego własność formalna to taka własność, która ,,wyłącznie i~bezpośrednio określa, jakie \textit{inne} własności mogą (lub nie) być przypisane przedmiotowi''\footnote{C.J. Insole, \textit{Why John Hick cannot, and should not, stay out of the jam pot}, ,,Religious Studies'', vol. 36 (2000), s.~28.}. Natomiast własności substancjalne nie zawierają bezpośrednio takiej informacji, choć niektóre z~nich -- takie, jak ,,wszechmocny'', ,,wszechmogący'' czy ,,wszechwiedzący'' -- mogą wymagać współwystępowania.

Jeszcze inaczej pojęcie własności formalnej rozumie Alvin Plantinga\index[names]{Plantinga, Alvin}. W~jego odczytaniu są to takie własności, które posiada \textit{wszystko} i~-- ponadto -- wszystko posiada je \textit{z~konieczności}. Aby własność została uznana za formalną, powinna ona spełniać oba te warunki. Przykładem będzie tutaj: ,,jest tożsamy z~samym sobą'', ,,posiada własności'', ,,posiada własności istotowe'', ,,jest taki, że 7+5=12''\footnote{Zob. A. Plantinga, \textit{Warranted Christian Belief}, Oxford University Press, New York 2000, s.~47.}.


\section{Dyskusja}

Jednakże to nie na niezdefiniowaniu kluczowych pojęć polega główny zarzut, jaki wyżej wymieni autorzy wysuwają przeciw słabej teorii Niewysłowionego. Hick\index[names]{Hick, John} twierdzi, że z~niewysławialności Rzeczywistego wynika, że nie można o~nim orzekać pewnych par własności, np. nie można powiedzieć o~nim, że jest dobry lub zły, czy że jest osobowy lub nieosobowy. Zdaniem Rowe'a\index[names]{Rowe, William}, Hick\index[names]{Hick, John} eksplikując w~taki sposób działanie swojej teorii używa dwóch rodzajów par własności:

\begin{itemize}
\item własności przeciwnych (np. dobry -- zły, czerwony -- zielony);
\item własności sprzecznych (np. osobowy -- nieosobowy, formalny -- nieformalny).
\end{itemize}
Tak jak zdania przeciwne nie mogą być jednocześnie prawdziwe -- tak własności przeciwnych nie można przypisać jednocześnie temu samemu obiektowi. Nie wyklucza to jednak sytuacji, w~której ani jedna, ani druga własność z~pary własności przeciwnych nie przysługuje danemu obiektowi, np. woda w~jeziorze może nie być gorąca ani zimna (lecz letnia), albo cytryna może nie być czerwona ani zielona (lecz żółta). Natomiast własności sprzeczne obarczone są silniejszym kryterium. Tak jak zadania sprzeczne nie tylko nie mogą być jednocześnie prawdziwe, lecz także nie mogą być jednocześnie fałszywe -- podobnie zawsze jedna z~pary własności sprzecznych musi danemu obiektowi przysługiwać. A~zatem główny zarzut Rowe'a\index[names]{Rowe, William} przeciw słabej teorii Niewysłowionego polega na tym, że nie widzi on sposobu, w~jaki Rzeczywiste może uniknąć posiadania jednej z~pary własności sprzecznych\footnote{W.L. Rowe, \textit{Religious pluralism}, dz. cyt., s.~146.} lub -- inaczej mówiąc -- że teoria ta nie pokazuje, w~jaki sposób i~dlaczego ogranicza się w~niej działanie prawa wyłączonego środka.


\begin{figure}[H]
\begin{center}

 \begin{tikzpicture}[node distance=5cm]

    \node (A) {\begin{tabular}{c} $P(a)$ \\ \end{tabular}};
    \node (E) [right=of A, xshift=2cm] {\begin{tabular}{c} $Q(a)$ \end{tabular}};
    \node (I) [below=of A] {\begin{tabular}{c}  $\neg Q(a)$ \end{tabular}};
    \node (O) at (I-|E) {\begin{tabular}{c}  $\neg P(a)$ \end{tabular}};

    \coordinate (CENTER) at ($(A)!0.5!(O)$);

    \node (contra) at (CENTER) {\begin{tabular}{c}Własności sprzeczne \end{tabular}};
    \path[-] (A) edge node[] {\begin{tabular}{c}Własności przeciwne\\ \ \end{tabular}} (E);
    \path[-] (I) edge node[] {\begin{tabular}{c}\ \\Własności podprzeciwne \end{tabular}} (O);
    \path[->] (A) edge node[rotate=90] {\begin{tabular}{c}Założenie\\ \ \end{tabular}} (I);
    \path[->] (E) edge node[rotate=-90] {\begin{tabular}{c}\ \\ \ \end{tabular}} (O);

    \path[-] (contra) edge (A);
    \path[-] (contra) edge (E);
    \path[-] (contra) edge (I);
    \path[-] (contra) edge (O);

\end{tikzpicture}

\caption[,,Kwadrat logiczny'' własności]{,,Kwadrat logiczny'' własności. Zależności zachodzą dla dwóch własności z~jednej domeny (np.~domeny koloru) lub po prostu dla dowolnych własności $P$ oraz $Q$, dla których zachodzi $\forall x (P(x) \to \neg Q(x))$.}\label{sil-hic-kwadrat}
\end{center}
\end{figure}

By zobrazować swoje wątpliwości, Rowe\index[names]{Rowe, William} podaje przykład pary własności sprzecznych: zielony -- nie-zielony i~liczby dwa. Czy liczba dwa może być zielona lub nie-zielona? Jeśli mielibyśmy pozostać w~obrębie koncepcji Hicka\index[names]{Hick, John}, należałoby stwierdzić, że -- ponieważ liczbom nie można przypisać własności koloru -- liczba dwa nie jest ani zielona, ani nie-zielona. Rowe\index[names]{Rowe, William} stanowczo odrzuca taki pogląd. Uważa on, że -- ponieważ nie jest możliwe, by liczba dwa była zielona -- jest ona z~konieczności nie-zielona, a~takie stwierdzenie wcale nie pociąga za sobą tezy, że liczbom można przypisywać własności koloru\footnote{Zob. tamże, ss.~147-149.}.

Podobny zarzut przeciw pomysłowi Hicka\index[names]{Hick, John} wysuwa Alvin Plantinga\index[names]{Plantinga, Alvin}:

\begin{quote}
Jeśli Hick\index[names]{Hick, John} uważa, że żaden z~naszych terminów nie może zostać użyty dosłownie na określenie Rzeczywistego, to nie jest możliwe, by to, co mówi, miało jakikolwiek sens. Zakładam, że termin trzykołowy nie przysługuje Rzeczywistemu; Rzeczywiste nie jest trzykołowe. Ale jeśli Rzeczywiste nie jest trzykołowe, to własność ,,nie jest trzykołowe'' przysługuje mu w~sposób dosłowny; jest nie-trzykołowe. Trudno byłoby nie być ani trzykołowym, ani nie nie-trzykołowym, ani nie uważam, że Hick\index[names]{Hick, John} chciałby sugerować, że jest to możliwe\footnote{A. Plantinga, \textit{Warranted Christian Belief}, dz. cyt., s.~45. Ocena apofatyzmu Hicka\index[names]{Hick, John} dokonana przez Plantingę\index[names]{Plantinga, Alvin} jest tak samo rozbudowana, co surowa -- por. tamże, ss.~43-63. Należy jednak stwierdzić, że uczynił on z~koncepcji Hicka\index[names]{Hick, John} pewien rodzaj słomianej kukły, w~wielu miejscach krytykując go za poglądy, których ten nigdy nie utrzymywał.}.
\end{quote}

Ponadto nie jest pewne, czy sama para drugorzędowych własności ,,substancjalny -- formalny'' jest parą własności przeciwnych, sprzecznych lub zupełnie od siebie niezależnych (z którą mamy do czynienia, gdy na przykład dzielimy kwiatki czerwone i~pachnące). Pozostawienie czytelnika z~prostym wyliczeniem kilku własności substancjalnych i~formalnych sprawia, że można niesprzecznie założyć, że istnieją własności, które nie są ani substancjalne, ani formalne. Podobnie, nietrudno wyobrazić sobie własność, która jest jednocześnie substancjalna, jak i~formalna -- choćby, nie szukając daleko, ,,jest sprzeczny'' oraz ,,jest taki, że można orzec o~nim dwa prawdziwe sądy, z~których jeden jest negacją drugiego''\footnote{Nawet jeśli komuś ten przykład nie przypadnie do gustu, to można niesprzecznie założyć, że istnieją takie własności.}. Inaczej mówiąc, nie jest jasne, czy podział zaproponowany przez Hicka\index[names]{Hick, John} jest pełny i~wyczerpujący, tj. dzieli zbiór własności na podzbiór właściwy i~jego dopełnienie. Jeśli istnieje jakaś własność substancjalna i~formalna zarazem, utrzymywanie WNT traci sens.

Zupełnie odmienny zarzut wobec pomysłu Hicka\index[names]{Hick, John} przedstawił Insole\index[names]{Insole, Christopher}. Uważa on, że ograniczenie mówienia o~Bogu wyłącznie do wypowiedzi o~charakterze czysto formalnym jest bezzasadne, ponieważ aby przypisać Rzeczywistemu samemu w~sobie jakąkolwiek własność formalną, musimy wpierw posiadać wiedzę na temat jego własności substancjalnych. W~rzeczywistości, przypisywanie jakiemukolwiek przedmiotowi własności formalnych wymaga od podmiotu posiadania dużo większej wiedzy niż przypisywanie mu wyłącznie własności substancjalnych. Skoro własność formalna mówi o~(nie)możliwości przypisania danemu przedmiotowi innych własności, aby móc ją w~sposób uprawniony orzec o~tym przedmiocie, musimy wpierw ustalić:

\begin{enumerate}[label = (\arabic*)]
\item jego ontologiczny typ (obiekt fizyczny, fikcyjny, rzeczywistość boska itp.),
\item jego ontologiczną naturę (prosta, złożona, osobowa, transcendentna, immanentna itp.),
\item nasz dostęp poznawczy do przedmiotów tego typu oraz
\item jakie typy własności można danemu przedmiotowi przypisywać w~sposób uprawniony na podstawie tego, co wiemy o~(1), (2) i~(3).
\end{enumerate}
Tymczasem do orzekania o~danym przedmiocie własności substancjalnych nie potrzeba posiadać tego rodzaju wiedzy. Wystarczy wiedzieć, że np. ,,ta sukienka jest czerwona'' lub że ,,Sherlock Holmes był bystry''\footnote{Zob. C.J. Insole, \textit{Why John Hick cannot}\ldots, dz. cyt., ss.~19-20. Por. P. Sikora, \textit{Logos Niepojęty}, dz. cyt., s.~121.}.

Wszyscy wymienieni wyżej autorzy podnoszą przeciw Hickowi\index[names]{Hick, John} również zarzut niekonsekwencji wskazując, że w~wielu miejscach na określenie Rzeczywistego \textit{an sich} używa on pojęć, które -- wedle jego własnego pomysłu -- należałoby zaliczyć do kategorii własności substancjalnych, a~nie formalnych. W~jednym ze swych dzieł Hick\index[names]{Hick, John} określa Rzeczywiste samo w~sobie jako ,,takie, że gdyby nie było prawdziwe, cała dziedzina doświadczeń religijnych, w~swojej różnorodności, byłaby czystą projekcją wyobraźni''\footnote{J. Hick, \textit{A~Christian Theology of Religions}\ldots, dz. cyt., ss.~59-60.}. Można byłoby uparcie twierdzić, że jest to własność formalna, ale by przysługiwała ona Rzeczywistemu, należy wpierw przyznać mu (substancjalną) własność ,,jest prawdziwe''. Oczywiście zakładając, że doświadczenia religijne nie są czystą projekcją wyobraźni\footnote{Por. W.L. Rowe, \textit{Religious pluralism}, dz. cyt., s.~145.}. W~innych miejscach Hick\index[names]{Hick, John} przypisuje Rzeczywistemu własności ,,bycia źródłem i~podstawą wszystkiego''\footnote{J. Hick, \textit{A~Christian Theology of Religions}\ldots, dz. cyt., s.~27.}, ,,bycia autentycznie doświadczanym zarówno jako fenomen teistyczny, jak i~nieteistyczny''\footnote{Tenże, \textit{An Interpretation of Religion}\ldots, dz. cyt., s.~242, 246-247.}, czy też ,,posiadania bogatej natury''\footnote{Tenże, \textit{A~Christian Theology of Religions}\ldots, dz. cyt., s.~62.} i~wiele innych określeń, które sprzeciwiają się głównej zasadzie słabej teorii Niewysłowionego, którą sam nałożył na mowę o~Bogu\footnote{Por. W.L. Rowe, \textit{Religious pluralism}, dz. cyt., s.~146; C.J. Insole, \textit{Why John Hick cannot}\ldots, dz. cyt., s.~26 oraz A. Plantinga, \textit{Warranted Christian Belief}, dz. cyt., ss.~44-45.}.

Oprócz wspomnianej wyżej niekonsekwencji, koncepcja Hicka\index[names]{Hick, John} oskarżana jest także o~zwykłą sprzeczność. Według Hicka\index[names]{Hick, John} niewysławialność Rzeczywistego samego w~sobie polega na tym, że jego natura ,,znajduje się poza zasięgiem siatki ludzkich pojęć''. Wszystko wskazuje na to, że własności formalne również należą do takiej kategorii. Gdyby trzymać się ściśle tego sformułowania, także one nie powinny zostać przypisane Rzeczywistemu \textit{an sich}.

\begin{quote}
Rzeczywiste w~sobie nie może być tym, za co ma je Hick\index[names]{Hick, John} -- rzeczywistością całkowicie przekraczającą sieć ludzkich pojęć. Stwierdzić powyższe, to zaprzeczyć, że Rzeczywiste przekracza wszystkie pojęcia, bo ,,przekracza'' też jest ludzkim pojęciem\footnote{W.L. Rowe, \textit{Religious pluralism}, dz. cyt., s.~145-146. Cytat w~j. polskim za P. Sikora, \textit{Logos Niepojęty}, dz. cyt., s.~120.}.
\end{quote}

Argument ten można próbować odeprzeć biorąc za dobrą monetę interpretację własności formalnych i~substancjalnych zaproponowaną przez Rowe'a\index[names]{Rowe, William}. Według niej własności substancjalne Rzeczywistego to własności istotowe, należące do jego natury. Ponieważ ta przekracza sieć ludzkich pojęć, nie mogą one zostać przypisane Rzeczywistemu. Należałoby jednak wtedy uznać, że własności formalne -- np. chętnie przez Hicka\index[names]{Hick, John} przypisywana Rzeczywistemu własność ,,taki, że nasze substancjalne pojęcia się do niego nie stosują'' -- nie należą do \textit{natury} Rzeczywistego. Próba ta ma szanse powodzenia. Wymagałaby jednak najpierw wyeksplikowania pojęcia \textit{natury} i~przedstawienia możliwości epistemologicznego dostępu do niej. Nietrudno zauważyć, że taka filozoficzna teoria miałaby kilka niepożądanych własności. Na przykład taką, że dopuszczałaby istnienie obiektów posiadających istotowe, należące do natury własności, które w~zasadzie im nie przysługują, lub których z~pewnych powodów nie można im przypisać. Ponadto, nie jest do końca jasne, dlaczego proste, bezpośrednie własności substancjalne Rzeczywistego \textit{an sich} mają znajdować się ,,poza zasięgiem siatki ludzkich pojęć'', a~bardziej złożone formalne własności mają już w~tej siatce się mieścić. Należy także odnotować, że podejście Rowe'a\index[names]{Rowe, William} jest niezgodne z~tą interpretacją teorii Niewysłowionego, zgodnie z~którą niewysławialność nie wynika z~samych tylko ograniczeń podmiotu poznającego, lecz jest istotną częścią transcendentnej natury Boga.

W~podobnym duchu wypowiada się Sebastian Gäb\index[names]{Gäb, Sebastian}\footnote{Zob. S. Gäb, \textit{Languages of ineffability: the rediscovery of apophaticism in contemporary} \textit{analytic philosophy of religion}, [w:] \textit{Negative Knowledge}, red. S. Hüsch i~in., Narr Francke Attempto Verlag, Tübingen 2020, ss.~191-206.} konstatując, że samą niewysławialność Rzeczywistego należałoby właściwie zaliczyć do jego własności substancjalnych. Jego zdaniem, gdy Hick\index[names]{Hick, John} nazywa Rzeczywiste niewysławialnym, jego celem jest nie tyle określenie zasad używania tego terminu, lecz raczej wyjaśnienie swojej koncepcji Rzeczywistego, a~bycie niewysławialnym jest istotną częścią tego pojęcia. Niewysławialność Rzeczywistego należy zatem do jego natury i~jest dokładnie tym, co odróżnia Rzeczywiste od innych obiektów. Z~tego powodu należy uznać, że niewysławialność jest jego istotową i~substancjalną własnością, lecz wtedy -- zgodnie z~samą słabą teorią Niewysłowionego -- nie może ona zostać przypisana Rzeczywistemu.

Kolejna grupa argumentów przeciwko koncepcji Hicka\index[names]{Hick, John} uderza w~nią jako teorię pluralizmu religijnego. Insole\index[names]{Insole, Christopher} podaje dwa rudymentarne warunki, jakie musi spełniać dobra hipoteza -- sprawiać, że dane zostają wyjaśnione 1) lepiej, niż gdyby to miało miejsce bez tej hipotezy oraz 2) lepiej niż w~świetle innych hipotez. W~takim wypadku słaba teoria Niewysłowionego Hicka\index[names]{Hick, John} byłaby dobrą hipotezą religijnego pluralizmu, gdyby wyjaśniała różnorodność religijnych doświadczeń 1) lepiej, niż gdyby to miało miejsce bez postulowania Rzeczywistego samego w~sobie oraz 2) lepiej niż inne teorie wyjaśniające ten fenomen. Według Insole'a\index[names]{Insole, Christopher}, postulowanie istnienia takiego $X$, o~którym możemy stwierdzić jedynie tyle, że nic o~nim nie może zostać stwierdzone, nie może stanowić wyjaśnienia czegokolwiek. A~zatem teoria Hicka\index[names]{Hick, John} nie spełnia żadnego z~tych warunków\footnote{Por. C.J. Insole, \textit{Why John Hick cannot}\ldots, dz. cyt., ss.~31-32.}.

Plantinga\index[names]{Plantinga, Alvin} idzie o~krok dalej i~oskarża Hicka\index[names]{Hick, John} o~intelektualny imperializm. Skoro wyjaśnieniem religijnego pluralizmu w~słabej teorii Niewysłowionego jest postulowanie Rzeczywistego samego w~sobie, któremu w~sposób dosłowny nie możemy przypisywać żadnej własności substancjalnej, Hick\index[names]{Hick, John} konkluduje, że standardowy język religijny, wszelkie dogmaty, tezy i~doktryny wszystkich religii należy uznać za fałszywe -- o~ile tylko traktujemy je dosłownie a~nie ,,mitologicznie''\footnote{Por. J. Hick, \textit{An Interpretation of Religion}\ldots, dz. cyt., s.~348.}. Nietrudno zauważyć, że motywacją stojącą za takim stwierdzeniem jest chęć zachowania równorzędności wszystkich wyznań. Jednakże, zdaniem Plantingi\index[names]{Plantinga, Alvin}, taka konstatacja przynosi zupełnie odmienny skutek:

\begin{quote}
Teraz deklarujemy, że wszyscy się mylą -- wszyscy, oprócz nas i~kilku oświeconych dusz. [\ldots] Wspaniałomyślnie uważamy, że reszta ludzkości błądzi; nie ulega wątpliwości, że chęci mają dobre, ale niestety mylą się co do tego, co uważają za najważniejsze i~najcenniejsze. Trudno mi uznać taką postawę za manifestację tolerancji i~intelektualnej pokory, wygląda to raczej na wyraz protekcjonalności\footnote{A. Plantinga, \textit{Warranted Christian Belief}, dz. cyt., s.~62.}.
\end{quote}

W~końcu, tak jak praca Jonesa\index[names]{Jones, John J.}, tak i~słaba teoria Niewysłowionego Hicka\index[names]{Hick, John} pozbawiona jest rozważań o~charakterze logicznym. Jak zauważa Insole\index[names]{Insole, Christopher}:

\begin{quote}
Mimo iż Hick\index[names]{Hick, John} nazywa własności formalne ,,generowanymi logicznie'', nie ma w~nich nic specjalnie ,,logicznego''. Stwierdzenie, że Bóg jest taki, że ,,żadne substancjalne pojęcia się do niego nie stosują'' nie ma charakteru analitycznego w~żadnym z~niekontrowersyjnych sensów intensji pojęcia ,,Bóg''. Jeśli istnieją jakieś ,,logiczne'' podstawy, które generowałyby takie formalne własności, Hick\index[names]{Hick, John} powinien je wyraźnie określić. Jednakże perspektywa wskazania takich podstaw nie wygląda obiecująco\footnote{C. J. Insole, \textit{Why John Hick cannot}\ldots, dz. cyt., s.~28.}.
\end{quote}
Okazuje się jednak, że praca Hicka\index[names]{Hick, John} potrafiła zainspirować innych autorów do przeprowadzenia rozważań, w~których teologia apofatyczna ujęta jest już w~bardziej formalne struktury. Przykładem takich rozważań jest Petera Küglera\index[names]{Kügler, Peter} propozycja sformułowania uniwersalnej i~egzystencjalnej zasady teologii negatywnej.




%%%%%%%%%%%%%%%%%%%%%%%%%%%%%%%%%%%%%%%%%%%%%%%%%%%%%%%%%%%%%%%%%%%%%%

%\part{Aspekt semantyczny}




\chapter{Uniwersalna i~egzystencjalna zasada teologii negatywnej}\label{sil-kugler}


\section{Metafora ciemności}

Punktem wyjścia rozważań Petera Küglera\index[names]{Kügler, Peter}\footnote{P. Kügler, \textit{The meaning of mystical ‘darkness}', ,,Religious Studies'', vol. 41 (2005), ss.~95-105.} są pisma Pseudo-Dionizego Areopagity\index[names]{Pseudo-Dionizy Areopagita}. Uważa on, że kluczową zarówno dla rozważań Dionizego\index[names]{Pseudo-Dionizy Areopagita}, jak i~dla całego nurtu religijnego mistycyzmu, jest metafora ciemności\footnote{,,Boska ciemność'' to tytuł pierwszego rozdziału \textit{Teologii mistycznej}.}. W~szczególności sądzi, że w~języku Areopagity\index[names]{Pseudo-Dionizy Areopagita} słowo ,,ciemność'' związane jest z~poznawczymi deficytami, które nazywa on ,,prawdziwie mistyczną ciemnością niepoznawalności''\footnote{Pseudo-Dionizy Areopagita, \textit{Teologia mistyczna}, I, 3.}. Skoro Bóg przekracza wszelkie pojmowanie i~wszelką wiedzę, nasza sytuacja poznawcza może zostać porównana do ciemności nocy, w~której nie możemy uzyskać wiedzy o~otaczającym nas świecie bazując wyłącznie na zmyśle wzroku. Zdaniem Küglera\index[names]{Kügler, Peter} podobieństwo tych przypadków jest na tyle wystarczające, by mówić sensownie o~,,ciemności'' jako metaforze Boga. Pod pewnym względem sytuacje te są jednak niepodobne. W~ciemności mimo wszystko możemy tworzyć pewne przekonania na temat otaczającego nas świata, jakkolwiek nie możemy uzasadnić tych przekonań poprzez odwołanie się do tego, co widzimy. Z~powodu tego ograniczenia, takie przekonania nigdy nie staną się wiedzą. Natomiast w~apofatycznej teologii Dionizego\index[names]{Pseudo-Dionizy Areopagita} nie tylko formułowanie wiedzy o~Bogu jest poza zasięgiem ludzkich możliwości, lecz nawet samo nabywanie przekonań o~nim wydaje się niemożliwe. Bóg jest niewysławialny i~niepojmowalny, jest ponad wszelkim stwierdzeniem i~ponad wszelkim zaprzeczeniem.

Przyjęcie takiej metafory Boga ma jednak pewne niepożądane konsekwencje. We współczesnej filozofii Boga uważa się, że metaforyczny dyskurs o~Bogu presuponuje posiadanie o~nim pewnych dosłownych przekonań\footnote{Na to, że za religijnym językiem metaforycznym powinny stać jakieś dosłowne przekonania, wskazuje wielu myślicieli. Na przykład Walter Stace\index[names]{Stace, Terence S.} twierdzi, że każda sensowna metafora powinna być oparta na \textit{podobieństwie} oraz przynajmniej teoretycznie powinna być ona \textit{przekładalna} na język dosłowny. Te dwa warunki odpowiadają dwóm popularnym teoriom metafor. Według pierwszej z~nich metafory to zakamuflowane porównania, według drugiej posiadają one to samo znaczenie, co odpowiadający im opis dosłowny -- por. W.T. Stace, \textit{Mysticism and Philosophy}, Macmillan \& Co Ltd, London 1961, ss.~284-306. Cytowany w~poprzedniej sekcji Christopher Insole\index[names]{Insole, Christopher} podaje nieco bardziej ogólną teorię (religijnych) metafor. W~jego koncepcji sensowna wypowiedź metaforyczna powinna spełniać trzy warunki: 1) jej odbiorca powinien pojąć, że dosłowna interpretacja nie jest odpowiednia; 3) powinien potrafić rozpoznać wszystkie metaforyczne znaczenia tego wyrażenia oraz 2) powinien móc wybrać właściwe znaczenie metaforyczne użyte w~danym kontekście -- por. C.J. Insole, \textit{Metaphor and the Impossibility of Failing to Speak about God}, ,,International Journal for Philosophy of Religion'', vol. 52 (2002), ss.~35-43. Podobne podejście do językowych metafor prezentuje John Searl\index[names]{Searl, John} -- por. J. Searl, \textit{Metaphor}, [w:] \textit{Metaphor and Thought}, red. A. Ortony, Cambridge University Press, Cambridge 1993, ss.~83-111.}. Jeśli przyjmiemy taką tezę, nietrudno wskazać paradoksalny charakter metafory ciemności. Mówi ona, że nabywanie i~utrzymywanie jakichkolwiek dosłownych przekonań o~Bogu nie jest możliwe, a~jednocześnie -- jak każda metafora -- zakłada posiadanie takich przekonań. Celem rozważań Küglera\index[names]{Kügler, Peter} jest uniknięcie tego paradoksu. Mówiąc dokładniej, chce on pokazać, że możliwe jest utrzymanie ,,ciemności'' jako metafory Boga przy jednoczesnym zachowaniu tezy, że religijny język metaforyczny presuponuje pewne dosłowne przekonania o~Bogu\footnote{Zob. P. Kügler, \textit{The meaning of mystical ‘darkness}', dz. cyt., s.~96.}.


\section{Dwie zasady teologii negatywnej i~ich paradoksalne konsekwencje}\label{sil-kug-zasady}

Zdaniem Küglera\index[names]{Kügler, Peter} dosłowną podstawą dla metafory ciemności jest centralna idea teologii apofatycznej rozumianej jako teologia milczenia -- mianowicie taka, że Bóg jest niewysławialny, nie można o~nim nic powiedzieć. Z~punktu widzenia niniejszej pracy interesujący jest fakt, że Kügler\index[names]{Kügler, Peter} stara się wyrazić ją we względnie formalny\footnote{On sam deklaruje, że jest to zapis półformalny (\textit{semiformal}) -- por. tamże, s.~98.} sposób. Pierwszą iteracją jego starań jest ogólna (uniwersalna) zasada teologii negatywnej (\ref{sil-kug-unt}):
%(UNT) Dla każdej własności Q, Bóg ani posiada, ani nie posiada Q.
\begin{flalign*}
		& \parbox[t]{.87\linewidth}{ 
		Dla każdej własności $Q$, Bóg ani posiada, ani nie posiada $Q$.} &\tag{UNT}\label{sil-kug-unt}
\end{flalign*}

Niestety, sąd wyrażony przez \ref{sil-kug-unt} dzieli z~teorią Niewysłowionego jej dobrze znane problemy z~zachowaniem spójności. Skoro \ref{sil-kug-unt} jest wyrażaniem wykorzystującym ogólną kwantyfikację przebiegającą po zbiorze własności, możemy ją egzemplifikować używając jakiejkolwiek własności. Rozważmy w~takim razie własność $B$~oznaczającą ,,jest niebieski''. Możemy następnie sformułować (bardziej złożoną) własność ,,ani jest, ani nie jest niebieski'' i~oznaczyć ją przez $B^*$. Teraz, powołując się na \ref{sil-kug-unt}, zmuszeni jesteśmy przyznać, że Bóg posiada własność $B^*$ i~jednocześnie -- zgodnie z~tą samą zasadą -- ani posiada, ani nie posiada $B^*$. Nietrudno zauważyć, że w~rzeczywistości powyższy argument zupełnie nie zależy od własności wyrażonej przez $B$~i przy jego pomocy możemy wyprodukować tyle sprzeczności, ile tylko istnieje własności. Z~tego powodu Kügler\index[names]{Kügler, Peter} proponuje zastąpić \ref{sil-kug-unt} szczegółową (egzystencjalną) zasadą teologii negatywnej (\ref{sil-kug-ent}).
%(\ref{sil-kug-ent}) Nie istnieje własność Q~taka, że Bóg posiada lub nie posiada Q.
\begin{flalign*}
		& \parbox[t]{.87\linewidth}{ 
		Nie istnieje własność $Q$~taka, że Bóg posiada lub nie posiada $Q$.} &\tag{ENT}\label{sil-kug-ent}
\end{flalign*}

Zanim posuniemy się dalej w~naszych rozważaniach, przedstawmy powyższe zasady w~nieco bardziej formalny sposób. Skoro \ref{sil-kug-unt} wykorzystuje kwantyfikację po własnościach, na potrzeby jej formalizacji najrozsądniej będzie zaadoptować logikę drugiego rzędu. W~takim ujęciu \ref{sil-kug-unt} przybierze postać
%(UNT'),
\begin{flalign*}
		& \forall_Q\ \neg (Q(g) \lor \neg Q(g)), &\tag{UNT'}\label{sil-kug-untprim}
\end{flalign*}
gdzie $Q$~jest symbolem predykatowym, a~$g$ stałą indywiduową oznaczającą Boga. W~takim sformułowaniu sprzeczność jest jeszcze bardziej oczywista:
\begin{flalign}
& \forall_Q\ \neg (Q(g) \lor \neg Q(g)) &  \eqref{sil-kug-untprim}\label{untprim1} \\
& \neg (B(g) \lor \neg B(g)) &  (\ref{untprim1},\ \forall \text{ elim.})\label{untprim2}  \\
& \neg B(g) \land B(g) & \qquad (\text{\ref{untprim2}, p. De Morgana\index[names]{De Morgan, Augustus}, }\neg\neg\text{ elim.})\label{untprim3}  \\
& B(g) & (\ref{untprim3},\ \land\text{ elim.})\label{untprim4}  \\
& \neg B(g) & ( -||- )\label{untprim5}  \\
& \qquad \text{contr. \ref{untprim4}, \ref{untprim5}} & \nonumber
\end{flalign}

Zinterpretowanie terminu ,,Bóg'' w~kategoriach deskrypcji określonej prowadziłoby do przeformułowania \ref{sil-kug-unt} w~następujący sposób:
%(UNT'')
%\begin{flalign*}
%		& G(x) \to \forall_Q\ \neg (Q(x) \lor \neg Q(x)). &\tag{UNT''}\label{sil-kug-untbis}
%\end{flalign*}
\begin{flalign*}
		& G(x) \equiv \forall_Q\ \neg (Q(x) \lor \neg Q(x)). &\tag{UNT''}\label{sil-kug-untbis}
\end{flalign*}
Jednakże w~takim wypadku oddanie naturalno-językowego terminu ,,Bóg'' przy pomocy predykatu prowadzi do dodatkowych niepożądanych konsekwencji. Jedną z~nich jest zdanie, że coś jest Bogiem wtedy i tylko wtedy, gdy nim nie jest:
%\begin{flalign}
%& G(x) to \forall_Q\ \neg (Q(x) \lor \neg Q(x)) & \eqref{sil-kug-untbis}\label{untprim6} \\
%& G(x) \to \neg (G(x) \lor \neg G(x)) & (\ref{untprim6},\ \forall \text{ elim., syl. hip.})\label{untprim7} \\
%& G(x) \to \neg G(x) \land G(x) & (\text{\ref{untprim7}, p. De Morgana, }\neg\neg\text{ elim., syl. hip.})\label{untprim8} \\
%& G(x) \to \neg G(x) & (\ref{untprim8},\ \land\text{ elim.}\label{untprim9})
%\end{flalign}
\begin{flalign}
& G(x) \equiv \forall_Q\ \neg (Q(x) \lor \neg Q(x)) & \eqref{sil-kug-untbis}\label{untprim6} \\
& G(x) \equiv \neg (G(x) \lor \neg G(x)) & (\ref{untprim6},\ \forall \text{ elim., syl. hip.})\label{untprim7} \\
& G(x) \equiv \neg G(x) \land G(x) & (\text{\ref{untprim7}, p. De Morgana\index[names]{De Morgan, Augustus}, }\neg\neg\text{ elim., syl. hip.})\label{untprim8} \\
& G(x) \equiv \neg G(x) & (\ref{untprim8},\ \land\text{ elim.})\label{untprim9}\\
& G(x)  & (\equiv \text{ elim., p. Claviusa }\ref{untprim9})\label{untprim10}\\
& \neg G(x) & ( -||- )\label{untprim11}\\
& \qquad \text{contr. \ref{untprim10}, \ref{untprim11}} & \nonumber
\end{flalign}

Z~powodu powyższych problemów w~tejże sekcji będę traktował termin ,,Bóg'' jako nazwę własną. Za takim ujęciem stoją jeszcze dwa argumenty. Po pierwsze, ułatwi to porównanie zasad Küglera\index[names]{Kügler, Peter} z~tymi formalnymi ujęciami teorii Niewysłowionego, w~których użycie predykatu w~celu oddania słowa ,,Bóg'' jest niemożliwe, nieintuicyjne lub prowadzi wprost do sprzeczności. Po drugie, skoro kwestia (nie)istnienia Boga nie jest centralnym zagadnieniem teologii apofatycznej -- przeciwnie, myśliciele apofatyczni raczej uznawali jego istnienie za pewnik -- przyjęcie tej kategorii językowej nie będzie kwestią kontrowersyjną\footnote{Por rozdz.~\ref{sil-kt-jez}.}. Odpowiadające takiemu ujęciu sformułowanie \ref{sil-kug-ent} przybierze następującą postać:
%(ENT')
\begin{flalign*}
		& \neg \exists_Q\ (Q(g) \lor \neg Q(g)). &\tag{ENT'}\label{sil-kug-entprim}
\end{flalign*}

Powodem, dla którego \ref{sil-kug-unt} zostało zastąpione przez \ref{sil-kug-ent} był fakt, że pierwsza z~tych zasad prowadziła do potencjalnie nieskończonej liczby paradoksów -- przy jej pomocy można było wygenerować paradoks samoodniesienia dla każdej własności obecnej w~języku. Nie oznacza to jednak, że \ref{sil-kug-ent} pozwala ostatecznie uniknąć tego problemu. Szczegółowa zasada teologii negatywnej także jest dosłowną wypowiedzią o~Bogu wyrażoną w~języku. Przypisuje ona Bogu pewną (złożoną) własność -- ,,taki, że nie ma żadnej własności, którą On posiada lub nie posiada''. Nazwijmy tę własność $G^*$. Ponownie jesteśmy zmuszeni stwierdzić, że z~\ref{sil-kug-ent} wynika, że Bóg posiada własność $G^*$, co jednocześnie jest sprzeczne z~\ref{sil-kug-ent}, ponieważ zgodnie z~jej treścią nie istnieje żadna własność, którą Bóg posiada (ani taka, której nie posiada). Jednakże, zdaniem Küglera\index[names]{Kügler, Peter}, logiczny problem \ref{sil-kug-ent} nie jest tak wielki jak trudności nękające \ref{sil-kug-unt}. Jest on jednak na tyle poważny, że w~ostatniej części pracy\footnote{Zob. P.~Kügler, \textit{The meaning of mystical ‘darkness}', dz. cyt., s.~99-103.} próbuje sobie z~nim poradzić. Rozważa on trzy strategie: ,,somopodważenie'', ,,ograniczenie'' i~,,samowykluczenie''\footnote{Org. odpowiednio: \textit{self-subversion}, \textit{restriction} i~\textit{self-exclusion}. Ten środkowy termin Piotr Sikora\index[names]{Sikora, Piotr} tłumaczy dosłownie, jako ,,restrykcję'', por. P. Sikora, \textit{Logos Niepojęty}, Wydawnictwo Universitas, Kraków 2010, ss.~121-122.}.


\section{Strategie Küglera. Somowykluczenie jako sposób na uniknięcie sprzeczności \ref{sil-kug-ent}}

Strategia somopodważenia polegałaby właściwe na rezygnacji z~metafory ciemności. Warto podkreślić, że jest ona bliska treści samej \textit{Teologii mistycznej} jak i~wielu interpretacji tego tekstu, wedle których doktryna apofatyczna Dionizego\index[names]{Pseudo-Dionizy Areopagita} w~takim samym stopniu kwestionuje zasadność afirmatywnego, jak i~negatywnego dyskursu o~Bogu. Wielu autorów nazywa to podejście ,,negacją negacji''\footnote{D. Turner, \textit{The Darkness of God: Negativity in Christian Mysticism}, Cambridge University Press, Cambridge 1995, s.~45; P. Rorem, \textit{Pseudo-Dionysius. A~Commentary on the Texts and an Introduction to Their Influence}, Oxford University Press, New York -- Oxford 1993, s.~213; D.~Brylla, \textit{Rozważania o~apofatycznej kategorii ,,negacja negacji''}, ,,Seminare'', vol. 38 (2017), nr 1, ss.~65-76.} i~wskazuje je jako najważniejszy powód, dla którego teologia negatywna jest teologią milczenia. Oferuje ona pewną drogę ku Bogu, lecz im dalej na niej się znajdujemy, tym większe trudności napotyka nasz język i~umysł. Na końcu tej drogi także i~ona sama musi zostać porzucona, ponieważ i~ona stanowi niewłaściwy sposób mówienia i~myślenia o~nim -- wobec nieopisywalnego Boga pozostaje wyłącznie milczenie. Zdaniem Küglera\index[names]{Kügler, Peter} jego \ref{sil-kug-ent} może przyczynić się do lepszego zrozumienia takiego stanowiska. Jego szczegółowa zasada teologii negatywnej pełniłaby tutaj rolę eksplanacyjną -- miałaby bowiem tłumaczyć, na czym dokładnie polegają trudności, które język ludzki napotyka przy próbie opisu transcendencji. Jednakże sam Kügler\index[names]{Kügler, Peter} nie jest przekonany co do słuszności takiego podejścia, ponieważ oznacza ono pogodzenie się z~paradoksalnym charakterem teorii Niewysłowionego, a~nie usunięcie go:

\begin{quote}
[\ldots] to, czy dojście do sprzeczności (i wywołane nim milczenie) zostanie uznane za najwspanialsze osiągnięcie czy też niepożądane niepowodzenie teologii jest kwestią filozoficznego temperamentu. Nie każdy będzie zachwycony strategią samopodważenia\footnote{P~Kügler, \textit{The meaning of mystical ‘darkness}', dz. cyt., s.~100.}.
\end{quote}
Najwyraźniej filozoficzny temperament Küglera\index[names]{Kügler, Peter} sprawia, że on sam zachwycony nią nie jest.

Druga z~rozważanych strategii polegałaby na podzieleniu zbioru własności na dwie klasy i~ograniczeniu stosowania \ref{sil-kug-ent} tylko do jednej z~nich. Przykładem takiego rozwiązania jest podział własności przypisywanych Bogu na substancjalne i~formalne, którego dokonał Hick\index[names]{Hick, John}\footnote{Zob. rozdz.~\ref{sil-slabatn}.}. Próbując obronić \ref{sil-kug-ent} przed paradoksem samoodniesienia można by założyć, że dotyczy ona tylko własności substancjalnych. Przy takim założeniu szczegółowa zasada teologii negatywnej Küglera\index[names]{Kügler, Peter} głosiłaby, że nie istnieje \textit{substancjalna} własność $Q$~taka, że Bóg posiada lub nie posiada $Q$. Natomiast ta własność, która wyrażona jest przez samą jej treść (a którą poprzednio oznaczyliśmy jako $G^*$), należy do kategorii własności formalnych. W~ten sposób unikamy sytuacji, w~której \ref{sil-kug-ent} odnosi się do samej siebie. Jednakże Kügler\index[names]{Kügler, Peter} nie jest przekonany także do takiego stanowiska. Uważa on, że podział na własności formalne i~substancjalne zaproponowany przez Hicka\index[names]{Hick, John}, a~także próby doprecyzowania tego podziału przez jego krytyków są albo beznadziejnie niejasne, albo -- w~najlepszym razie -- przebiegają w~nieodpowiednich miejscach.

W~zamian za to proponuje on trzecią strategię, którą nazywa samowykluczeniem. W~myśl tego rozwiązania szczegółowa zasada teologii negatywnej dotyczy wszystkich własności przypisywanych Bogu poza tą, którą sama ta zasada wyraża. Innymi słowy, z~zasięgu obowiązywania \ref{sil-kug-ent} wykluczamy własność $G^*$ (,,taki, że nie ma żadnej własności, którą On posiada lub nie posiada''), czyli tę właśnie, która jest przypisywana Bogu przez \ref{sil-kug-ent}. Zdaniem Küglera\index[names]{Kügler, Peter} taki zabieg najlepiej służy przedstawieniu \ref{sil-kug-ent} jako dosłownej podstawy dla metafory ciemności.


\section{Dyskusja}\label{sil-kug-dyskusja}

Warto zauważyć, że \ref{sil-kug-entprim} (a także naturalno-językowe sformułowanie zawarte w~\ref{sil-kug-ent}) przyjmuje pewną postać negacji prawa wyłączonego środka. Mając to na uwadze, przy próbie ujęcia rozważań Küglera\index[names]{Kügler, Peter} w~formalne struktury powinniśmy zrezygnować ze stosowania logiki klasycznej z~co najmniej dwóch powodów. Po pierwsze, w~klasycznych przypadkach zasada \textit{tertium non datur} musi obowiązywać. Po drugie, zastępowania (\ref{sil-kug-untprim}) przez (\ref{sil-kug-entprim}) na gruncie logiki klasycznej traci jakikolwiek sens, ponieważ wyrażenia te są równoważne (nietrudno zauważyć, że równoważność obu zasad konstytuowana jest przez prawo De Morgana\index[names]{De Morgan, Augustus})\footnote{Warto dodać, że nie ma znaczenia tutaj fakt, że dla formalizacji zasad Küglera\index[names]{Kügler, Peter} zastosowaliśmy logikę drugiego rzędu. Prawa rządzące tą równoważnością zostają zachowane.}. Kügler\index[names]{Kügler, Peter} jest świadomy tego faktu i~sam proponuje, by do przedstawienia apofatycznego dyskursu zastosować jedną z~logik nieklasycznych\footnote{Zob. P~Kügler, \textit{The meaning of mystical ‘darkness}', dz. cyt., s.~98-99.}. Niestety, nie podaje on wprost żadnego konkretnego rachunku, który mógłby służyć jako odpowiednia rama dla jego rozważań. Spróbujmy jednak przyjrzeć się bliżej tej kwestii.

Na pierwszy rzut oka wydawałoby się, że najlepszą kandydatką do oddania formalnej struktury poglądów Küglera\index[names]{Kügler, Peter} jest logika intuicjonistyczna. Powstała ona jako rezultat formalizacji pewnych poglądów dotyczących podstaw matematyki (zwanych intuicjonizmem). Poglądy te były rozwijane w~pierwszych dekadach ubiegłego stulecia i~podobnie jak większość nurtów filozoficznych badań nad podstawami matematyki tamtego okresu, powstały w~reakcji na pewne antynomie dostrzeżone w~teorii mnogości\footnote{Historię i~okoliczności powstania logiki intuicjonistycznej opisuję w~P. Urbańczyk, \textit{Geneza intuicjonistycznego rachunku zdań i~Twierdzenie Gliwienki}, ,,Zagadnienia Filozoficzne w~Nauce'', nr 56 (2014), ss.~33-56.}. Od samego początku oczywistym było, że logika intuicjonistyczna, jako próba sprecyzowania intuicjonistycznych sposobów wnioskowań pochodzących z~odmiennej, zrekonstruowanej matematyki, nie może akceptować wszystkich praw stosowanych na gruncie rachunku klasycznego. Do jej najbardziej znanych własności należy fakt, że odrzuca ona prawo wyłączonego środka. Dziś można śmiało powiedzieć, że logika intuicjonistyczna jest najlepiej ugruntowaną i~najszerzej zbadaną logiką nieklasyczną posiadającą tę własność -- co dla celów formalizacji \ref{sil-kug-ent} byłoby pożądaną cechą.

Powiedzieliśmy jednak, że rachunek logiczny, który by miał leżeć u~podstaw rekonstrukcji \ref{sil-kug-ent} powinien także blokować równoważność \ref{sil-kug-untprim} i~\ref{sil-kug-entprim}. Problem w~tym, że logika intuicjonistyczna nie spełnia tego kryterium. W~rzeczywistości odrzuca ona ,,jedną czwartą'' praw De Morgana\index[names]{De Morgan, Augustus}, to znaczy jedną implikację jednego z~tych praw, mianowicie:
%Not in Int (chyba nie to, co w~artykule).
\begin{flalign*}
&\nvdash_{\text{INT}} \neg \forall x A \to \exists x \neg A.&
\end{flalign*}
Implikacja w~drugą stronę oraz drugie z~praw De Morgana\index[names]{De Morgan, Augustus} stanowiące o~tym, że \ref{sil-kug-untprim} pociąga za sobą \ref{sil-kug-entprim} i~\textit{vice versa}, w~logice intuicjonistycznej zachodzi:
%In int x3.
\begin{flalign*}
&\vdash_{\text{INT}} \exists x \neg A \to \neg \forall x A.&\\
&\vdash_{\text{INT}}  \forall  x \neg A \to \neg \exists x  A.&\\
&\vdash_{\text{INT}}  \neg \exists x  A \to  \forall x \neg A.&
\end{flalign*}

Z~tego powodu także logika intuicjonistyczna nie może stać się właściwym narzędziem do oddania struktury wnioskowań Küglera\index[names]{Kügler, Peter}. Można jeszcze próbować osłabiać intuicjonistyczną negację na tyle, by w~ten sposób utworzony rachunek zachował własność $\nvdash A \lor \neg A$~i, dodatkowo, posiadał własność $\nvdash \neg (A \land B) \to \neg A \lor \neg B$. Okazuje się jednak, że te same prawa De Morgana\index[names]{De Morgan, Augustus} zostają zachowane w~słabszej od logiki intuicjonistycznej logice minimalnej Johanssona\index[names]{Johansson, Ingebrigt}, a~nawet w~logice subminimalnej z~najsłabszą intuicjonistyczną negacją\footnote{Por. J.M. Dunn, \textit{Generalized Ortho Negation}, [w:] \textit{Negation: A~Notion in Focus}, red. H.~Wansing, Walter de Gruyter, Berlin -- New York 1996, ss.~3-26; S.P. Odintsov, \textit{Constructive Negations and Paraconsistency}, Trends in Logic, vol. 26, Springer-Verlag, New York 2008; S.P. Odintsov, \textit{On the structure of paraconsistent extensions of Johansson's logic}, ,,Journal of Applied Logic'', vol. 3 (2005), ss. 43-65; Y.~Shramko, \textit{Dual Intuitionistic Logic and a~Variety of Negations: The Logic of Scientific Research}, ,,Studia Logica: An International Journal for Symbolic Logic'', vol. 80 (2005), ss.~347-367; A. Colacito i~in., \textit{Subminimal negation}, ,,Soft Computing'', vol. 21 (2016), ss.~165-174; N. Bezhanishvili i in., \textit{A~Study of Subminimal Logics of Negation and Their Modal Companions}, [w:] \textit{Language, Logic, and Computation}, red. A. Silva i~in., Springer, Berlin -- Heidelberg 2019, ss.~21-41.}. Filozof badający teorie zawierające sprzeczności powinien także ucieszyć się z~faktu, że obie te logiki posiadają parakonsystentne własności. Być może istnieje jakaś odmiennie budowana logika parakonsystentna, która blokuje zarówno prawo wyłączonego środka, jak i~to prawo De Morgana\index[names]{De Morgan, Augustus}, które prowadzi do niechcianych przez Küglera\index[names]{Kügler, Peter} równoważności \ref{sil-kug-unt} i~\ref{sil-kug-ent}. Niestety, Kügler\index[names]{Kügler, Peter} -- poza uwagą o~nieklasyczności rachunku, w~ramach którego \ref{sil-kug-ent} stanowiłoby formalną teorię teologii milczenia -- nie dokłada starań, by taki rachunek wskazać lub skonstruować. Mimo tego utrzymuje, że taka właśnie nieklasyczna (ale nieokreślona) logika jest \textit{implicite} wykorzystywana przez Pseudo-Dionizego Areopagitę\index[names]{Pseudo-Dionizy Areopagita} w~ostatnim rozdziale \textit{Teologii mistycznej}\footnote{Zob. P~Kügler, \textit{The meaning of mystical ‘darkness}', dz. cyt., s.~98.}.

Skoro w~dyskusji pojawił się już wątek logik parakosystentnych, warto wspomnieć, że motywacją dla wprowadzenia \ref{sil-kug-ent} był dla Küglera\index[names]{Kügler, Peter} fakt, że \ref{sil-kug-unt} generowało nieskończenie wiele paradoksów -- dokładnie tyle, ile własności (lub predykatów) znajduje się w~języku. Po zastąpieniu \ref{sil-kug-unt} przez \ref{sil-kug-ent} sprzeczność pojawiała się tylko w~przypadku użycia tej własności, którą \ref{sil-kug-ent} wyraża. Nie jest jednak do końca jasne, dlaczego nieskończenie wiele sprzeczności miałoby być gorsze niż jedna. Jak wspomnieliśmy w~rozdziale \ref{sil-int-par}, głównym powodem, dla którego w~teoriach unika się sprzeczności, jest fakt, że pociągają one za sobą wszystko -- w~systemach, w~których pojawia się para zdań sprzecznych, można dowieść dowolne zdanie\footnote{Zob. rozdz.~\ref{sil-int-par} a~także G. Priest, \textit{What so bad about contradictions}?, ,,The Journal of Philosophy'', vol. 95 (1998), nr 8, ss.~410-426.}. Mówiąc bardziej ścisłym językiem, systemy takie ulegają przepełnieniu. O~ile nie poruszamy się w~sferze logik parakonsystentnych, do przepełnienia systemu wystarczy pojawienie się jednej pary zdań sprzecznych. Nie potrzeba do tego ich nieskończonej liczby.

Warto także zwrócić uwagę na sposób, w~jaki Kügler\index[names]{Kügler, Peter} próbuje poradzić sobie ze sprzecznością. Strategia ,,samowykluczenia'', którą proponuje, nakazuje wyłączyć z~zasięgu oddziaływania \ref{sil-kug-ent} własność, którą sama ta zasada wyraża. Kügler\index[names]{Kügler, Peter} usiłuje uzasadnić zastosowanie takiej strategii uzusem takich naturalno-językowych wyrażeń, jak np. ,,nie ufaj nikomu'' (w domyśle, używający takiego zwrotu miałby chcieć przekazać: ,,nie ufaj nikomu prócz mnie mówiącemu Ci teraz, byś nie ufał nikomu'') lub pojawianiem się podobnych strategii w~pewnych interpretacjami klasycznego paradoksu kłamcy. Nietrudno jednak uznać samowykluczenie za zwykłe rozwiązanie \textit{ad hoc}, które -- zamiast rozwiązywać -- ucieka od paradoksu samoodniesienia generowanego przez szczegółową zasadę teologii negatywnej i~jako takie nie może zostać uznane za satysfakcjonującą strategię radzenia sobie ze sprzecznościami teologii milczenia.


\chapter{Semantyczna zasada teologii negatywnej}\label{sil-gell}

Obiekcje podobne do tych podniesionych w~poprzednim rozdziale mogą równie dobrze dotyczyć interpretacji teologii negatywnej, którą przedstawił Jerome I.~Gellman\index[names]{Gellman, Jerome I.}\footnote{Zob. J.I. Gellman, \textit{The Meta-Philosophy of Religious Language}, ,,Nous'', vol. 11 (1971), ss.~151-161.}. W~jego rozważaniach teologia apofatyczna jest teorią, w~ramach której jakikolwiek predykat $P$~języka ,,skończonych bytów'' nie może być prawdziwie orzekany o~Bogu. Główną tezą tej teorii jest, że Bóg nie należy do przedmiotowego zakresu odniesienia żadnego z~predykatów naszego języka. Jeśli teolog apofatyczny neguje posiadanie przez Boga jakiejś własności, ma on na myśli raczej negacją \textit{wykluczającą} niż negację \textit{wyboru}. Jego zamiarem jest wyłącznie stwierdzenie, że to nieprawda, że predykat $P$~przysługuje Bogu, niekoniecznie sugerując, że można o~nim orzec dopełnienie tego predykatu, czyli nie-$P$. Rozważania te doprowadziły Gellmana\index[names]{Gellman, Jerome I.} do sformułowania pewnego wariantu zasady teologii negatywnej, który (przynajmniej na poziomie syntaktycznym) tożsamy jest z~\ref{sil-kug-unt} przedstawioną w~poprzednim paragrafie. Mimo podobnych zarzutów wysuwanych w~stronę obu autorów, punkt wyjścia teorii Gellmana\index[names]{Gellman, Jerome I.} jest zgoła odmienny od tego, który przyświecał Küglerowi\index[names]{Kügler, Peter}.


\section{Wewnętrzne i~zewnętrzne problemy języka religijnego}

Gellman\index[names]{Gellman, Jerome I.} ocenia adekwatność różnych analiz znaczenia języka religijnego. Wedle tej oceny, takie analizy powinny być klasyfikowane i~porządkowane nie ze względu na to, jak wiele z~języka religijnego mogą objąć, lecz ze względu na to, których jego elementów dotyczą. Gellman\index[names]{Gellman, Jerome I.} zauważa, że zdania języka religijnego nie mają równego statusu. Jedne z~nich są bardziej istotne od drugich. Np. w~przypadku tradycji chrześcijańskiej, najważniejszym fragmentem języka religijnego są teksty biblijne i~pisma Ojców Kościoła, a~-- dajmy na to -- teksty publikowane we współczesnych poradnikach kaznodziejskich czy nawet czasopismach teologicznych stanowią jego drugorzędną część. Podobnych podziałów można dokonać w~języku religijnym większości zachodnich religii. Na tej podstawie Gellman\index[names]{Gellman, Jerome I.} wyróżnia dwa rodzaje problemów religijnych: \textit{zewnętrzne} i~\textit{wewnętrzne}.

Problemem religijnym jest dowolny problem sformułowany w~języku religijnym, który na pierwszy rzut oka przedstawia wierzeniom religijnym jakiekolwiek trudności, np. kontestuje je -- najczęściej zawierając parę zdań przeciwnych lub sprzecznych, ale jednakowo ugruntowanych w~języku religijnym\footnote{Gellman\index[names]{Gellman, Jerome I.} używa określenia ,,niekompatybilnych''. Czy jest to ,,niekompatybilność'' o~charakterze logicznym, czy nie, pozostawia kwestą otwartą, zob. tamże, s.~153.}. \textit{Wewnętrzny} problem religijny musi być rozpoznany w~tej części języka religijnego, która stanowi jego podstawę i~najistotniejszy fragment. Wszystkie inne problemy religijne są \textit{zewnętrzne}. Przykładem \textit{zewnętrznego} problemu religijnego jest niezgodność twierdzenia o~wszechwiedzy Boga (w szczególności wiedzy dotyczącej przyszłości) z~tezą o~wolnej woli. Zagadnienie to nie pojawia się w~podstawowych tekstach chrześcijańskich czy judaistycznych -- zostało wypracowane dopiero przez filozofów scholastycznych, którzy te teksty komentowali. Przykładem religijnego problemu \textit{wewnętrznego} jest twierdzenie o~wszechmocy, wszechwiedzy i~dobroci Boga w~obliczu twierdzenia, że na świecie istnieje zło i~niesprawiedliwość.

\begin{figure}[H]
\begin{center}

 \begin{tikzpicture}[node distance=2cm]
 
% \clip [rounded corners=.5cm] (0,0) rectangle (4,8);

    \node (A) {\begin{tabular}{c} $D$ \\ \end{tabular}};
    \node (E) [right=of A, xshift=5cm] {\begin{tabular}{c} $C$ \end{tabular}};
    \node (I) [below=of A] {\begin{tabular}{c}  $A$ \end{tabular}};
    \node (O) at (I-|E) {\begin{tabular}{c}  $B$ \end{tabular}};
    \node (Z) [below=of I, yshift=2cm, xshift=6cm] {\begin{tabular}{c} \textbf{Twardy rdzeń języka religijnego} \end{tabular}};

%    \coordinate (CENTER) at ($(A)!0.5!(O)$);

%    \node (contra) at (CENTER) {\begin{tabular}{c}Własności sprzeczne \end{tabular}};
    \path[-] (A) edge node[] {\begin{tabular}{c}Problem zewnętrzny\\ \ \end{tabular}} (E);
    \path[-] (I) edge node[] {\begin{tabular}{c}Problem wewnętrzny\\ \  \end{tabular}} (O);
%    \path[->] (A) edge node[rotate=90] {\begin{tabular}{c}Założenie\\ \ \end{tabular}} (I);
%    \path[->] (E) edge node[rotate=-90] {\begin{tabular}{c}Założenie\\ \ \end{tabular}} (O);

%    \path[-] (O) edge (A);
    \path[-] (I) edge node[rotate=20] {\begin{tabular}{c}\qquad\quad Problem zewnętrzny\\ \ \end{tabular}} (E);
%    \path[-] () edge (I);
%    \path[-] () edge (O);
    
%    \draw [red,line width=1pt,rounded corners=.3cm]
%            ([shift={(0.5\pgflinewidth,0.5\pgflinewidth)}]0,0) rectangle
%            ([shift={(-0.5\pgflinewidth,-0.5\pgflinewidth)}]4,8);

	\draw[thick,rounded corners=.3cm]     ($(I)+(-1,1)$) rectangle ($(O)+(1.5,-1.3)$);

\end{tikzpicture}

\caption[Wewnętrzne i~zewnętrzne problemy religijne według Gellmana]{Wewnętrzne i~zewnętrzne problemy religijne. W~koncepcji Gellmana\index[names]{Gellman, Jerome I.} wewnętrzny problem religijny może wystąpić tylko w przypadku ,,konfliktu'' pomiędzy dwoma tezami należącymi do twardego rdzenia języka religijnego. Na powyższym schemacie \textit{wewnętrzny} problem religijny wystąpiłby tylko w przypadku ,,niezgodności'' między zdaniami $A$ oraz $B$. W~przypadku potencjalnych sprzeczności między zdaniami $A$ i~$C$, $D$ i~$C$ (a~także, ewentualnie, między $A$ i~$D$ oraz $C$ i~$B$) mówimy o~\textit{zewnętrznych} problemach religijnych.}\label{sil-gell-prob}
\end{center}
\end{figure}
%Gellmaan-pic z~opisem

Według Gellmana\index[names]{Gellman, Jerome I.} każda adekwatna analiza języka religijnego nie powinna usuwać wewnętrznych problemów danej religii. Innymi słowy, jeśli przy analizie znaczenia języka religijnego jakiś rozpoznany \textit{wewnętrzny} problem religijny nie wystąpi lub nie da o~sobie znać, taka analiza będzie musiała zostać uznana za nieadekwatną (przynajmniej w~tym zakresie).


\section{Bóg jako obiekt poza zakresem przedmiotowym predykatów ludzkiego języka}

Gellman\index[names]{Gellman, Jerome I.} rozważa krótko kazus teologii negatywnej jako jeden z~przykładów teorii języka religijnego (obok formalnej analizy języka religijnego Józefa Marii Bocheńskiego\index[names]{Bocheński, Józef Maria} i~Tomaszowej\index[names]{Tomasz z~Akwinu@Tomasz z~Akwinu, \textit{św}.} teorii analogii). Kojarzy ją jednak wyłącznie z~teologami żydowskimi i~arabskimi\footnote{Wśród podanych przez niego przykładów teologów negatywnych pojawia się Mojżesz Majmonides\index[names]{Majmonides, Mojżesz}, Abraham ibn Daud\index[names]{Daud@ibn Daud, Abraham} oraz Bahya ibn Paquda\index[names]{Paquda@ibn Paquda, Bahya}. Powiązanie teologii negatywnej wyłącznie z~myślicielami żydowskimi i~arabskimi wydaje się jednak niewłaściwe.}. Podobnie jak Kügler\index[names]{Kügler, Peter} uważa on, że przypisywanie Bogu jakiegokolwiek atrybutu na gruncie teologii negatywnej wiąże się z~popełnieniem błędu (przesunięcia) kategorialnego. Innymi słowy, w~rozważaniach Gellmana\index[names]{Gellman, Jerome I.} teologia negatywna jest teorią, w~myśl której jakikolwiek predykat $P$~języka ,,skończonych bytów'' nie może być prawdziwie orzekany o~Bogu. Nie oznacza to jednak, że możemy o~nim orzec negacje wszystkich predykatów. By wykluczyć taką możliwość, Gellman\index[names]{Gellman, Jerome I.} wprowadza definicję zakresu przedmiotowego predykatu\footnote{Tę własność można nazwać także ,,zakresem rodzajowym'' lub ,,zakresem odniesienia'' predykatu (org. \textit{sortal range}).}.

%Definicja SR. Zakres przedmiotowy predykatu P~to dziedzina obiektów, o~których można znacząco orzec P~lub jego dopełnienie.
\begin{defin}[Zakres przedmiotowy predykatu]\label{sil-gell-srdef}
Zakres przedmiotowy predykatu $P$~to dziedzina obiektów, o~których można znacząco orzec $P$~lub jego dopełnienie.
\end{defin}
\noindent
W~świetle tej definicji, główna teza teologii negatywnej w~interpretacji Gellmana\index[names]{Gellman, Jerome I.} głosi, że
%(SNT) Bóg nie należy do zakresu przedmiotowego żadnego z~predykatów naszego języka.
\begin{flalign*}
		& \parbox[t]{.87\linewidth}{ 
		Bóg nie należy do zakresu przedmiotowego żadnego z~predykatów naszego języka.} &\tag{SNT}\label{sil-gell-snt}
\end{flalign*}

Motywacje stojące za wprowadzeniem takiej zasady są oczywiste -- pozwala ona na zaprzeczanie, że Bóg posiada jakąś własność $P$~bez jednoczesnego uznawania, że posiada on własność nie-$P$ i~\textit{vice versa}. Według Gellmana\index[names]{Gellman, Jerome I.}, gdy mówimy na przykład, że Bóg nie jest mądry, mamy na myśli raczej negacje \textit{wykluczającą}, niż negację \textit{wyboru}\footnote{Potrzebę wprowadzenia podziału na negację \textit{wykluczającą} i~negację \textit{wyboru} na gruncie języka naturalnego zauważył Gerrit Mannoury w~G. Mannoury, \textit{Les fondements psycho-linguistiques des mathématiques}, Éditions du Griffon, Neuchâtel 1947. Pewne próby rozwinięcia tego tematu pojawiały się kilkukrotnie w~logice i~jej filozofii, np. pewną ciekawą propozycję podaje Fred Sommers, zob. F. Sommers, \textit{Predicability}, [w:] \textit{Philosophy in America}, red. M. Black, Routledge, London 2002, ss.~262-281. Mimo niezgodności dat, istnieje duże prawdopodobieństwo, że Gellman\index[names]{Gellman, Jerome I.} opierał swoją koncepcję na tekście: R.H. Thomason, \textit{A~semantic theory of sortal incorrectness}, ,,Journal of Philosophical Logic'', vol. 1 (1972), ss.~209-258, który stanowi próbę formalnego podejścia zarówno do kwestii negacji \textit{wykluczającej} i~negacji \textit{wyboru}, jak również zagadnienia zakresu przedmiotowego oddziaływania predykatów. Gellman\index[names]{Gellman, Jerome I.} jednak bezpośrednio nie odwołuje się do tej pracy. Nie podaje też żadnych definicji (formalnych bądź nieformalnych) tak określonych negacji.}. Naszym zamiarem jest stwierdzenie tylko, że to nieprawda, że predykat $P$~przysługuje Bogu, niekoniecznie sugerując jednocześnie, że można o~nim orzec dopełnienie tego predykatu\footnote{J.I. Gellman, \textit{The Meta-Philosophy of Religious Language}, dz. cyt., s.~158.}. Bóg jest poza jego gatunkowym zakresem, to znaczy, że nie należy do zbioru obiektów, o~których można orzec $P$ lub nie-$P$.

I~w~drugą stronę -- zdanie ,,Bóg jest potężny'' myśliciel apofatyczny zrozumie jako negację dopełnienia predykatu ,,jest potężny''. W~innym kontekście oznaczałoby to przypisanie obiektowi, o~którym mowa, tej właśnie własności. Jednakże w~przypadku Boga negowanie dopełnienia predykatu $P$~nie oznacza przypisywania mu $P$, ponieważ dopełnienie dane jest innym rodzajem negacji -- negacją \textit{wykluczającą}. Mówiąc ogólnie, zdania języka religijnego negują dopełnienia wszystkich wymienianych przez nie własności Boga, który jest poza zakresem wszystkich naszych predykatów. One, z~kolei -- będąc predykatami ,,języka skończonych bytów'' -- z~konieczności muszą oznaczać niedoskonałe własności\footnote{Zob. tamże.}.

W~końcu, Gellman\index[names]{Gellman, Jerome I.} odrzuca teologię negatywną -- nie jako teorię niespójną i~wewnętrznie sprzeczną, lecz jako teorię nieadekwatną. Po pierwsze, ze względu na wynikającą z~niej niepoznawalność Boga. Po drugie, dlatego, że w~ramach tej teorii nie można postawić wewnętrznych problemów teologicznych. Na przykład mówienie o~wszechmocy, wszechwiedzy i~dobroci Boga -- w~Gellmana\index[names]{Gellman, Jerome I.} interpretacji teologii apofatycznej -- oznacza jedynie, iż nie przypisujemy mu takich własności, jak słabość, głupota czy moralna niedoskonałość. Nie twierdzimy przy tym, że można o~nim orzekać takie predykaty, jak moc, wiedza czy dobroć. W~takim wypadku nie ma mowy o~\textit{wewnętrznym} problemie religijnym wynikającym z~uznania boskiej wszechmocy w~obliczu obecnej w~świecie niesprawiedliwości. Skreśla to, w~oczach Gellmana\index[names]{Gellman, Jerome I.}, teologię apofatyczną z~listy adekwatnych teorii służących do analizy znaczenia języka religijnego.


\section{Dyskusja}

Trudno nie zwrócić uwagi na to, że w~swoim naturalno-językowym sformułowaniu \ref{sil-gell-snt} przyjmuje pewną postać zasady o~treści zbliżonej do tej wyrażonej przez \ref{sil-kug-unt}. Sam Gellman\index[names]{Gellman, Jerome I.} zauważa, że jeśli traktujemy istnienie jako kwantyfikator a~nie jako predykat (co bynajmniej nie jest kontrowersyjnym podejściem), stwierdzenie istnienia Boga polegałoby w~gruncie rzeczy na stwierdzeniu istnienia obiektu poza rodzajowym zakresem wszystkich naszych predykatów\footnote{Tamże.}. Sens tak przedstawionej zasady można próbować zapisać syntaktycznie w~logice drugiego rzędu w~następujący sposób:
%(SNT')${\exists}$\textit{x}${\forall}$\textit{Q}${\neq}$(Q(\textit{x})${\vee}$${\neq}$Q(\textit{x})),
\begin{flalign*}
		&  \exists x \forall Q\ \neg(Q(x) \lor \neg Q(x)). &\tag{SNT'}\label{sil-gell-sntprim}
\end{flalign*}
gdzie $Q$~jest symbolem predykatowym. Nietrudno zaobserwować, że na gruncie klasycznej logiki drugiego rzędu wyrażenie to jest równoważne \ref{sil-kug-untprim}
\begin{flalign}
&\text{\ref{sil-kug-untprim}} \equiv \text{\ref{sil-gell-sntprim}}.&
\end{flalign}

Ta prosta obserwacja pozwala stwierdzić, że większość zarzutów przedstawionych w~poprzednim rozdziale\footnote{Zob. rozdz.~\ref{sil-kug-dyskusja}.} jest zasadna i~trafna także wtedy, gdy podniesie się je przeciw rozważaniom Gellmana\index[names]{Gellman, Jerome I.}. W~szczególności, jego ujęcie teologii negatywnej nie może być wolne od zarzutu sprzeczności.

Jak wspominałem powyżej, naturalno-językowy termin ,,Bóg'' nie ma jednoznacznie określonej kategorii językowej i~można go przedstawiać zarówno za pomocą stałej indywiduowej, jak i~predykatu $G(x)$ rozumianego jako ,,$x$ jest Bogiem''\footnote{Por. rozdz.~\ref{sil-kt-jez}.}. Podobnie jak w~przypadku poprzednich rozważań, by uniknąć dodatkowych niepożądanych konsekwencji przy formalizacji \ref{sil-gell-snt}, nie możemy reprezentować terminu ,,Bóg'' przy użyciu predykatu, ponieważ -- jak głosi \ref{sil-gell-snt} -- Bóg nie należy do przedmiotowego zakresu żadnego z~predykatów naszego języka, w~szczególności do zakresu odniesienia predykatu $G$. Taka próba doprowadziłaby do natychmiastowej sprzeczności\footnote{Skoro \ref{sil-gell-sntprim} $\equiv$ \ref{sil-kug-untprim}, dowód tego faktu został przedstawiony w~rozdziale \ref{sil-kug-zasady}.}.

Jednakże, by otrzymać parę zdań sprzecznych, wcale nie trzeba decydować się na taki sposób reprezentowania terminu ,,Bóg''. Pozostając wciąż na stosunkowo nieformalnym poziomie możemy z~łatwością uzyskać dobrze nam już znany paradoks samoodniesienia. Ostatecznie \ref{sil-gell-snt}, podobnie jak \ref{sil-kug-unt}, jest pewnym stwierdzeniem o~Bogu wyrażonym w~dosłownym języku. Głosi ono, że Bóg nie może znaleźć się w~zakresie przedmiotowym żadnej własności. Niech $B$~będzie taką własnością. Rozważmy teraz własność bycia poza zakresem przedmiotowym własności $B$~i oznaczmy ją przez $B^{**}$. Przyjmując \ref{sil-gell-snt} zmuszeni jesteśmy przyznać, że Bóg posiada własność $B^{**}$, podczas gdy -- wedle tej samej zasady -- nie może on znajdować się w~jej zakresie.

Powyższe argumenty, jakkolwiek przemyślne, sformułowane zostały na dosyć wieloznacznym poziomie języka, a~próba bardziej precyzyjnego ujęcia rozważań Gellmana\index[names]{Gellman, Jerome I.} w~postaci \ref{sil-gell-sntprim} może nie wydać się dla wszystkich satysfakcjonująca. \ref{sil-gell-sntprim} zdaje się nie wyrażać dostatecznie dobrze sensu \ref{sil-gell-snt}, ponieważ definicja \ref{sil-gell-srdef} przedmiotowego zakresu predykatu, na którym ta zasada została oparta, posiada wyraźny komponent semantyczny. Można zatem pokusić się o~rekonstrukcję \ref{sil-gell-snt} przy użyciu pojęć semantycznych. Jednakże i~w takim ujęciu teologia apofatyczna w~rozumieniu Gellmana\index[names]{Gellman, Jerome I.} nie zdoła uchronić się przed sprzecznością.

Niech $M = \langle U, \Delta\rangle$ będzie interpretacją języka taką, że $U$~jest zbiorem niepustym (naszym universum), a~$\Delta$ jest funkcją określoną na zbiorze poszczególnych stałych indywiduowych. Na potrzeby naszej analizy wspomnę tylko, że w~ramach interpretacji utożsamiamy własności ze zbiorami obiektów w~naszym uniwersum, które spełniają odpowiednią relację jednoargumentową, a~dla dowolnej stałej indywiduowej $a$, $\Delta (a) \in U$. Niech $\Delta(B)$ będzie zbiorem obiektów posiadających własność $B$. Oczywiście, $\Delta(B)$ jest podzbiorem $U$ ($\Delta(B)
\subset U$, w~ekstremalnych przypadkach $\Delta(B)$ może być zbiorem pustym). Zauważmy teraz, że zbiór obiektów posiadających własność $\neg B$~identyfikuje się z~dopełnieniem zbioru $\Delta(B)$
%($\overline{\Delta(B)} = U - \Delta(B)$).
($-\Delta(B) = U \setminus \Delta(B)$).
Niech $SR_B$ oznacza zakres przedmiotowy predykatu $B$. Zgodnie z~definicją \ref{sil-gell-srdef}:
\begin{flalign}
%&SR_B = \Delta(B) \cup \overline{\Delta(B)}.&
&SR_B = \Delta(B) \cup -\Delta(B).&
\end{flalign}

Posługując się tak zapisaną definicją można sformułować semantyczną zasadę teologii negatywnej (w rozumieniu Gellmana\index[names]{Gellman, Jerome I.}):
%(SNT'') xxx,
\begin{flalign*}
%		& \Delta(g) \notin SR_B = \Delta(B) \cup \overline{\Delta(B)} = U &\tag{SNT''}\label{sil-gell-sntbis}
		& \Delta(g) \notin SR_B = \Delta(B) \cup -\Delta(B) = U, &\tag{SNT''}\label{sil-gell-sntbis}
\end{flalign*}
gdzie przez stałą indywiduową $g$~oznaczyliśmy wyróżniony obiekt z~naszego uniwersum -- Boga. \ref{sil-gell-sntbis} stoi jednak w~sprzeczności z~samą definicją interpretacji semantycznej, zgodnie z~którą $\Delta(g) \in U$. Jeśli więc jesteśmy skłonni podtrzymywać takie rozumienie teologii negatywnej, musimy ponownie wyjść poza ramy logiki klasycznej. Niemniej jednak, w~tym przypadku nie mamy do dyspozycji wielu alternatyw, ponieważ metalogika jakiejkolwiek logiki, także logik nieklasycznych, jest zasadniczo logiką klasyczną\footnote{Por. np. B. Czernecka-Rej, \textit{On Four Types of Argumentation For Classical Logic}, ,,Roczniki Filozoficzne'', vol. 68, nr 4 (2020), ss.~284-285.}.


\chapter{Elementy logiki modalnej w~formalnej rekonstrukcji tezy o~niewysławialności}\label{sil-jac}

Zadanie obrony teologii milczenia przed zarzutami o~bycie teorią logicznie sprzeczną postawił przed sobą Jonathan D. Jacobs\index[names]{Jacobs, Jonathan D.} w~wyróżnionej nagrodą Sandersa\index[names]{Sanders, Marc} pracy The \textit{Ineffable, Inconceivable, and Incomprehensible God Fundamentality and Apophatic Theology}\footnote{J.D. Jacobs, \textit{The Ineffable, Inconceivable, and Incomprehensible God. Fundamentality and Apophatic Theology}, [w:] \textit{Oxford Studies in Philosophy of Religion}, vol.~6, red. J.~Kvanvig, Oxford University Press, Oxford 2015, ss.~158- 176. W~ostatnich latach ta właśnie praca zwróciła uwagę analitycznych metafizyków i~filozofów religii na teologię apofatyczną.}.

Warto nadmienić, że wedle zapewnień Jacobsa\index[names]{Jacobs, Jonathan D.} jego praca nie stanowi próby interpretacji jakiejkolwiek doktryny apofatycznej żadnego konkretnego teologia negatywnego -- czy to współczesnego, czy historycznego. Nie jest też teoretycznym rozwinięciem żadnego ze stanowisk teorii Niewysłowionego, lecz właśnie próbą \textit{obrony} takiej teorii. Co ciekawe, jest to próba dokonana z~dwoma dodatkowymi założeniami. Po pierwsze, teza o~niewysławialności Boga wzięta jest najzupełniej poważnie i~broniona w~swojej ,,pełnokrwistej'', silnej wersji. Jacobs\index[names]{Jacobs, Jonathan D.} nie uważa, że niewysławialność Boga wynika z~ułomności czy niedoskonałości ludzkiego systemu poznawczego. Bóg jest niewysławialny ze swej natury, jest to jego wewnętrzna własność. Po drugie, próba ta dokonywana jest w~kontekście doktryny chrześcijańskiej i~przy założeniu, że cały zbiór tez tej doktryny stanowi zbiór tez prawdziwych. Oznacza to, że obrony przed zarzutem sprzeczności wymaga nie tylko sama teoria Niewysłowionego. Przed paradoksem należy uchronić także jednoczesne utrzymywanie, że Bóg jest niewysławialny i~ jest ,,jakiś'', że nic o~nim nie można powiedzieć przy jednoczesnym twierdzeniu, że -- na przykład -- jest jeden w~trzech osobach itp.


\section{Prawdy fundamentalne i~przygodne}

Do obrony teorii Niewysłowionego Jacobs\index[names]{Jacobs, Jonathan D.} angażuje popularne w~ostatnich latach w~metametafizyce\footnote{\textit{Sic}! Dyscyplina ta nazywana jest również metaontologią.} pojęcie \textit{fundamentalności}\footnote{Por. R. Bliss, \textit{Fundamentality}, [w:] \textit{The Routledge Handbook of Metametaphysics}, red. tenże, J.T.M. Miller, Routledge, Abingdon -- New York 2020, ss.~211-221; R. Bliss, G. Priest (red.), \textit{Reality and its Structure: Essays in Fundamentality}, Oxford University Press, Oxford 2018; a~także T.E. Tahko, \textit{Fundamentality}, [w:] \textit{The Stanford Encyclopedia of Philosophy}, wyd. jesień 2018, red. E.N. Zalta, {\textless}\url{https://plato.stanford.edu/archives/fall2018/entries/fundamentality/}{\textgreater}.}
. Choć samo w~sobie pojęcie to ma na celu uchwycenie idei, wedle której w~świecie istnieją rzeczy podstawowe i~pierwotne, Jacobs\index[names]{Jacobs, Jonathan D.} koncentruje się raczej na tych rozumieniach fundamentalności, które dotyczą sposobów, w~jaki nasze reprezentacje próbują oddać strukturę świata. Otóż, zdaniem wielu współczesnych metafizyków analitycznych\footnote{Jacobs\index[names]{Jacobs, Jonathan D.} powołuje się wprost na trzech autorów: Kita Fine'a\index[names]{Fine, Kit}, Rossa Camerona\index[names]{Cameron, Ross} oraz Theodore'a Sidera\index[names]{Sider, Theodore}. Jak się jeszcze okaże, to ten ostatni wywarł największy wpływ na przedstawianą tu obronę teorii Niewysłowionego. Por. K. Fine, \textit{The Question of Ontology}, [w:] \textit{Metametaphysics. New Essays on the Foundations of Ontology}, red. D. Chalmers i~in., Oxford University Press, Oxford -- New York 2009, ss.~157-177; R.P. Cameron, \textit{Truthmakers and ontological commitment: or how to deal with complex objects and mathematical ontology without getting into trouble}, ,,Philosophical Studies: An International Journal for Philosophy in the Analytic Tradition'', vol. 140 (2008), nr 1, Selected Papers from the 2007 Bellingham Summer Philosophy Conference, ss.~1-18; T. Sider, \textit{Writing the Book of the World}, Oxford University Press, Oxford 2012.}, o~ile wszystkie prawdziwe reprezentacje odwzorowują świat, niektóre robią to w~specjalny sposób -- przedstawiają, jaki świat jest w~rzeczywistości, jak jest fundamentalnie (na fundamentalnym poziomie), jaka jest prawdziwa natura rzeczy lub też, posługując się platońską metaforą, ,,kroją rzeczywistość na części pierwsze''\footnote{Dosłownie ,,dzielą członki rzeczywistości w~stawach'', org. \textit{carves nature at its joints}. Ta pochodząca z~\textit{Fajdrosa} (265e) metafora przyjęła się idiomatycznie się w~j. angielskim, gdzie oznacza poprawną klasyfikację czy taksonomię lub podział względem właściwych i~trafnych \textit{fundamenta divisionis}.}. Pozwala to Jacobsowi\index[names]{Jacobs, Jonathan D.} wprowadzić podział na prawdy \textit{fundamentalne}, z~którymi mamy do czynienia, gdy prawdziwy sąd odwzorowuje świat w~taki właśnie fundamentalny sposób, oraz -- gdy tak się nie dzieje -- prawdy \textit{niefundamentalne} bądź \textit{przygodne}.

W~celu zilustrowania różnicy pomiędzy prawdami fundamentalnymi i~przygodnymi, Jacobs\index[names]{Jacobs, Jonathan D.} podpiera się przykładem zaczerpniętym od Theodore'a Sidera\footnote{T. Sider, \textit{Writing the Book of the World}, dz. cyt., ss.~1-2. Por. także J.D. Jacobs, \textit{The Ineffable}\ldots, dz.cyt., ss.~161-162.}: wyobraźmy sobie prostokąt utworzony poprzez połączenie dwóch kwadratów, z~których ten po lewej stronie jest biały, a~ten po prawej czarny (rys. \ref{sil-jac-siderpic}). Możemy zgodnie z~prawdą stwierdzić, że połowa prostokąta jest biała, a~połowa czarna oraz że fragment pola powierzchni prostokąta zamalowany kolorem białym jest równy polu powierzchni zamalowanej kolorem czarnym.
%Sider-pic
\begin{figure}[H]
\begin{center}

 \begin{tikzpicture}[every node/.style={inner sep=0,outer sep=0}, line cap=rect,node distance=4cm]

    \node (tl) {};
    \node (tm) [right=of tl] {};
    \node (tr) [right=of tm] {};
    \node (bl) [below=of tl] {};
    \node (bm) [below=of tm] {};
    \node (br) at (bl-|tr) {};

    \coordinate (CENTER) at ($(tl)!0.5!(br)$);
    \node (contra) at (CENTER) {};
    
    \draw[line width=1pt]     (tl) rectangle (br);
    \draw[fill]     (tm) rectangle (br);

    \path[-,color=white,dashed, line width=1pt,inner sep=0,outer sep=0] (contra) edge (tr);
    \path[-,dashed,line width=1pt,inner sep=0,outer sep=0]  (contra) edge (bl);
%    \draw [inner sep=0,outer sep=0](contra) --++ (bl);

\end{tikzpicture}

\caption[Eksperyment myślowy Sidera]{Eksperyment myślowy Sidera\index[names]{Sider, Theodore}. Dwukolorowy prostokąt, którego części zamalowane każdym z kolorów mają równe pola powierzchni.}\label{sil-jac-siderpic}
\end{center}
\end{figure}

Jednakże, rozważmy teraz pewną hipotetyczną społeczność językową, która nie posiada pojęcia ,,czarny'' oraz ,,biały''. Ma ona odmienne pojęcia kolorów -- i~tak, na przykład, zamiast rozpatrywać powyższy prostokąt jako podzielony \textit{pod względem koloru} na dwie równe części wzdłuż osi przechodzącej przez środki jego dłuższych boków, uznaje ona, że prostokąt podzielony jest \textit{kolorystycznie} wzdłuż osi przechodzącej przez jego dwa przeciwstawne wierzchołki. Według tej hipotetycznej społeczności językowej powstałe w~wyniku takiego podziału części prostokąta posiadają odmienne kolory. Mianowicie kolor w~ten sposób wydzielonej lewej górnej części prostokąta nazywają ,,biarnym'', a~kolor prawej dolnej części określają jako ,,czały''. Załóżmy teraz, że jeden z~członków takiej społeczności stwierdza, że ,,pole powierzchni prostokąta zamalowane kolorem czałym jest równe polu powierzchni prostokąta zamalowanemu kolorem biarnym''. Czy twierdzenie takie jest prawdziwe? Zdaniem Sidera\index[names]{Sider, Theodore} należy uznać, że tak. Uwzględniając znaczenia pojęć, jakich używa wypowiadająca taki sąd osoba, sąd ten jest prawdziwy -- biarna część prostokąta nie jest ani większa, ani mniejsza od czałej, ich pola powierzchni są równe. Jednakże, według Sidera\index[names]{Sider, Theodore}, ,,ciężko oprzeć się pokusie stwierdzenia, że ludzi ci się mylą''\footnote{Por. tamże.}. Mimo iż używając swoich specyficznych pojęć kolorów wypowiadają oni sądy prawdziwe, cały czas zdają się być niezdolni do uchwycenia obiektywnej struktury rzeczywistości.

Podążając za tymi intuicjami, Jacobs\index[names]{Jacobs, Jonathan D.} proponuje wprowadzenie do języka (modalnego) operatora zdaniowego $\mathscr{F}$, który zestawiony z~dowolnym zdaniem oznacza, że sąd wyrażony przez to zdanie jest sądem \textit{fundamentalnym}. Innymi słowy, wyrażenie $\mathscr{F}A$ powinniśmy czytać, jako ,,Fundamentalnie $A$'', ,,W rzeczywistości $A$'', ,,Sąd wyrażony przez $A$~odwzorowuje obiektywną wewnętrzną strukturę rzeczywistości'', ,,$A$ oddaje prawdziwą naturę rzeczy'' lub nawet ,,$A$ kroi rzeczywistość na części pierwsze''\footnote{Pomysł na wprowadzenie takiego operatora został zaczerpnięty częściowo od Kita Fine'a\index[names]{Fine, Kit} -- zob. K.~Fine, \textit{The Question of Ontology}, dz. cyt., s.~26 -- bezpośrednio zaś od Theodore'a Sidera\index[names]{Sider, Theodore}. Operator proponowany przez Jacobsa\index[names]{Jacobs, Jonathan D.} został jednak znacząco zmodyfikowany. Sider\index[names]{Sider, Theodore} sugeruje, by taki symbol mógł stać przy dowolnej ,,kategorii gramatycznej'' dowolnego wyrażenia językowego. W~szczególności dopuszcza on, by operator ten w~danym wyrażeniu wiązał jakiś funktor, na przykład koniunkcję -- zob. T. Sider, dz. cyt., ss.~216-238. W~mojej opinii, ograniczenie funktora $\mathscr{F}$~do pełnienia roli wyłącznie operatora zdaniowego jest słuszne. Jacobs\index[names]{Jacobs, Jonathan D.} wykazuje się tu względną ogładą logiczną i~chroni tę propozycję przed zarzutami o~brak większego logicznego sensu -- por. J. Jacobs, \textit{The Ineffable}\ldots, dz. cyt., s.~162.}.

Niestety, Jacobs\index[names]{Jacobs, Jonathan D.} nie rozwodzi się wystarczająco wnikliwie na temat precyzyjnego logicznego znaczenia tego operatora, jakie nadane by mu mogło zostać poprzez podanie określonych rządzących nim reguł lub aksjomatów w~obrębie któregoś z~modalnych systemów. Wiemy jedynie, jak zachowuje się fundamentalność ($\mathscr{F}$) względem negacji (brak ,,rozdzielności''):
%z~artykułu,\label{sil-jac-fundneg}
\begin{flalign}
& \nvdash \neg \mathscr{F} A \to \mathscr{F} \neg A & \label{sil-jac-fundneg}
\end{flalign}
oraz że zachowana jest tzw. reguła (\textsf{T}), tzn. sądy fundamentalne muszą być prawdzie:
%druga reg. z~art.\label{sil-jac-modalK}
\begin{flalign}
& \vdash \mathscr{F}  A \to A. & \label{sil-jac-modalK}
\end{flalign}

Prawdy fundamentalne, czyli prawdziwe sądy, których ,,struktura idealnie odzwierciedla strukturę rzeczywistości'' możemy teraz zapisywać jako $\mathscr{F}A$\footnote{Jacobs\index[names]{Jacobs, Jonathan D.} prawdy fundamentalne zapisuje używając koniunkcji: $A \land \mathscr{F}A$. Jest to oczywista redundancja -- $A$~wynika bezpośrednio z~$\mathscr{F}A$ na mocy prawa \ref{sil-jac-modalK}.}. Zdania wyrażające prawdziwe sądy, których struktura nie odzwierciedla rzeczywistości lub odzwierciedla w~ją w~sposób niedostateczny, nazywane praz Jacobsa\index[names]{Jacobs, Jonathan D.} prawdami niefundamentalnymi lub przygodnymi, będziemy zapisywać jako $A \land \neg \mathscr{F}A$.


\section{Fundamentalna zasada teologii negatywnej i~milczenie w~,,teologicznym pokoju''}

Mając zdefiniowane pojęcia prawd fundamentalnych i~prawd przygodnych, Jacobs\index[names]{Jacobs, Jonathan D.} powołuje do życia kolejną wersję zasady teologii negatywnej. Niech $\mathcal{G}$~będzie zbiorem wszystkich prawdziwych zdań o~tym, jaki Bóg jest wewnętrznie, jaki jest ze swojej natury.
%FNT
\begin{flalign*}
	& \forall_{A \in \mathcal{G}}\ \neg \mathscr{F} A \land \neg \mathscr{F} \neg A. &\tag{FNT}\label{sil-jac-fnt}
\end{flalign*}
W~konsekwencji, nie istnieje żaden prawdziwy \textit{fundamentalny} sąd o~Bogu i~jego naturze -- każdy prawdziwy sąd na ten temat może być co najwyżej prawdą \textit{przygodną}.

Jak powyższa zasada wiąże się z~teologią milczenia? Otóż, jeśli ograniczymy się do wypowiadania o~Bogu wyłącznie prawd fundamentalnych, to nic o~nim nie będziemy mogli powiedzieć. Jacobs\index[names]{Jacobs, Jonathan D.} wyjaśnia ten związek poprzez wprowadzenie metafory ,,pokoju teologicznego''\footnote{Lub ,,pokoju teologii'' (org. \textit{theology room}). Ta metafora także została pożyczona od Sidera\index[names]{Sider, Theodore}, który do zilustrowania swoich przemyśleń o~fundamentalności i~ugruntowaniu używa przenośni ,,pokoju metafizyki''. Por. T. Sider, \textit{Writing the Book of the World}, dz. cyt., ss.~74-77.}. Wchodzimy do pokoju teologicznego zastrzegając, że to, co mówimy o~Bogu, wyraża wyłącznie sądy fundamentalne i~nic ponadto.

\begin{quote}
Jeśli istnieje jakaś fundamentalna prawda wystarczająco bliska temu, co mamy na myśli, możemy ją wypowiedzieć. Jeśli istnieje fundamentalny sąd, ale jest on fundamentalnie fałszywy, możemy go wypowiedzieć i~coś stwierdzić, choć to, co stwierdzimy, będzie fałszywe. Jeśli natomiast nie ma żadnego fundamentalnego sądu wystarczająco bliskiego temu, co mamy na myśli, [\ldots] nie możemy nic stwierdzić. [\ldots]

Jeśli Teza o~Niewyrażalności [\ref{sil-jac-fnt} -- P.U.] jest prawdziwa, a~my wchodzimy do pokoju teologicznego, nie pozostaje nam nic innego, jak zachować milczenie. Nie możemy nic powiedzieć. Jeśli chcielibyśmy opisać Boga w~jakikolwiek sposób -- jako kochającego, miłosiernego czy cierpiącego -- musielibyśmy opuścić pokój teologiczny\footnote{J. Jacobs, \textit{The Ineffable}\ldots, dz. cyt., s.~166.}.
\end{quote}

Teologia apofatyczna w~rozumieniu Jacobsa\index[names]{Jacobs, Jonathan D.} polega więc na tym, że nie możemy konstruować, utrzymywać i~wypowiadać żadnych fundamentalnych sądów o~Bogu i~jego naturze. Możemy konstruować (i konstruujemy) o~nim wyłącznie sądy przygodne. Mogą one być prawdziwe bądź nie, ale każdy, nawet prawdziwy sąd o~Bogu, nie będzie nigdy sądem fundamentalnie prawdziwym. Wchodząc do ,,teologicznego pokoju'' musimy zachować milczenie.


\section{Dyskusja}

Przypomnijmy, że głównym celem pracy Jacobsa\index[names]{Jacobs, Jonathan D.} jest obrona teologii milczenia przed dwoma rodzajami sprzeczności: 1)~,,wewnętrzną'' sprzecznością tej teorii, polegającą na przywoływanym już wielokrotnie paradoksie samoodniesienia oraz 2)~,,zewnętrzną'' sprzecznością, czyli absurdem polegającym na twierdzeniu, że o~Bogu nic nie można powiedzieć, przy jednoczesnym twierdzeniu, że jest \textit{jakiś}, na przykład miłosierny, jeden w~trzech osobach itp.

Zdaniem Jacobsa\index[names]{Jacobs, Jonathan D.} opisana powyższej strategia ukrycia par zdań sprzecznych za (zanegowanym) operatorem fundamentalności $\mathscr{F}$ w~obrębie \ref{sil-jac-fnt} wraz z~ograniczeniem rozdzielności negacji względem fundamentalności \eqref{sil-jac-fundneg} pozwala zachować spójność teorii Niewysłowionego w~węższym, ,,wewnętrznym'' sensie. Zacznijmy jednak od próby obrony teorii niewysłowionego przed sprzecznościami w~tym drugim, szerszym, ,,zewnętrznym'' sensie, ponieważ próba ta, w~mojej opinii, wydaje się bardziej skuteczna. Podział prawd na fundamentalne i~przygodne sprawia, że niesprzeczne wydaje się utrzymywanie i~wypowiadanie dowolnego sądu o~naturze Boga z~jednoczesnym zachowaniem teorii Niewysłowionego, to znaczy z~równoczesnym twierdzeniem, że Bóg jest niewysławialny. Zgodnie z~pierwotnym założeniem Jacobsa\index[names]{Jacobs, Jonathan D.}, można niesprzecznie z~\ref{sil-jac-fnt} uznać, że cały zbiór tez doktryny chrześcijańskiej jest zbiorem zdań prawdziwych. Zdania te są prawdziwe, choć wyrażają jedynie sądy niefundamentalne. Są prawdziwe, ale prawdziwe przygodnie.

Mimo iż Jacobs\index[names]{Jacobs, Jonathan D.} zastrzegał, że jego rozważania nie stanowią interpretacji żadnej z~doktryn teologii negatywnej któregokolwiek z~jej protagonistów -- czy to historycznych, czy współczesnych -- sam wskazuje, że jego pomysł może służyć do modelowania teologii Pseudo-Dionizego Areopagity\index[names]{Pseudo-Dionizy Areopagita}. W~ramach tej propozycji możemy reprezentować trzystopniowe ,,wspinanie się po apofatycznej drabinie''. Zaczynamy od stwierdzenia jakiejś (przygodnej) prawdy w~ramach teologii katafatycznej, na przykład, że Bóg jest jeden w~trzech osobach ($p$). Następnie przechodzimy do pierwszego stopnia zaprzeczenia -- stwierdzamy, że nie jest prawdą, że fundamentalnie Bóg jest jeden w~trzech osobach ($\neg \mathscr{F}p$). W~ostatnim kroku akceptujemy zaprzeczenie zaprzeczenia -- uznajemy, że nieprawda, że fundamentalnie nie jest tak, że Bóg jest jeden w~trzech osobach ($\neg \mathscr{F} \neg p$). Wedle propozycji Jacobsa\index[names]{Jacobs, Jonathan D.} możemy więc przyjmować dowolną doktrynę religijną -- na przykład chrześcijańską doktrynę o~trzech hipostazach jednej boskiej substancji -- pod warunkiem, że tezy takiej doktryny uznamy wyłącznie za prawdy przygodne. Fundamentalnej prawdy o~Bogu nigdy nie poznamy. Bóg Jacobsa\index[names]{Jacobs, Jonathan D.} jest wysławialny, choć przygodnie. Fundamentalnie zaś pozostaje niewysławialny.

Warto zauważyć, że w~konsekwencji w~obronę teorii Niewysłowionego przed paradoksem Jacobs\index[names]{Jacobs, Jonathan D.} angażuje pewien rodzaj teorii podwójnej prawdy. W~obrębie jego rozważań taka obrona jest możliwa tylko, gdy przyjmie się, że \textit{w~pewnym sensie} tezy doktryny religijnej (np. teza o~Trójcy Świętej) są prawdziwe, w~\textit{innym} są fałszywe. Uwaga ta jest o~tyle interesująca, o~ile koncepcja podwójnej prawdy pojawiła się w~historii myśli po raz pierwszy w~kontekście relacji i~związków rozumu i~wiary, a~konkretniej przy potępieniu\footnote{Zob. H. Thijssen, \textit{Condemnation of 1277}, [w:] \textit{The Stanford Encyclopedia of Philosophy}, wyd. zima 2018, red. E.N. Zalta, {\textless}\url{https://plato.stanford.edu/archives/win2018/entries/condemnation/}{\textgreater}.} prób pogodzenia z~doktryną chrześcijańską nowej -- przynajmniej dla ówczesnych, średniowiecznych myślicieli -- filozofii arystotelesowskiej\index[names]{Arystoteles}. Jak pokazuje Bartosz Brożek\index[names]{Brożek, Bartosz}\footnote{Zob. B. Brożek, \textit{The Double Truth Controversy: An Analytical Essay}, Copernicus Center Press, Kraków 2010.}, sama teoria podwójnej prawdy generuje swoje filozoficzne problemy i~jest ciekawym i~wdzięcznym do logicznych analiz zagadnieniem. W~tym miejscu jednak nie będę roztrząsał tych filozoficznych problemów i~logicznych zagadnień. Mogę natomiast przyznać rację Jacobsowi\index[names]{Jacobs, Jonathan D.}\index[names]{Jacobs, Jonathan D.}, że zastosowanie takiego ,,wytrychu'' rozwiązuje problemy z~,,zewnętrznym'' rodzajem sprzeczności teologii milczenia. Trudno jednak nie oprzeć się wrażeniu, że rozwiązuje ten problem w~sposób trywialny -- podobne konstatacje mogą wyrugować każdą sprzeczność z~dowolnej teorii. Dodatkowo, robi to w~sposób, który nie może zadowolić żadnej ze stron teologicznego spektrum -- ani myśliciela apofatycznego, ani teologa katafatycznego.

Teolog katafatyczny (zwłaszcza apologeta lub taki, którego celem jest obrona prawomyślności wiary) z~pewnością nie będzie doceniał faktu, że w~doktrynie Jacobsa\index[names]{Jacobs, Jonathan D.} wszystkie prawdy religijne uznaje się za niefundamentalne, a~do wyrażenia natury Boga wystarczyć muszą sądy o~charakterze przygodnym. Jeśliby przyjąć takie rozumienie prawd fundamentalnych i~przygodnych, jakie jest tu proponowane, należy stwierdzić, że żadna z~ksiąg teologicznych nie mówi o~niczym, co odpowiadałoby rzeczywistości -- cokolwiek jakikolwiek teolog ustali na temat Boga, jakkolwiek go nie opisze, trzeba sądzić, że \textit{w~rzeczywistości} tak nie jest. Z~drugiej strony, przedstawiona w~powyższy sposób teologia negatywna traci swój ,,apofatyczny pazur''. Bóg przestaje być niewysławialny, niewyrażalny i~nieopisywalny. W~gruncie rzeczy możemy -- nie naruszając ducha tak sformatowanego apofatyzmu -- twierdzić o~nim cokolwiek godząc się jedynie na to, że mówiąc o~Bogu wyrażamy co najwyżej sądy niefundamentalne.

Można jednak spróbować wziąć za dobrą monetę takie rozumienie apofatyzmu powołując się na fakt, że rozważania Jacobsa\index[names]{Jacobs, Jonathan D.} są dobrze ugruntowane w~filozoficznych ustaleniach Sidera\index[names]{Sider, Theodore}. Problem w~tym, że ustalenia te Jacobs\index[names]{Jacobs, Jonathan D.} przyjmuje właściwie bezkrytycznie i~bezrefleksyjnie -- do pewnego stopnia w~celowy i~świadomy sposób. Tymczasem nietrudno przedstawić argumenty przeciw tej metametafizycznej wizji. Jednym z~takich argumentów może być zwrócenie uwagi na fakt, że podział na prawdy fundamentalne i~przygodne sam nie wydaje się fundamentalny. Trudno oprzeć się wrażeniu, że to dychotomia fundamentalne–przygodne odpowiada parze przymiotników ,,czałe''–,,biarne'' z~eksperymentu myślowego Sidera\index[names]{Sider, Theodore} (rys. \ref{sil-jac-siderpic}), a~dużo bliższy ,,struktury rzeczywistości'' wydaje się być, na przykład, tradycyjny podział na sądy prawdziwe i~fałszywe. Nawet jeśli tak nie jest, to nie do końca wiadomo, jakie są kryteria oceny fundamentalności pojęć, prawd i~sądów. Dlaczego i~na jakiej podstawie jedne sądy mamy uważać za fundamentalne, a~innym odmawiać odzwierciedlania struktury rzeczywistości? Od pytania o~kryterium fundamentalności niedaleko już do oskarżenia metametafizycznej teorii Sidera\index[names]{Sider, Theodore} o~epistemiczną arogancję i~metafizyczny imperializm -- dlaczego mamy uważać, że to akurat nasze pojęcia oddają rzeczywistość taką, jaka jest, a~inne pojęcia (na przykład pojęcia jakiejś hipotetycznej odmiennej społeczności językowej) są niedoskonałe i~,,czegoś im brakuje''\footnote{T. Sider, \textit{Writing the Book of the World}, dz. cyt., s.~2.}. Ewidentnie, Siderowi\index[names]{Sider, Theodore} nie wystarczyło tej epistemicznej pokory, którą wykazał się Jacobs\index[names]{Jacobs, Jonathan D.} zakładając, że może istnieć rzeczywistość, do której nigdy nie będziemy mieli ,,fundamentalnego'' dostępu. Jednakże, przedstawiając swoje rozważania na temat prawd o~charakterze fundamentalnym i~przygodnym, Jacobs\index[names]{Jacobs, Jonathan D.} odcina się od takich oraz szeregu innych filozoficznych dyskusji. Na przykład, świadomie nie rozstrzyga kwestii ontologicznego charakteru samych reprezentacji, sądów czy postaw propozycjonalnych. Równie dobrze mogą to być obiekty abstrakcyjne, językowe czy intencjonalne stany mentalne. Za pierwotne wzięte tu zostało pojęcie struktury rzeczywistości, zatem po prostu ,,mówimy, że \guillemotleft$\mathscr{F}A$\guillemotright, gdy $A$~idealnie odwzorowuje [\ldots] strukturę rzeczywistości''\footnote{J. Jacobs, \textit{The Ineffable}\ldots, dz. cyt., s.~163.}. Podobnie, nie argumentuje on za słusznością przyjętej przez niego wersji metametafizycznej koncepcji fundamentalności oraz jej konsekwencji. Jak sam zauważa, mogą z~niej wynikać pewne istotne, choć wysoce dyskusyjne z~filozoficznego punktu widzenia, twierdzenia -- na przykład takie, że istnieją sądy jednocześnie prawdziwe i~niefundamentalne. Mimo iż jest przekonany, że taki sposób myślenia o~tym, jak nasze reprezentacje odwzorowują świat, jest ,,prawdopodobnie prawdziwy''\footnote{Tamże, s.~164.}, celowo powstrzymuje się on przed argumentowaniem za tym, że jest
%to pogląd słuszny.
tak w~istocie.
Do osiągnięcia zamierzonego celu -- czyli obrony teologii milczenia przed zarzutami o~bycie logicznie sprzeczną -- nie potrzebuje on wykazywania prawdziwości czy słuszności podziału na prawdy fundamentalne i~niefundamentalne czy teorii, jaka za takim podziałem stoi. Wystarczy mu uznanie, że podział ten i~teoria są logicznie spójne. Jednakże nie wszyscy autorzy zgadzają się z~nim w~tej kwestii.

Michael Rea\index[names]{Rea, Michael C.} i~Samuel Lebens\index[names]{Lebens, Samuel R.} sugerują, że w~rozważaniach Jacobsa\index[names]{Jacobs, Jonathan D.} kryje się sprzeczność\footnote{Zob. S.R. Lebens, \textit{Why so negative about negative theology? The search for a~Plantinga-proof apophaticism}, ,,International Journal for Philosophy of Religion'', vol. 76 (2014), nr 3, przypis 1; M.C. Rea, \textit{Essays in Analytic Theology}, vol.~1, ser. \textit{Oxford Studies in Analytic Theology}, Oxford University Press,  Oxford 2021, ss. 120-138.}, choć sami wskazują ją w~niewłaściwym miejscu. Zauważają oni, że -- choć Jacobs\index[names]{Jacobs, Jonathan D.} akceptuje wprost prawa logiki klasycznej -- konstrukcja \ref{sil-jac-fnt} sprawia, że w~obrębie prawd fundamentalnych naruszone zostaje prawo wyłączonego środka\footnote{Które sami nazywają prawem ,,dwuwartościowości'', a~w tym kontekście nawet ,,fundamentalnym prawem dwuwartościowości''. Por. tamże.}
\begin{flalign}
&\mathscr{F} A \lor \mathscr{F} \neg A.&\label{sil-jac-fundtertium}
\end{flalign}
Następnie Lebens\index[names]{Lebens, Samuel R.} próbuje wykazać, że
\begin{flalign}
&\neg (\mathscr{F} A \lor \mathscr{F} \neg A) \vdash \mathscr{F} A \land \mathscr{F} \neg A,&\label{sil-jac-leb}
\end{flalign}
ale przedstawione przez niego wnioskowanie zawiera szkolny błąd\footnote{Lebens\index[names]{Lebens, Samuel R.} sam przyznaje się do tego błędu w~pracy opublikowanej trzy lata później. Por. S.R.~Lebens, \textit{Negative Theology as Illuminating and/or Therapeutic Falsehood}, [w:] \textit{Negative Theology as Jewish Modernity}, red. M. Fagenblat, Indiana University Press, Bloomington 2017, przypis 29. Co ciekawe, mimo przyznania, że przedstawione przez niego wcześniej wnioskowanie zawierało błąd, Lebens\index[names]{Lebens, Samuel R.} wciąż utrzymuje, że naruszenie zasady wyłączonego środka w~obrębie prawd fundamentalnych musi doprowadzić do sprzeczności.}. Nie da się wykazać takiego wynikania, jeśli ograniczymy się -- jak robi to Lebens\index[names]{Lebens, Samuel R.} -- wyłącznie do praw logiki klasycznej. Może jednak ono zajść przy odpowiedniej interpretacji operatora $\mathscr{F}$. Kwestia ta zostanie jeszcze poruszona w~dalszej części niniejszego paragrafu. Na razie zauważmy, że -- choć dowód formuły \ref{sil-jac-leb} podany przez Lebensa\index[names]{Lebens, Samuel R.} nie jest poprawny -- rzeczywiście, ,,fundamentalne'' prawo wyłączonego środka \eqref{sil-jac-fundtertium} nie tyle zostaje naruszone, co jego negacja jest prostą logiczną konsekwencją \ref{sil-jac-fnt} -- i~można tego dowieść (tym razem skutecznie) na gruncie wyłącznie praw logiki klasycznej.
\begin{flalign}
& \neg \mathscr{F} A \land \neg \mathscr{F} \neg A &  \eqref{sil-jac-fnt}\label{negf3nd1} \\
& (\neg \mathscr{F} A \land \neg \mathscr{F} \neg A) \to \neg (\mathscr{F} A \lor \mathscr{F} \neg A) & \qquad\qquad \text{(p. De Morgana\index[names]{De Morgan, Augustus})}\label{negf3nd2}  \\
& \neg (\mathscr{F} A \lor \mathscr{F} \neg A) & (\text{MP }\ref{negf3nd1},\ref{negf3nd2})\label{negf3nd3}
\end{flalign}

Odnotujmy w~takim razie, że po raz kolejny próba podania sformalizowanej teorii teologii negatywnej prowadzi do przyjęcia pewnej formy zanegowanego prawa wyłączonego środka. W~konsekwencji, stosowne uwagi krytyczne sformułowane w~poprzednich rozdziałach będą miały zastosowanie także do teorii Jacobsa\index[names]{Jacobs, Jonathan D.}. By jednak wykazać, że anonsowana przez Jacobsa\index[names]{Jacobs, Jonathan D.} obrona teologii apofatycznej jest nieefektywna i~sama prowadzi do sprzeczności, spróbujmy wpierw dokonać rekonstrukcji (modalnego) rachunku logicznego, będącego podstawą dla właściwej interpretacji operatora fundamentalności.

Oddajmy Jacobsowi\index[names]{Jacobs, Jonathan D.} to, że podczas oznaczania sądów fundamentalnych i~przygodnych oraz wyrażania tezy o~niewysławialności (\ref{sil-jac-fnt}) posługuje się względnie formalną notacją. Wprowadza do języka operator zdaniowy ($\mathscr{F}$) o~wyraźnie modalnych cechach. Nie podaje on jednak dokładnej charakterystyki rachunku, w~ramach którego jego teoria jest konstruowana. Mimo tego, fakt, że w~pracy zawarto dwie uwagi dotyczące działania operatora $\mathscr{F}$ -- prawa \eqref{sil-jac-fundneg} oraz \eqref{sil-jac-modalK} -- sprawia, że podjęcie próby
%rekonstrukcji
identyfikacji
takiego rachunku nie musi pozostać bezowocne. Zauważmy najpierw, że -- znów na gruncie samej logiki klasycznej -- \ref{sil-jac-fnt} pociąga za sobą negację prawa dystrybucji negacji względem $\mathscr{F}$. %(\ref{sil-jac-fundneg}).
\begin{flalign}
& \neg \mathscr{F} A \land \neg \mathscr{F} \neg A &  \eqref{sil-jac-fnt}\label{negax1} \\
& (\neg \mathscr{F} A \land \neg \mathscr{F} \neg A) \to \neg(\neg \mathscr{F} A \to \mathscr{F} \neg A) & \qquad \text{(p. negacji implikacji)}\label{negax2}  \\
& \neg(\neg \mathscr{F} A \to \mathscr{F} \neg A) & (\text{MP }\ref{negax1},\ref{negax2})\label{negax3}
\end{flalign}
Przyjmując \ref{sil-jac-fnt} możemy więc powstrzymać się od bezpośredniego stosowania \eqref{sil-jac-fundneg} w~celu scharakteryzowania operatora $\mathscr{F}$, przynajmniej w~obrębie zbioru zdań o~boskiej naturze. Zatem która z~logik modalnych będzie właściwa dla oddania charakterystyki $\mathscr{F}$? Należy lojalnie przyznać, że sprawa ta nie jest ostatecznie rozstrzygnięta nawet w~przypadku drobiazgowo analizowanej aletycznej logiki modalnej, w~której operatorów modalnych używa się na oznaczenie pojęć możliwości i~konieczności. Samą konieczność, dużo lepiej ugruntowaną w~filozoficznych rozważaniach i~uważaną za pierwotną motywację do powstania logik modalnych, można logiczne wyrażać w odmiennych rachunkach i~na różne sposoby. Spróbujmy zatem pozostać uczciwym w~stosunku do autora krytykowanej tu pracy i~powstrzymać się od tworzenia ,,słomianej kukły'', z~którą on sam by się nie mógł utożsamiać. Na podstawie wszystkich powyższych uwag można przyjąć, że najlepszym kandydatem na rachunek, w~ramach którego powinno się dokonywać rekonstrukcji rozważań Jacobsa\index[names]{Jacobs, Jonathan D.}, jest najsłabsza z~normalnych logik modalnych zawierająca aksjomat \eqref{sil-jac-modalK}, czyli system K\footnote{System K~to system modalny, który posiada prawo dystrybucji, regułę wymuszania oraz \ref{sil-jac-modalK}. Por. J. Garson, \textit{Modal Logic}, [w:] \textit{The Stanford Encyclopedia of Philosophy}, wyd. lato 2021, red. E.N. Zalta, {\textless}\url{https://plato.stanford.edu/archives/sum2021/entries/logic-modal/}{\textgreater}. Oczywiście, wybór systemu K~nie jest niekontrowersyjny, ponieważ nie jest do końca jasne, jaki system modalny życzyłby sobie stosować Jacobs\index[names]{Jacobs, Jonathan D.}. W~ostatnich paragrafach, w~których odpierane są hipotetyczne zarzuty, zdaje się on wyrażać chęć zablokowania także $ \mathscr{F} p \to \mathscr{F} \neg \neg p$~lub nawet $ \mathscr{F} p \to \mathscr{F} (p \lor q)$. W~kontekście rozważań Jacobsa\index[names]{Jacobs, Jonathan D.} kontrowersyjna może wydać się także, obecna we wszystkich systemach normalnej logiki modalnej, reguła wymuszania.} (wzbogacony, rzecz jasna, o~\ref{sil-jac-fnt}).

W~celu zbadania spójności ,,fundamentalnej'' wersji teologii milczenia, odpowiedzmy wpierw na pytanie, czy samo \ref{sil-jac-fnt}, tak sformułowane, jest sądem fundamentalnym, czy przygodnym. Rozważania Jacobsa\index[names]{Jacobs, Jonathan D.} zostawiają pole do interpretacji ciągnących w~stronę obu tych rozwiązań. Z~jednej strony uważa on, że ,,Bóg jest niewysławialny fundamentalnie'' (a ,,wysławialny przygodnie''\footnote{J. Jacobs, \textit{The Ineffable}\ldots, dz. cyt., s.~167.}), zatem
\begin{flalign}
&\mathscr{F} \eqref{sil-jac-fnt}. \label{sil-jac-funFNT}&
\end{flalign}
Ostatecznie, przecież \ref{sil-jac-fnt} jest jakimś sądem na temat boskiej natury. Jak twierdzi Jacobs\index[names]{Jacobs, Jonathan D.}, ,,Bóg jest niewysławialny, niemożliwy do zrozumienia i~niepojmowalny \textit{sam w~sobie}, jest taki wewnętrznie''\footnote{Tamże, s.~165. Omar Fakhri próbuję rozwinąć pomysł Jacobsa\index[names]{Jacobs, Jonathan D.} w~stronę, która sugerowałaby, że niewysławialność Boga jest ugruntowana raczej w~niedoskonałościach ludzkiego języka: O. Fakhri, \textit{The ineffability of God}, ,,International Journal for Philosophy of Religion'', vol. 89 (2021), ss.~25-41.}. O~ile zakładamy, że \ref{sil-jac-fnt} jest prawdziwe (a wynika to bezpośrednio z~\ref{sil-jac-funFNT}\ oraz \ref{sil-jac-modalK}), to $\text{\ref{sil-jac-fnt}} \in \mathcal{G}$. Zatem, na mocy samego \ref{sil-jac-fnt},
\begin{flalign}
&\neg \mathscr{F} \eqref{sil-jac-fnt}.&
\end{flalign}


Można próbować jeszcze bronić teorię Jacobsa\index[names]{Jacobs, Jonathan D.} przyjmując, że \ref{sil-jac-fnt} jest prawdą niefundamentalną. Niemniej, jeśli poruszamy się w~obrębie systemu K, na mocy reguły wymuszania otrzymamy $\mathscr{F} \eqref{sil-jac-fnt}$ lądując w~bezpośredniej sprzeczności. Oczywiście, obrońca doktryny Jacobsa\index[names]{Jacobs, Jonathan D.} może podnieść zarzut, że reguła wymuszania, obecna w~systemie K (i~wszystkich normalnych logikach modalnych), blokuje pożądaną możliwość istnienia prawd prawdziwych, choć niefundamentalnych. Problem w~tym, że taki obrońca niespecjalnie miałby co zaoferować w~zamian. Alternatywy, jakie miałby w zanadrzu, nie są atrakcyjne z~logicznego punktu widzenia. Są tu dwie drogi wyjścia. Można 1) zrezygnować z~pełności rachunku, który przyjmujemy za ramę naszych rozważań. Taką alternatywę odrzucam ze względu na jej nieracjonalność. Gdybyśmy do formalizacji twierdzeń teologii apofatycznej dopuszczali rachunki, co do których nie zachodzi twierdzenie o~pełności, nasze rozwiązania zostałyby strywializowane i~pozbawione sensu. Można też 2) zaproponować w~zamian jakiś rachunek modalny nieposiadający reguły wymuszania i~prawa dystrybucji. Postępujący tą drogą obrony zwolennik doktryny Jacobsa\index[names]{Jacobs, Jonathan D.} miałby do dyspozycji tzw. nienormalną logikę modalną, która posiada taką właśnie charakterystykę. Problem w~tym, że ten wybór zablokuje z~kolei inne pożądane cechy systemu, np. podawane \textit{explicite} prawo \eqref{sil-jac-modalK}\footnote{Zresztą, do nienormalnej logiki modalnej można podnosić też pewne wątpliwości natury filozoficznej, zob. M. Klonowski, K. Krawczyk, \textit{Problem wszechwiedzy logicznej. Krytyka światów nienormalnych i~propozycja nowego rozwiązania}, ,,Filozofia Nauki'', vol. 27 (2019), nr 1, ss.~27-48.}. Zatem, nawet gdy założymy, że \ref{sil-jac-fnt} jest tylko prawdą przygodną, problem samozwrotności generujący sprzeczności wciąż pozostanie aktualny. Skoro twierdzimy, że apofatyzm polega na ograniczeniu się do wypowiadania wyłącznie fundamentalnych prawd i~sądów, musimy także zrezygnować z~utrzymania \ref{sil-jac-fnt}. Mówiąc krótko, przy takim założeniu musimy przemilczeć także ten fakt, że Bóg jest niewyrażalny. Używając metafory Jacobsa\index[names]{Jacobs, Jonathan D.}, aby wejść do ,,teologicznego pokoju'', musimy powstrzymać się od wchodzenia do środka i~pozostać na zewnątrz.


\chapter{Klasyczna teoria niewysławialności}\label{sil-boch}

Swoje rozważania dotyczące teologii apofatycznej Józef Maria Bocheński\index[names]{Bocheński, Józef Maria} zawarł w~dziele \textit{Logika religii}\footnote{J.M. Bocheński, \textit{Logika religii}, tłum. S. Magala, Instytut wydawniczy PAX, Warszawa 1990. Wyd. org.: J.M. Bocheński, \textit{The Logic of Religion}, New York University Press, New York 1965.}. Przeprowadza je w~kontekście analizy dyskursu religijnego -- jego języka, logiki oraz znaczenia, jakie przez taki dyskurs jest przekazywane. W~tym kontekście Bocheński\index[names]{Bocheński, Józef Maria} odróżnia teologię negatywną od teorii tego, co niewysłowione. Ta pierwsza przyznaje dyskursowi religijnemu jakieś znaczenie, choć ogranicza je do ,,czystych negacji''. Ta druga prowadzi do stwierdzenia, że dyskurs religijny nie posiada żadnego znaczenia. W~obu przypadkach Bocheński\index[names]{Bocheński, Józef Maria} próbuje wykazać, że teorie te wcale nie muszą zawierać bezpośredniej sprzeczności i~pokazuje warunki, przy których należy je uznać za spójne. Jednakże ostatecznie odrzuca je -- przyznając, że mimo wszystko nie mogą one stanowić właściwych teorii dyskursu religijnego -- z~innych, pozalogicznych powodów.


\section{Typologia znaczeń i~możliwych teorii religii}\label{sil-boch-znaczteol}

O~ile posługiwanie się logiką w~celu modelowania i~formalnych badań rozmaitych problemów z~zakresu filozofii czy nauki jest generalnie akceptowalną metodą badawczą, o~tyle stosowanie jej na gruncie religii bywało często kwestionowane -- także z~powodu pewnych antylogicznych tendencji obecnych właściwie we wszystkich religiach. Bocheński\index[names]{Bocheński, Józef Maria}, badając prawomocność logiki religii, podaje typologię znaczeń (rys. \ref{sil-boch-typ}) niesionych przez dyskurs (także dyskurs religijny) oraz generowane i~porządkowane przez tę typologię możliwe\footnote{Bocheński\index[names]{Bocheński, Józef Maria} mówi o~wszystkich możliwych \textit{a~priori} teoriach religii zakładając, że mogą znaleźć się wśród nich także takie teorie, co do których sam twierdzi, że nikt ich \textit{explicite} nie utrzymywał. Por. J.M. Bocheński, \textit{Logika religii}, dz. cyt., ss.~347-352.} teorie religii. Wedle tej typologii wypowiedź może posiadać lub nie posiadać znaczenia. Znaczenie niesione przez wypowiedź może być subiektywne lub obiektywne oraz komunikowalne lub niekomunikowalne. Komunikowalne znaczenie obiektywne możne być pełne albo niepełne. Pełne znaczenia mogą przybrać postać twierdzenia lub inną (performatywy, modlitwy, nakazu, reguły itd.).
%Bochenski-typologia-znaczenia-pic \label{sil-boch-typ}
\begin{figure}[H]
\begin{center}
 \begin{tikzpicture}[node distance=1cm]

    \node (1) {znaczenie};
    \node (21) [below=of 1, xshift=-2cm] {subiektywne};
    \node (22) [below=of 1, xshift=2cm] {obiektywne};
    \node (31b) [below=of 22, xshift=-2cm] {niekomunikowalne};
    \node (32b) [below=of 22, xshift=2cm] {komunikowalne};
    \node (41) [below=of 32b, xshift=-2cm] {niepełne};
    \node (42) [below=of 32b, xshift=2cm] {pełne};
    \node (51) [below=of 42, xshift=-2cm] {twierdzenia};
    \node (52) [below=of 42, xshift=2cm] {inne};

    \path[-] (1) edge (21);
    \path[-] (1) edge (22);
    \path[-] (22) edge (31b);
    \path[-] (22) edge (32b);
    \path[-] (32b) edge (41);
    \path[-] (32b) edge (42);
    \path[-] (42) edge (51);
    \path[-] (42) edge (52);


\end{tikzpicture}

\caption[Typologia znaczeń według Bocheńskiego]{Typologia znaczeń według Bocheńskiego\index[names]{Bocheński, Józef Maria}\footnotemark.}\label{sil-boch-typ}
\end{center}
\end{figure}
\footnotetext{Por. tamże, s.~348.}


Powyższa typologia pozwala Bocheńskiemu\index[names]{Bocheński, Józef Maria} podzielić i~uporządkować teorie religii na:

\begin{enumerate}
\item Teorie nonsensu głoszące, że dyskurs religijny nie posiada żadnego znaczenia;
\item Teorie emocjonalistyczne głoszące, że znaczenie dyskursu religijnego jest subiektywne i~czysto uczuciowe;
\item Teorie niekomunikowalności, według których dyskurs religijny posiada obiektywne znaczenie, choć jest ono niekomunikowalne;
\item Teorie znaczeń częściowych, według których obiektywne znaczenie komunikowalne w~obrębie dyskursu religijnego jest niepełne;
\item Teorie komunikowalności nietwierdzeniowej głoszące, że dyskurs religijny posiada obiektywne komunikowalne znaczenie, ale nie wyraża się ono w~twierdzeniach;
\item Teorie twierdzeniowe zakładające, że przynamniej pewna część dyskursu religijnego składa się ze zdań mających znaczenie.
\end{enumerate}
Warto dodać, że powyższa lista w~pewnym sensie stanowi częściowy porządek. Z~perspektywy teorii $n$, cokolwiek twierdzi się w~ramach teorii $m<n$ może być uznane za prawdziwe w~stosunku do pewnych fragmentów dyskursu religijnego. Teoria $n$~głosi ponadto, że istnieją jakieś partie dyskursu religijnego cechujące się odmiennym rodzajem znaczenia. Na przykład, z~punktu widzenia teorii twierdzeniowej można uznać, że pewne fragmenty dyskursu religijnego zawierają wyłącznie reguły, inne są niekomunikowalne, w~jeszcze innych liczą się wyłącznie uczucia i~nie ma tam żadnego znaczenia ponad to subiektywne i~związane z~emocjami. Co więcej, z~perspektywy teorii twierdzeniowej można również przyznać, że pewne fragmenty dyskursu religijnego w~ogóle nie niosą żadnego znaczenia. Zatem teorie te są uporządkowane względem wzrastającej mocy. Co ciekawe, w~dociekaniach Bocheńskiego\index[names]{Bocheński, Józef Maria} teoria Niewysłowionego leży u~podstaw tej hierarchii, natomiast teologia negatywna należy do teorii znajdujących się u~jej szczytu -- Bocheński\index[names]{Bocheński, Józef Maria} umieszcza ją wśród teorii znaczeń niepełnych.


\section{Tajemnica częściowego znaczenia}\label{sil-boch-tajem}

W~rozważaniach Bocheńskiego\index[names]{Bocheński, Józef Maria} teologia negatywna jest zatem alternatywnym dla teologii milczenia sposobem przedstawiania i~badania dyskursu religijnego. Terminem, który często pojawia się w~kontekście obu tych doktryn, jest ,,tajemnica''. Bocheński\index[names]{Bocheński, Józef Maria} twierdzi, że choć to o~przedmiocie religijnym twierdzi się, że jest tajemniczy (i~to w~tajemniczy w~sposób szczególny, ,,wyższy''), własność ,,tajemniczości'' nie przysługuje przedmiotom, lecz zdaniom. Z~tego powodu mówi się na przykład o~,,tajemnicach wiary''\footnote{Według Bocheńskiego\index[names]{Bocheński, Józef Maria} mówienie o~tajemnicach wiary to wypowiedzi metajęzykowe. Por. tamże, s.~413.}. Bocheński\index[names]{Bocheński, Józef Maria} pojęcie tajemnicy opiera na pojęciu rozumienia.

\begin{defin}[Tajemnica]
,,Tajemniczy'' w~sensie ogólnym oznacza stan rzeczy, którego nie daje się w~pełni uchwycić. Mówiąc ściślej, zdanie $p$ jest tajemnicze dla podmiotu $x$ wtedy, i~tylko wtedy, jeżeli $x$ nie w~pełni rozumie $p$\footnote{Ta i~poniższe definicje tajemniczości pochodzą bezpośrednio od Bocheńskiego\index[names]{Bocheński, Józef Maria}. Zob. tamże, ss.~413-414.}. Zatem, w~zależności od znaczenia ,,rozumienia'':

\begin{enumerate}[label = (\arabic*)]
\item $p$ jest tajemnicze dla $x$ wtedy, i~tylko wtedy, jeżeli $x$~dobrze (nawet w~pełni) rozumie znaczenie $p$, ale nie potrafi o~własnych siłach (intuicja lub wnioskowanie) przesądzić prawdziwości $p$.
\item $p$ jest tajemnicze dla $x$ wtedy, i~tylko wtedy, gdy $x$ rozumie znaczenie $p$ jako takie, ale nie zna jego aksjomatycznych powiązań z~innymi zdaniami, jakie akceptuje.
\item $p$ jest tajemnicze dla $x$ wtedy, i~tylko wtedy, gdy $p$ zawiera przynamniej jeden termin \textit{a} taki, że \textit{a} stosowane jest w~$p$ w~sposób tylko częściowo pokrywający się z~zastosowaniem \textit{a} w~dyskursie świeckim.
\item $p$ jest tajemnicze dla $x$ wtedy, i~tylko wtedy, gdy w~$p$ istnieje przynajmniej jeden taki termin \textit{a}, który jest całkowicie pozbawiony znaczenia dla $x$.
\end{enumerate}
\end{defin}

Powyższa definicja wyznacza cztery klasy teorii dotyczących tajemnicy\footnote{Por. typologia znaczeń i~możliwych teorii religii w~sekcji \ref{sil-boch-znaczteol}.}. Według pierwszej z~nich (1) prawd tajemnic religijnych nie da się poznać -- można je przyjąć wyłącznie na drodze wiary. Mogą one zostać przyjęte i~zaakceptowane wyłącznie na mocy autorytetu, a~niezależnie od niego nie da się ich uzasadnić lub określić ich wartości prawdziwościowych. Bocheński\index[names]{Bocheński, Józef Maria} ten rodzaj tajemnicy nazywa ,,tajemnicą prawdziwości'' i~uważa go w~gruncie rzeczy za tezę dotyczącą struktury dyskursu religijnego. Według drugiej z~nich (2) wierny nie rozumie prawd wiary, ponieważ każdorazowo rozumienie ich zależy od kontekstu aksjomatycznego. Twierdzi się tutaj, że słowa, w~których wyrażono prawdy wiary, nie wyczerpują nieskończonego bogactwa Boga, ponieważ istnieją powiązania aksjomatyczne zdań wyrażających prawdy wiary z~innymi zdaniami, których wierny nie zna. Tę odmianę tajemnicy Bocheński\index[names]{Bocheński, Józef Maria} nazywa ,,tajemnicą aksjomatyczną'' i~przekonuje, że nie jest to zjawisko odosobnione i~wyjątkowe dla dyskursu religijnego. W~końcu, w~trzeciej klasie teorii dotyczących tajemnicy (3) twierdzi się, że terminy, których używa się do wyrażenia prawd wiary, w~obrębie dyskursu religijnego nie przekazują pełni znaczenia, jakie jest im przypisywane na gruncie dyskursu świeckiego. W~myśl tej teorii prawdy wiary mają wciąż jeszcze jakieś znaczenie, choć już tylko częściowe. Z~tego powodu Bocheński\index[names]{Bocheński, Józef Maria} nazywa to ,,tajemnicą częściowego znaczenia'', a~wśród przykładów tej klasy teorii wymienia teologię negatywną (i teorię analogii)\footnote{Zob. J.M. Bocheński, \textit{Logika religii}, dz. cyt., ss.~414-415.}. Do czwartej klasy teorii (4) wyznaczanej przez ,,tajemnicę nonsensu'' należy teoria Niewysłowionego, zgodnie z~którą wszystkie terminy dyskursu religijnego są pozbawione znaczenia. Teoria ta zostanie omówiona w~kolejnej sekcji (\ref{sil-boch-nonsens}).

W~obrębie teologii negatywnej uważa się więc, że istnieje tylko częściowa tożsamość między znaczeniem danego terminu na gruncie dyskursu świeckiego, a~znaczeniem tego samego terminu użytego w~dyskursie religijnym. Zatem, w~przeciwieństwie do teorii nonsensu, według teologii negatywnej nie jest tak, że dyskurs religijny nic nie znaczy, jednakże jakiekolwiek ma on znaczenie, ma je na drodze ,,czystej negacji''. Jak zauważa Bocheński\index[names]{Bocheński, Józef Maria}, żaden ze zwolenników teologii negatywnej nie starał się sformułować takiej teorii w~dostatecznie precyzyjny sposób, a~przez swoją niejasną postać prowokuje ona do krytyki. Większość uwag krytycznych przedstawionych przez Bocheńskiego\index[names]{Bocheński, Józef Maria} wskazuje na jej paradoksalny charakter.

Po pierwsze, mając wyrażenie ,,$x$ jest niebiałe'' i~orzekając negację tego wyrażenia o~przedmiocie religii, otrzymamy negację niebiałości. Będzie to oznaczać, że przedmiot religii jest biały -- w~konsekwencji otrzymamy własność całkowicie pozytywną. Po drugie, jeśli dopuścimy do tego, by o~przedmiocie religii orzekać wszystkie negacje, popadniemy w~sprzeczność. Możemy bowiem mu przypisać własność bycia niebiałym (czyli negację własności bycia białym) oraz własność bycia nie-niebiałym (czyli negację bycia niebiałym), a~zatem także własność bycia białym, o~ile utrzymujemy silne prawo podwójnej negacji.

Według Bocheńskiego\index[names]{Bocheński, Józef Maria} można uniknąć tych sprzeczności, gdy ograniczy się zakres teorii do klasy własności pozytywnych. Wpierw jednak należałoby tę klasę zdefiniować. Bocheński\index[names]{Bocheński, Józef Maria} nie podaje jednak odpowiedniej definicji tego rodzaju własności. By nadać swoim rozważaniom nieco więcej precyzji, przedstawia następującą propozycję indukcyjnej definicji własności pozytywnych:
\begin{defin}[Włsność pozytywna\footnote{Tamże, s.~416.}]\label{sil-boch-pozytywna}\hfill\ 
\begin{enumerate}%[label = (\arabic*)]
\item Własność postrzegana bezpośrednio jest własnością pozytywną.
\item Własność definiowana za pomocą formuły zawierającej wyłącznie symbole własności pozytywnych i~terminów logiki pozytywnej jest własnością pozytywną.
\end{enumerate}
\end{defin}
Od razu jednak dodaje, że nie jest ona zadawalająca z~co najmniej dwóch powodów. Po pierwsze, definicja ta jest za wąska -- klasa własności pozytywnych ograniczona jest dużo surowiej niż wymagaliby tego zwolennicy teologii apofatycznej. Po drugie, pojęcie własności postrzeganej bezpośrednio jest bardzo nieścisłe. Obrazując to jego własnym przykładem -- dlaczego nie można by postrzegać bezpośrednio, że krowa nie jest niebieska? Bocheński\index[names]{Bocheński, Józef Maria} dodaje jednak, że problemy z~podaniem dostatecznie precyzyjnej i~trafnej definicji własności pozytywnych nie są najpoważniejszymi z~tych, które nękają teologię negatywną. Dlatego, na potrzeby dalszych rozważań zakłada on, że pojęcie własności pozytywnej zostało zdefiniowane poprawnie.

Przyjmując to założenie Bocheński\index[names]{Bocheński, Józef Maria} próbuje formalnie zrekonstruować znaczenie teologii negatywnej w~taki sposób, który zadowoliłby jej potencjalnych zwolenników. Niech $t$~będzie terminem występującym w~dyskursie świeckim w~jakimś znaczeniu. W~obrębie tak sformatowanej teorii mamy prawo utrzymywać, że znaczenie $t$~jest inne, gdy przypisywane jest przedmiotowi religii, i~używając $t$~w dyskursie religijnym mamy na myśli coś innego. Niech zapis $M(t,\pi ,\phi)$ oznacza ,,$t$ jest terminem występującym w~dyskursie świeckim w~znaczeniu $\phi $''. $\alpha $ natomiast oznacza klasę własności pozytywnych (jakkolwiek zdefiniowanych). Treść teologii negatywnej można zatem wyrazić w~następujący sposób postać:
\begin{flalign*}
		& \forall t\footnotemark  M(t, \pi, \phi) \land
		\phi \in \alpha
		\to \neg \phi (g), &\tag{NNT}\label{sil-boch-NNT}
\end{flalign*}
%\begin{align}
%\forall t\footnote{}  M(t, \pi, \phi) \land
%\phi \in \alpha
%\to \neg \phi (g),\label{sil-boch-NNT}
%\end{align}
gdzie $g$~oznacza przedmiot religii, czyli Boga. W~takim wypadku dyskurs religijny może zawierać wyłącznie twierdzenia, które orzekają o~Bogu negację własności pozytywnych wyrażonych w~terminach świeckich. Deskrypcja określona skojarzona z~predykatem ,,jest Bogiem'' przedstawiać się będzie następująco:\footnotetext{Zarówno w~oryginalnej pracy, jak i~w jej polskim tłumaczeniu kwantyfikator ten wiąże zmienną(?) oznaczoną symbolem $F$, która nie ma żadnych wystąpień w~niniejszej formule. Za dobrą monetę przyjmuję, że jest to oczywista pomyłka drukarska. Wskazuje na to także poprawny, jak się wydaje, zapis zawarty w~\ref{sil-boch-NNTbis} poniżej.}
\begin{flalign*}
		\begin{split}
		 G(x) \equiv_{\text{def}}\ &\exists x \big((
		\forall t  M(t, \pi, \phi) \land
		\phi \in \alpha
		\to \neg \phi (x))\ 
		\land\\
		&\forall y ((\forall t
		M(t, \pi, \phi) \land
		\phi \in \alpha
		\to \neg \phi (y))
%		\to
		\equiv
		y=x)\big)\footnotemark.
		\end{split}\tag{NNT'}\label{sil-boch-NNTbis}
\end{flalign*}
%\begin{align}
%\begin{split}
%G(x) \equiv_{\text{def}}\ &\exists x \big((
%\forall t  M(t, \pi, \phi) \land
%\phi \in \alpha
%\to \neg \phi (x))\\
%\land\ &\forall y ((\forall t
%M(t, \pi, \phi) \land
%\phi \in \alpha
%\to \neg \phi (y))
%\to y=x)\big)\footnote{}.\label{sil-boch-NNTbis}
%\end{split}
%\end{align}
\footnotetext{Nie da się ukryć, że zapis M(t,$\pi $,$\varphi $) niewiele wnosi do tak skonstruowanej zasady. Jakkolwiek istnieją pewne motywacje za rozróżnianiem znaczeń terminów występujących w~dyskursie religijnym i~świeckim, wszystko wskazuje na to, że w~rozważaniach dotyczących teologii negatywnej można z~tego podziału zrezygnować, mówiąc o~znaczeniach terminów bez osadzania ich w~konkretnych dyskursach. Tak czy owak, rezygnacja z~tego podziału i~uwspólnienie dyskursu doprowadziłyby do znacznego uproszczenia tej formuły. Stopień skomplikowania tego zapisu wynika również z~tego, że Bocheński\index[names]{Bocheński, Józef Maria} korzysta z~notacji używanej w~\textit{Principia Mathematica}, a~prezentowana tu formuła stanowi jej tłumaczenie na bardziej współczesną notację. Por. A.N. Whitehead, B. Russell, \textit{Principia Mathematica}, vol. 1, Cambridge University Press, Cambridge 1910 oraz B. Linsky, \textit{The Notation in Principia Mathematica}, [w:] \textit{The Stanford Encyclopedia of Philosophy}, wyd. zima 2021, red. E.N. Zalta, <\url{https://plato.stanford.edu/archives/win2021/entries/pm-notation/}>.} 

Dzięki ograniczeniu teorii do klasy własności pozytywnych, tak sformułowana teologia negatywna może zostać pozbawiona sprzeczności. Zdaniem Bocheńskiego\index[names]{Bocheński, Józef Maria} własność przypisywana Bogu w~teorii wyrażonej zasadą \ref{sil-boch-NNT} nie jest własnością metajęzykową (mimo użycia metajęzykowych terminów), lecz pewną własnością prze\-dmio\-to\-wo-językową drugiego rzędu -- taką, że nie można przypisać mu żadnej prze\-dmio\-to\-wo-językowej własności pierwszego rzędu. Przy odpowiedniej definicji klasy własności pozytywnych ($\alpha$) z~teorii tej można wyrugować sprzeczności. Mimo tego Bocheński\index[names]{Bocheński, Józef Maria} nie jest usatysfakcjonowany tym osiągnięciem. Fakt, że w~teologii negatywnej nie można przypisać Bogu żadnej przedmiotowo-językowej własności pierwszego rzędu wyklucza ją z~roli adekwatnej teorii dyskursu religijnego z~co najmniej dwóch powodów. Po pierwsze dlatego, że istnieje niepusty (a nawet zawierający więcej niż jeden element) zbiór własności przedmiotowo-językowych pierwszego stopnia przypisywanych Bogu na gruncie dyskursu religijnego. Po drugie wydaje się, że przedmiot, któremu możemy przypisać wyłącznie własność przedmio\-to\-wo-językową drugiego stopnia -- taką, że nie można przypisać mu żadnej przedmiotowo-językowej własności pierwszego stopnia -- nie może być obiektem czci i~chwały. Mówiąc wprost, Bocheński\index[names]{Bocheński, Józef Maria} odrzuca teologię negatywną z~powodu jej uwikłania nie w~,,wewnętrzny'', lecz oba ,,zewnętrzne'' paradoksy\footnote{Por. rozdz.~\ref{sil-int-par}.}.


\section{Teoria nonsensu}\label{sil-boch-nonsens}

W~typologii Bocheńskiego\index[names]{Bocheński, Józef Maria} teoria tego, co niewysłowione, jest teorią \textit{nonsensu}, zgodnie z~którą dyskurs religijny jest pozbawiony jakiegokolwiek znaczenia. Zwraca on uwagę, że według wielu komentatorów teoria ta jest wewnętrznie sprzeczna. Z~reguły argumentacja za taką tezą polega na wskazaniu paradoksu Niewyrażalnego -- twierdząc, że nie da się niczego powiedzieć o~Bogu, teoria ta sama coś o~nim mówi, a~zatem jest sprzeczna i~należy ją odrzucić. Bocheński\index[names]{Bocheński, Józef Maria} występuje przeciwko takiemu przedstawianiu teologii milczenia. Twierdzi on, że da się ją uratować od sprzeczności, lecz nawet mimo tego, nie odpowiada ona potrzebom dyskursu religijnego\footnote{Zob. J.M. Bocheński, \textit{Logika religii}, dz. cyt., ss.~353-356.}.

Bocheński\index[names]{Bocheński, Józef Maria} uważa, że jeśli przestrzega się pewnych obowiązujących w~logice konwencji, zarzut sprzeczności stawiany teorii Niewysłowionego przestanie obowiązywać. Należałoby wpierw dowieść, że w~danym ,,układzie odniesienia'' teoria ta prowadzi do sprzeczności, tymczasem nikt takiego dowodu nie przedstawił. Według Bocheńskiego\index[names]{Bocheński, Józef Maria} jest zupełnie przeciwnie -- nietrudno wykazać, że teoria tego, co niewysłowione, jest spójna. Poniżej przedstawię jego argumentację.

Załóżmy, że dwuargumentowy predykat ${N_w}(x,\mathcal{L})$ oznacza ,,$x$ jest niewyrażalne w~języku $\mathcal{L}$''. Zapiszmy teraz formułę zawierającą ten predykat
\begin{flalign}
&\exists x\exists \mathcal{L}\ N_w(x,\mathcal{L}).&\label{sil-boch-prenw}
\end{flalign}


Wydaje się, że nie tylko można ją wypowiedzieć nie popadając w~sprzeczność, lecz także jest ona prawdziwa -- nietrudno znaleźć taki obiekt $x$ i~taki język \textit{l}, które spełniałyby zapisany wyżej warunek. (Bocheński\index[names]{Bocheński, Józef Maria} podaje przykład krowy i~języka szachów: nie da się opisać krowy w~języku szachów).

Możemy powyższy przykład uogólnić i~sformułować metajęzykową definicję Boga o~następującej postaci:
\begin{flalign*}
&\forall \mathcal{L}\ N_w(g,\mathcal{L}),\tag{MNT}\label{sil-boch-MNT}&
\end{flalign*}
%${\forall}$lNw(g,l),\label{sil-boch-MNT}
gdzie $g$ jest stałą oznaczającą Boga, lub zgodnie z~podejściem Bocheńskiego\index[names]{Bocheński, Józef Maria} do wyboru kategorii językowej terminu Bóg:
\begin{flalign*}
&G(x) \equiv_{\text{def}} \exists x \forall \mathcal{L} \big(N_w(x,\mathcal{L}) \land \forall y (N_w(y,\mathcal{L})
%\to
\equiv
x = y)\big),\tag{MNT'}\label{sil-boch-MNTbis}&
\end{flalign*}
%G(x) {\textbackslash}equiv\_\{def\} {\textbackslash}exists x~{\textbackslash}forall l~(Nw(x,l) {\textbackslash}land {\textbackslash}forall y~(Nw(y,l) {\textbackslash}equiv x~= y)), \label{sil-boch-MNTbis}
gdzie $G(x)$ jest predykatem oznaczającym ,,$x$ jest Bogiem''. Na pierwszy rzut oka wydaje się, że te formuły są bardziej problematyczne -- twierdzenie, że $x$~jest niewysłowione w~żadnym języku zdaje się prowadzić do sprzeczności. Bocheński\index[names]{Bocheński, Józef Maria} próbuje jednak uniknąć tego problemu, stosując zwykłe konwencje wykorzystywane do pozbywania się antynomii semantycznych. Należy założyć, że żadne zdanie traktujące o~pewnej klasie języków, nie jest formułowane w~żadnym z~tych języków. Aby było pozbawione sprzeczności, musi zostać sformułowane w~innym języku, czyli odpowiednim metajęzyku. Możemy więc założyć, że klasa języków wspominana w~\ref{sil-boch-MNTbis} i~\ref{sil-boch-MNT} jest klasą języków przedmiotowych. W~takim wypadku zasady te stają się zdaniami metajęzyka pierwszego stopnia. Po takim zabiegu, sformułowana definicja jest znacząca i~pozbawiona sprzeczności. Nie ma bowiem niespójności w~twierdzeniu, że coś nie daje się wysłowić w~jakimś języku, lub nawet w~klasie języków, o~ile twierdzenie to jest wyrażone w~języku nienależącym do tej klasy. Według Bocheńskiego, przy takim założeniu -- standardowym z~punktu widzenia logiki ogólnej -- teoria tego, co niewysłowione, pozostaje relewantna i~spójna, a~zarzut sprzeczności zostaje oddalony.

Bocheński\index[names]{Bocheński, Józef Maria} odrzuca jednak teorię niewysłowionego z~tych samych powodów, dla których odrzucał teologię negatywną. Po pierwsze, na mocy \ref{sil-boch-MNT} nie można przypisać Bogu jakiekolwiek własności przedmiotowo-językowej. Jedyną własnością, jaką możemy mu przypisać, jest metajęzykowa własność bycia niewysłowionym w~żadnym z~języków przedmiotowych. W~takim wypadku wierny nie mógłby akceptować żadnego zdania dyskursu religijnego, które przypisałoby Bogu jakąkolwiek własność przedmiotowo-językową. Wydaje się to niespójne z~faktycznym dyskursem religijnym. Po drugie, niemożliwe byłoby oddawanie czci obiektowi, o~którym wiemy tylko i~wyłącznie to, że nie można o~nim nic powiedzieć. Jeśli wierny miałby czcić obiekt pozbawiony własności przedmiotowo-językowych, równie dobrze tym obiektem mógłby być nie Bóg a~szatan, Ludwik XIV\index[names]{Ludwik XIV} lub Homer\index[names]{Homer}\footnote{Przykłady te pochodzą odpowiednio z~tamże, s.~356 oraz S. Gäb, \textit{Languages of ineffability: the rediscovery of apophaticism in contemporary analytic philosophy of religion}, [w:] \textit{Negative Knowledge}, red. S. Hüsch i~in., Narr, Tübingen 2020, ss.~191-206.}. Mówiąc krótko, zdaniem Bocheńskiego\index[names]{Bocheński, Józef Maria} opisany powyżej metajęzykowy zabieg zdaje się chronić teologię milczenia przed ,,wewnętrznym'' paradoksem Niewyrażalnego. Pozostaje on jednak nieskuteczny w~likwidowaniu jej ,,zewnętrznych'' paradoksów -- zarówno w~wersji twierdzeniowej, jak i~nietwierdzeniowej. Z~tego powodu Bocheński\index[names]{Bocheński, Józef Maria} postuluje odrzucenie tej teorii.


\section{Dyskusja}\label{sil-boch-dyskusja}

W~\textit{Logice religii} Bocheński\index[names]{Bocheński, Józef Maria} przedstawia w~gruncie rzeczy dwie niezależne interpretacje teologii apofatycznej, które nazywa teologią negatywną oraz teorią tego, co niewysłowione. Ta pierwsza dopuszcza, by język religii posiadał jakieś znaczenie, choć jest ono ograniczane do przypisywania Bogu negacji pozytywnych własności przedmiotowo-językowych pierwszego rzędu. Uniknicie sprzeczności teologii negatywnej dokonuje się poprzez kolejne ograniczenie -- ograniczenie nałożone na klasę tychże właśnie własności pozytywnych. Teoria tego, co niewysłowione, w~typologii Bocheńskiego\index[names]{Bocheński, Józef Maria} pozbawia dyskurs religijny jakiegokolwiek znaczenia. W~tej teorii Bóg jest niewysławialny i~nie da się przypisać mu żadnej przedmiotowo-językowej własności. Uniknięcie jej paradoksalnego charakteru dokonuje się przez zastosowanie metajęzykowego zabiegu (w stylu Tarskiego\index[names]{Tarski, Alfred}). Uznaje się, że ,,niewysławialność'' jest własnością metajęzykową mówiącą o~tym, że Boga nie da się wyrazić w~żadnym z~języków przedmiotowych.

Olbrzymią słabością teologii negatywnej w~rozumieniu Bocheńskiego\index[names]{Bocheński, Józef Maria} jest fakt, że trudno jest znaleźć w~historii myśli jakiegokolwiek jej reprezentanta. On sam zauważa, że nikt tej teorii nie starał się ,,sformułować w~terminach dostatecznie precyzyjnych''\footnote{J.M. Bocheński, \textit{Logika religii}, dz. cyt., s.~416.}. Wydaje się jednak, że w~rzeczywistości po prostu nikt jej nie utrzymywał. Gdy teolog apofatyczny mówi, że do Boga nie można stosować żadnych określeń językowych, nie twierdzi, że trzeba zaprzeczyć każdemu pozytywnemu predykatowi, lecz każdemu predykatowi w~ogóle. Oczywiście, we współczesnej literaturze można znaleźć pewne próby zreinterpretowania wybranych tez autentycznych przedstawicieli teologii apofatycznej w~taki sposób, by tworzyły one coś na kształt proponowanego przez Bocheńskiego\index[names]{Bocheński, Józef Maria} rozwiązania\footnote{Por. P. Rojek, \textit{Logika teologii negatywnej}, ,,Pressje'', nr 29 (2012), a~także rozdz.~\ref{rojek-bochenski}.}. Ostatecznie jednak, są to próby spreparowania sztucznej teorii, a~nie rekonstrukcja rzeczywistej doktryny apofatycznej.

Spróbujmy jednak ustalić charakter własności przypisywanej Bogu w~\ref{sil-boch-NNT}. Mimo występujących w~niej terminów metajęzykowych Bocheński\index[names]{Bocheński, Józef Maria} uważa, że -- w~przeciwieństwie do formalizmu użytego w~teorii tego, co niewysłowione -- nie jest to własność metajęzykowa, lecz własność przedmiotowo-językowa drugiego stopnia. Skoro tak, obecność w~tej formule terminów metajęzykowych budzi pewne zastrzeżenia. W~istocie trudno oprzeć się wrażeniu, że zapis $\lceil \forall t M(t,\pi,\varphi) \rceil$  niewiele wnosi do skonstruowanej na potrzeby teologii negatywnej zasady. Rozróżnianie dyskursu świeckiego i~religijnego ma pewne znaczenie dla rozważań Bocheńskiego\index[names]{Bocheński, Józef Maria} w~ogólności, ale w~kontekście samej teologii negatywnej można z~niego zrezygnować bez straty sensu proponowanych rozstrzygnięć -- mówiąc po prostu o~znaczeniu terminów (bez osadzania ich w~konkretnych dyskursach). Tak czy owak, rezygnacja z~tego podziału i~uwspólnienie dyskursu nie tylko znacznie uprościłoby \ref{sil-boch-NNT} (a zwłaszcza \ref{sil-boch-NNTbis}), lecz także uczyniłoby te zasady bliższymi jakiejkolwiek rzeczywistej doktryny apofatycznej.

Dość nieoczekiwaną konsekwencją użycia $\lceil \forall t M(t,\pi,\varphi) \rceil$ w~formułowaniu zasad teologii negatywnej jest fakt, że w~dyskursie religijnym musimy ograniczyć się wyłącznie do terminów znanych z~dyskursu świeckiego. Inaczej mówiąc, w~obrębie języka religii nie powinniśmy używać jakichkolwiek nowych terminów, niewystępujących w~obrębie dyskursu świeckiego, na przykład jakichś własnych i~wewnętrznych terminów dyskursu religijnego. Gdy przyjdzie nam przypisać Bogu jakąś własność wyrażoną terminem niewystępującym w~jakimś znaczeniu w~języku dyskursu świeckiego, nie będziemy mogli mu przepisać ani tej własności, ani jej negacji, nawet gdyby należała do klasy własności pozytywnych.

Jednakże największą bolączką teorii nazywanej przez Bocheńskiego\index[names]{Bocheński, Józef Maria} teologią negatywną jest brak adekwatnej definicji własności pozytywnych. Skoro w~ostateczności teologia negatywna ma polegać na przypisywaniu Bogu negacji własności pozytywnych, a~na definicji tychże własności ma opierać się obrona tej teorii przed sprzecznościami, mamy prawo oczekiwać, by zostały one dostatecznie trafnie i~precyzyjnie zdefiniowane. Tymczasem epistemiczna definicja podana przez Bocheńskiego\index[names]{Bocheński, Józef Maria} opierająca się na ,,bezpośrednim postrzeganiu'' własności pozytywnych jest przez niego samego uważana za zbyt wąską i~nieścisłą. By móc skutecznie chronić teologię negatywną przed sprzecznościami, definicja ta przede wszystkim nie może dopuszczać wzajemnego określania jednej pozytywnej własności przez negację drugiej (oraz -- w~zależności od konkretnej postaci ewentualnej definicji -- ograniczać w~obrębie stosowania własności pozytywnych działanie silnego prawa podwójnej negacji). Rzeczywiście wydaje się, że dokonana przez Bocheńskiego\index[names]{Bocheński, Józef Maria} próba nadania tej teorii większej precyzji jest nieskuteczna we wskazanym tu zakresie. Dodatkowo, by uniknąć oczywistej sprzeczności w~\ref{sil-boch-MNTbis} z~klasy własności pozytywnych trzeba usunąć jeszcze dwie konkretne własności. Po pierwsze, należy z~tego grona wykluczyć własność wyrażoną predykatem $G(x)$ -- ,,$x$ jest Bogiem''. Po drugie, należy uznać, że pewna własność przedmiotowo-językowa drugiego rzędu -- ,,taki, że nie można przypisać mu żadnej przedmiotowo-językowej własności pierwszego rzędu'' -- także nie należy do własności pozytywnych.

Z~punku widzenia przedstawionych w~tej części pracy rozważań, teologia negatywna w~rozumieniu Bocheńskiego\index[names]{Bocheński, Józef Maria} jest jednak jedynie kompromisowym rozwiązaniem. Ogranicza ona możliwość mówienia o~Bogu, ale w~sposób niepełny -- redukuje mowę o~Bogu do przypisywania mu wyłącznie negacji własności pozytywnych. Bardziej konsekwentna w~tym zakresie jest opisywana przez niego teoria Niewysłowionego, która zabrania mówienia o~Bogu czegokolwiek -- przynajmniej w~obrębie języków przedmiotowych -- czyniąc, zdaniem Bocheńskiego\index[names]{Bocheński, Józef Maria}, dyskurs religijny pozbawionym jakiekolwiek bezpośredniego znaczenia.

Na pierwszy rzut oka wydaje się, że ,,metajęzykowy zabieg'', jaki Bocheński\index[names]{Bocheński, Józef Maria} zastosował w~modelu teorii tego, co niewysłowione, oddala zarzut niespójności tej doktryny i~jest to akceptowalne logicznie rozwiązanie. Należy jednak przyznać, że nie jest to rozwiązanie bez skazy. Pomysłem Bocheńskiego\index[names]{Bocheński, Józef Maria} na usunięcie problematycznego somoodniesieniowego paradoksu Niewyrażalnego jest uznanie, że ,,niewysławialność'' jest metajęzykowym dwuargumentowym predykatem orzekanym o~jakimś obiekcie i~pewnym języku przedmiotowym. Oznacza to, że gdy mówimy o~Bogu, że jest niewysławialny w~żadnym z~języków (przedmiotowych), nie wypowiadamy tego sądu w~żadnym z~języków, o~których mówimy, ale w~odpowiednim metajęzyku lub, inaczej mówiąc, wyłączamy metajęzyk z~zasięgu kwantyfikacji obecnej w wypowiadanym w~ten sposób sądzie. Inspiracji Bocheńskiego dla takiego rozumienia teologii milczenia należy szukać w~logicznych ustaleniach z~pierwszej połowy XX wieku, zwłaszcza w~pracach Alfreda Tarskiego\index[names]{Tarski, Alfred}\footnote{A. Tarski, \textit{Pojęcie prawdy w~językach nauk dedukcyjnych}, Prace Towarzystwa Naukowego Warszawskiego, Wydział III Nauk Matematyczno-Fizycznych, vol. 34, Warszawa 1933.} poświęconych klasycznej teorii prawdy, które powstały w~wyniku zmagań z~paradoksami semantycznymi, w~szczególności z~paradoksem kłamcy\footnote{Por. rozdz.~\ref{sil-int-par}.}.

Według klasycznej teorii prawdy zdania, w~których formułowane są semantyczne paradoksy samoodniesienia -- takie, jak analizowany w~tej części pracy paradoks Niewyrażalnego czy paradoks kłamcy -- są błędnie skonstruowane. Miesza się w~nich dwa poziomy językowe: język przedmiotowy i~metajęzyk. W~języku przedmiotowym mówimy o~świecie i~obiektach znajdujących się w~świecie, metajęzyk jest bogatszy -- możemy w~nim wyrażać także sądy o~języku przedmiotowym. Zatem, by móc mówić o~prawdziwości zdań (lub, w~przypadku teologii milczenia w~ujęciu Bocheńskiego\index[names]{Bocheński, Józef Maria}, niewysławialności obiektów w~danym języku), narzuca się stratyfikację języków i~zabrania ich mieszania. Dla przykładu załóżmy, że mamy zdanie \ref{sil-boch-T1} o~treści ,,Śnieg jest biały''.
%T1Śnieg jest biały.
\begin{flalign}
& \text{Śnieg jest biały.} &&\tag{T1}\label{sil-boch-T1}
\end{flalign}
Możemy teraz o~nim wypowiedzieć się w~metajęzyku, na przykład przy pomocy następującego zdania \ref{sil-boch-T2}:
%T2Zdanie T1 jest prawdziwe.
\begin{flalign}
& \text{Zdanie \ref{sil-boch-T1} jest prawdziwe.} &&\tag{T2}\label{sil-boch-T2}
\end{flalign}
Zdanie T2 sformułowane jest w~metajęzyku nadbudowanym nad językiem przedmiotowym, w~którym wyrażono \ref{sil-boch-T1}. Poziomy języka nie są więc naruszane i~zdanie jest poprawnie skonstruowane. Zgodnie z~koncepcją Tarskiego\index[names]{Tarski, Alfred}, warunek prawdziwości zdań (a~zatem i~\ref{sil-boch-T1}) można określić za pomocą tzw. cząstkowych definicji prawdy zwanych także T-równoważnościami. Taka cząstkowa definicja prawdy dla \ref{sil-boch-T1} przybierze postać:
%T2'Zadanie \ref{sil-boch-T1} jest prawdziwe wtedy i~tylko wtedy, gdy śnieg jest biały.
\begin{flalign}
& \text{Zadanie \ref{sil-boch-T1} jest prawdziwe wtedy i~tylko wtedy, gdy śnieg jest biały.} &&\tag{T2'}\label{sil-boch-T2prim}
\end{flalign}
Równoważność ta należy do metajęzyka, mimo iż na jej końcu pojawiają się wyrażenia równokształtne z~\ref{sil-boch-T1} -- stanowią one tłumaczenie terminów i~zdań z~języka przedmiotowego na metajęzyk. Jeśli natomiast chcielibyśmy powiedzieć coś o~zdaniu \ref{sil-boch-T2}, musielibyśmy znów użyć innego języka i~orzec w~nim na przykład, że
%T3Zdanie \ref{sil-boch-T2} jest prawdziwe.
\begin{flalign}
& \text{Zdanie \ref{sil-boch-T2} jest prawdziwe.} &&\tag{T3}\label{sil-boch-T3}
\end{flalign}
Zatem \ref{sil-boch-T3} nie jest już elementem metajęzyka, lecz języka wyższego poziomu -- metametajęzyka. Jeśli chcemy wyrazić warunek prawdziwości zdania \ref{sil-boch-T2}, również musimy dokonać tego na poziomie metametajęzyka -- T-równoważność dla \ref{sil-boch-T2} przyjmuje formę:
%T3'Zdanie \ref{sil-boch-T2} jest prawdziwe wtedy i~tylko wtedy, gdy zadanie \ref{sil-boch-T1} jest prawdziwe.
\begin{flalign}
& \parbox[t]{.89\linewidth}{\strut Zdanie \ref{sil-boch-T2} jest prawdziwe wtedy i~tylko wtedy, gdy zadanie \ref{sil-boch-T1} jest prawdziwe.\strut} &&\tag{T3'}\label{sil-boch-T3prim}
\end{flalign}
Z~poziomu metametajęzyka, możemy mówić o~wszystkich językach niższego poziomu, a~więc nie tylko o~metajęzyku, lecz także o~języku przedmiotowym.
%T3''Zdanie \ref{sil-boch-T2} jest prawdziwe wtedy i~tylko wtedy, gdy zadanie \ref{sil-boch-T1} jest prawdziwe, to znaczy wtedy i~tylko wtedy, gdy śnieg jest biały.
\begin{flalign}
& \parbox[t]{.87\linewidth}{\strut Zdanie \ref{sil-boch-T2} jest prawdziwe wtedy i~tylko wtedy, gdy zadanie \ref{sil-boch-T1} jest prawdziwe, to znaczy wtedy i~tylko wtedy, gdy śnieg jest biały.\strut } &&\tag{T3''}\label{sil-boch-T3bis}
\end{flalign}
Słowem, nie możemy zdefiniować prawdy dla danego języka w~tym języku. Nie da się określić prawdziwości zdań w~języku, w~którym zdania te zostały sformułowane. By to zrobić, musimy wejść na poziom metajęzyka, na którym można mówić nie tylko o~wszystkim, o~czym mówiliśmy w~języku przedmiotowym, lecz również i~o tym języku. W~taki sposób tworzy się nieskończona hierarchia języków. Na każdym z~wyższych poziomów możemy wyrazić wszystko, co było możliwe do wyrażenia w~językach niższego poziomu, oraz sądy o~zdaniach języków niższego poziomu.

Sposób generowania nieskończonej hierarchii języków w~alternatywnym do propozycji Bocheńskiego\index[names]{Bocheński, Józef Maria} modelu teorii Niewysłowionego przedstawia Bartosz Brożek\index[names]{Brożek, Bartosz}\footnote{Zob. B. Brożek, \textit{Marzenie Leibniza. Rzecz o~języku religii}, Copernicus Center Press, Kraków 2016, rozdz.~4.}. W~ramach tego ujęcia teza o~niewysławialności Boga została zredukowana do metajęzykowego stwierdzenia, że wszystkie wypowiedzi o~Bogu są bezsensowne.

Rozważmy zdanie
%D1Bóg jest intelektem.
\begin{flalign}
& \text{Bóg jest intelektem.} &&\tag{D1}\label{sil-boch-D1}
\end{flalign}
W~analizie Brożka\index[names]{Brożek, Bartosz}, gdy Dionizy\index[names]{Pseudo-Dionizy Areopagita} zaprzecza tego rodzaju zdaniom, chce jedynie stwierdzić, że nie można tak mówić, że z~takim zdaniami

\begin{quote}
jest coś nie tak, nie możemy w~ten sposób się wyrazić. Gdyby Dionizy\index[names]{Pseudo-Dionizy Areopagita} znał odróżnienie języka przedmiotowego i~metajęzyka, zamiast pisać ,,Bóg nie jest intelektem'' sformułowałby odpowiednią wypowiedź metajęzykową\footnote{Tamże.}.
\end{quote}
Zgodnie z~interpretacją Brożka\index[names]{Brożek, Bartosz}, taka metajęzykowa wypowiedź Dionizego\index[names]{Pseudo-Dionizy Areopagita} o~zdaniu \ref{sil-boch-D1} przybrałaby postać:
%D2Zdanie \ref{sil-boch-D1} jest bezsensowne.
\begin{flalign}
& \text{Zdanie \ref{sil-boch-D1} jest bezsensowne.} &&\tag{D2}\label{sil-boch-D2}
\end{flalign}
Zdanie uważa się za pozbawione sensu, gdy nawet hipotetycznie nie da się jednoznacznie określić, czy jest prawdziwe, czy fałszywe. Brożek\index[names]{Brożek, Bartosz} formułuje więc odpowiednią równoważność (analogiczną do T-równoważności Tarskiego\index[names]{Tarski, Alfred}):
%D2'Zdanie \ref{sil-boch-D1} jest bezsensowne wtedy i~tylko wtedy, gdy \ref{sil-boch-D1} nie jest prawdziwe i~\ref{sil-boch-D1} nie jest fałszywe.
\begin{flalign}
& \parbox[t]{.87\linewidth}{\strut Zdanie \ref{sil-boch-D1} jest bezsensowne wtedy i~tylko wtedy, gdy \ref{sil-boch-D1} nie jest prawdziwe i~\ref{sil-boch-D1} nie jest fałszywe.\strut} &&\tag{D2'}\label{sil-boch-D2prim}
\end{flalign}
Zakładając dwuwartościowość\footnote{Tzn. zakładając, że zdanie jest prawdziwe wtedy i~tylko wtedy, gdy nie jest fałszywe oraz że zdanie jest fałszywe wtedy i~tylko wtedy, gdy nie jest prawdziwe.} oraz przyjmując T-równoważność dla \ref{sil-boch-D1} i~jej odpowiednik dla zdań fałszywych (,,Zdanie \ref{sil-boch-D1} jest fałszywe wtedy i~tylko wtedy, gdy Bóg nie jest intelektem''), otrzymamy:
%D2''Zdanie \ref{sil-boch-D1} jest bezsensowne wtedy i~tylko wtedy, gdy nie jest tak, że Bóg jest intelektem i~nie jest tak, że Bóg nie jest intelektem.
\begin{flalign}
& \parbox[t]{.87\linewidth}{\strut Zdanie \ref{sil-boch-D1} jest bezsensowne wtedy i~tylko wtedy, gdy nie jest tak, że Bóg jest intelektem i~nie jest tak, że Bóg nie jest intelektem.\strut} &&\tag{D2''}\label{sil-boch-D2bis}
\end{flalign}
Powyższa równoważność należy do metajęzyka. Oczywiście, zawiera ona wyrażenie ,,Bóg jest intelektem'' równokształtne z~\ref{sil-boch-D1}, ale wyrażenie to stanowi tylko tłumaczenie \ref{sil-boch-D1} na metajęzyk. Trudno nie zauważyć, że równoważność ta prowadzi do sprzeczności. Warunkiem bezsensowności zdania \ref{sil-boch-D1} jest jednoczesna prawdziwość dwóch metajęzykowych zdań, z~których jedno jest negacją drugiego. Z~perspektywy przedstawianego tu argumentu istotniejsze jest jednak to, że w~wyniku analiz zdanie \ref{sil-boch-D2} okazało się kolejną (metajęzykową) wypowiedzią o~Bogu. A~zatem, z~punktu widzenia dionizyjskiej teologii milczenia musimy uznać, że jest bezsensowne.
%D3 Zdanie \ref{sil-boch-D2} jest bezsensowne.
\begin{flalign}
& \text{Zdanie \ref{sil-boch-D2} jest bezsensowne.} &&\tag{D3}\label{sil-boch-D3}
\end{flalign}
Otwiera to drogę do nieskończonej hierarchii języków, w~których kolejno stwierdzamy, że zdania o~Bogu z~niższego poziomu są bezsensowne. Zdaniem Brożka\index[names]{Brożek, Bartosz} taka intuicja stoi za tą interpretacją teologii apofatycznej, która każe twierdzić, że o~Bogu nie można mówić w~żadnym języku.

\begin{quote}
Niezależnie od tego, jak bogaty język skonstruujemy, nie będzie on wystarczającym narzędziem do opisu Transcendencji -- zawsze będziemy zmuszeni przenieść się ,,poziom wyżej'', gdzie przeżyjemy \textit{déjà vu}, znów okaże się, że język, którym dysponujemy, jest niewydolny\footnote{Tamże.}.
\end{quote}

Wydaje się, że Bocheński\index[names]{Bocheński, Józef Maria} próbował uniknąć tego progresu w~nieskończoność formułując swoją zasadę teologii milczenia w~postaci \ref{sil-boch-MNT}. Problem w~tym, że teorię zbudowaną na tej zasadzie bynajmniej nie można nazwać konsekwentnym apofatyzmem. Co prawda, w~oczach Bocheńskiego\index[names]{Bocheński, Józef Maria} jest ona bardziej konsekwentna od teologii negatywnej \textit{tout court} i~dostatecznie ekstremalna -- pozbawia dyskurs religijny jakiegokolwiek znaczenia i~generuje tzw. ,,zewnętrzne'' paradoksy. Jednakże teolog apofatyczny z~pewnością z~większym entuzjazmem przyjmie powyższą konstatację Brożka\index[names]{Brożek, Bartosz}, niż zadowoli się modelem Bocheńskiego\index[names]{Bocheński, Józef Maria}, który w~jego ocenie może mieć nieco zachowawczy charakter. Inaczej mówiąc, teolog apofatyczny doceni obecność dużego kwantyfikatora w~sformułowaniu \ref{sil-boch-MNT}, wolałby jednak, by ten kwantyfikator swoim zasięgiem obejmował \textit{wszystkie} języki, nie tylko języki przedmiotowe. Za takim postawieniem sprawy przemawiają także inne, filozoficzno-językowe argumenty związane ze strukturą teorii budowanych analogicznie do klasycznej koncepcji prawdy Tarskiego\index[names]{Tarski, Alfred}, o~których poniżej. Tymczasem spróbujmy jeszcze bliżej przyjrzeć się modelowi zaproponowanemu przez Bocheńskiego\index[names]{Bocheński, Józef Maria}.

W~koncepcji Bocheńskiego\index[names]{Bocheński, Józef Maria} \ref{sil-boch-MNT} jest metajęzykowym zdaniem mówiącym o~niewysławialności pewnego obiektu $g$ w~klasie języków przedmiotowych. Dla ułatwienia, załóżmy, że nie mamy do czynienia z~całą klasą języków przedmiotowych, lecz z~jednym takim językiem. Zresztą, nie jest to kontrowersyjne posunięcie, a~mówienie o~,,języku szachów'' czy innych tego typu językach, które z~pewną dozą dodatkowych założeń można nazwać językami przedmiotowymi, trochę zaciemnia obraz. Jeśli język przedmiotowy oznaczymy symbolem $\mathcal{L}_0$, metajęzyk możemy oznaczyć symbolem $\mathcal{L}_1$, metametajęzyk symbolem $\mathcal{L}_2$ itd. Zauważmy, że w~języku $\mathcal{L}_{n+1}$ mamy prawo mówić o~wszystkim, o~czym mówiliśmy w~języku $\mathcal{L}_{m\leq n}$, $\mathcal{L}_{n+1}$ zawiera jednak dodatkowe terminy, które pozwalają nam wypowiadać zdania dotyczące języków o~indeksach $m \leq n$. Na przykład, zgodnie z~koncepcją Tarskiego\index[names]{Tarski, Alfred}, język $\mathcal{L}_{n+1}$ będzie zawierał predykat ,,jest prawdziwy w~$\mathcal{L}_{n}$'', w~języku $\mathcal{L}_{n+2}$ dostępny jest predykat ,,jest prawdziwy w~języku $\mathcal{L}_{n+1}$'' itd. Konsekwencją tego jest fakt, że nie istnieje jedna definicja prawdy -- każda jest zrelatywizowana do danego języka i~może zostać wyrażona tylko w~języku wyższego poziomu\footnote{Stąd w~klasycznej teorii prawdy Tarskiego\index[names]{Tarski, Alfred} T-równoważności nazywa się ,,cząstkowymi'' definicjami.}. Podobnie zachowuje się predykat ,,jest bezsensowny w~$\mathcal{L}_{n}$'' w~modelu Brożka\index[names]{Brożek, Bartosz} -- nie mówi się tu o~bezsensowności zdania w~ogóle, lecz o~bezsensowności w~danym języku niższego poziomu. Wydaje się więc słusznym uznać, że predykat ,,jest niewysławialny'' nie jest predykatem dwuargumentowym, lecz predykatem jednoargumentowym zrelatywizowanym do konkretnego języka. Będziemy zatem mówić, że coś jest ,,niewysławialne w~$\mathcal{L}_{0}$'', ,,niewysławialne w~$\mathcal{L}_{1}$'', czy ,,niewysławialne w~$\mathcal{L}_{n}$'', co oznaczymy za pomocą litery predykatowej z~odpowiednim indeksem. Przy takim założeniu model teorii Niewysłowionego przybierze postać
%MNT2Nw\_\{L0\}(g). \label{sil-boch-MNT2}
\begin{flalign*}
&N_{w\mathcal{L}_{0}}(g).&\tag{MNT2}\label{sil-boch-MNT2}
\end{flalign*}
Zdanie to, oczywiście, pozostaje zdaniem metajęzykowym (czyli wypowiedzianym w~języku $\mathcal{L}_{1}$). Wydaje się więc, że model Bocheńskiego\index[names]{Bocheński, Józef Maria} nie narusza zasad wyznaczonych przez Tarskiego\index[names]{Tarski, Alfred} -- przyjmuje stratyfikację języka i~nie miesza jego poziomów. Dokładniej mówiąc, ogranicza pojęcie niewysławialności do niewysławialności w~danym języku, mówiąc o~tym w~języku wyższego poziomu. Wciąż jednak pozostawia wiele wątpliwości. Są one dwojakiego rodzaju. Po pierwsze, w~przeciwieństwie do koncepcji Tarskiego\index[names]{Tarski, Alfred} i~Brożka\index[names]{Brożek, Bartosz}, pojęcie, które u~Bocheńskiego wzbogaca każdy kolejny język -- ,,niewysławialny w~$\mathcal{L}_{n}$'', nie dotyczy zdań, lecz obiektów. Z~tego powodu analiza tego modelu staje się kłopotliwa z~logicznego punktu widzenia, bo to zdania, nie obiekty, są nośnikami sensu. Po drugie, z~tego samego powodu nie wiadomo, jak miałaby wyglądać cząstkowa definicja niewysławialności, czyli jakiś odpowiednik T-równoważności dla tej teorii.
%MNT2' Nw \mathcal{L}_{0}(g) wtw A. \label{sil-boch-MNT2bis}.
\begin{flalign*}
&N_{w\mathcal{L}_{0}}(g) \equiv A.&&\tag{MNT2'}\label{sil-boch-MNT2prim}
\end{flalign*}
Inaczej mówiąc, nie do końca wiadomo, jakie wyrażenie, formuła lub zdanie miałyby znaleźć się na poziomie języka przedmiotowego $\mathcal{L}_{0}$ i~motywować przejście na poziom metajęzyka $\mathcal{L}_{1}$, by na nim wypowiedzieć \ref{sil-boch-MNT2}\footnote{Ta i~poniższe uwagi dotyczą, rzecz jasna, także \ref{sil-boch-MNT} i~wszystkich innych wersji tej zasady.}. Być może adwokat modelu Bocheńskiego\index[names]{Bocheński, Józef Maria} z~radością przyjmie taką uwagę -- jest to pożądany efekt, w~końcu \ref{sil-boch-MNT2} mówi o~tym, że Bóg jest niewysławialny. W~rzeczywistości jednak nie ucieknie on od tego problemu i~będzie musiał się z~nim zmierzyć w~odpowiedzi na drugi zarzut. Mianowicie, nie jest jasne, jak ma wyglądać (cząstkowa) definicja niewysławialności. Przyjrzyjmy się jeszcze raz rozwiązaniom obecnym w~teoriach Tarskiego\index[names]{Tarski, Alfred} i~Brożka\index[names]{Brożek, Bartosz}. Częstkowa definicja prawdy mówi, że
%T2''Zadanie ,,Śnieg jest biały'' jest prawdziwe wtedy i~tylko wtedy, gdy śnieg jest biały.
\begin{flalign}
& \parbox[t]{.89\linewidth}{\strut Zdanie ,,Śnieg jest biały'' jest prawdziwe wtedy i~tylko wtedy, gdy śnieg jest biały.\strut} &&\tag{T2''}\label{sil-boch-T2bis}
\end{flalign}
Cząstkowa definicja bezsensowności ma postać
%D2'''Zdanie ,,Bóg jest intelektem'' jest bezsensowne wtedy i~tylko wtedy, gdy nie jest tak, że Bóg jest intelektem i~nie jest tak, że Bóg nie jest intelektem.
\begin{flalign}
& \parbox[t]{.88\linewidth}{\strut Zdanie ,,Bóg jest intelektem'' jest bezsensowne wtedy i~tylko wtedy, gdy nie jest tak, że Bóg jest intelektem i~nie jest tak, że Bóg nie jest intelektem.\strut} &&\tag{D2{'}{'}{'}}\label{sil-boch-D2ter}
\end{flalign}
Prawe człony tych równoważności zawierają tłumaczenia zdań z~języka przedmiotowego na metajęzyk lub metajęzykową eksplikację definiowanego pojęcia w~terminach tych zdań. Co więc powinniśmy podstawić za $A$~w \ref{sil-boch-MNT2prim}? Jedną z~propozycji na odpowiednik \mbox{T-równoważności} dla niewysławialności może być pewien rodzaj metajęzykowej tożsamości. W~takim wypadku za $A$~powinniśmy podstawić
%A'Nw\_\{\mathcal{L}_{0}\}(g).
\begin{flalign*}
&N_{w\mathcal{L}_{0}}(g).&&\tag{$A$'}\label{sil-boch-Aprim}
\end{flalign*}
Możemy też prześledzić, jak sam Bocheński\index[names]{Bocheński, Józef Maria} rozumie niewysławialność i~spróbować przedstawić jakąś definicję na bazie jego rozważań. A~twierdzi on często, że w~teorii Niewysławialnego Bóg pozbawiony jest wszelkich własności. Alternatywę dla prawego członu \ref{sil-boch-MNT2prim} mogłoby więc stanowić pewne zdanie wyrażone w~logice drugiego rzędu o~postaci
%$$A''\neg \exists Q \mathcal{L}_{0} Q \mathcal{L}_{0}(g).$$
\begin{flalign*}
&\neg \exists Q_{\mathcal{L}_{0}}\ Q_{\mathcal{L}_{0}}(g).&&\tag{$A$''}\label{sil-boch-Abis}
\end{flalign*}


Niezależnie od naszego wyboru należy uznać, że zdania te albo są przekładem z~języka przedmiotowego (co oznaczałoby, że posiadają swoje równokształtne odpowiedniki w~$\mathcal{L}_{0}$), albo stanowią \textit{definiens} pojęcia niewysławialności w~terminach zdań, formuł lub wyrażeń pochodzących z~$\mathcal{L}_{0}$. W~konsekwencji, w~zależności od tego, jaki kształt przybierze wypowiedź z~języka przedmiotowego tłumaczona w~prawym członie \ref{sil-boch-MNT2prim}\ na metajęzyk, albo ponownie mieszamy poziomy języka i~wracamy do paradoksu Niewyrażalnego na poziomie języka $\mathcal{L}_{1}$, albo dopuszczamy, by język ten zawierał wypowiedzi bezsensowne i~nieniosące znaczenia.

Do modelu Bocheńskiego\index[names]{Bocheński, Józef Maria} można wysuwać dodatkowe, pozalogiczne zastrzeżenia. Dotyczą one przede wszystkim zagadnienia niewysławialności i~postulowania (choćby potencjalnego) istnienia niewysławialnych obiektów. To, czy takie obiekty istnieją, jest przedmiotem badań filozoficznych, a~formuła \eqref{sil-boch-prenw} bynajmniej nie zostałaby jednomyślnie uznana za prawdziwą wśród metafizyków. Przeciw takiemu założeniu wystąpiliby choćby zwolennicy stosunkowo popularnej a~wyjątkowo zachowawczej w~kwestii nadmiarowego poszerzania ,,uniwersum dyskursu'' teorii zobowiązań ontologicznych Willarda Van Ormana Quine'a\index[names]{Quine, Willard V.O.}\footnote{Zob. W.V.O. Quine, \textit{O~tym, co istnieje}, [w:] tenże, \textit{Z~punktu widzenia logiki}, tłum. B. Stanosz, Aletheia, Warszawa 2000. Zob. także B.~Brożek, A. Olszewski, \textit{Kilka uwag o~kryterium Quine'a}, ,,Filozofia Nauki'', vol 18 (2010), nr 1, ss.~5-15 oraz K. Wójtowicz, \textit{O~pojęciu ,,zobowiązania ontologicznego}'', ,,Przegląd Filozoficzny — Nowa Seria'', r.~X (2001), nr 1(37), ss.~121-138.}. Można też sądzić, że postulowanie istnienia obiektów, o~których nie można nic powiedzieć, stoi na opak z~wielowiekową regułą zwaną brzytwą Ockhama\index[names]{Ockham, William}\footnote{Por. A. Baker, \textit{Simplicity}, [w:]~\textit{The Stanford Encyclopedia of Philosophy},~wyd. zima 2016, red. E.N. Zalta, {\textless}\url{https://plato.stanford.edu/archives/win2016/entries/simplicity/}{\textgreater}.}.

Zresztą przykład podawany przez Bocheńskiego\index[names]{Bocheński, Józef Maria} w~celu przekonania czytelnika co do prawdziwości \eqref{sil-boch-prenw}, z~dzisiejszego punktu widzenia wydaje się dosyć niefortunny. Bocheński zapewnia, że ,,rzecz całkiem oczywista, że krowa jest niewysłowiona w~języku szachów, tzn. że w~tym języku nie da się o~niej nic powiedzieć''. Jakkolwiek Bocheński\index[names]{Bocheński, Józef Maria} rozumie ,,język szachów'', motywacja stojąca za wykorzystaniem go w~przykładzie jest oczywista -- Bocheński\index[names]{Bocheński, Józef Maria} chce zestawić ze sobą dosyć prostą syntaktykę z~abstrakcyjnym, symbolicznym alfabetem i~ograniczonym zestawem reguł oraz losowy obiekt ze świata rzeczywistego i~przekonać nas do niemocy tego typu języka w~próbach opisu takich obiektów. Tymczasem najnowsza historia pokazuje, że mając do dyspozycji maszyny wyposażone w~prymitywny, abstrakcyjny alfabet składający się tylko z~dwóch symboli -- \{0, 1\} -- i~niewielki zestaw reguł manipulujących tymi symbolami, możemy wyrazić \textit{quicquidlibet} i~rozmawiać \textit{de omni re scibili}.

Nie do końca klarowne są natomiast motywacje, dla których Bocheński\index[names]{Bocheński, Józef Maria} swój model buduje wokół pojęcia niewysławialności, a~nie ontologicznie prostszej, ,,pozytywnej'' własności wysławialności (opatrzonej w~stosownych miejscach negacją). Nie jest to może najcięższy z~zarzutów przeciwko przedstawionej powyżej interpretacji Bocheńskiego\index[names]{Bocheński, Józef Maria} (o ile w~ogóle można to traktować jako zarzut). Jednakże uwaga ta jest warta odnotowania o~tyle, o~ile w~innych miejscach swojej pracy przykłada on wagę do takiego rozróżnienia i~opiera o~nie obronę teologii negatywnej przed paradoksem Niewyrażalnego.

Tak czy owak, to stratyfikacja i~tworzenie hierarchii języków wywołuje największe opory przed przyjęciem rozwiązań w~stylu klasycznej teorii prawdy. Język naturalny, w~którym formułowana jest teologia apofatyczna, nie zawiera jakiegoś eksplicytnego rozwarstwienia poziomów i~na próżno wyróżniać i~wskazywać w~nim język przedmiotowy, metajęzyk itd. Posługując się terminologią Tarskiego\index[names]{Tarski, Alfred}, język naturalny jest semantycznie zamknięty, czyli zawiera już wszystkie pojęcia semantyczne, które go opisują. To kolejny powód, dla którego teolog apofatyczny nie przychylałby się do ograniczenia mówienia o~Bogu wyłącznie do języków przedmiotowych. Jeśli język ma jakieś inne poziomy, nie ma żadnych powodów, by dopuścić do tego, by Bóg był na tych poziomach opisywalny. W~każdym razie, hierarchia poziomów języka nie jest częścią zwykłego dyskursu. Z~tego powodu wprowadzanie jej do formalnych ustaleń w~celu ominięcia paradoksów semantycznych uznawane jest często za zbyt drastyczne i~przesadne podejście oraz za rozwiązanie \textit{ad hoc}.

Poza tym, hierarchia języków wprowadza dodatkowe techniczne problemy, które nie występują, gdy mamy do czynienia z~językiem semantycznie zamkniętym. Nawet jeśli w~wyniku metajęzykowego zabiegu udaje nam się pozbyć paradoksu Niewyrażalnego, to razem z~nim usuwamy również wiele nieparadoksalnych przypadków samoodniesienia. Rozważmy na przykład następującą parę zdań\footnote{Poniższy argument jest trawestacją przykładu pochodzącego od Kripkego\index[names]{Kripke, Saul}. Por. S. Kripke, \textit{Outline of a~Theory of Truth}, ,,The Journal of Philosophy'', vol. 72 (1975), ss.~690–716.}:
%(R)Żaden obiekt nie jest wyrażalny w~języku dyskursu świeckiego.\label{sil-boch-relig}
%(Ś)Większość obiektów jest niewyrażalna w~języku dyskursu religijnego.\label{sil-boch-swiec}
\begin{flalign*}
&\text{Żaden obiekt nie jest wyrażalny w~języku dyskursu świeckiego.}&\tag*{(R)}\label{sil-boch-relig}\\
&\text{Większość obiektów jest niewyrażalna w~języku dyskursu religijnego.}&\tag*{(Ś)}\label{sil-boch-swiec}
\end{flalign*}

Załóżmy, że zdanie \ref{sil-boch-relig} zostało wypowiedziane w~języku dyskursu religijnego, a~zdanie \ref{sil-boch-swiec} w~języku dyskursu świeckiego. Które jest nich jest wyżej w~hierarchii języków? Przyjmując rozwiązanie Tarskiego\index[names]{Tarski, Alfred} należałoby uznać, że zdanie \ref{sil-boch-relig} jest na wyższym poziomie niż wszystkie wypowiedzi języka świeckiego i, odwrotnie, zdanie \ref{sil-boch-swiec} musi być na wyższym poziomie niż wszystkie wypowiedzi języka religijnego. Ponieważ \ref{sil-boch-relig} jest wypowiedzią w~języku religijnym, a~\ref{sil-boch-swiec} jest wypowiedzią w~języku świeckim, \ref{sil-boch-relig} musiałoby być na wyższym poziomie niż \ref{sil-boch-swiec}, a~\ref{sil-boch-swiec} na wyższym niż \ref{sil-boch-relig}. To oczywiście jest niemożliwe, więc w~teoriach wykorzystujących metajęzykowy zabieg w~stylu Tarskiego\index[names]{Tarski, Alfred} zdania te są uważane za błędnie skonstruowane. \ref{sil-boch-relig} i~\ref{sil-boch-swiec} są w~rzeczywistości pośrednio autoreferencyjne, ponieważ \ref{sil-boch-relig} odwołuje się do całości wypowiedzi w~języku świeckim, w~tym do \ref{sil-boch-swiec}, a~\ref{sil-boch-swiec} odwołuje się do większości zdań języka religijnego, w~tym do \ref{sil-boch-relig}. Niemniej jednak w~większości przypadków \ref{sil-boch-relig} i~\ref{sil-boch-swiec} są nieszkodliwe i~nie pociągają sprzeczności.

Z~powodu tych problemów w~dzisiejszej logice proponuje się inne, nowsze rozwiązania pomagające uniknąć paradoksów semantycznych (czy też, mówiąc bardziej ogólnie, paradoksów samozwrotności). Niemniej, przykłady modelów Bocheńskiego\index[names]{Bocheński, Józef Maria} i~Brożka\index[names]{Brożek, Bartosz} pokazują, że rozwiązania zaczerpnięte z~klasycznej teorii prawdy mogą być owocne na polu logicznej analizy teologii milczenia. To, który z~tych modeli zostanie wybrany za bardziej trafny, zależeć będzie od filozoficznego temperamentu i~apofatycznego zacięcia. Model Brożka\index[names]{Brożek, Bartosz} nie pomaga uniknąć paradoksów. Wręcz przeciwnie -- buduje nieskończoną hierarchię języków, w~której sprzeczność pojawia się na każdym z~językowych poziomów. Model Bocheńskiego\index[names]{Bocheński, Józef Maria} pozostawia niejasności co do rozumienia centralnego pojęcia tej teorii -- niewysławialności, prowadzając do podejrzenia, że sprzeczności nie zostały wcale usunięte, lecz wystarczająco dobrze zakamuflowane. Ponadto, z~teologicznego punktu widzenia, ,,negatywność'' tego rozwiązania pozostawia dużo do życzenia. Co prawda Bóg jest tutaj uznany za niewysławialnego, lecz tylko na podstawowym, przedmiotowym poziomie języka. Na każdym z~wyższych poziomów języka pozostaje on wysławialny. Mimo wszystko, przy dostatecznie dużej dozie dobrej woli można uznać, że próba obrony teologii milczenia przed sprzecznością, którą w~\textit{Logice religii} dokonuje Bocheński\index[names]{Bocheński, Józef Maria}, jest trafna lub chociaż logicznie akceptowalna, przynajmniej w~zakresie ,,wewnętrznego'' paradoksu Niewyrażalnego\footnote{Całą baterię argumentów przeciwko podejściu ,,w stylu Tarskiego\index[names]{Tarski, Alfred}'' do niewysławialności, częściowo pokrywających się z~przedstawionymi w~tym rozdziale, można znaleźć w: A. Kukla, \textit{Ineffability and Philosophy}, Routledge, London -- New York 1998, rozdz.~1.}.

