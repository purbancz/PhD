%$\textsc{x}R_{p}\textsc{y}$\\
%$\textsc{a}R_{n}\textsc{b}$\\
%$\textsc{a}R_{s}\textsc{b}$


\chapter{Teologia negatywna jako mistycyzm inspirowany neoplatonizmem}\label{neopl}

Często podkreśla się, że teologia negatywna jest spuścizną filozofii platońskiej i~neoplatońskiej, przynajmniej w~obrębie tradycji zachodniej. Rzeczywiście, możemy wydobyć wiele fragmentów pism Platona\index[names]{Platon}, Plotyna\index[names]{Plotyn}, Proklosa\index[names]{Proklos} i~innych neoplatończyków, które -- poprzez umieszczenie terminu ,,Bóg'' w~miejsce centralnego pojęcia neoplatońskiej filozofii, ,,Jedni'' (\textgreek{t`o <'En}) -- byłyby nieodróżnialne od chrześcijańskiej doktryny apofatycznej. Niniejsza część pracy czyni zadość tym interpretacjom teologii apofatycznej, które widzą w~niej głównie kontynuację myśli neoplatońskiej, a~także daje asumpt do przedstawienia nietypowego formalizmu, który w~zamyśle i~intencjach jego autora -- Uwe Meixnera\index[names]{Meixner, Uwe}\footnote{U. Meixner, \textit{Negative Theology, Coincidentia Oppositorum, and Boolean Algebra}, ,,History of Philosophy \& Logical Analysis'', vol. 1 (1998), nr 1, ss.~75-89.} -- jest narzędziem rekonstrukcji zarówno neoplatońskich, jak i~apofatycznych rozważań o~najwyższym, pierwotnym i~absolutnym bycie.

W~teologii apofatycznej rozumianej jako mistycyzm o~pochodzeniu neoplatońskim transcendencja Boga wyłania się ze ściśle zhierarchizowanej ontologii. W obrębie teologii milczenia uznaje się transcendencję Boga, ponieważ nie da się o~nim nic powiedzieć. Teologiczny sceptycyzm przesuwa apofatyczne akcenty w~stronę wiedzy i~poznania uznając Boga za transcendentnego, bo istotowo niepoznawalnego. W~mistycyzmie o~neoplatońskiej proweniencji Jednia nie jest ,,ponad'' bytami dlatego, że nie można o~niej mówić i~myśleć, tak jak mówi i~myśli się o~bytach. Przeciwnie: nie można o~niej mówić i~myśleć dlatego, że znajduje się na samym szczycie w wyrafinowanej metafizycznej konstrukcji.


\section{Neoplatońskie źródła teologii apofatycznej}

Pewne ślady myślenia apofatycznego można odnaleźć już w~pismach Platona\index[names]{Platon}\footnote{E. Dodds, \textit{The Parmenides of Plato and the Origin of the Neoplatonic ‘One'}, ,,The Classical Quarterly'', vol. 22 (1928), nr 3-4, ss.~129-142.}. Dialogiem, który najsilniej oddziałał na sposób myślenia neoplatończyków i~-- za ich pośrednictwem -- teologów negatywnych, jest \textit{Parmenides}. Dialog ten wywołuje niemało problemów interpretacyjnych. Nawet sama forma i~cel powstania tekstu nie jest jasna dla współczesnego czytelnika i~-- jak twierdzą historycy\footnote{Por. D. Carabine, \textit{The Unknown God. Negative Theology in the Platonic Tradition: Plato to Eriugena}, \textit{Louvain Theological and Pastoral Monographs}, Peeters Press, Louvain 1995, ss.~35-50.} -- nie była klarowna nawet w~pierwszym okresie istnienia Akademii, niedługo po powstaniu dzieła. Dialog ten uważany bywa za poważną filozoficzną wykładnię o~transcendentnej i~niewysłowionej Jedni, lecz czasem także za ćwiczenie z~tworzenia sofizmatów lub nawet zgryźliwy i~ponury żart greckiego filozofa. Władysław Witwicki\index[names]{Witwicki, Władysław}, polski tłumacz Platona\index[names]{Platon} znany z~kąśliwych uwag wysuwanych w~kierunku klasycznego filozofia, twierdzi, że Platon\index[names]{Platon} w~\textit{Parmenidesie}

\begin{quote}
o~nic nie kruszy kopii i~niczego nie głosi z~przekonaniem, tylko rozpatruje sobie i~roztrząsa takie i~inne zdania na chłodno i~od niechcenia, i~z uśmiechem przekory. Chce znaleźć sprzeczności i~u siebie, i~u Parmenidesa, i~znajduje ich więcej, niż było trzeba. Jak ten, co burzy domki z~kart, które sam budował długi czas. Kończy dialog sprzecznymi zdaniami, jakby sprzeczność nie miała w~sobie nic niepokojącego, jakby nie była nieomylnym wskaźnikiem jakiegoś fałszu. I~tak zostawia czytelnika. Jakby mu język pokazał na końcu\footnote{W. Witwicki, \textit{Przedmowa}, [w:] Platon, \textit{Parmenides}, Wydawnictwo Marek Derewiecki, Kęty 2021.}.
\end{quote}

Rzeczywiście, nie potrzeba wiele złej woli, by dojść do podobnych przekonań, bowiem już w~tym tekście, którego powstanie dzieli od działalności Pseudo-Dionizego Areopagity\index[names]{Pseudo-Dionizy Areopagita} co najmniej dziesięć stuleci, można odnaleźć dobrze nam znane apofatyczne sformułowania, według których Jednia jest niewysłowiona (\textgreek{>'agghtos}), niepojęta (\textgreek{o>u m`hn o>ut`o l'egomen}), nieokreślona (\textgreek{akathgo\~ito}) czy ponad wszelkim rozumem (\textgreek{o>'ute a>'isthsic o>'ute >epist'hmh})\footnote{Wszystkie określenia pochodzą z~Platon, \textit{Parmenides}, 142a. Por. R. Mortley, \textit{Negative Theology and Abstraction in Plotinus}, ,,The American Journal of Philology'', vol. 96 (1975), nr 4, s.~372.}. Wnioski, jakie niektórzy komentatorzy wyciągają z~tego dialogu, przypominają do złudzenia tezy wyjęte z~\textit{Teologii mistycznej}. Według Meixnera\index[names]{Meixner, Uwe}\footnote{U. Meixner, \textit{Negative Theology}\ldots, dz. cyt., s.~76.} fragment 137c-142a należy streścić w~zdaniu głoszącym, że \textit{O~Jedni nie można powiedzieć niczego}, natomiast kolejne części dialogu (142b-155e) każą sądzić, że \textit{O~jedni można powiedzieć cokolwiek}.

\begin{quote}
Trudno wyobrazić sobie poważniejszy atak na racjonalność niż stwierdzenie obu tych tez na raz. Pierwsze stwierdzenie jest sprzeczne z~prawem wyłączonego środka, drugie z~prawem niesprzeczności, oba stwierdzenia sprzeciwiają się sobie i~ewidentnie są wzajemnie sprzeczne\footnote{Tamże.}.
\end{quote}

Do zaskakująco podobnych wniosków dochodzi Zbigniew Król\index[names]{Król, Zbigniew}, który rekonstruując pojęcia negacji w~\textit{Parmenidesie}\footnote{Z. Król, \textit{The Implicit Logic of Plato's Parmenides}, ,,Filozofia Nauki'', vol. 21 (2013), nr 1, ss.~121-135.}, próbuje pogodzić je z~dwoma hipotezami wysuniętymi na podstawie tych samych fragmentów dialogu. Pierwsza sformułowana przez niego hipoteza głosi, że \textit{Jednia nie uczestniczy w~niczym} (w szczególności nie uczestniczy w~wielości, części, całości, wielkości, miejscu ruchu itd.), druga hipoteza natomiast stwierdza, że \textit{Jednia uczestniczy w~bycie}, co nieuchronnie prowadzi do wniosku, że posiada jakieś własności\footnote{Tamże, ss.~132-133.}.

Spośród wszystkich pism Platona\index[names]{Platon} to \textit{Parmenides} wywarł największy wpływ na Plotyna\index[names]{Plotyn} -- pochodzącego z~terenów Egiptu greckiego filozofa, twórcę neoplatonizmu. Natomiast najsilniej oddziałującą na niego platońską koncepcją była teoria idei oraz wynikający z~niej dualizm między światem idei i~światem zjawisk, a~w konsekwencji także ontologiczne powiązania między tymi światami. Według Platona\index[names]{Platon} bowiem prawdziwie istnieją wyłącznie pewne doskonałe byty nazywane przez niego ideami. Zjawiska, rzeczy oraz ich własności istnieją wyłącznie dzięki uczestnictwie w~ideach. Na przykład rzeczy dobre są dobre dzięki temu, że uczestniczą w~idei dobra, a~piękne zjawiska są piękne dzięki obecności w~nich idei piękna. Na tych pomysłach bazował Plotyn\index[names]{Plotyn} budując swą teoretyczną konstrukcję, choć czerpał także bez oporów z~myśli Arystotelesa\index[names]{Arystoteles}, Filona Aleksandryjskiego\index[names]{Filon z~Aleksandrii} oraz presokratyków. Z~takiej mieszanki powstał silnie ustrukturyzowany i~spójny system ontologiczny, w~którym centralne (a właściwie najwyższe) miejsce zajmuje Jednia.

Neoplatońska ontologia Plotyna\index[names]{Plotyn} ustanawia więc pełną hierarchię bytów. Na jej szczycie znajduje się pierwotna, najdoskonalsza, niepodzielna Jednia. Niżej w~hierarchii umieszcza się byt, który dzielony jest na ożywiony i~nieożywiony. Spośród bytów ożywionych doskonalsze są te, które odznaczają się duszą rozumną (\textgreek{no\~uc}). Świat \textgreek{no\~uc}, czyli świat idealny, stanowi więc drugi szczebel po Jedni w~ontologicznej drabinie Plotyna\index[names]{Plotyn}. Trzecim jest świat psychiczny, a~czwartym świat materii nieożywionej. Jednia jest wolna od jakichkolwiek przeciwieństw i~podziałów, bezwzględna, absolutna, ontologicznie niezależna od niczego\footnote{Por. C.M. Cohoe, \textit{Why the One cannot have parts: Plotinus on divine simplicity, ontological independence, and perfect being theology}, ,,Philosophical Quarterly'', vol. 67 (2017), nr 269, ss.~751-771.}, jest szczytem dobra, piękna, prawdy i~jedności oraz źródłem wszystkiego, co istnieje. Wszystkie byty niższych szczebli hierarchii uczestniczą w~bytach doskonalszych, z~kolei te doskonalsze emanują i~tworzą byty mniej doskonałe, nazwane hipostazami. Hierarchia Plotyna\index[names]{Plotyn} jest więc nie tylko porządkiem zmniejszającej się doskonałości, ale i~mocy twórczej, która wyczerpuje się na poziomie materii będącej kresem procesu emanacyjnego.

Jednia może być zatem nazwana źródłem istnienia bytów, ale poza tą własnością nie da się wskazać i~przypisać jej żadnej innej. Leży ona poza dziedziną rzeczy, które można ogarnąć myślą, poznać lub nazwać. Jest niepoznawalna, znajduje się nie tylko poza bytem i~substancją (\textgreek{>ep'ekeina to\~u >'ontos}\footnote{Plotinus, \textit{Ennead}, VI. 2, red. i~tłum. A.H. Armstrong, Loeb Classical Library, Harvard University Press, Cambridge 1984.}, \textgreek{>ep'ekeina o>us'ias}\footnote{Tamże, I. 7.}), ale i~poza umysłem i~rozumieniem (\textgreek{>ep'ekeina no\~u  ka`i no'hsewc}\footnote{Tamże.}). Natomiast celem życia człowieka ma być wspięcie się w~górę hierarchii bytów i~zjednoczenie z~absolutną Jednią, które może się dokonać poprzez wysiłek moralny, intelektualny oraz przez sztukę. Filozofia, estetyka i~etyka są więc dla Plotyna\index[names]{Plotyn} trzema drogami pozwalającymi wzbić się ponad byt i~w mistycznym uniesieniu bezpośrednio ,,dotknąć'' doskonałego bytu\footnote{Por. L. Gerson, \textit{Plotinus}, [w:] \textit{The Stanford Encyclopedia of Philosophy}, wyd. jesień 2018, red. E.N. Zalta, {\textless}https://plato.stanford.edu/archives/fall2018/entries/plotinus/{\textgreater}.}.

Zbudowanej w~ten sposób konstrukcji teoretycznej nie jest wcale daleko do religii, zwłaszcza takiej, którą ujmuje się w~teoretyczne ramy teologii apofatycznej. Zauważają to nie tylko badacze myśli neoplatońskiej, lecz także komentatorzy pism należących do \textit{corpus} teologii negatywnej. Na przykład Paul Rorem\index[names]{Rorem, Paul}, który zaproponował najbardziej agnostyczną interpretację dionizyjskiej teologii negatywnej, podkreśla jej podobieństwo z~późną filozofią neoplatońską. W~obu tych doktrynach byty uporządkowane są względem pewnej hierarchii i~w celu dotarcia do bytu absolutnego należy ,,wspiąć się'' po takiej ,,drabinie'' bytów. U~Dionizego\index[names]{Pseudo-Dionizy Areopagita}, by spotkać Boga należy wpierw zanegować nasze związane z~materią wyrażenia i~wyobrażenia i~przekroczyć je, by dojść do ich pojęciowych znaczeń. Następnie zanegowane zostać powinny także owe znaczenia oraz wszelkie inne pojęcia umysłu, ponieważ przekroczenie naszej wiedzy prowadzi do Niepoznawalnego, do cichego zjednoczenia z~Bogiem. Innymi słowy, Rorem\index[names]{Rorem, Paul} zwraca uwagę, że u~Areopagity drogą do Boga jest zaprzeczenie wszystkich bytów. Dionizy\index[names]{Pseudo-Dionizy Areopagita} jednak wielokrotnie stwierdza, że Bóg jest także ponad wszelkim zaprzeczeniem. Ostatecznie więc należy zanegować także wszelkie negacje, nie pozostawając już z~żadnym pojęciem Boga\footnote{Por. P. Rorem, \textit{Pseudo-Dionysius. A~Commentary on the Texts and an Introduction to Their Influence}, Oxford University Press, Oxford -- New York 1993, ss.~210-211.}. Jak zauważa Rorem\index[names]{Rorem, Paul}, Dionizy\index[names]{Pseudo-Dionizy Areopagita}

\begin{quote}
zaprzecza i~wykracza poza wszystkie nasze pojęcia lub ,,pojęciowe'' atrybuty Boga i~kończy na odrzuceniu wszelkiego mówienia i~myślenia, nawet negatywnego\footnote{P. Rorem, C. Luibheid, \textit{Dionysius the Areopagite. The Complete Work}, Paulist Press, New York 1987, s.~99. Cytat w~j. polskim za: P. Rojek, \textit{Logika teologii negatywnej}, ,,Pressje'', nr 29 (2012), s.~222.}.
\end{quote}

Rzeczywiście, element niewysławialności został mocno wyeksponowany u~niektórych filozofów późnego neoplatonizmu. Dobrym przykładem takiego rozwoju doktryny jest Damascjusz\index[names]{Damascjusz} należący do ateńskiej szkoły neoplatończyków. Zgodnie z~jej duchem rozciągał on ontologiczną hierarchię mnożąc hipostazy, w~wyniku czego ponad Jednią, którą możemy jeszcze w~jakiś niedoskonały sposób poznać i~nazwać, umieścił Niewysłowione, które jest absolutnie transcendentne, i~którego nie można nazwać, pojąć ani pomyśleć.

\begin{quote}
Jest to w~istocie tak, jakby ktoś, niewidomy od urodzenia, twierdził, że ciepło nie leży u~podstaw koloru. Czyż nie powie on raczej, że kolor nie jest ciepły, skoro ciepło jest czymś, co można dotknąć i~co zna on dzięki dotykowi, a~w żaden sposób nie zna koloru, jak tylko w~ten, że nie jest on przedmiotem dotyku? Bo wie on jedynie, czego nie wie, i~nie jest to jego wiedza o~kolorze, ale świadomość własnej niewiedzy. W~ten sam sposób my, mówiąc, że [Niewysłowione] jest niepoznawalne, nie stwierdzamy czegoś, co się do niego odnosi, ale wyrażamy naszą relację do niego. Bo dla niewidomego ślepota nie jest w~kolorze, ale w~nim samym; z~pewnością zatem niewiedza o~[Niewysłowionym] jest w~nas, bo wiedza o~tym, co znane, jest w~poznającym podmiocie, a~nie w~tym, co jest przedmiotem poznania\footnote{Damascius, \textit{Problems \& Solutions Concerning First Principles}, I.I.6, tłum. S. Ahbel-Rappe, Oxford University Press, Oxford 2010, s.~74. Cytat za: B. Brożek, \textit{Marzenie Leibniza. Rzecz o~języku religii}, Copernicus Center Press, Kraków 2016, rozdz.~4.}.
\end{quote}

Jak zauważa Bartosz Brożek\index[names]{Brożek, Bartosz}\footnote{B. Brożek, \textit{Marzenie Leibniza}\ldots, dz. cyt., rozdz.~4.}, w~podejściu Damascjusza\index[names]{Damascjusz} można także wskazać podstawowe różnice między neoplatońską ontologią a~teologią apofatyczną, przynajmniej w~wydaniu chrześcijańskim. Przede wszystkim, w~obrębie neoplatonizmu (w przeciwieństwie do teologii negatywnej) niewysławialność absolutu nie wynika z~jego transcendentnej natury, lecz z~ograniczeń ludzkiego poznania, wiedzy i~języka. W~tym kontekście Brożek\index[names]{Brożek, Bartosz} sugeruje, że należy bardziej niż o~teologii negatywnej mówić o~,,antropologii'' lub ,,epistemologii negatywnej'' Damascjusza\index[names]{Damascjusz}. Podobnie, rozważania Pseudo-Dionizego Areopagity\index[names]{Pseudo-Dionizy Areopagita} wynikają z~egzystencjalnego dramatu wierzącego człowieka, który nie może wysłowić tego, co stanowi o~istocie jego egzystencji, i~z tego powodu należy je uznać za teologię \textit{par excellence}. Rozważania Damascjusza\index[names]{Damascjusz} natomiast sprawiają wrażenie ,,wyrafinowanej gry intelektualnej'' oraz próby ,,wyciągnięcia ostatecznych wniosków z~imponującej konstrukcji teoretycznej''.

\begin{quote}
Przeczucie Niewysłowionego jest zatem ,,efektem'' towarzyszącym skomplikowanej konstrukcji teoretycznej. Teologia negatywna neoplatończyków jest nadbudowana nad skonstruowaną przez nich ontologią -- bez niej byłaby niezrozumiałą grą słów. Zastanawiające, że to wyrafinowane narzędzie teoretyczne w~kontekście myśli chrześcijańskiej znalazło swe naturalne środowisko\footnote{Tamże.}.
\end{quote}

Nie oznacza to jednak, że neoplatończycy nie wikłają się w~dobrze nam znany paradoks Niewysławialnego. Paradoksalnego charakteru mówienia o~czymś, co określa się słowem ,,Niewysłowione'', oraz nazywania czegoś, co miało nie posiadać nazwy, świadomy jest sam Damascjusz\index[names]{Damascjusz} i próbuje uniknąć tego problemu. Z~jednej strony przekonuje, że mówiąc o~Niewysłowionym nie mówimy o~nim, lecz wyrażamy refleksję o~granicach naszego języka i~umysłu. Z~drugiej strony mówienie o~Niewysłowionym próbuje on ugruntować w~tym, że jest w~nas obecny ,,jakiś element niewysłowienia'' pochodzący z~absolutu.

\begin{quote}
A~co do nas, jak możemy czynić jakiekolwiek założenia odnośnie do [Niewysłowionego], gdyby nie było w~nas jakiegoś jego śladu [\ldots]. Być może należałoby zatem powiedzieć, że ten byt -- Niewysłowione jako takie -- przekazuje wszystkim rzeczom niewysłowioną partycypację, poprzez którą jest w~każdym z~nas jakiś element niewysłowienia? Jest przecież tak, że rozpoznajemy, że niektóre rzeczy są bardziej niewysłowione niż inne -- Jednia bardziej niż Byt, Byt bardziej niż Życie, Życie bardziej niż Intelekt, i~tak dalej\footnote{Damascius, \textit{Problems \& Solutions}\ldots, dz. cyt., I.I.8, Cytat za: B. Brożek, \textit{Marzenie Leibniza}\ldots, dz. cyt.}.
\end{quote}

Zatem znów poruszamy się po hierarchicznej drabinie neoplatońskiej ontologii. Możemy więc stwierdzić, że niewysławialność idzie w~parze z~doskonałością bytu. Najprościej nam mówić i~opisywać rzeczy konkretne. Łatwość ta rozmywa się jednak, gdy zaczynamy rozważać rzeczy bardziej doskonałe. Trudniej mówić nam o~rzeczach abstrakcyjnych, jeszcze trudniej o~duchowych itd. Ostatecznym wnioskiem z~takiej postępującej gradacji musi być uznanie, że istnieje coś najdoskonalszego i~zupełnie niewyrażalnego. W~świetle takiej obserwacji łatwiej już pojąć, co Brożek\index[names]{Brożek, Bartosz} ma na myśli twierdząc, że w~filozofii neoplatońskiej niewysławialność jest wynikiem rozbudowanej teoretycznej konstrukcji\footnote{Por. B. Brożek, \textit{Marzenie Leibniza}\ldots, dz. cyt.}.

%Zatem tak, jak teologia milczenia akcentowała transcendencję Boga w kontekście języka twierdząc, że jest on niewyrażalny a sceptycyzm teologiczny

%W~świetle powyższych rozważań należy stwierdzić, że w~teologii apofatycznej rozumianej jako mistycyzm o~pochodzeniu neoplatońskim transcendencja Boga wyłania się ze ściśle zhierarchizowanej ontologii. W obrębie teologii apofatycznej rozumianej jako teologia milczenia uznaje się transcendencję Boga, ponieważ nie da się o~nim nic powiedzieć. Teologiczny sceptycyzm przesuwa apofatyczne akcenty w~stronę wiedzy i~poznania uznając Boga za transcendentnego, bo istotowo niepoznawalnego i~niepojmowalnego. W mistycyzmie o~neoplatońskiej proweniencji Jednia nie jest ,,ponad'' bytami dlatego, że nie można o~niej mówić i~myśleć, tak jak mówi i~myśli się o~bytach. Przeciwnie: nie można o~niej mówić i~myśleć dlatego, że znajduje się na samym szczycie w wyrafinowanej metafizycznej konstrukcji.



\section{Algebra zbioru potęgowego jako formalna podstawa neoplatońskich rozważań o~Jedni i~apofatycznych rozważań o~Bogu}

Logiczny szkielet dla powyżej naszkicowanego neoplatońskiego mistycyzmu próbuje podać Meixner\index[names]{Meixner, Uwe} w~krótkiej pracy \textit{Negative Theology, Conincindentia Oppositorum and Boolean Algebra}, która jest wypadkową jego ogólniejszych rozważań w~zakresie dziedziny nazywanej często ontologią formalną\footnote{U. Meixner, \textit{Axiomatic formal ontology}, ser. \textit{Synthese Library}, red. J. Hintikka i~in., vol. 264, Springer, Dordrecht 1997.}. Choć zawarta w~tej pracy rekonstrukcja dotyczy wprost neoplatońskich koncepcji zawartych w~klasycznych pismach filozoficznych, Mei\-xner deklaruje dodatkowo, że zaprezentowany przez niego model pozwala zinterpretować w~sposób logicznie spójny także teologiczne wywody Pseudo-Dionizego Areopagity\index[names]{Pseudo-Dionizy Areopagita}. Ponadto przekonuje też, że sam sposób rekonstrukcji w~pewien sposób odwzorowuje idee neoplatońskie.

\begin{quote}
[...] nie chodzi wyłącznie o~to, że jego [Pseudo-Dionizego\index[names]{Pseudo-Dionizy Areopagita} -- P.U.] wypowiedzi mogą być przez nas zrekonstruowane logicznie, a~interpretacja narzucona na te wypowiedzi w~wyniku logicznej rekonstrukcji musi być uznana za zupełnie obcą temu, co Pseudo-Dionizy\index[names]{Pseudo-Dionizy Areopagita} miał na myśli. Koncepcje, które wykorzystuje ta logiczna rekonstrukcja, były zasadniczo obecne w~starożytnej logice i~ontologii formalnej już od czasów Platona\index[names]{Platon}. Co więcej, były istotną częścią atmosfery intelektualnej tradycji neoplatońskiej, w~której sytuuje się Pseudo-Dionizy\index[names]{Pseudo-Dionizy Areopagita}, a~która bardzo szanowała logikę i~ontologię formalną, nawet gdy poszukiwała mistycznej ekstazy. Arystotelesowski\index[names]{Arystoteles} wpływ na tę tradycję przysłużył się jedynie silniejszemu ich ugruntowaniu\footnote{U. Meixner, \textit{Negative Theology}\ldots, dz. cyt., ss.~82-83.}.
\end{quote}
Jest to zatem kolejna referowana w~niniejszej pracy próba obrony apofatycznej doktryny Areopagity\index[names]{Pseudo-Dionizy Areopagita} przed zarzutami o~bycie teorią paradoksalną i~sprzeczną.

\subsection{Własności jako relacje część-całość}

Swoją logiczną rekonstrukcję Meixner\index[names]{Meixner, Uwe} rozpoczyna od podważenia dobrze ugruntowanego, tradycyjnego założenia o~kategorialnej odróżnialności podmiotu i~predykatu. Tradycyjnie logicy (a także metafizycy) przyjmują arystotelesowskie\index[names]{Arystoteles} rozróżnienie między własnościami a~przedmiotami (indywiduami). Inaczej mówiąc, na ogół uważa się, że przedmioty i~własności należą do dwóch odrębnych, nieprzecinających się kategorii. Własności nie istnieją samodzielnie, mogą być jedynie egzemplifikowane przez przedmioty. Z~drugiej strony indywidua nie są własnościami i~nie mogą być o~niczym orzekane. Meixner\index[names]{Meixner, Uwe} przekonuje\footnote{Tamże, s.~72.} jednak, że w~filozofii Platona\index[names]{Platon} rozróżnienie to jest nieostre -- własności są ideami, czyli niezależnymi od umysłu, ponadczasowymi bytami. Mogą istnieć samodzielnie i~są co najmniej tak samo realne jak indywidua. Dlatego postuluje, by zaprzestać rozdzielania tych dwóch klas i~w~duchu zinterpretowanej przez niego tradycji platońskiej mówić o~jednej kategorii ,,przedmioto-własności'' (nazwanej dalej po prostu ,,własnościami'').

Jakkolwiek podejście to wydaje się kuriozalne, nie jest to pomysł odosobniony w~historii ludzkiej myśli. Dwudziestowieczne rozważania nad nierozróżnialnością przedmiotów i~własności w~logice zapoczątkował Frank Plumpton Ramsey\index[names]{Ramsey, Frank P.}\footnote{W~swoim młodzieńczym artykule z~1925 roku przedrukowanym w: F.P. Ramsey, \textit{Universals}, [w:] Tenże, \textit{Philosophical Papers}, red. D.H. Mellor, Cambridge University Press, Cambridge -- New York -- Melbourne 1990, ss.~8-32.}. Stawia on tezę o~kategorialnej nierozróżnialności podmiotu i~predykatu, a~w konsekwencji znaczeniowej nieodróżnialności własności i~indywiduów:

\begin{quote}
[\ldots] wydaje mi się tak jasne, jak to tylko możliwe w~filozofii, że dwa zdania: ,,Sokrates\index[names]{Sokrates} jest mądry'' oraz ,,Mądrość jest własnością Sokratesa\index[names]{Sokrates}'' stwierdzają ten sam fakt i~wyrażają ten sam sąd. Nie są one, oczywiście, tym samym zdaniem, ale mają to samo znaczenie, tak jak dwa zdania w~dwóch różnych językach mogą mieć to samo znaczenie. To, którego zdania użyjemy, jest kwestią stylu literackiego lub punktu widzenia [\ldots]; ale niezależnie od tego, które zdanie wypowiemy, mamy na myśli to samo. W~jednym z~tych zdań podmiotem jest ,,Sokrates''\index[names]{Sokrates}, w~drugim ,,mądrość''; a~zatem to, które z~tych dwóch jest podmiotem, a~które predykatem, zależy od tego, jakiego zdania użyjemy do wyrażenia naszego sądu, i~nie ma nic wspólnego z~logiczną naturą Sokratesa\index[names]{Sokrates} czy mądrości, lecz jest wyłącznie kwestią dla gramatyków. W~ten sam sposób, przy wystarczająco elastycznym języku, dowolny sąd może zostać wyrażony w~taki sposób, że dowolny z~jego terminów stanie się podmiotem. Nie ma zatem zasadniczej różnicy między przedmiotem sądu a~jego predykatem i~nie można na tym rozróżnieniu oprzeć żadnej fundamentalnej klasyfikacji przedmiotów\footnote{Tamże, s.~12.}.
\end{quote}
Okazuje się także, że sama logika, jako niezinterpretowany system formalny, nie daje wyraźnych podstaw do formalnego odróżnienia przedmiotów i~własności i~trudno uznać ją za formalną ontologię predykacji, jak chcieliby niektórzy. W~rzeczywistości, można mówić o~wielu formalnych ontologiach przedmiotów i~własności -- neutralnej, ściśle przedmiotowej, atrybutywnej, przedmiotowo-atrybutywnej i~atrybutywno-przedmiotowej -- często izomorficznych ze sobą\footnote{Por. J. Paśniczek, \textit{Predykacja. Elementy ontologii formalnej przedmiotów, własności i~sytuacji}, Copernicus Center Press, Kraków 2014, rozdz.~1.}.

Wróćmy jednak do sposobu, w~jaki Meixner\index[names]{Meixner, Uwe} dokonuje rekonstrukcji teorii Platona\index[names]{Platon} i~neoplatończyków. Powołując się na drugą część \textit{Parmenidesa} sugeruje, że Platon\index[names]{Platon} rozważa możliwość analizowania predykatów poprzez relacje część-całość. Ta obserwacja prowadzi Meixnera\index[names]{Meixner, Uwe} do intensjonalnej teorii własności.

\begin{quote}
Własności są w~oczywisty sposób częściami innych własności: treść jednej własności jest zawarta w~treści innej własności. Tak więc, na przykład, własność Nie-bycia-żonatym-w-1991-roku jest w~taki właśnie sposób elementem, to znaczy intensjonalną częścią, własności Bycia-kawalerem-w-1991-roku\footnote{U. Meixner, \textit{Negative Theology\ldots}, dz. cyt., s.~77.}.
\end{quote}
Warto przypomnieć, że w~obrębie interpretacji Meixnera\index[names]{Meixner, Uwe} powyższa obserwacja dotyczyć musi całej kategorii przedmioto-własności.

\subsection{Porządki na klasie przedmioto-własności}

Meixner\index[names]{Meixner, Uwe} twierdzi, że strukturą odpowiedzialną za ,,neoplatońską'' ontologię formalną jest struktura mereologiczna (ponieważ mereologia stanowi właściwą teorię relacji część-całość). Jak
%przypadkowo
udowodnił Tarski\index[names]{Tarski, Alfred}, o strukturze mereologicznej można myśleć, jak o~zupełnej algebrze Boole'a\index[names]{Boole, George} pozbawionej elementu zerowego\footnote{A. Tarski, \textit{Zur Grundlegung der Boole'schen Algebra I}, ,,Fundamenta Mathematicae'', vol. 24 (1935), nr 1, ss.~177-198. Zob. także A. Pietruszczak, \textit{Pieces of mereology}, ,,Logic and Logical Philosophy'', vol. 14 (2005), nr 2, ss.~229-233.}. Dodatkowo pewna wersja twierdzenia o~reprezentacji Stone'a\index[names]{Stone, Marshall H.} mówi, że zupełne algebry Boole'a są algebrami zbioru potęgowego\footnote{M.H. Stone, \textit{The Theory of Representation for Boolean Algebras}, ,,Transactions of the American Mathematical Society'', vol. 40 (1963), nr 1, ss.~1-37. Zob. także C. Pontow, R. Schubert, \textit{A~mathematical analysis of theories of parthood}, ,,Data and Knowledge Engineering'', vol. 59 (2006), nr 1, s.~108 oraz B. Skowron, \textit{Część i~całość. W~stronę topoontologii}, Oficyna Wydawnicza Politechniki Warszawskiej, Warszawa 2021, ss.~8-11.}.
W~rzeczywistości to właśnie tę ostatnią strukturę wykorzystuje Meixner\index[names]{Meixner, Uwe} do rekonstrukcji neoplatońskiej hierarchii bytów. Inaczej mówiąc, przyjmując jego ustalenia dotyczące własności oraz relacji między nimi, możemy uporządkować klasę własności według relacji ,,jest częścią''. W~ten sposób otrzymujemy strukturę izomorficzną do algebry zbioru potęgowego. Meixner\index[names]{Meixner, Uwe} nie podaje wprost aksjomatyki algebry Boole'a\index[names]{Boole, George}. Zamiast tego przedstawia tę strukturę za pomocą poniższego zbioru aksjomatów, twierdzeń i~definicji. Taki sposób prezentacji motywuje opinią, według której to właśnie te twierdzenia są w~rzeczywistości wykorzystywane przez ontologów zainspirowanych Platonem\index[names]{Platon}\footnote{Zob. U.~Meixner, \textit{Negative Theology\ldots}, dz. cyt., s.~77.}.
\begin{flalign}
& \parbox[t]{.9\linewidth}{
Dla każdej własności \textsc{x, y, z}: jeśli \textsc{x}~jest częścią \textsc{y}~i \textsc{y}~jest częścią \textsc{z}, to \textsc{x}~jest częścią \textsc{z}.} &\tag{M\textsubscript{1}}\label{mei1}\\
& \text{Dla każdej własności \textsc{x}: \textsc{x}~jest częścią \textsc{x}.} &\tag{M\textsubscript{2}}\label{mei2}\\
& \parbox[t]{.9\linewidth}{
Dla każdej własności \textsc{x}, \textsc{y}: jeśli \textsc{x}~jest częścią \textsc{y}~i \textsc{y}~jest częścią \textsc{x}, to \textsc{x}~jest identyczny z~\textsc{y}.} &\tag{M\textsubscript{3}}\label{mei3}\\
& \parbox[t]{.9\linewidth}{
Dla wszystkich \textsc{x}: jeśli \textsc{x}~jest własnością, to nie-\textsc{x} (negacja \textsc{x}) też jest własnością.} &\tag{M\textsubscript{4}}\label{mei4}\\
& \text{Istnieje własność, która jest częścią każdej własności.} &\tag{M\textsubscript{5}}\label{mei5}\\
& \text{Istnieje własność, której częściami są wszystkie własności.} &\tag{M\textsubscript{6}}\label{mei6}
\end{flalign}
%
%1. Dla każdej własności \textsc{x, y, z}: jeśli \textsc{x}~jest częścią \textsc{y}~i \textsc{y}~jest częścią \textsc{z}, to \textsc{x}~jest częścią \textsc{z}.
%
%2. Dla każdej własności \textsc{x}: \textsc{x}~jest częścią \textsc{x}.
%
%3. Dla każdej własności \textsc{x}, \textsc{y}: jeśli \textsc{x}~jest częścią \textsc{y}~i \textsc{y}~jest częścią \textsc{x}, to \textsc{x}~jest identyczny z~\textsc{y}.
%
%4. Dla wszystkich \textsc{x}: jeśli \textsc{x}~jest własnością, to nie-\textsc{x} (negacja \textsc{x}) też jest własnością.
%
%5. Istnieje własność, która jest częścią każdej własności.
%
%6. Istnieje własność, której częściami są wszystkie własności.
%
Dalej Meixner\index[names]{Meixner, Uwe} zauważa, że:
\begin{flalign}
& \parbox[t]{.87\linewidth}{Istnieje dokładnie jedna własność, która jest częścią każdej\\własności. \hfill (\ref{mei5}, \ref{mei3}) } &  \label{mei7}\\
& \parbox[t]{.87\linewidth}{
Istnieje dokładnie jedna własność, której wszystkie własności są\\częściami.\hfill(\ref{mei6}, \ref{mei3})} &\label{mei8}
\end{flalign}
%\begin{flalign}
%& \text{Istnieje dokładnie jedna własność, która jest częścią każdej} & \\
%& \text{własności.} & \omit\hfill (\ref{mei5}, \ref{mei3})  \nonumber  \label{mei7}\\
%& \text{Istnieje dokładnie jedna własność, której wszystkie własności są} & \\
%& \text{częściami.} & \omit\hfill (\ref{mei6}, \ref{mei3}) \nonumber \label{mei8}
%\end{flalign}
%
%7. Istnieje dokładnie jedna własność, która jest częścią każdej własności.(5, 3)
%
%8. Istnieje dokładnie jedna własność, której wszystkie własności są częściami.(6, 3)
%
Wprowadza do swojego systemu także (dosyć wieloznaczną) formułę, która może powiedzieć nieco więcej o~pojęciu negacji własności:
\begin{flalign}
& \parbox[t]{.9\linewidth}{
Własność, która jest częścią każdej własności, jest (identyczna z) negacją własności, której częściami są wszystkie własności\footnotemark.}
&\tag{M\textsubscript{7}}\label{mei9}
\end{flalign}
\footnotetext{W~oryginalnym ujęciu Meixnera\index[names]{Meixner, Uwe}: ,,Maksymalność jest (identyczna z) negacją minimalności'' (patrz przypis \ref{przyp-mei-niesc}). Jest to jeden z~jedynie dwóch aksjomatów rządzących negacją, jakie Meixner\index[names]{Meixner, Uwe} podaje wprost w~swoim systemie (drugi to \ref{mei4}). Wybranie takich aksjomatów sprawia, że pojęcie negacji można interpretować wieloznacznie. Zapewne Meixnerowi\index[names]{Meixner, Uwe} idzie o~dopełnienie, ale formuły te dopuszczają także inne rozumienie (np. brak części wspólnych). W~przypisie dodaje jednak dwie dodatkowe formuły, o~których, jak przekonuje, ,,warto pamiętać'':
\begingroup
\setlength{\abovedisplayskip}{0pt}
\setlength{\belowdisplayskip}{0pt}
\setlength{\abovedisplayshortskip}{0pt}
\setlength{\belowdisplayshortskip}{0pt}
\begin{flalign}
& \parbox[t]{.9\linewidth}{
Dla każdej własności \textsc{x}: \textsc{x}~jest identyczne z~negacją negacji \textsc{x}.}
&\tag{M\textsubscript{7.1}}\\
& \parbox[t]{.9\linewidth}{
Dla dowolnych własności \textsc{x, y}: jeśli zarówno \textsc{y}~jak i~nie-\textsc{y} są częściami \textsc{x}, to każda własność jest częścią \textsc{x}.}
&\tag{M\textsubscript{7.2}}
\end{flalign}
\endgroup
}
%9.1 Dla każdej własności \textsc{x}: \textsc{x}~jest identyczne z~negacją negacji \textsc{x}. 9.2 Dla każdej własności \textsc{x, y}: jeśli zarówno \textsc{y}~jaki nie-\textsc{y} są częściami \textsc{x}, to każda własność jest częścią \textsc{x}

%9. Własność, która jest częścią każdej własności, jest (identyczna z) negacją własności, której częściami są wszystkie własności\footnote{W~oryginalnym ujęciu Meixnera: ,,Maksymalność jest (identyczna z) negacją minimalności'' (patrz przypis ref następny). Jest to jeden z~jedynie dwóch aksjomatów rządzących negacją, jakie Meixner podaje wprost w~swoim systemie (drugi to {\textbackslash}eqref). Wybranie takich aksjomatów sprawia, że pojęcie negacji można interpretować wieloznacznie. Zapewne Meixnerowi idzie o~dopełnienie, ale formuły te dopuszczają także inne rozumienie (np. brak części wspólnych). W~przypisie dodaje jednak dwie dodatkowe formuły, o~których, jak przekonuje, ,,warto pamiętać'': 9.1 Dla każdej własności X: X~jest identyczne z~negacją negacji X. 9.2 Dla każdej własności X, Y: jeśli zarówno Y~jaki nie-Y są częściami X, to każda własność jest częścią X.}.
%
W~końcu Meixner\index[names]{Meixner, Uwe} wzbogaca swoją teorię o~komponent nacechowany metafizycznie, nadając specjalne znaczenie ekstremom porządku.
\begin{defin}\label{mei-def1}
Byt := własność, która jest częścią każdej własności (tzn. element najmniejszy porządku).
\end{defin}
\begin{defin}\label{mei-def2}
Jednia := własność, której wszystkie własności są częściami (tzn. element największy porządku)\footnote{\label{przyp-mei-niesc}Zasadniczo aksjomaty, twierdzenia i~definicje Meixnera\index[names]{Meixner, Uwe} staram się podać na tyle wiernie, na ile jest to możliwe bez wprowadzania czytelnika w~błąd. W~tym miejscu chciałbym jednak zawrzeć kilka uwag. Po pierwsze, Meixner\index[names]{Meixner, Uwe} nie używa terminów ,,element największy'' oraz ,,najmniejszy'' lub ,,1'' i~,,0'', lecz ,,maksymalność'' oraz ,,minimalność'' i~dopiero te utożsamia z~Jednią i~bytem. Usuwam tę warstwę formalizacji z~jego systemu, ponieważ jest redundantna oraz prowadzi do nieporozumień. Element maksymalny nie jest tym samym, czym jest element największy, o~który idzie Meixnerowi\index[names]{Meixner, Uwe}. Podobnie rzecz się ma z~elementem najmniejszym (lub zerowym) i~minimalnym. Nie chciałbym jednak z~góry odrzucać tego systemu wyłącznie z~powodu niepoprawnej terminologii. Zasadniczo jednak uważam, że Meixnera\index[names]{Meixner, Uwe}, owszem, można nazwać ontologiem formalnym, ale jego sposób prezentacji ontologii formalnej pokazuje, że bliżej mu do pierwszego, niż do drugiego członu tej nazwy.}.
\end{defin}
%
%Definicja 1. Byt := własność, która jest częścią każdej własności (tzn. element najmniejszy porządku).
%
%Definicja 2. Jednia := własność, której wszystkie własności są częściami (tzn. element największy porządku)\footnote{Zasadniczo aksjomaty, twierdzenia i~definicje Meixnera staram się podać na tyle wiernie, na ile jest to możliwe bez wprowadzania czytelnika w~błąd. W~tym miejscu chciałbym jednak zawrzeć kilka uwag. Po pierwsze, Meixner nie używa terminów ,,element największy'' oraz ,,najmniejszy'' lub ,,1'' i~,,0'', lecz ,,maksymalność'' oraz ,,minimalność'' i~dopiero te utożsamia z~Jednią i~bytem. Usuwam tę warstwę formalizacji z~jego systemu, ponieważ jest redundantna oraz prowadzi do nieporozumień. Element największy nie jest tym samym, czym jest element maksymalny, o~który idzie Meixnerowi. Podobnie rzecz się ma z~elementem najmniejszym (lub zerowym) i~minimalnym. Nie chciałbym jednak z~góry odrzucać tego systemu wyłącznie z~powodu niepoprawnej terminologii. Zasadniczo jednak uważam, że Meixnera, owszem, można nazwać ontologiem formalnym, ale jego sposób prezentacji ontologii formalnej pokazuje, że bliżej mu do pierwszego, niż do drugiego członu tej nazwy.}.
%
\vskip -0.5cm
\begingroup
\setlength{\abovedisplayskip}{0pt}
\setlength{\abovedisplayshortskip}{0pt}
\begin{flalign}
& \text{
Jednia jest negacją bytu.} & \text{(\ref{mei9}, def. \ref{mei-def1}, def. \ref{mei-def2})}  \label{mei10}
\end{flalign}
\endgroup
%10. Jednia jest negacją bytu (9, def 1, def 2).

Meixner\index[names]{Meixner, Uwe} przekonuje, że jednym z~głównych celów Platona\index[names]{Platon} w~\textit{Parmenidesie} było podanie zadawalającej analizy użycia czasownika ,,być''. Wśród tych użyć Meixner\index[names]{Meixner, Uwe} wyróżnia ,,jest'' predykatywne -- takie, które w~języku naturalnym wiąże się z~przypisaniem czemuś jakiegoś predykatu, na przykład ,,Afrodyta jest piękna'' -- obok ,,jest'' tożsamościowego (stwierdzającego identyczność, na przykład ,,Samuel Longhorn Clemens\index[names]{Clemens, Samuel L.|see{Twain, Mark}} jest Markiem Twainem\index[names]{Twain, Mark}''), oraz ,,jest'' podporządkowania (gdy mówimy o~przynależności do rodzaju lub gatunku, na przykład ,,szczur wędrowny jest ssakiem'')\footnote{Nie jest odosobniony w~takim twierdzeniu. Por. klasyczny dziś tekst C.H. Kahn, \textit{Byt u~Parmenidesa i~Platona}, ,,Przegląd Filozoficzny -- Nowa Seria'', vol. 1 (1992), nr 4, ss.~95-116.}. Według Meixnera\index[names]{Meixner, Uwe} druga część \textit{Parmenidesa} może dawać przesłanki ku temu, by uważać, że Platon\index[names]{Platon} analizował relację wyznaczoną przez ,,jest'' predykatywne jako relację część-całość\footnote{U. Meixner, \textit{Negative Theology}\ldots, dz. cyt., ss.~75-76.}. Tak czy owak, ta relacja jest na tyle istotna w~formalno-ontologicznym systemie Meixnera\index[names]{Meixner, Uwe}, że podaje aż trzy jej definicje (z~których jedną formułuje w~bezpośrednim nawiązaniu do Pseudo-Dionizego Areopagity\index[names]{Pseudo-Dionizy Areopagita}):
\begin{defin}\label{mei-def3}
\textsc{x}~jest predykatywnie \textsc{y}~:= \textsc{x}~jest własnością i~\textsc{y} jest własnością, oraz \textsc{y}~jest częścią \textsc{x}.
\end{defin}
\begin{defin}\label{mei-def4}
\textsc{x}~jest ściśle predykatywnie\footnote{W~oryginalnym brzmieniu obie definicje mówią o~,,jest'' predykatywnym, bez dodatkowych określeń (\textit{\textit{sic}}!) -- stanowią dwie wersje jednej definicji, choć w~treści pracy w~tym drugim przypadku jest czasem mowa o~\textit{normalnym} ,,jest'' predykatywnym.} \textsc{y}~:= \textsc{x}~jest własnością i~\textsc{y} jest własnością, \textsc{y}~jest częścią \textsc{x}, ale nie-\textsc{y} nie jest częścią \textsc{x}.
\end{defin}
\begin{defin}\label{mei-def5}
\textsc{x}~jest superpredykatywnie \textsc{y}~:= \textsc{x}~jest własnością i~\textsc{y}~jest własnością, \textsc{x}~nie jest ściśle predykatywnie \textsc{y}, ale \textsc{y}~jest częścią \textsc{x}.
\end{defin}
%Definicja 3. X~jest predykatywnie Y~:= X~jest własnością i~Y jest własnością, oraz Y~jest częścią X.
%
%Definicja 4. X~jest ściśle predykatywnie\footnote{W~oryginalnym brzmieniu obie definicje mówią o~,,jest'' predykatywnym, bez dodatkowych określeń (sic!) -- stanowią dwie wersje jednej definicji, choć w~treści pracy w~tym drugim przypadki jest czasem mowa o~\textit{normalnym} ,,jest'' predykatywnym.} Y~:= X~jest własnością i~Y jest własnością, Y~jest częścią X, ale nie-Y nie jest częścią X.
%
%Definicja 5. X~jest superpredykatywnie Y~:= X~jest własnością i~Y jest własnością, X~nie jest ściśle predykatywnie Y, ale Y~jest częścią X.
%
\noindent Mając te definicje, można produkować takie zasady teologii negatywnej, jakie tylko są potrzebne do obrony przyjętych założeń:
\begin{flalign}
& \parbox[t]{.87\linewidth}{Jednia jest predykatywnie każdą własnością. \hspace*{\fill}(\ref{mei8}, def. \ref{mei-def1}, \ref{mei-def3})} &\tag{ONT}\label{mei-ONT}\\
& \parbox[t]{.87\linewidth}{Dla dowolnej własności X: Jednia jest predykatywnie zarówno X, jak\\
i~nie-X. \hfill (\ref{mei-ONT}, \ref{mei4})} & \label{mei11}\\
& \parbox[t]{.87\linewidth}{
Dla dowolnej Y, dla wszystkich X: X~nie jest ściśle predykatywnie zarówno\\
Y, jak i~nie-Y. \hfill (def. \ref{mei-def4})} & \label{mei12}\\
& \parbox[t]{.87\linewidth}{Jednia nie jest ściśle predykatywnie żadną własnością.\hfill (\ref{mei12})} & \tag{ONT'}\label{mei-ONTprim}
\end{flalign}
%\begin{flalign}
%& \text{Jednia jest predykatywnie każdą własnością.} &\tag{ONT}\label{mei-ONT}\\
%& \text{Dla dowolnej własności X: Jednia jest predykatywnie zarówno X, jak} &  \label{mei11}\\ 
%& \text{i~nie-X.} & \nonumber\\
%& \text{Dla dowolnej Y, dla wszystkich X: X~nie jest ściśle predykatywnie ani Y,} & \label{mei12} \\
%& \text{ani nie-Y.} & \nonumber\\
%& \text{Jednia nie jest ściśle predykatywnie żadną własnością} & \tag{ONT'}\label{mei-ONTprim}
%\end{flalign}
%ONT Jednia jest predykatywnie każdą własnością. (8, def 1, 2 i~3)
%
%11. Dla dowolnej własności X: Jednia jest predykatywnie zarówno X, jak i~nie-X. (ONT, 4)
%
%12. Dla dowolnej Y, dla wszystkich X: X~nie jest ściśle predykatywnie ani Y, ani nie-Y (def 4).
%
%ONT' Jednia nie jest ściśle predykatywnie żadną własnością. (12)
%
oraz:
%\begin{flalign}
%& \parbox[t]{.86\linewidth}{Jednia jest superpredykatywnie każdą własnością.\\\hspace*{\fill} (\ref{mei-ONT}, \ref{mei-ONTprim}, def. \ref{mei-def5})} & \tag{ONT''}\label{mei-ONTbis}
%\end{flalign}
\begin{flalign}
& \parbox[t]{.87\linewidth}{Jednia jest superpredykatywnie każdą własnością.\hfill (\ref{mei-ONT}, \ref{mei-ONTprim}, def. \ref{mei-def5})} & \tag{ONT''}\label{mei-ONTbis}
\end{flalign}
%ONT'' Jednia jest superpredykatywnie każdą własnością. (ONT, ONT', def 5)

Zdaniem Meixnera\index[names]{Meixner, Uwe} powyższa struktura stanowi logiczną ramę dla mistycyzmu inspirowanego neoplatonizmem. W~jej obrębie można uznać, że Jednia jest superpredykatywnie bytem, boskością, dobrem, a~jednocześnie niesprzecznie uznać, że ściśle predykatywnie nie posiada żadnej z~tych własności. Co więcej, Jednia jest jedynym obiektem z~klasy przedmioto-własności, który spełnia zasady \ref{mei-ONT}, \ref{mei-ONTprim} oraz \ref{mei-ONTbis}.

\section{Dyskusja}

Algebra zbioru potęgowego, którą Meixner\index[names]{Meixner, Uwe} proponuje jako centralne narządzie rekonstrukcji teologii negatywnej rozumianej jako spuścizna po koncepcjach Platona\index[names]{Platon} i~filozofii neoplatońskiej, jest interesującą i~atrakcyjną strukturą. Obecne w~niej częściowe porządki rzeczywiście wydają się -- przynajmniej na pierwszy rzut oka -- przypominać hierarchiczną ontologię Plotyna\index[names]{Plotyn}. Propozycję tę z~pewnością można uznać za nieszablonową i~wartą uwagi, a~metalogiczne związki algebry zbioru potęgowego z~mereologią oraz zastosowaniami tejże w~ontologii formalnej sprawiają jedynie, że pomysł Meixnera\index[names]{Meixner, Uwe} jest jeszcze ciekawszy. Odnosząc się jednak bezpośrednio do jego propozycji trudno oprzeć się wrażeniu, że struktura algebraiczna mająca rekonstruować apofatyczne i~neoplatońskie rozważania została przedstawiona w sposób co najmniej niekompletny. Spróbujmy na samym początku nadrobić te braki. Za taką algebrę będziemy uważać piątkę uporządkowaną $\mathfrak{A} = \langle \mathcal{P}(\textsc{g}), \cup, \cap, -, \varnothing, \textsc{g}\rangle$, w~której $\mathcal{P}(\textsc{g})$ jest zbiorem (potęgowym zbioru \textsc{g}), $\cup$ oraz $\cap$ są pewnymi dwuargumentowymi działaniami $\mathcal{P}(\textsc{g}) \times \mathcal{P}(\textsc{g}) \to \mathcal{P}(\textsc{g})$ a~$-$ jednoargumentowym działaniem $\mathcal{P}(\textsc{g}) \to \mathcal{P}(\textsc{g})$, natomiast zbiory $\varnothing$ oraz \textsc{g}~są pewnymi wyróżnionymi elementami $\mathcal{P}(\textsc{g})$, a~która dla wszystkich $\textsc{x,y,z}\in\mathcal{P}(\textsc{g})$ spełnia następujące aksjomaty\footnote{Aksjomatykę algebry Boole'a\index[names]{Boole, George} podaję za A.~Pietruszczak, \textit{Pieces of mereology}, dz. cyt., s. 230 (poza oczywistą pomyłką drukarską -- Pietruszczak w A\textsubscript{3} podaje dwukrotnie tę samą formułę).}:
\begin{align}
& \textsc{x} \cup \textsc{y} = \textsc{y} \cup \textsc{x}, & \textsc{x} \cap \textsc{y} = \textsc{y} \cap \textsc{x}, \tag{A\textsubscript{1}}\label{meiA1} \\
& \textsc{x} \cup (\textsc{y} \cup \textsc{z}) = (\textsc{x} \cup \textsc{y}) \cup \textsc{z}, & \textsc{x} \cap (\textsc{y} \cap \textsc{z}) = (\textsc{x} \cap \textsc{y}) \cap \textsc{z}, \tag{A\textsubscript{2}}\label{meiA2} \\
& \textsc{x} \cup (\textsc{y} \cap \textsc{z}) = (\textsc{x} \cup \textsc{y}) \cap (\textsc{x} \cup \textsc{z}), & \textsc{x} \cap (\textsc{y} \cup \textsc{z}) = (\textsc{x} \cap \textsc{y}) \cup (\textsc{x} \cap \textsc{z}), \tag{A\textsubscript{3}}\label{meiA3} \\
& \textsc{x} \cup \varnothing = \textsc{x}, & \textsc{x} \cap \textsc{g} = \textsc{x}, \tag{A\textsubscript{4}}\label{meiA4} \\
& \textsc{x} \cup -\textsc{x} = \textsc{g}. & \textsc{x} \cap -\textsc{x} = \varnothing. \tag{A\textsubscript{5}}\label{meiA5}
\end{align}

Działanie $\cup$ jest zatem operacją sumy a~$\cap$ operacją iloczynu zbiorów.

Jeśli $\textsc{g} \neq \varnothing$, algebrę $\mathfrak{A}$~nazwiemy niezdegenerowaną.

Między elementami $\mathcal{P}(\textsc{g})$ możemy zdefiniować relację inkluzji (zawierania).
\begin{defin}
$(\textsc{x} \subseteq \textsc{y}) \equiv (\textsc{x} \cup \textsc{y} = \textsc{y}) \equiv (\textsc{x} \cap \textsc{y} = \textsc{x}).$
\end{defin}
%Definicja X~subset Y~equiv X~sum Y~= Y~equiv X~iloczyn Y~= X.
%
\noindent Relacja ta tworzy częściowe porządki na zbiorze $\mathcal{P}(\textsc{g})$, co znaczy, że cechuje ją:
\begin{flalign}
& \forall_{\textsc{x}\in\mathcal{P}(\textsc{g})}\ \textsc{x} \subseteq \textsc{x}, & \tag{zwrotność}\label{mei-zw} \\
& \forall_{\textsc{x,y,z}\in\mathcal{P}(\textsc{g})}\ \textsc{x} \subseteq \textsc{y} \land \textsc{y} \subseteq \textsc{z} \to \textsc{x} \subseteq \textsc{z}, &  \tag{przechodniość}\label{mei-przech} \\
& \forall_{\textsc{x,y}\in\mathcal{P}(\textsc{g})}\ \textsc{x} \subseteq \textsc{y} \land \textsc{y} \subseteq \textsc{x} \to \textsc{x} = \textsc{y}. &  \tag{słaboantysymetryczność}\label{mei-santsym}
\end{flalign}
%(zwrotność)
%
%(przechodniość)
%
%(słaboantysymetryczność)

Ponadto \textsc{g}~oraz $\varnothing$ stanowią odpowiednio największy i~najmniejszy element porządku w~$\mathcal{P}(\textsc{g})$.
\begin{flalign}
& \forall_{\textsc{x}\in\mathcal{P}(\textsc{g})}\ \textsc{x} \subseteq \textsc{g}, & \label{mei-najw} \\
& \forall_{\textsc{x}\in\mathcal{P}(\textsc{g})}\ \varnothing \subseteq \textsc{x}, & \label{mei-najm}
\end{flalign}
%forall X~in mathcal P~(G) X~subset G,
%
%forall X~in mathcal P~(G) emptyset subset X.


\begin{figure}[H]
\begin{center}
 \begin{tikzpicture}[node distance=2cm]

    \node (1) {\{\textsc{x,y,z}\}};
    \node (dub2) [below=of 1] {\{\textsc{x,z}\}};
    \node (dub3) [left=of dub2] {\{\textsc{x,y}\}};
    \node (dub1) [right=of dub2] {\{\textsc{y,z}\}};
    \node (sing3) [below=of dub3] {\{\textsc{x}\}};
    \node (sing2) [below=of dub2] {\{\textsc{y}\}};
    \node (sing1) [below=of dub1] {\{\textsc{z}\}};
    \node (0) [below=of sing2] {$\varnothing$};


    \path[-latex] (0) edge (sing1) edge (sing2) edge (sing3);
    \path[-latex] (sing1) edge (dub1) edge (dub2);
    \path[-latex] (sing2) edge (dub1) edge (dub3);
    \path[-latex] (sing3) edge (dub2) edge (dub3);
    \path[-latex] (dub1) edge (1);
    \path[-latex] (dub2) edge (1);
    \path[-latex] (dub3) edge (1);

\end{tikzpicture}

\caption[Krata zbioru potęgowego $\mathcal{P}(\{\textsc{x,y,z}\})$]{Elementy zbioru potęgowego $\mathcal{P}(\{\textsc{x,y,z}\})$ uporządkowane relacją inkluzji.}\label{mai-poset-rys}
\end{center}
\end{figure}
%Rys. Krata zbioru potęgowego $\mathcal{P}(\{\textsc{x,y,z}\})$ uporządkowana relacją inkluzji.
Aksjomat \eqref{meiA5} definiuje działanie $-$ jako operację dopełnienia zbioru\footnote{Warto zauważyć, że zachodzi także
\begingroup
\setlength{\abovedisplayskip}{0pt}
\setlength{\belowdisplayskip}{0pt}
\setlength{\abovedisplayshortskip}{0pt}
\setlength{\belowdisplayshortskip}{0pt}
\begin{flalign}
& \forall_{\textsc{x}\in\mathcal{P}(\textsc{g})}\ - - \textsc{x} = \textsc{x} \text{ oraz} & \\
& \forall_{\textsc{x,y}\in\mathcal{P}(\textsc{g})}\ \textsc{x} \subseteq \textsc{y} \equiv -\textsc{y} \subseteq -\textsc{x}. &
\end{flalign}%
\endgroup%
\vspace*{-.25cm}
}.
%X~subset Y~equiv -X subset -Y,
%
%{}-(-X) = X.

Możemy teraz, zgodnie z~intencją Meixnera\index[names]{Meixner, Uwe}, skojarzyć zbiór $\mathcal{P}(\textsc{g})$ z~klasą ,,przedmioto-własności'', element największy \textsc{g}~z Jednią\footnote{Być może jakąś inspiracją dla Meixnera\index[names]{Meixner, Uwe} był fakt, że element maksymalny w~algebrze Boole'a\index[names]{Boole, George} często oznacza się za pomocą cyfry ,,1'' (a minimalny za pomocą ,,0'').}, element najmniejszy $\varnothing$~z~bytem, negację (przedmioto-)własności z~dopełnieniem reprezentującego ją zbioru, relację ,,jest częścią'' z~relacją inkluzji $\subseteq$, a także podać trzy definicje ,,jest'' predykatywnego wykorzystujące tę relację.
\begin{defin}
$\textsc{x}R_p\textsc{y} \equiv_{\text{def}}
%\forall_{\textsc{x,y}\in\mathcal{P}(\textsc{g})}\ 
\textsc{y} \subseteq \textsc{x}$.\\
,,$\textsc{x}R_p\textsc{y}$'' czytamy jako ,,\textsc{x} jest predykatywnie \textsc{y}''.
\end{defin}
\begin{defin}
$\textsc{x}R_n\textsc{y} \equiv_{\text{def}}
%\forall_{\textsc{x,y}\in\mathcal{P}(\textsc{g})}\ 
\textsc{y} \subseteq \textsc{x} \land -\textsc{y} \nsubseteq \textsc{x}$.\\
,,$\textsc{x}R_n\textsc{y}$'' czytamy jako ,,\textsc{x} jest ściśle predykatywnie \textsc{y}''.
\end{defin}
\begin{defin}
$\textsc{x}R_s\textsc{y} \equiv_{\text{def}}
%\forall_{\textsc{x,y}\in\mathcal{P}(\textsc{g})}\ 
\neg \textsc{x}R_n\textsc{y} \land \textsc{x}R_p\textsc{y}$.\\
,,$\textsc{x}R_s\textsc{y}$'' czytamy jako ,,\textsc{x} jest superpredykatywnie \textsc{y}''.
\end{defin}%
%Def X~Rp Y~:= forall X,Y in P(G) Y~subset X. ,,X Rp Y'' czytamy jako ,,X jest predykatywnie Y''.
%
%Def X~Rn Y~:= forall X,Y in P(G) Y~subset X~land -Y notsubset X. ,,X Rn Y'' czytamy jako ,,X jest ściśle predykatywnie Y''.
%
%Def X~Rs Y~:= forall X,Y in P(G) neg X~Rn Y~land X~Rp Y. ,,X Rs Y'' czytamy jako ,,X jest super predykatywnie Y''.
%
\noindent Oczywiście, w~powyżej przedstawionych terminach można sformułować odpowiedniki zasad \ref{mei-ONT}, \ref{mei-ONTprim} oraz \ref{mei-ONTbis}.

Wiele jednak wskazuje na to, że proponowanie takiej struktury do formalnej rekonstrukcji mistycyzmu inspirowanego neoplatonizmem -- przynajmniej w~sposób, w~jaki robi to Meixner\index[names]{Meixner, Uwe} -- pozbawione jest jakiegoś przekonywającego teologicznego i~filozoficznego uzasadnienia, a~logicznie prowadzi do pewnej wersji paradoksu samozwrotności.

Propozycja Meixnera\index[names]{Meixner, Uwe} wydaje się nieadekwatną teorią z~punktu widzenia samej teologii negatywnej, a~przynajmniej -- wbrew przekonywaniom Meixnera\index[names]{Meixner, Uwe} -- można uważać, że jest formalną idealizacją\footnote{Zob. P. Urbańczyk, ``\textit{Internal'' Problems of Normative Theories of Thinking and Reasoning}, ,,Zagadnienia Filozoficzne w~Nauce'', nr 60 (2016), ss.~43-44.} trawestującą teologiczne dystynkcje doktryny Areopagity\index[names]{Pseudo-Dionizy Areopagita}. Na przykład, w~\textit{Teologii mistycznej}, Pseudo-Dionizy\index[names]{Pseudo-Dionizy Areopagita} twierdzi, że ,,powinniśmy zakładać i~przypisywać jej [przyczynie wszystkich bytów, a~zatem Bogu, Jedni -- PU] wszystkie twierdzenia (\textgreek{kataf'askein j'eseic}), które czynimy w~odniesieniu do bytów''\footnote{Pseudo-Dionizy Areopagita, \textit{Teologia mistyczna}: I, 2, tłum. własne. Oczywiście, Dionizy\index[names]{Pseudo-Dionizy Areopagita} natychmiast dodaje, że powinniśmy je także zanegować.}. Problem polega na tym, że na gruncie interpretacji Meixnera\index[names]{Meixner, Uwe} nie możemy zrobić niczego podobnego. W~modelu zbudowanym na algebrze zbioru potęgowego byt jest najmniejszym elementem porządku i~nie ma żadnej własności, która byłaby częścią tej własności ($\neg \exists_Q\ Q \subseteq \varnothing \land Q \neq \varnothing$). Nie możemy przypisać bytowi żadnej własności poza nim samym ($\varnothing \subseteq \varnothing$). Oznacza to, że jesteśmy w~stanie wyartykułować jedynie tautologiczne stwierdzenie, że byt jest (predykatywnie) bytem ($\varnothing R_p \varnothing$). Innych własności byt nie posiada.

,,Teologiczne'' niezgrabności tej rekonstrukcji można mnożyć na wielu różnych płaszczyznach. W~ramach kolejnego przykładu zauważmy, że w~obrębie modelu Meixnera\index[names]{Meixner, Uwe} cokolwiek nie powiedzielibyśmy o~czymkolwiek, orzeklibyśmy o~tym czymś ,,część Boga''. Obserwacja ta pozwala oskarżyć teorię Meixnera\index[names]{Meixner, Uwe} o~popadanie w~jakąś formę panteizmu, co być może -- dzięki koncepcji emanacji -- zgadzałoby się z~przynajmniej pewnymi wersjami doktryny neoplatońskiej, jednakże byłoby dalekie od większości ustaleń teologów apofatycznych, nie tylko tych wywodzących się z~tradycji chrześcijańskiej.

Inną kłopotliwą cechą interpretacji Meixnera\index[names]{Meixner, Uwe} jest to, że utożsamia ona Boga z~własnością (zagadnienie to nie powinno być mylone z~problemem niejednoznaczności terminu ,,Bóg'' w~sensie jego kategorii językowej). Co więcej, w~tak skonstruowanym systemie jest to bardzo szczególny rodzaj własności -- intensjonalnie zawiera w~sobie wszystkie inne własności, co przeczy neoplatońskiej (i apofatycznej) tezie, że Jednia (lub Bóg) jest bytem doskonałym i~całkowicie prostym. W~następstwie czego model Meixnera\index[names]{Meixner, Uwe} przeczy także tym (doktrynalnie istotnym) tezom teologii negatywnej, zgodnie z~którymi Bóg jest ,,poza'' i~,,ponad'' sferą własności. Z~drugiej strony, konsekwencją przyjętego modelu jest także stwierdzenie, że Bóg nie może być nieskończony. Izomorfizm między algebrą Boole'a\index[names]{Boole, George} a~algebrą zbioru potęgowego dotyczy wyłącznie zbiorów potęgowych zbiorów skończonych\footnote{Choć ma pewną wersję dla zbiorów nieskończonych, ale wtedy mówi się już o~podzbiorach skończonych i~podalgebrach. Por. M.H. Stone, \textit{The Theory of Representation for Boolean Algebras}, dz. cyt.}.

Ta ostatnia uwaga pozwala także wskazać pewną ,,filozoficzną'' niezgrabność rekonstrukcji Meixnera\index[names]{Meixner, Uwe}. Mianowicie, w~obrębie tak zdefiniowanej struktury zmuszeni jesteśmy przyjąć, że istnieje zawsze wyłącznie $2^{n-1}-2$ własności (nie licząc największego i~najmniejszego elementu porządku) oraz taka sama liczba ich negacji. Nie jest do końca jasne, dlaczego liczba modelowanych własności powinna zostać w~taki sposób ograniczona i~nie mogłaby być dowolna.

Ponadto, przyjmując ontologię formalną Meixnera\index[names]{Meixner, Uwe} należałoby podać dobrze uzasadnione filozoficznie powody, dla których porządek między własnościami nie mógłby mieć więcej niż jeden element minimalny lub więcej niż jeden element maksymalny. Inaczej mówiąc, wydaje się, że nie ma dostatecznie uzasadnionych argumentów, by uznać, że porządek wyznaczony relacją ,,jest częścią'' na elementach zbioru własności miałby posiadać element największy lub najmniejszy. Nietrudno wyobrazić sobie kratę własności uporządkowaną taką relacją, ale pozbawioną absolutnych ekstremów. W~tym przypadku porządek między własnościami pozostałby częściowy, ale twierdzenia \eqref{mei7} i \eqref{mei8} nie zachodziłyby. Tak czy owak, każdy inny rodzaj porządku nałożonego na zbiór własności lub brak absolutnych ekstremów porządku powoduje, że model Meixnera\index[names]{Meixner, Uwe} staje się nieadekwatny i~chybiony.

Podobnie, jak miało to miejsce z~zastrzeżeniami o~charakterze teologicznym, także filozoficzne niezgrabności tej rekonstrukcji można wskazywać na wielu różnych płaszczyznach. Na przykład wygląda na to, że Meixner\index[names]{Meixner, Uwe} w~swoim modelu całkowicie zignorował istnienie predykatów o~argumentowości większej niż jeden, a~przynajmniej nie jest jasne, jakie miejsce zajmowałyby takie predykaty w~formalnej strukturze mającej w~jego rozumieniu oddawać neoplatońską hierarchię bytów (czy też, jak je reinterpretuje Meixner\index[names]{Meixner, Uwe}, ,,przedmioto-własności''). Język naturalny nie jest pozbawiony predykatów wieloargumentowych. Oczywiście im wyższa argumentowość, tym trudniej odnaleźć jakiegoś rzeczywistego reprezentanta, ale zasadniczo nie ma większych trudności, by wskazywać naturalno-językowe predykaty na przykład dwu- czy trójargumentowe. Wydaje się, że nie da się ich wpasować w~hierarchię własności Meixnera\index[names]{Meixner, Uwe}, a~przynajmniej trudno zidentyfikować obiekt, któremu własności wyrażone predykatami wieloargumentowymi odpowiadałyby w~tak skonstruowanym modelu\footnote{Najpewniej byłyby to wieloargumentowe relacje $R_n(\textsc{x}_1, \textsc{x}_2,\ldots, \textsc{x}_n)$, ale ich definicja w~terminach inkluzji między elementami $\mathcal{P}(\textsc{g})$ wydaje się problematyczna.}.

Najwięcej zastrzeżeń budzi jednak sprowadzenie przedmiotów i~własności do jednej kategorii, zwłaszcza w~połączeniu z~analizą ,,jest'' predykatywnego. Intensjonalna teoria własności w~wydaniu zaprezentowanym przez Meixnera\index[names]{Meixner, Uwe} wydaje się być u~swoich podstaw teorią nieadekwatną w~rzeczywistych przypadkach orzekania. Można zgodzić się jeszcze, że istnieją pewne relacje część-całość między samymi własnościami, ale z~pewnością akceptacja zachodzenia takiej relacji między własnościami a~indywiduami musi się spotkać z~dużym oporem. Upieranie się przy tym, że wyrażenia typu ,,Aleksandra jest piękna'' w~istocie oznaczają ,,Piękno jest częścią Aleksandry'', wydaje się zdecydowanie dziwne i~karkołomne, nawet przy ograniczeniu takich sądów do wyrażeń o~charakterze metaforycznym. Co więcej, problem z~taką analizą wcale nie jest mniejszy, gdy przejdziemy do wyrażeń wyższego rzędu i~zaczniemy orzekać o~własnościach. Wyrażenie ,,Czerwony jest częścią koloru'' niełatwo jest sensownie zinterpretować, nawet korzystając z~mereologicznego sensu relacji ,,jest częścią'', z~jakim sympatyzuje Meixner\index[names]{Meixner, Uwe}. Nie wspominając już, że -- z~powodu przechodniości relacji ,,jest częścią'' \eqref{mei1} -- ze zdań ,,Lewy policzek Aleksandry jest czerwony'' oraz ,,Czerwony jest kolorem'' wynika natychmiast, że ,,Lewy policzek Aleksandry jest kolorem'', przynajmniej w~obrębie ontologii wyznaczonej definicją \ref{mei-def3}.

Innym związanym z~decyzją o~umieszczeniu indywiduów i~własności w~jednej klasie problemem ,,filozoficznym'' jest kwestia negacji przedmiotów. W~formalnej interpretacji Meixnera\index[names]{Meixner, Uwe} klasa przedmioto-własności uporządkowana jest relacją ,,jest częścią'' i~tworzy strukturę, której modelem jest algebra zbioru potęgowego. Oznacza to, że wspomnianej relacji odpowiada relacja inkluzji ($\subseteq$), która zachodzi między elementami $\mathcal{P}(\textsc{g})$, a~negacja własności charakteryzowana jest operacją dopełnienia odpowiadającego jej zbioru w~sposób, o~jakim mówi \eqref{meiA5}. W~takiej strukturze każdy element, łącznie z~elementem największym i~najmniejszym, posiada swoje dopełnienie. Jak jednak interpretować negację indywiduów? Nawet odchodząc od neoplatonizmu, który stanowił punkt wyjścia rozważań Meixnera\index[names]{Meixner, Uwe}, trudno o~zadawalającą ontologiczną teorię uzasadniającą obecność tego typu obiektów w~modelu.

Prawdopodobnie największym zarzutem przeciwko tej decyzji jest spostrzeżenie, że potencjalnie prowadzi ona do takich samych paradoksów samozwrotności, z~jakimi mieliśmy do czynienia w~dwóch pozostałych częściach pracy. Do tej pory nie było mowy o~tym, jakie elementy mogą składać się na zbiory, które w~teorii Meixnera\index[names]{Meixner, Uwe} odpowiadają przedmioto-własnościom. Zasadniczo model ten mógłby być rozważany w~oderwaniu od ustaleń w~tym zakresie. Jednakże umieszczenie własności i~przedmiotów w~obrębie jednej kategorii pozwala sądzić, że elementami takich zbiorów mogą być inne zbiory. Jeśli się na to zgodzimy, załóżmy, że istnieje jakaś złożona własność (lub przedmiot), której elementami są wszystkie inne własności\footnote{Innym podejściem do ujawnienia samozwrotności i~paradoksalnego charakteru teorii Meixnera\index[names]{Meixner, Uwe} jest wykazanie, że nie rozróżnia on między byciem elementem zbioru a~podzbiorem. Podejrzenie to nie jest bezpodstawne, bowiem za relacją ,,jest częścią'' ($\sqsubseteq$) w~obrębie samej mereologii, na którą powołuje się Meixner\index[names]{Meixner, Uwe} uzasadniając swoją analizę ,,jest'' predykatywnego, stoją obie te intuicje. Por. R. Gruszczyński, R. Pietruszczak, \textit{How to define a~mereological (collective) set}, ,,Logic and Logical Philosophy'', vol. 19 (2010), nr 4, ss.~309-314.} i~skojarzmy ją ze zbiorem $\mathcal{P}(\textsc{g})$. Zgodnie z~\eqref{mei-najw} musimy uznać, że
\begin{flalign}
& \mathcal{P}(\textsc{g}) \subseteq \textsc{g}. & \label{mei-par-prod}
\end{flalign}
Od takiej konstatacji wiedzie już więcej niż jedna bezpośrednia droga do wykazania sprzeczności. Zauważmy na przykład, że w~zdefiniowanej powyżej strukturze \linebreak
\mbox{$\mathfrak{A} = \langle \mathcal{P}(\textsc{g}), \cup, \cap, -, \varnothing, \textsc{g}\rangle$}, \textsc{g}~jest zbiorem o~największej mocy i prawdziwa jest ogólna własność $\forall_{\textsc{x,y}\in\mathcal{P}(\textsc{g})}\ \textsc{x} \subseteq \textsc{y} \to \left|\textsc{x}\right| \leq \left|\textsc{y}\right|$.
Z~twierdzenia Cantora\index[names]{Cantor, Georg} wiemy, że $\left|\mathcal{P}(\textsc{g})\right| > \left|\textsc{g}\right|$, co właściwie od razu daje nam parę zdań sprzecznych. Jednakże struktura dowodu tego twierdzenia ujawnia samozwrotny charakter rozważań Meixnera\index[names]{Meixner, Uwe} i~dobrze byłoby odtworzyć to rozumowanie w~obrębie tego modelu. Zatem skoro moc zbioru \textsc{g}~ma być największa spośród wszystkich zbiorów w~klasie własności, a~dopuszczamy istnienie własności wyrażonej zbiorem $\mathcal{P}(\textsc{g})$, o~którym twierdzenie Cantora\index[names]{Cantor, Georg} mówi, że jego moc jest większa od mocy \textsc{g}, załóżmy roboczo, że $|\mathcal{P}(\textsc{g})| = |\textsc{g}|$. Jeśli tak, to istnieje bijekcja $f: \textsc{g} \to \mathcal{P}(\textsc{g})$. Rozważmy zbiór $\textsc{r} = \{\textsc{x}: \textsc{x} \notin f(\textsc{x})\}$. Z~$\textsc{r} \subseteq \textsc{g}$ wiemy, że $\textsc{r} \in \mathcal{P}(\textsc{g})$. Skoro $f$~jest surjekcją, istnieje takie $\textsc{s} \in \textsc{g}$, że $f(\textsc{s}) = \textsc{r}$, z~czego wnioskujemy, że $\textsc{s} \in \textsc{r}$ wtedy i~tylko wtedy, gdy $\textsc{s} \notin \textsc{r}$\footnote{Alternatywnie, antynomie w~stylu antynomii Russella\index[names]{Russell, Bertrand} można produkować korzystając z~obserwacji wynikającej z \eqref{mei-par-prod}, że $\textsc{g} \in \mathcal{P}(\textsc{g}) \subseteq \textsc{g}$.}.

Fakt, że akurat ten fragment propozycji Meixnera\index[names]{Meixner, Uwe} potencjalnie prowadzi do samozwrotności, bez wątpienia można uznać za interesującą obserwację. Jak przy okazji argumentowania za pewnym wyborem kategorii językowej terminu ,,Bóg'' zauważył Walter Terence Stace\index[names]{Stace, Terence S.}, to właśnie zacieranie rozróżnienia między indywiduami a~własnościami związane jest z~niepojmowalnością i~niewysławialnością, które -- jak przekonywałem w~poprzednich częściach pracy -- zaangażowane są w semantyczne i epistemiczne paradoksy teologii apofatycznej.

\begin{quote}
Opozycja podmiot-przedmiot wynika z~samej natury intelektu. Ale doświadczenie mistyczne jest ponad tym rozróżnieniem i~przez to intelekt nie jest w~stanie go zrozumieć. Z~tego powodu jest ono niepojmowalne i~niewyrażalne\footnote{W.T. Stace, \textit{Time and Eternity: An Essay in the Philosophy of Religion}, Princeton University Press, Princeton 1952, s.~40 -- cyt. za W.P. Alston, \textit{Ineffability}, ,,The Philosophical Review'', vol 65, nr 4 (1956), s.~518.}.
\end{quote}

Nawet jeśli uznamy, że powyższe wnioskowanie oparte jest o~fałszywą przesłankę, należy zauważyć, że w~teorii Meixnera\index[names]{Meixner, Uwe} o~Bogu możemy orzec dowolne ,,predykatywne'' sprzeczności. Zgodnie z~\eqref{mei11}, dla dowolnej własności Bóg jednocześnie posiada (predykatywnie) tę własność, jak i~jej negację. W~szczególności, w~ramach teorii Meixnera\index[names]{Meixner, Uwe} można stwierdzić, że Bóg jest (predykatywnie) jednocześnie Bogiem oraz negacją Boga. Mając na względzie fakt, że zgodnie z~twierdzeniami o~reprezentacji Stone'a\index[names]{Stone, Marshall H.} algebra zbioru potęgowego, na której Meixner\index[names]{Meixner, Uwe} oparł swój model, jest algebrą Boole'a\index[names]{Boole, George}, należałoby podejrzewać, że posiadająca taką własność teoria musi okazać się niespójna.

Trudno także nie oskarżyć teorii Meixnera\index[names]{Meixner, Uwe} o~pewną oczywistą niekonsekwencję. Deklaruje on, że w wyniku jego analiz  będzie można niesprzecznie utrzymywać dwie platońskie\index[names]{Platon} tezy wyabstrahowane z~\textit{Parmenidesa}: \textit{O~Jedni nie można powiedzieć niczego} oraz \textit{O~jedni można powiedzieć cokolwiek}. Trudno byłoby przyznać, że zadanie to zostało wykonane z~powodzeniem. Meixner\index[names]{Meixner, Uwe} pokazał jedynie, że w~\textit{pewnym} sensie (dokładnie rzecz biorąc w~sensie definicji \ref{mei-def3}) Bogu można przypisać wszystkie własności, w~\textit{innym} zaś sensie (w sensie definicji \ref{mei-def4}) nie można mu przypisać żadnej własności. Inaczej mówiąc, Bóg jest predykatywnie każdą własnością, a~ściśle predykatywnie nie jest żadną własnością. Ciężko pozbyć się wrażenia, że jest to kolejne omawiane w~niniejszej pracy rozwiązanie wykorzystujące koncepcję ,,podwójnej prawdy'' oraz posiadające charakter rozwiązania \textit{ad hoc}.

Podobnym charakterem odznaczają się również inne pomysły Meixnera\index[names]{Meixner, Uwe}. Na przykład, za jedno z~tego typu rozwiązań \textit{ad hoc} uznać można wprowadzenie trzeciej definicji ,,jest'' (super)predykatywnego. W~modelu Meixnera\index[names]{Meixner, Uwe} taki rodzaj orzekania w~sposób trywialny i~oczywisty może być używany wyłącznie w~przypadku Boga (Jedni). Próbuje on jednak uzasadnić wprowadzenie tej definicji pewnymi określeniami Boga z~prefiksem ,,super-'' obecnymi u~Pseudo-Dionizego Areopagity\index[names]{Pseudo-Dionizy Areopagita}. Argumentuje, że dzięki takiej definicji możemy utrzymywać, że Bóg (i tylko Bóg) jest super-bytem\footnote{Meixner\index[names]{Meixner, Uwe} zdaje się zapominać, że łacińskie \textit{super} oznacza ,,ponad''.}, super-boskością czy super-dobrem, zapominając, że w~tym samym modelu należy przypisać Bogu także wszystkie negatywne (choć nie w~apofatycznym sensie) lub neutralne i~losowe własności i~uznać, że jest on na przykład super-pijakiem, super-lwem\footnote{Przypominam, że ,,przykłady'' określeń Boga, takie jak ,,pijak'' czy ,,lew'' pochodzą od samego Dionizego. Zob. Pseudo-Dionizy Areopagita, \textit{Imiona Boskie}: III, 1.} lub super-złem. W~ogóle, cała konstrukcja teorii Meixnera\index[names]{Meixner, Uwe} sprawia wrażenie cokolwiek ,,ociężałej''. Wydaje się, że chciałby on podać stosunkowo prosty model w~celu zrekonstruowania, wyjaśnienia i~rozwiązania problemów wielu doktryn -- zarówno filozoficznych, jak i~teologicznych. W~istocie efekt jego rekonstrukcji jest zgoła odwrotny -- odnosi się wrażenie, że zaproponowana przez niego struktura jest nader rozbudowana i~bogata, a wykorzystując ją do modelowania neoplatońskiej teorii i towarzyszącego jej mistycyzmu i mnożąc definicje często tych samych pojęć, w~rzeczywistości osiąga on wyniki trywialne, niewielkie lub pozorne.

Należy jednak docenić pomysł Meixnera\index[names]{Meixner, Uwe} za jego oryginalność oraz dodać, że w~świetle powyższych uwag ostatecznie zrezygnowanie z~mereologii na rzecz algebry (zbioru potęgowego) odbyło się ze szkodą dla proponowanego przez niego modelu filozofii neoplatońskiej oraz teologii negatywnej. Istnieje wiele przesłanek pozwalających przypuszczać, że przynajmniej część zarzutów podniesionych w~niniejszej dyskusji zostałoby odpartych (a przyjemniej straciłoby na sile), gdyby zostały wysuwane przeciwko teorii podpartej modelem będącym strukturą mereologiczną.


