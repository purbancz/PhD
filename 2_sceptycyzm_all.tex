
%\part{Aspekt epistemiczny}

\chapter{Teologia apofatyczna jako teologiczny sceptycyzm}\label{scep}
%\section{Wprowadzenie}


%\chapter{Wprowadzenie}


Kolejna interpretacja teologii negatywnej przenosi apofatyczne akcenty z~mówienia na myślenie i~z języka na wiedzę. Jej centralnym twierdzeniem jest teza, że Bóg jest całkowicie niepoznawalny, niepojęty i~niemożliwy do skonceptualizowania czy zrozumienia (w przeciwieństwie do takich apofatycznych określeń znanych z~rozważań z~zakresu teologii milczenia, jak ,,niewyrażalny'', ,,nieopisywalny'' czy ,,niewysławialny''). W~związku z~tym interpretację tę nazywa się ,,teorią Niepoznawalnego''. W~literaturze można ją spotkać również pod nazwą ,,agnostycznej teologii negatywnej''\footnote{Takiej nazwy używa Paweł Rojek. Zob. P. Rojek, \textit{Logika teologii negatywnej}, ,,Pressje'', 29 (2012), ss.~216-230.}, ale ja wolę określać ją mianem ,,teologicznego sceptycyzmu''\footnote{W~nawiązaniu do klasycznego sceptycyzmu, którego przedstawiciele podważali możliwość prawdziwego poznawania lub nabywania wiedzy w~ogóle.}, ponieważ kwestionuje ona możliwość zdobycia jakiejkolwiek wiedzy o~Bogu. Oczywiście, przez sceptycyzm nie rozumiem żadnej formy niewiary czy wiarołomstwa. Teologowie negatywni, którzy podkreślali niepoznawalność Boga, byli -- jak się wydaje -- głęboko wierzącymi myślicielami religijnymi.

Wysunięcie propozycji takiej interpretacji każe zadawać zaangażowane filozoficznie pytania o~relacje między językiem a~wiedzą i~poznaniem. Podobne pytania otwierają także ciekawe wątki rozważań z~zakresu kognitywistyki i~nauk o~poznaniu. Zasadniczo, w~zależności od przyjętego stanowiska filozoficznego, konsekwencją tych ustaleń będzie wniosek, zgodnie z~którym teoria Niepojmowalnego miałaby wynikać z~teorii Niewysławialnego lub odwrotnie lub też teorie te miałyby być równospójne a~ich zasady równoważne. Jednakże niezależnie od tych ustaleń nie wypada nie zgodzić się z~tezą głoszącą, że teoria Niepojmowalnego stanowi pełnoprawną, odrębną i~dojrzałą interpretację teologii apofatycznej.

%***

Także i~w przypadku tej interpretacji można wskazać przykłady podobnego sposobu myślenia wśród dwudziestowiecznych teologów i~filozofów religii nieutożsamianych wprost z~nurtem apofatycznym. Wątki mówiące o~tym, że Bóg przekracza wszystko, co możemy o~nim pomyśleć ponownie znajdziemy u~Rudolfa Oto, Karla Bartha czy Karla Rahnera i~innych wiodących współczesnych teologów. Wydaje się, że z~wymienionej trójki to u~Bartha są one najbardziej widoczne.

Barth w~monumentalnej \textit{Dogmatyce Kościelnej}\footnote{K. Barth, \textit{Church Dogmatics}, tom 2, cz. 1, tłum. T.F. Torrance, G.W. Bromiley, T\&T Clark, Edinburgh 1957, ss.~179-204.} przedstawia alternatywne rozumienie transcendencji, którą pojmuje już nie tylko jako ontologiczne oddzielenie i~rozróżnienie między stworzeniem a~Stwórcą, lecz także jako niezdolność ludzkich pojęć do uchwycenia istoty Boga, choć w~tym kontekście woli on mówić raczej o~,,tajemnicy'' albo ,,ukryciu'' Boga. Według Bartha człowiek siłą własnego rozumu nigdy nie utworzy właściwego pojęcia Boga -- nawet przez analogię do pojęć, które potrafi zrozumieć, takich jak ,,pan'', ,,stwórca'' czy ,,zbawca''. Konsekwencją zaprzeczenia możliwości poznania Boga za pomocą rozumu jest odrzucenie teologii naturalnej jako przedsięwzięcia pozbawionego jakichkolwiek szans na powodzenie\footnote{Zob. tamże.}. Z~punktu widzenia tej pracy interesujący jest fakt, że u~Bartha niepoznawalności Boga nie pociąga za sobą Jego niewysławialności. W~teologii Bartha nasze ludzkie pojęcia i~słowa mogą odnosić się do Boga, o~ile ugruntowane w~objawieniu\footnote{Oznacza to, że w~teologii Bartha możliwe jest jednak zdobycie \textit{jakiejś} wiedzy o~Bogu. Można Go poznać przez podobieństwo i~analogię, choć nie za pomocą naturalnych zdolności poznawczych -- dokonuje się to w~darmowym, niezasłużonym akcie łaski, jakim jest objawianie. Zob. K.E. Johnson, \textit{Divine Transcendence, Religious Pluralism and Barth's Doctrine of God}, ,,International Journal of Systematic Theology'', vol. 5 (2003), nr 2, ss.200-224.}.

%***

Przyjęcie, że zasadnicza teza teologii negatywnej głosi, że Bóg przekracza wszystko, co możemy o~nim pomyśleć, stanowi popularne podejście w~interpretacji pism tych samych autorów, o~których wspominaliśmy już w~poprzedniej części pracy: Grzegorza z~Nyssy, Pseudo-Dionizego Areopagity, Tomasza z~Akwinu, Mikołaja z~Kuzy, czy Mistrza Eckharta. Jednakże zwolennicy teologicznego sceptycyzmu odwołują się najchętniej do pism Rambama, czyli Mojżesza Majmonidesa.

Mimo wszystkich różnic, podobieństwa między teologią milczenia i~teorią Niepojmowalnego są wciąż znaczne. Największym z~nich jest wynikający z~samozwrotności paradoksalny charakter, który dzielą obie teorie. Paradoks Niepoznawalnego tworzymy w~ten sam sposób i~na takich samych zasadach, co paradoks Niewysłowionego.


\section{Niewysławialność vs niepoznawalność. Uwaga o~relacjach między językiem a~umysłem w~kontekście teologii negatywnej}\label{scep-werapo}

Na pierwszy rzut oka sceptycyzm teologiczny wydaje się tożsamy z~teologią milczenia (lub bardzo do niej zbliżony). Istnieją pewne powody, aby sądzić, że przynajmniej jedna z~tych teorii pociąga za sobą drugą. To, w~którą stronę przebiegnie taka implikacja, zależeć może od filozoficznych przekonań dotyczących związków i~zależności między językiem a~umysłem, mową a~wiedzą czy myśleniem a~sposobem i~możliwością formułowania i~wyrażania sądów. Niemniej, obie teorie są do siebie -- przynajmniej \textit{prima facie} -- znacząco podobne. Wniosek ten może pojawić się także w~wyniku lektury tekstów źródłowych teologii negatywnej -- rzeczywiście wydaje się, że dla wielu autorów apofatyzm można wyrazić tak samo skutecznie i~trafnie tezą o~niewyrażalności, jak i~tezą o~niepojmowalności. Granica między niemożnością powiedzenia czegokolwiek o~Bogu a~brakiem możliwości posiadania o~nim jakiejkolwiek wiedzy zaciera się choćby u~bohatera poprzedniej części pracy -- Pseudo-Dionizego Areopagity, ale i~u wielu innych myślicieli apofatycznych.

Twierdzę jednak, że obie teorie i~związane z~nimi interpretacje teologii negatywnej nie mogą być równoważne. Po pierwsze ze względu na ich konsekwencje. Podczas gdy teologia milczenia czyni dyskurs religijny bezsensownym, teologiczny sceptycyzm (przynajmniej hipotetycznie) nie niesie ze sobą takich konkluzji. Przyjmując teorię Niewysłowionego nie możemy sensownie powiedzieć czegokolwiek o~Bogu. Nie wolno nam ani potwierdzić, ani zaprzeczyć jakiemukolwiek przymiotowi Boga. W~ramach teorii Niepoznawalnego dyskurs religijny może mieć znaczenie, choć ograniczone -- możemy przynajmniej stwierdzić, czego o~nim nie wiemy. Druga uwaga jest bardziej ogólna i~dotyczy wprost różnic i~związków między wiedzą i~poznaniem a~językiem i~mową\footnote{Mam świadomość, że z~jednej strony wiedza, umysł, poznanie i~myślenie, z~drugiej język, mowa, mówienie i~generowanie mowy niekoniecznie muszą być uznane za terminy tożsame i~równoważne. Jednakże w~przybliżeniu, jakie wymagane jest do przeprowadzenia rozważań przedstawianych w~tym rozdziale, są na tyle bliskoznaczne, że będą często stosowane zamiennie (o ile kontekst ich użycia nie wskaże, że jest inaczej).}. Szczegółowe zbadanie tych związków i~różnic nie jest celem niniejszej pracy. Niemniej jednak wydaje się uzasadnionym, by twierdzić, że z~jednej strony możemy myśleć i~mówić o~rzeczach, których nie znamy, a~z drugiej strony możemy rozumieć więcej, niż jesteśmy w~stanie wyrazić słowami. Spróbujmy przyjrzeć się bliżej tym kwestiom.

W~poprzedniej części pracy przedstawialiśmy dwa podejścia do źródeł niewysławialności i~niepojmowalności Boga\footnote{Por. rozdz.~\ref{sil-int-nazw}.}. Zasadniczo większość teologów negatywnych i~badaczy zajmujących się apofatyczną doktryną uważa, że niemożliwość zrozumienia i~poznania Boga wynika z~jego transcendentnego charakteru i~jest jego istotną własnością, elementem jego natury. Istnieją jednak myśliciele przekonani, że powodem niepoznawalności Boga są ograniczenia ludzkiego umysłu i~niewystarczające zdolności poznawcze człowieka, które sprawiają, że nie może on posiąść o~Bogu żadnej autentycznej wiedzy. Można pokusić się o~sformułowanie dwóch wersji takiego stanowiska -- słabszej i~silniejszej. Według pierwszej z~nich ograniczenia ludzkiego umysłu i~języka mogłyby zostać przezwyciężone, natomiast druga głosiłaby tezę przeciwną -- co prawda niewysławialność/niepoznawalność Boga wynika z~ograniczeń ludzkiego języka/umysłu, ale nigdy nie będzie on na tyle doskonały, by wyrazić/poznać naturę Boga. Ta druga, silna wersja niewiele się różni (przynajmniej, gdy idzie o~logiczne i~teologiczne konsekwencje) od podejścia, zgodnie z~którym niepoznawalność Boga należy do jego natury.

By zilustrować tę pierwszą, wyobraźmy sobie, że w~niedalekiej przyszłości rodzi się religijny geniusz, który potrafi poznać i~pojąć w~zupełności nieskończone bogactwo natury Boga. Czy taki religijny geniusz będzie w~stanie należycie sformułować, określić i~opisać tę naturę? Jeśli tak, czy będzie mógł ją odpowiednio zakomunikować innym ludziom? W~końcu, czy taki komunikat miałby szansę, by u~swoich odbiorców wywołać poprawny obraz Boga w~taki sposób, by i~oni poznali tę naturę w~pełni?

W~innym hipotetycznym scenariuszu wyobraźmy sobie, że w~dalekiej przyszłości ludzkość na drodze rozwoju pokonała swoje ograniczenia i~wykształciła umiejętności poznawcze do tego stopnia, że każdy człowiek -- wskutek długotrwałej medytacji oraz czytania mistycznych ksiąg -- jest w~stanie poznać i~pojąć w~całości bezkresną naturę Boga. Czy ludzie, którzy w~medytacji poznali tę naturę, mogą rozmawiać o~Bogu w~sposób sensowny i~znaczący? Czy, gdy dwaj przedstawiciele takiego społeczeństwa przyszłości będą wypowiadać słowo ,,Bóg'', w~ich umysłach aktywizują się tożsame reprezentacje mentalne?


\subsection{Problem niekomunikowalności języka religijnego}\label{scep-nkom}

Powyższe eksperymenty myślowe mają stanowić ilustracje różnych problemów dotyczących zależności między językiem a~umysłem w~kontekście teologii negatywnej. Nie zamierzam udzielać definitywnych odpowiedzi na pytania, jakie się pojawiają w~ich obrębie, choć temat ten zostanie jeszcze podjęty. Tymczasem zwróćmy uwagę na jeszcze jedno zagadnienie, na które wskazują te ilustracje -- problem (nie)komunikowalności znaczenia w~dyskursie religijnym. Generalnie jest to problem z~zakresu pragmatyki i~trzeba przyznać, że -- nawet w~jej obrębie, bez kontekstu religijnego -- nie należy on do najchętniej podejmowanych zagadnień. Jednym z~autorów, którzy wprost próbowali przedstawić i~zbadać je w~formalny sposób, jest Józef Maria Bocheński. Poświęca on temu problemowi krótki paragraf w~\textit{Logice religii}\footnote{J.M. Bocheński, \textit{Logika religii}, tłum. S. Magala, Instytut wydawniczy PAX, Warszawa 1990, §12.}.

W~rozważaniach Bocheńskiego teoria niekomunikowalności dyskursu religijnego to teoria, na gruncie której uważa się, że wyrażenia języka religijnego niosą jakieś obiektywne znaczenie, ale nie da się tego znaczenia przekazać -- nie można go zakomunikować innym użytkownikom dyskursu religijnego. By wskazać jakieś inne, pozareligijne warunki, w~których taka sytuacja może mieć miejsce, Bocheński podaje przykład więźnia, który w~ramach zachowania wspomnień, rozmawia sam ze sobą w~języku, którego nie zna żaden ze współwięźniów znajdujących się z~nim w~celi.

Główna zasada takiej teorii w~rekonstrukcji Bocheńskiego zawiera dwa trójargumentowe predykaty skonstruowane w~obrębie pragmatycznego układu odniesienia: $U(x,t,\phi)$ oznaczający ,,$x$ używa terminu $t$~w znaczeniu $\phi$'' oraz $M(t, \phi, x)$ wyrażający funkcję zdaniową ,,termin $t$~znaczy $\phi$ dla osobnika $x$''. Zasada ta przedstawiona jest zdaniem:
\begin{flalign}
		& \forall x \forall t \big(U(x,t,\phi) \to \forall y (M(t, \phi, y) \to y=x)\big). &\label{scep-nkom-form}
\end{flalign}

W~radykalnej interpretacji teoria niekomunikowalności Bocheńskiego głosi, że gdy dwie osoby, $a$ i~$b$, posługują się językiem religijnym, $a$ nie rozumie wypowiedzi $b$ i~\textit{vice versa} -- $b$ nie wie, co $a$ ma na myśli. Inaczej mówiąc, dyskurs religijny jest pozbawiony znaczenia dla odbiorcy, choć jednocześnie całkowicie znaczący dla nadawcy komunikatu. Bocheński odrzuca tę teorię z~przyczyn empirycznych -- to, co wiemy o~zachowaniu wiernych temu przeczy. Wierni zdają się rozumieć wypowiedzi religijne i~podobnie na nie reagować. Jednakże za jakąś jej wersją mogą przemawiać pewne podejścia, zgodnie z~którymi wiedza o~Bogu jest wiedzą osobistą i~wiedzą ,,o osobie'' i~jako taka pozbawiona jest twierdzeniowego (propozycjonalnego) charakteru\footnote{Próby badania ,,wiedzy o~osobie'', czyli właściwie rozwinięcia alternatywnej epistemologii, odmiennej od tradycyjnych studiów nad wiedzą o~obiektach lub zdaniach, pojawiają się od jakiegoś czasu w~literaturze. Por. M.A. Benton, \textit{Epistemology Personalized}, ,,The Philosophical Quarterly'' 67 (2017), nr 269, ss.~813–834; w~kontekście wiedzy o~Bogu: M.A. Benton, \textit{God and Interpersonal Knowledge}, ,,Res Philosophica'', vol. 95 (2018), nr 3, ss.~421-447 oraz L.J. Keller, \textit{Divine ineffability and franciscan knowledge}, ,,Res Philosophica'', vol. 95 (2018), nr 3, ss.~347-370.}. Tak czy owak, Bocheński odrzuca teorię niekomunikowalności powołując się na obserwacje zachowań użytkowników języka religijnego. Tak się składa, że omawiany tu problem najpełniejsze odzwierciedlenie w~kontekście filozoficznym (a przynajmniej pozareligijnym) znajduje właśnie na gruncie lingwistyki behawioralnej, a~konkretnie tezy o~niezdeterminowaniu przekładu Willarda Van Ormana Quine'a\footnote{Zob. W.V.O. Quine, \textit{Three Indeterminacies}, [w:] \textit{Perspectives on Quine}, red. R. Barret, R. Gibson, Blackwell, Basil 1990, ss 1-16 oraz Tenże, \textit{Indeterminacy of Translation Again}, ,,Journal of Philosophy'', vol. 84 (2012), nr 1, 5-10.}.

Punktem wyjścia tezy o~niezdeterminowaniu przekładu jest kolejny eksperyment myślowy\footnote{Quine nazywa ten eksperyment ,,przekładem radykalnym''.}. W~jego ramach wyobraźmy sobie lingwistę, który rozpoczął badania terenowe, by jako pierwszy opracować przekład nieznanego do tej pory języka używanego przez ludzi żyjących przed jego przybyciem w~izolacji. Nazwijmy ten język językiem źródłowym. Dla języka źródłowego nie istnieje jeszcze żaden słownik czy podręcznik -- słowem, nie ma do niego żadnego dostępu, nawet zapośredniczonego w~którymś ze znanych do tej pory języków. Jedyny materiał, jaki może zebrać lingwista tworzący pierwszy podręcznik języka źródłowego, to wypowiedzi jego użytkowników oraz obserwacja towarzyszących tym wypowiedziom okoliczności i~zachowań. Mając takie dane lingwista będzie próbował wyróżnić zbiór zdań obserwacyjnych\footnote{Zdania obserwacyjne to istotny element koncepcji Quine'a. Zob. A. Kubić, ,,\textit{Niezdeterminowanie'' granic poznania w~znaturalizowanym relatywizmie pojęciowym W.V.O. Quine'a}, ,,Annales Universitatis Mariae Curie-Sklodowska, sectio I~-- Philosophia-Sociologia'', vol. 43 (209), nr 2, ss.~73-92}, czyli wypowiedzi typu ,,Pada deszcz'' albo ,,To jest królik''. Na tej samej zasadzie będzie próbował zidentyfikować klasę nazw i~kryjących się za nimi znaczeń. By zweryfikować swoje przypuszczenia nasz lingwista będzie używał tych wypowiedzi w~podobnych okolicznościach, w~których je usłyszał, i~badał reakcje słyszących jego wypowiedzi tubylców. Wszelkie potwierdzenia i~zaprzeczenia a~także spodziewane lub nieoczekiwane reakcje użytkowników języka źródłowego będą utwierdzały naszego badacza w~jego przypuszczeniach albo kazały mu przesunąć znaczenia lub zmienić kontekst testowanych wypowiedzi. Dzięki takiej procedurze będzie on mógł stopniowo poszerzać swój zakres znajomości języka źródłowego o~coraz to nowe, bardziej złożone rodzaje wyrażeń. W~końcu, jego tworzony w~terenie słownik i~podręcznik (zbiór zasad przekładu) przybiorą formę kompletnych opracowań a~on sam będzie mógł całkiem efektywnie komunikować się z~pierwotnymi użytkownikami języka źródłowego.

Quine argumentuje, że taki projekt jest zawsze skazany na niepowodzenie i, choć do znaczenia mamy dostęp jedynie przez obserwację zachowań, stworzony w~taki sposób przekład nigdy nie odda tubylczego obrazu świata czy też -- co dość istotne w~systemie Quine'a -- stojącej za wypowiedzią języka źródłowego ontologii. Załóżmy, że nasz badacz obserwował, jak tubylcy wypowiadają ,,gavagai'' na widok przebiegającego nieopodal królika. Seria podobnych obserwacji, a~także wielokrotne użycie tego wyrażenia w~różnych kontekstach, utwierdziła badacza w~przekonaniu, że wypowiedź ta znaczy po prostu ,,królik''. Tymczasem nie ma żadnej pewności, czy rzeczywistym odniesieniem tego wyrażenia nie jest ,,futrzak'', ,,potencjalna kolacja dla niewielkiej liczny osób'' czy ,,fragmenty przestrzeni wypełnione niedużym zwierzęciem''. Załóżmy dodatkowo, że przed publikacją pierwszego podręcznika języka źródłowego do ludzi nim się posługujących przybył drugi lingwista i~wykonał tę samą pracę korzystając z~tych samych procedur. Nie ma przeszkód, by sądzić, że przekłady dokonane za pomocą tych dwóch różnych zbiorów zasad, choć dadzą poprawne przewidywania dotyczące zachowań użytkowników języka, będą diametralnie różne (a potencjalnie niespójne) jeśli idzie o~ich znaczenie w~języku przekładu. Możemy także założyć, że tekst przełożony na język źródłowy za pomocą pierwszego zbioru zasad, a~potem przetłumaczony z~powrotem na podstawie drugiego podręcznika, będzie diametralnie różny od oryginału\footnote{Wydaje się, że by dojść do takiego wniosku, nie potrzeba \textit{przekładu radykalnego}, wystarczy skorzystać ze współczesnych narzędzi automatycznych służących do tłumaczeń z~rzeczywistych języków.}.

Powyższe rozważania pozwalają uznać argument Bocheńskiego przeciwko teorii niekomunikowalności za bezzasadny. Co prawda w~przypadku teorii niekomunikowalności wypowiedzi dyskursu religijnego formułowane są w~obrębie jednego języka, ale jej konstrukcja sprawia, że sytuacja między dwoma uczestnikami dyskursu religijnego jest tożsama z~tą, w~jakiej się znalazł lingwista z~eksperymentu myślowego Quine'a. Wskazuje na to treść głównej zasady ujętej w~postaci wyrażenia \eqref{scep-nkom-form} -- znaczenie znane jest tylko autorowi wypowiadanego komunikatu, odbiorca nie ma do niego dostępu. Na pełną analogię z~przekładem radykalnym Quine'a wskazuje również podany przez Bocheńskiego przykład pozareligijnego kontekstu niekomunikowalności. Zatem, skoro mamy do czynienia z~tym samym zjawiskiem i~zgodzimy się z~przekazem Quine'a, do odrzucenia teorii niekomunikowalności nie wystarczy wyłącznie obserwacja zachowań wiernych. Mimo iż behawior wiernych zdaje się sugerować, że rozumieją oni wypowiedzi dyskursu religijnego a~obserwacje ich zachowań prowadzą do wniosków, że odpowiednio reagują na poszczególne wypowiedzi, niekoniecznie musi to świadczyć o~tym, że mają oni pełne rozumienie zdań języka religijnego oraz właściwe pojęcia terminów tego języka.

Oczywiście, argument Bocheńskiego można bronić (a zarazem teorię niekomunikowalności można odrzucać) na wiele różnych sposobów\footnote{Jednym z~wektorów obrony może zostać chociażby przywołanie konsekwencji kolejnego eksperymentu myślowego Hilarego Putnama, zwanego ,,bliźniaczą Ziemią''. Eksperyment ten ma przekonywać, że znaczenia ,,nie są w~głowach'', pokazując przy okazji, że do momentu obserwacji podważającej dotychczasowe rozumienie zdań czy odniesienia terminów, terminy znaczą to, co nam się wydaje, że znaczą.% Por rozdz.~\ref{end}.
} podobnie, jak na wiele sposobów można argumentować za tezą o~niezdeterminowaniu przekładu w~kontekście dyskursu religijnego. Jednakże prowadzenie takiej dyskusji nie jest to celem niniejszego opracowania. Nic jednak nie stoi na przeszkodzie, by zauważyć, że problem (nie)komunikowalności znaczenia w~języku religijnym, pojawiający się okazji rozważań nad teologią apofatyczną, obnaża kolejne ciekawe związki między językiem a~umysłem. Zgodnie z~zapowiedzią, związkom tym przyjrzymy się dokładniej w~następnej sekcji.

\subsection{Języka a~poznanie}

Znaleźliśmy się w~miejscu, w~którym mamy do dyspozycji dwie interpretacje teologii negatywnej, dwie tworzone w~ich ramach klasy teorii i~dwie towarzyszące im apofatyczne tezy wyrażające transcendencję Boga -- tezę o~niewysławialności (NW) i~tezę o~niepojmowalności (NP). W~ramach podsumowania doczasowych rozważań wyobraźmy sobie cztery teologiczne postawy względem tych tez: 1) solidarnie odrzucenie obu naraz; przyjęcie jednej z~nich z~jednoczesnym odrzuceniem drugiej, czyli 2) akceptację tezy o~niewysławialności i~odrzucenie tezy o~niepojmowalności lub odwrotnie 3) odrzucenie tezy o~niewysławialności i~przyjęcie tezy o~niepojmowalności; 4) przyjęcie obu tez uznając, że Bóg jest niewysławialny i~niepojmowalny.

\begin{enumerate}[label = \arabic*), itemindent=6mm, labelwidth=4mm, labelsep=2mm, itemsep=1em, leftmargin=0mm]
\item $\neg \text{NW} \land \neg \text{NP}$

Jednoczesne odrzucenie tezy o~niewysławialności i~tezy o~niepojmowalności to, oczywiście, podejście redukujące. Sprowadziłoby nas do pozycji teologa katafatycznego, według którego człowiek jest w~stanie pozytywnie wypowiadać się jaki Bóg jest i~rozumieć to, co wypowiada.

\item $\text{NW} \land \neg \text{NP}$

Możemy utrzymywać tezę o~niewysławialności jednocześnie odrzucając tezę o~niepojmowalności. Oznaczałoby to, że uznajemy, że Bóg jest pojmowalny, choć nie możemy go wyrazić i~opisać. Stanowisko to można sprowadzić do teorii niekomunikowalności, którą rozważaliśmy w~sekcji powyżej\footnote{Zob. rozdz.~\ref{scep-nkom}.}.

\item $\neg \text{NW} \land \text{NP}$

Sytuacja, w~której odrzucamy tezę o~niewysławialności a~akceptujemy tezę o~niepojmowalności, także jest logicznie do pomyślenia. Przedstawicielem tego podejścia jest na przykład Karl Barth\footnote{K. Barth, \textit{Church Dogmatics}, tom 2, cz. 1, tłum. T.F. Torrance, G.W. Bromiley, T\&T Clark, Edinburgh 1957, ss.~179-204.}, według którego nie możemy posiadać właściwego pojęcia Boga, ale (dzięki objawieniu) możemy jednak o~nim mówić. Można także wskazać takie wątki u~teologów negatywnych (na przykład u~Mojżesza Majmonidesa, Tomasza z~Akwinu czy Mikołaja z~Kuzy), które wskazywałyby, że rozwijają oni teorię Niepoznawalnego pozbawioną językowego komponentu. Zatem podobną postawę możemy odnaleźć zarówno na gruncie teologii negatywnej, jak u~myślicieli niezaliczanych wprost do nurtu apofatycznego.

\item $\text{NW} \land \text{NP}$

Przyjęcie tezy o~niewysławialności i~tezy o~niepojmowalności naraz prowadzi nas do pełnego apofatyzmu. Takie stanowisko jest najczęściej reprezentowane wśród teologów tego nurtu. Wszystko wskazuje na to, że różnicy między niepojmowalnością a~niewysławialnością nie dostrzegał choćby protagonista poprzedniej części niniejszej pracy, Pseudo-Dionizy Areopagita\footnote{Zob. rozdz.~\ref{sil-dionizy}.}. Wydaje się ono także stać w~zgodzie z~tym, jak teoretycznie i~zdroworozsądkowo traktujemy związki między językiem a~poznaniem.
\end{enumerate}

Dodatkowo, możemy wyróżnić (hipotetycznie) możliwe trzy (cztery) podejścia do inferencyjnych powiązań między NW a~NP: równoważność, wynikanie i~niezależność.

\begin{enumerate}[label = \arabic*), itemindent=6mm, labelwidth=4mm, labelsep=2mm, itemsep=1em, leftmargin=0mm]
\item $\text{NW} \equiv \text{NP}$

Według pierwszego podejścia tezy te są równoważne\footnote{Oczywiście, możemy mówić tylko o~równoważności zdań. W~przypadku teorii mówi się raczej o~równospójności. Dla uproszczenia wywodu pozostaniemy na razie na poziomie tez.}, co oznacza, że przyjęcie jednej z~nich pociąga za sobie natychmiastowo do przyjęcie drugiej i~odwrotnie (a także odrzucenie jednej z~nich musi prowadzić natychmiastowo do odrzucenia drugiej i~odwrotnie). Prawdopodobnie jest to ciche założenie większości teologów negatywnych, dla których różnica między wiedzą a~językiem była transparentna (o~ile można wysunąć takie przypuszczenie bez popadania w~anachronizm).

\item $\text{NW} \rightarrow \text{NP}$

Według drugiego podejścia teza o~niewysławialności pociąga za sobą tezę o~niepojmowalności. (jednocześnie odrzucenie niepojmowalności Boga musi prowadzić do odrzucenia jego niewysławialności). Teologowie apofatyczni przyjmując to stanowisko będą uznawali pojmowanie Boga za bardziej pierwotny fenomen niż zdolności do jego opisu. Zarazem będą twierdzić, że skoro nie można wyrazić Boga, nie ma żadnych możliwości, by posiąść o~nim jakąkolwiek wiedzę.

\item $\text{NW} \leftarrow \text{NP}$

Według trzeciego podejścia teza o~niepojmowalności jest silniejsza i~pociąga za sobą tezę o~niewysławialności (jednocześnie założenie, że Bóg jest wysławialny, prowadzi do uznania, że można Go zrozumieć i~pojąć). Teologowie negatywni sympatyzujący z~tym stanowiskiem będą twierdzić, że to mówienie o~Bogu jest bardziej pierwotnym fenomenem niż posiadanie o~Nim jakiejkolwiek wiedzy. Będą oni jednocześnie twierdzić, że skoro nie można pojąć natury Boga, nie można o~nim także nic powiedzieć.

\item $\neg (\text{NW} \rightarrow \text{NP}) \land \neg (\text{NP} \rightarrow \text{NW})$

Intuicyjnie można sobie wyobrazić sytuację, w~której teza o~niepojmowalności i~teza o~niewysławialności przyjęte są na podstawie odrębnych zbiorów przesłanek w~obrębie dwóch niezależnych, niemających części wspólnych teorii i~nie są one powiązane inferencyjnie, tj. żadna z~nich nie jest konsekwencją drugiej. Jednakże jest to założenie logicznie nie do przyjęcia (zdanie umieszczone powyżej jest kontrtautologią i~-- przynajmniej na gruncie logiki klasycznej -- będzie zawsze fałszywe) oraz bardzo mało prawdopodobne zarówno z~punktu widzenia samej teologii, jaki i~nauk o~poznaniu. Z~tego powodu takie stanowisko nie jest tutaj w~ogóle brane pod uwagę\footnote{Warto zauważyć, że wyróżnione związki inferencyjne mogą wynikać logicznie wprost z~przedstawionych powyżej postaw względem tez o~niewysławialności i~niepojmowalności. Wydaje się jednak, że wynikanie logicznie nie musi oznaczać, że teolog przyjmujący daną postawę koniecznie utrzymuje loginie równoważną z~nią zależność między tymi dwoma tezami. Z~jednej strony ma to związek z~charakterystyką działania umysłu, którą normatywne teorie myślenia i~rozumowania nie potrafią poprawnie uchwycić. Por. P. Urbańczyk, ``\textit{Internal'' Problems of Normative Theories of Thinking and Reasoning}, ,,Zagadnienia Filozoficzne w~Nauce'', 2016, nr 60, ss.~35-52. Z~drugiej strony, wiąże się to z~faktem, że mamy tu do czynienia raczej z~całymi teoriami, niż z~pojedynczymi tezami. Sprowadzenie dwóch apofatycznych doktryn do krótkich, konkretnych tez jest idealizacją mającą na celu ułatwienia przeprowadzenia rozważań.}.
\end{enumerate}

Wybór stanowiska określającego zależności między tezą o~niewysławialności i~tezą o~niepojmowalności z~pewnością nie będzie się dokonywał w~obrębie samej teologii. Musi zależeć przede wszystkim od bardziej ogólnych, filozoficznych upodobań -- objęcia pozycji teoretycznej wyznaczającej relację między językiem a~myśleniem a~także, co istotne ze współczesnego punktu widzenia, wyboru zestawu danych eksperymentalnych faworyzujących interpretacje na korzyść danego rodzaju relacji. W~szczególności (i analogicznie do rozważanych powyżej zależności między niewysławialnością i~niepojmowalnością) można wyróżnić trzy klasy interpretacji relacji między językiem a~poznaniem:

\begin{enumerate}[label = (\arabic*)]
\item  interpretacje, zgodnie z~którymi poznanie kształtuje język,

\item  interpretacje, według których język kształtuje poznanie oraz

\item  stanowiska, na gruncie których interpretacje (1) i~(2) nie są sprzeczne i~się nie wykluczają.
\end{enumerate}

Pierwszy sposób interpretacji to, jak się wydaje, przeważające i~zdroworozsądkowe założenie, zgodnie z~którym człowiek posiada jakiś system poznawczy (wewnętrzny system reprezentacji), który manifestuje się na wiele różnych sposobów, a~jednym z~nich jest język\footnote{Z. Kövecses, \textit{Język, umysł, kultura}, tłum. A. Kowlacze-Pawlik, M. Buchta, Wydawnictwo Universitas, Kraków 2011, s.~473.}. Druga interpretacja znana jest w~literaturze jako hipoteza relatywizmu językowego lub hipoteza Sapira-Whorfa, rzadziej jako ,,whorfianizm''\footnote{Zob. B.C. Scholz, F.J. Pelletier, G.K. Pullum, R. Nefdt, \textit{Philosophy of Linguistics}, [w:] \textit{The Stanford Encyclopedia of Philosophy}, wyd. wiosna 2022, red. E.N. Zalta, <\url{https://plato.stanford.edu/archives/spr2022/entries/linguistics/}>.}. Teorie trzeciej klasy można uzyskać wykazując, że język i~poznanie to przejawy tego samego zjawiska lub procesu, albo wypracowując kompromis\footnote{Współcześnie trzy klasy stanowisk w~obrębie filozofii języka i~w kontekście jego związków z~umysłem dzieli się raczej na stanowiska \textit{eksternalistyczne}, zgodnie z~którymi pierwotnym językowym fenomenem są rzeczywiste wypowiedzi użytkowników języka, \textit{emergentystyczne}, według których podstawowym fenomenem jest komunikacja i~poznanie społecznie oraz \textit{esencjalistyczne}, zgodnie z~którymi bazowy fenomen stanowi z~reguły wrodzona intuicja gramatyki i~znaczenia terminów. Zob. tamże. Jednakże na potrzeby naszych rozważań przedstawiony powyżej podział jest bardziej trafny i~w zupełności wystarczający.} mówiący, że zarówno poznanie kształtuje ekspresję języka, jak i~język wpływa na to, jak myślimy.

Jako przykład takiej koncyliacyjnej interpretacji należącej do trzeciej kategorii można wskazać dziś już trącącą myszką i~dość niszową koncepcję Johna Fodora zwaną hipotezą języka myśli. Zgodnie z~tą hiopotezą myślenie odbywa się w~języku mentalnym, zwanym czasem językiem \textit{myśleńskim}. Myśleński przypomina język mówiony w~wielu aspektach -- na przykład składa się ze słów i~zdań a~znaczenie zdań zależy od znaczenia jego komponentowych składowych. Warto dodać, że od lat siedemdziesiątych ubiegłego stulecia podejścia w~stylu Fodora wiązały się z~najczęściej z~przyjęciem obliczeniowego modelu umysłu. Szczegóły tej koncepcji nie są istotne dla naszych rozważań\footnote{Dobry wgląd w~koncepcję Fodora daje lektura M. Rescorla, \textit{The Language of Thought Hypothesis}, [w:] \textit{The Stanford Encyclopedia of Philosophy}, wyd. lato 2019, red. E.N. Zalta, <\url{https://plato.stanford.edu/archives/sum2019/entries/language-thought/}>.}. Zauważmy jedynie, że głosi ona, że umysł i~język są w~pewnym sensie tożsame -- myślenie, to w~gruncie rzeczy język a~język (myśleński) jest sposobem organizacji i~działania naszego umysłu.

W~rzeczywistości większość współczesnych i~historycznych teorii dotyczących tego, jak działa nasz umysł budowana jest w~zgodzie z~pierwszą klasą interpretacji. Teza mówiąca, że poznanie kształtuje język, stanowi albo milczące założenie takich teorii, albo wyraźnie wyartykułowaną hipotezę roboczą, zgodnie z~którą badanie i~próba opisu systemu poznawczego dokonuje się przez studiowanie języka\footnote{Choć, oczywiście, inne, pozajęzykowe źródła przesłanek do budowania takich teorii również brane są pod uwagę -- zwłaszcza we współczesnych naukach kognitywnych, które więcej niż chętnie sięgają po dane od stowarzyszonej z~nimi psychologii poznawczej.}. Według takiego paradygmatu powstawała chociażby popularna w~poprzedniej dekadzie i~do dzisiaj mająca liczne grono zwolenników hipoteza umysłu ucieleśnionego. Mówi ona, że nasze poznanie kształtowane jest na drodze interakcji, w~jakie wchodzimy z~otoczeniem, poruszając się w~nim i~używając zmysłów. Podstawowe pojęcia tworzone są na bazie naszej orientacji i~działania w~świecie, np. to, co lepsze, umieszczane jest ,,pojęciowo'' \textit{powyżej} albo \textit{na górze} -- to, co gorsze, natomiast \textit{na dole} lub \textit{poniżej}. Bardziej złożone i~abstrakcyjne pojęcia budowane są na bazie tych ,,cielesnych'' i~podstawowych. I~tak, na przykład, \textit{wspinamy się} po szczeblach kariery, ale \textit{popadamy} w~depresję itp. Jeden z~autorów tej idei, George Lakoff, podaje cały szereg tzw. metafor\footnote{Zob. np. G. Lakoff, M. Johnson, \textit{Metafory w~naszym życiu}, Wydawnictwo Aletheia, Warszawa 2010; G. Lakoff, R.E. Núñez, \textit{Where Mathematics Comes From. How the Embodied Mind Brings Mathematics into Being}, Basic Books, New York 2000.}, które w~tej koncepcji stanowią mechanizm poznawczy przenoszący znaczenie z~dziedzin wiążących się z~interakcjami naszych ciał ze środowiskiem na zupełnie nowe dziedziny, dopiero podlegające procesowi rozumienia. Znów -- szczegóły koncepcji ucieleśnienia są irrelewantne dla naszych rozważań. Zauważmy jedynie, że według tej teorii nasze myślenie (ukształtowane przez interakcje ze środowiskiem) wypływa na to, jak mówimy, a~badanie języka i~jego metafor ma nam dawać obraz tego, jak myślimy.

Zupełnie odmienne stanowisko reprezentują zwolennicy hipotezy relatywizmu językowego, zgodnie z~którą język kształtuje nie tylko to, jak opisujemy, lecz także jak postrzegamy i~pojmujemy świat. W~następstwie tego podejścia należy uznać, że umysły osób przynależących do odmiennych społeczności językowych będą się różnić w~takim zakresie, w~jakim różnią się języki tych społeczności. Pomysły te próbowano podeprzeć anegdotycznymi przykładami z~zakresu językoznawstwa, z~których chyba najbardziej popularny dotyczył niezwykle rozbudowanego słownika związanego z~rozpoznawaniem śniegu w~językach Inuitów\footnote{W~różnych pracach podawano różną liczbę terminów z~języków ,,eskimoskich'' mających oznaczać różne rodzaje śniegu -- od 3 do ok. 300. Zob. B. Kortmann, G.K. Pullum, \textit{The Great Eskimo Vocabulary Hoax and Other Irreverent Essays on The Study of Language}, University of Chicago Press, Chicago and London 1991, ss.~159-171.}. Mimo to hipoteza relatywizmu językowego, przynajmniej w~swojej radykalnej wersji, odniosła druzgocącą porażkę i~została odparta pod presją miażdżącej przewagi sprzecznych z~nią interpretacji wyników badań empirycznych. Należy jednak przyznać, że kontrowersje, jakie wokół niej narastały, stymulowały rozwój zarówno teoretycznych, jak i~empirycznych badań nad językiem. Pojawiały się też, zwłaszcza ostatnio, wyniki badań zdające się faworyzować jakąś słabszą wersję tej hipotezy.

Najprawdopodobniej pierwszym poważnym ciosem w~teorie głoszące, że język jest zjawiskiem bardziej pierwotnym i~fundamentalnym niż poznanie, są badania Brenta Berlina i~Paula Kaya\footnote{B. Berlin, P. Kay, \textit{Basic Color Terms: Their Universality and Evolution}, University of California Press, Berkeley~1991.} dotyczące percepcji kolorów. Wynika z~nich, że nazwy kolorów podstawowych mogą opierać się na uniwersalnych aspektach fizjologii widzenia barwnego, a~zatem to raczej postrzeganie determinuje używaną terminologię, a~nie odwrotnie. Inne dane zdają się wskazywać, że takie ,,antyrelatywistyczne'' efekty, które zostały zaobserwowane w~obrębie percepcji i~nazywania kolorów, mogą nie występować w~innych domenach. Znana seria eksperymentów przeprowadzona pod przewodnictwem Lery Boroditsky sugeruje, że język może jednak wpływać na (choć raczej nie determinować) nasze poznanie i~sposób myślenia o~rzeczywistości. Badania Boroditsky dotyczą m.in. różnych koncepcji czasu wśród użytkowników języka mandaryńskiego i~angielskiego\footnote{L. Boroditsky, \textit{Does Language Shape Thought? Mandarin and English Speakers' Conceptions of Time}, ,,Cognitive Psychology'', vol. 43 (2001), ss.~1-22.}, czy wpływu rodzaju gramatycznego rzeczowników na sposób mentalnych reprezentacji obiektów nieożywionych\footnote{L. Boroditsky, L.A. Schmidt, W. Phillips, \textit{Sex, Syntax, and Semantics}, [w:] \textit{Language in Mind: Advances in the Study of Language and Thought}, red. D. Gentner, S. Goldin-Meadow, The MIT Press, Cambridge -- London 2003.}. W~podobnym duchu, choć na gruncie teoretycznym, argumentuje Dan Slobin. Jego koncepcja ,,myślenia dla mówienia''\footnote{D.I. Slobin, \textit{From ``thought and language'' to ``thinking for speaking}'', [w:] \textit{Rethinking linguistic relativity}, red. J.J. Gumperz, S.C. Levinson, Cambridge University Press, Cambridge 1996, ss.~70-96.} przedstawia stosunkowo słaby wpływ języka na poznanie głosząc, że język subiektywizuje i~ukierunkowuje nasze doświadczenie i~kształtuje sposób, w~jaki myślimy, przynajmniej podczas generowania mowy. Za jeszcze bardziej ograniczonym wpływie języka na myślenie opowiada się Ronald Langacker\footnote{R. Langacker, \textit{Gramatyka kognitywna}. Wprowadzenie, Universitas, Kraków 2011.} zauważając, że w~większości języków stosuje się różne formy gramatyczne do wyrażenia tej samej treści poznawczej.

Oczywiście, powyższe krótkie wyliczenie kilku badań i~argumentów pojawiających się po obu stronach sporu o~relatywizm językowy nie rości pretensji do bycia dogłębnym i~rzetelnym opracowaniem. Ma ono na celu pokazanie, że za naszym teologicznym zagadnieniem rozróżnienia interpretacji apofatycznych doktryn może kryć się dojrzały problem filozoficzny mający poważne reperkusje w~empirycznych i~teoretycznych naukach o~poznaniu. W~niniejszej pracy nie chcę jednoznacznie określać prymatu tezy o~niewysławialności i~nad tezą o~niepojmowalności lub \textit{vice versa}. Tym bardziej nie zamierzam rozstrzygać kwestii powiązań między językiem a~umysłem. Mam jednak nadzieję, że mimo to wywód ten pozwala sądzić, że teologia milczenia i~sceptycyzm teologiczny mogą stanowić dwie odrębne interpretacje teologii negatywnej, mające rozróżnialne motywacje filozoficzne i~prowadzące do odmiennych konsekwencji. Z~punktu widzenia niniejszej pracy istotne jest również to, że obie interpretacje, w~celu dokonania ich formalnej rekonstrukcji, domagają się użycia odmiennych środków logicznych. Mimo iż dla wielu reprezentantów teologii negatywnej różnica między tymi doktrynami nie była wystarczająco wyraźna, można wskazać grupę apofatycznych myślicieli, którzy zadawali się świadomie rozwijać teorię Niepoznawalnego abstrahując od jej ewentualnego językowego odpowiednika. Najbardziej znamiennym ich reprezentantem jest Mojżesz ben Majmon, czyli Majmonides.

\section{Mojżesz Majmonides}

Mojżesz Majmonides (zwany także Mojżeszem ben Majmonem a~przez hebrajskojęzyczną część badaczy Rambamem), najprawdopodobniej najwybitniejszy filozof żydowski był niewątpliwie także jednym z~najbardziej radykalnych średniowiecznych przedstawicieli teologii negatywnej. Urodził się w~1138 roku w~znajdującej się wtedy pod islamskim panowaniem hiszpańskiej Kordobie. Jednakże, z~powodu zaostrzenia polityki przeciwko niemuzułmańskim mieszkańcom Hiszpanii, już w~młodzieńczym wieku zmuszony był uciekać do północnej Afryki. Tam powstała większość jego dzieł, z~których najbardziej znaczącym dla przedstawianych w~niniejszej pracy rozważań jest ukończony w~1190 roku \textit{Przewodnik błądzących}\footnote{Majmonides, \textit{The Guide of the Perplexed}, vol. 1-2, tłum. S. Pines, Chicago University Press, Chicago -- London 1963; Polskie tłumaczenie (pierwszej części): Tenże, \textit{Przewodnik błądzących}, dz. cyt. Z~punktu widzenia teologii negatywnej interesujące są zwłaszcza rozdziały 51-60 pierwszej części \textit{Przewodnika}.}, w~którym zawarł m.in. swoją apofatyczną doktrynę. Majmonides był dla judaizmu tym, kim Tomasz z~Akwinu dla chrześcijaństwa -- w~swoich pracach próbował wyjaśniać istotę judaizmu w~sposób zgodny z~myślą i~logiką neoplatoników i~Arystotelesa, których pisma tłumaczone z~greki na język arabski zaczęły przedostawać się do europejskiego kręgu kulturowego. Początkowo prace Rambama były uznawane za kontrowersyjne i~w pewnych ośrodkach rabinicznych zakazywano ich czytania. Szybko jednak zdobyły uznanie, by następnie stać się najbardziej wpływowymi w~historii myśli judaistycznej. Do dziś wśród badaczy myśli żydowskiej funkcjonuje powiedzenie mówiące, że ,,od Mojżesza do Mojżesza nie było nikogo jak Mojżesz''\footnote{K. Seeskin, \textit{Maimonides}, [w:] \textit{The Stanford Encyclopedia of Philosophy}, wyd. wiosna 2021, red. E.N. Zalta, <\url{https://plato.stanford.edu/archives/spr2021/entries/maimonides/}>.}.

Punktem wyjścia apofatycznej doktryny Majmonidesa jest uznanie, że Bóg jest bytem koniecznym, który nie posiada żadnej przyczyny -- ani dla swojego istnienia, ani dla swojej istoty. Jest bytem absolutnie prostym i~nieporównywalnym do czegokolwiek innego, co istnieje. Z~tego powodu nie możemy zrozumieć żadnego pojęcia, gdy jest ono użyte do jego opisu. Każde imię nadane mu w~Biblii, każdy przypisany mu predykat nie jest użyty ani dosłownie, ani metaforycznie, ani nawet analogicznie. Majmonides przekonuje, że użycie to jest jedynie wieloznaczne. Inaczej mówiąc, nawet w~Torze przymioty nadawane są Bogu nie inaczej, jak przez ekwiwokację. Ponieważ w~efekcie stanowią one homonimy, nie wiemy, co tak naprawdę oznaczają (gdy orzekane są o~Bogu). W~rzeczywistości musimy je wszystkie zanegować, aby uchronić nasz nieporadny i~nieświadomy umysł przed niebezpieczeństwem błędu bałwochwalstwa. Ale nawet negacja ,,żadną miarą nie daje poznania prawdziwej rzeczywistości rzeczy, o~której dany sąd jest negowany''\footnote{Majmonides, \textit{The Guide}…, dz. cyt., cz. 1, rozdz.~59.}. Dlatego Bóg pozostaje w~zupełności poza naszym pojmowaniem.

Nie możemy zrozumieć żadnego pojęcia, gdy jest ono użyte do opisania Boga. W~teologii Majmonidesa Bogu nie można przypisać żadnych własności. Jedyne atrybuty, jakie można o~nim ewentualnie orzec, dotyczą jego akcji i~ich konsekwencji (ale nie można twierdzić, że takie atrybuty wyrażają lub opisują jego naturę). Mówiąc bardziej szczegółowo, Majmonides wyróżnia pięć rodzajów predykatów -- dzieli je na takie, które oznaczają (i) definicje, (ii) definicje cząstkowe, (iii) własności (w tym ilość, położenie, dyspozycje itp.), (iv) relacje oraz (v) działania. Pierwsze cztery rodzaje predykatów opisują daną rzecz w~aspekcie jej istoty lub przypadłości. Predykaty opisujące działanie odnoszą się natomiast do wyników lub konsekwencji aktów danego agenta. Podają one opis tego, co agent zrobił, a~nie tego, czym jest. Jak przekonuje Majmonides, prawdziwie orzekane o~Bogu mogą być wyłącznie te predykaty wyrażające działanie (v). Nie dostarczają one jednak żadnej wiedzy o~istocie Boga. Innych rodzajów własności, a~mianowicie (i)-(iv), należy Bogu odmówić\footnote{Por. J.A. Buijs, \textit{The negative theology of Maimonides and Aquinas}, ``Review of metaphysics'', vol. 41 (1988), nr 164, s.~729.}. Orzekanie o~nim takich predykatów stałoby w~sprzeczności z~jego absolutnie prostą naturą.

W~konsekwencji, nie można posiąść żadnego pojęcia Boga ani żadnej o~nim wiedzy. Nie sposób przedstawić ani tzw. klasycznej równościowej definicji Boga (w której \textit{definiens} tworzy się podając rodzaj i~różnicę gatunkową), ani żadnej z~definicji cząstkowych. Możliwość skonstruowania klasycznej definicji Boga oznaczałaby, że istnieje jakiś wyższy od Boga rodzaj, pod który trzeba by było go podporządkować. Bóg, który nie ma żadnych przyczyn i~nad którym nie ma wyższego rodzaju, nie może zostać w~taki sposób zdefiniowany. W~podobny sposób nieuprawniona jest także żadna z~tzw. nieklasycznych równościowych definicji (w której \textit{definiens} to wyliczenie pojęć, które w~sumie składają się na \textit{definiendum}). W~przypadku Boga podanie jakiejkolwiek definicji cząstkowej musiałoby oznaczać, że ma on złożoną naturę i~nie jest absolutnie prosty. Z~tego samego powodu powstrzymujemy się przed przypisywaniem Bogu jakichkolwiek własności czy opisywaniem jego natury\footnote{Por. I. Franck, \textit{Maimonides and Aquinas on man's knowledge of God: A~twentieth century perspective}, ``Review of Metaphysics'', vol. 38 (1985), nr 3, s.~594.}. Wszelkie takie własności są niedoskonałymi ludzkimi pojęciami i~nie mogą reprezentować natury Boga.

Wyraźne inspiracje ideami Mojżesza ben Majmona widoczne są w~pracach najbardziej wpływowego średniowiecznego filozofa -- Tomasza z~Akwinu. To najprawdopodobniej z~tych inspiracji pochodzą najważniejsze oryginalne rozważania Tomasza, zawarte w~\textit{O~bycie i~istocie}\footnote{Tomasz z~Akwinu, \textit{De ente et essentia.} \textit{O~bycie i~istocie}, [wydanie dwujęzyczne] tłum. polskie M. Krąpiec, Redakcja Wydawnictw KUL, Lublin 1981.}
czy w~\textit{Sumie teologicznej}\footnote{Tomasz z~Akwinu, \textit{Suma teologiczna}, tłum. P. Bełch, tom 1-34, Katolicki Ośrodek Wydawniczy ,,Veritas'', Londyn 1962-1986.}
. Od Majmonidesa zdają się pochodzić choćby te Tomaszowe koncepcje, które każą sądzić, że -- choć możemy zrozumieć i~poznać, że Bóg istnieje -- natura czy też istota boska pozostaje zupełnie nieznana (\textit{penitus manet ignotum}) ludzkiemu umysłowi. Choć wydaje się, że apofatyzm Tomasza był słabszy niż Majmonidesa\footnote{Por. J.A. Buijs, \textit{Comments on Maimonides' negative theology}, ,,The New Scholasticism'', vol. 49 (1975), nr 1, ss.~87-93.}, często wymieniany jest on jednym tchem, obok Rambama i~Pseudo-Dionizego, wśród średniowiecznych teologów negatywnych.

Jednym z~powodów, dla którego apofatyzm Majmonidesa jest uważany za bardziej radykalny od Tomaszowego, jest podejście tego pierwszego do kwestii orzekania o~Bogu przez analogię. Majmonides uważa, że absolutne rozróżnienie między Bogiem a~stworzeniem logicznie wyklucza taką możliwość, natomiast Tomasz ją dopuszcza, choć nie bez zastrzeżeń. Odrzucenie orzekania przez analogię może posiadać jedną z~dwóch poniższych motywacji:
\begin{enumerate}[label = (\arabic*)]
\item Aby jakiś atrybut mógł zostać przypisany przez analogię do dwóch różnych bytów, np. do Boga i~człowieka, koniecznym jest, by atrybut ten przypisany tymże bytom miał znaczenie tożsame, podobne lub aby jego znaczenia się pokrywały. Nie może być mowy o~analogii, jeśli takie podobieństwo lub nakładanie się na siebie znaczeń nie występuje. Jednakże, skoro nie możemy wiedzieć, co znaczy jakikolwiek predykat, gdy wprost orzekamy go o~Bogu, nie mamy podstaw także do przypisania go Bogu na podstawie orzekania przez analogię.

\item Doktryna o~absolutnej i~nieredukowalnej różnicy między Bogiem a~stworzeniem zakłada, że jakikolwiek atrybut orzekany o~Bogu -- gdyby takie przypisanie było możliwe -- byłby całkowicie i~zupełnie różny od predykatu wyrażonego tym samym terminem użytym w~odniesieniu do człowieka, a~zatem żadna analogia w~tym wypadku nie jest możliwa.
\end{enumerate}

Kolejną różnicą miedzy apofatyzmem Rambama i~Tomasza jest ich podejście do tez o~niewysławialności i~niepojmowalności oraz do zależności między nimi. Jak przekonuje Joseph A. Buijs\footnote{Zob. J.A. Buijs, \textit{The negative theology}…, dz. cyt., ss.~734-735.}, Tomasz obie tezy uważał za autonomiczne i~obie wywodził niezależnie ze swojej metafizyki. Jeśliby już szukać u~niego jakichś wskazówek determinujących wynikanie w~którąkolwiek ze stron, należałoby raczej przeprowadzić je, w~kontrapozycji, z~języka do myśli -- skoro pewne predykaty mogą (w sposób niedoskonały, ale prawdziwy) opisywać istotę Boga, możemy posiadać (niedoskonałą, choć prawdziwą) wiedzę o~Bogu. Dla Majmonidesa to epistemiczna forma apofatyzmu była bez wątpienia tą pierwotną i~bardziej podstawową. To ją wywodził ze swojej metafizyki, a~dopiero za jej konsekwencję uznawał semantyczną wersję apofatyzmu w~postaci teologii milczenia -- skoro istota Boga pozostaje dla nas niepozwalana, nie możemy żadną miarą też jej opisać w~sposób znaczący. Z~kolei Ehud Z. Benor idzie o~krok dalej w~interpretacji myśli Majmonidesa argumentując, że średniowieczny filozof odrzuca tezę o~niewyrażalności, rozwijając teorię Niepoznawalnego abstrahując od jej ewentualnych językowych konsekwencji. ,,Nie docenilibyśmy problemu, przed którym stanął Majmonides, gdybyśmy umiejscowili go w~niezdolności języka do wyrażenia trudnych idei metafizycznych''\footnote{E.Z. Benor, Meaning and reference In Maimonides' negative theology, ,,Harvard Theological Review'', vol. 88 (1995), nr 3, ss.~352-353.}. Hilary Putnam ujmuje tę myśl jeszcze dosadniej:

\begin{quote}
[...] nie chodzi tylko o~to, że ktoś czuje (jeśli jest religijny), że nie może właściwie wyrazić tego, co ma na myśli, używając do opisania Boga terminów, których dostarcza nasz język; chodzi o~to, że ktoś czuje, iż nie może mieć na myśli tego, co powinien mieć na myśli\footnote{H. Putnam, \textit{On negative theology}, ,,Faith and Philosophy'', vol. 14 (1997), nr 4, s.~410.}.
\end{quote}

Na koniec warto dodać, że Majmonides dopuszcza tezę, według której natura boska jest w~pewnym sensie poznawalna -- swoją istotę może zrozumieć i~pojąć wyłącznie Bóg. Poświęca on temu zagadnieniu spory fragment pierwszej części \textit{Przewodnika}\footnote{Zob. np. Majmonides, \textit{The Guide}…, dz. cyt., cz. 1, rozdz.~58.}
. Ben Majmon twierdzi, że wiedza boska i~ludzka są heterotypiczne -- są czymś całkowicie innym, mają zupełnie inne charaktery. Dla człowieka próba zrozumienia istoty Boga, byłaby ostatecznie próbą nabycia boskiej wiedzy i~zdolności rozumienia na boskim poziomie. Człowiek, który chciałby zrozumieć Boga, chciałby w~efekcie stać się takim, jak on. Pomijając te konsekwencje rozważań Majmonidesa możemy stwierdzić, że jego doktryna nie stanowi apofatyzmu kompletnego. Nie jest nim o~tyle, o~ile istnieje co najmniej jeden byt, który może pojąć i~zrozumieć boską istotę -- sam Bóg\footnote{Uwagę tę zamieszczam na marginesie zawartych tu rozważań. Wydała mi się interesująca, ponieważ potencjalnie i~od zupełnie innej strony wprowadza do doktryny Majmonidesa kolejne paradoksy samozwrotności.}.

\section{Paradoksalny charakter sceptycyzmu teologicznego}

Pomimo wszystkich różnic między teologią milczenia a~sceptycyzmem teologicznym, ciąży na nich ten sam problem samozwrotności. Obie interpretacje teologii negatywnej są na tyle podobne, by większość argumentów przeciw teorii Niewysławialnego bazujących na wskazaniu jej paradoksalnego charakteru można było przeformułować \textit{mutatis mutandis} przeciw teorii Niepojmowalnego. Zresztą cytaty ilustrujące sprzeczność i~samoodniesienie teologii milczenia przywoływane w~pierwszej części niniejszej pracy odnoszą się tak samo skutecznie, o~ile nawet nie trafniej, do epistemicznego aspektu apofatyzmu\footnote{Por. rozdz.~\ref{sil-int-par}.}. Możemy zatem przedstawić propozycję paradoksu teorii Niepojmowalnego sformułowanego w~sposób, który odpowiadałby paradoksom semantycznym z~poprzedniej części pracy.

\subsubsection{Paradoks Niepoznawalnego}

Zgodnie z~wykładnią teologicznego sceptycyzmu o~Bogu możemy wiedzieć jedynie to, że jest niepoznawalny.

Czy zatem -- zgodnie z~tą interpretacją teologii negatywnej -- Bóg jest czy nie jest niepoznawalny? Jeśli przyjmiemy, że nie jest on niepoznawalny, to nie ma poza tym żadnej innej wiedzy o~Bogu, a~zatem jest on niepoznawalny. Jeśli natomiast założymy, że jest on niepoznawalny, wiemy już coś o~Bogu -- mianowicie to, że jest niepoznawalny, co w~konsekwencji pociąga za sobą sprzeczność. Mianowicie wynika z~tego, że i~Bóg nie jest niepoznawalny wtedy i~tylko wtedy, gdy jest niepoznawalny.

W~ten sam sposób w~kontekście teorii Niepojmowalnego możemy budować także ,,zewnętrzne'' paradoksy teologii negatywnej z~uwzględnieniem wszystkich istniejących różnic między interpretacją zorientowaną na aspekty językowe a~tą, która podkreśla epistemiczny charakter apofatyzmu. Jeśli utrzymujemy, że o~Bogu nie możemy posiadać żadnej wiedzy, w~jaki sposób możemy upierać się, że jest, na przykład, jeden w~trzech osobach lub że jest sędzią sprawiedliwym? Podobnie, jak możemy zwracać się do czego lub przypisywać wartość czemuś, czego nie potrafimy pojąć?\footnote{Por. zewnętrzny twierdzeniowy oraz nietwierdzeniowy (performatywny) paradoks teologii milczenia tamże.} Te ,,zewnętrzne'' paradoksy również zachowują ważność w~sceptycyzmie teologicznym.

\subsubsection{Posiadanie i~nabywanie wiedzy jako reprezentant procesów mentalnych w~kontekście teologicznego sceptycyzmu}

Zasadniczo paleta terminów w~apofatycznym słowniku teologicznego sceptycyzmu jest nieco szersza. W~obrębie tej interpretacji teologii negatywnej mówimy, że Bóg jest całkowicie niepoznawalny, niepojęty i~niemożliwy do skonceptualizowania czy zrozumienia, nie można posiąść o~nim żadnej wiedzy. Oczywiście, taki paradoks powinniśmy bez trudu móc skonstruować dla każdego z~wymienionych tu apofatycznych przymiotników\footnote{Nie wszystkie z~wymienionych tu apofatycznych terminów są gramatycznymi przymiotnikami, lecz źródłem tego faktu są \textit{notabene} ograniczenia języka polskiego. Na przykład w~języku angielskim każdą z~tych cech już da się wyrazić taką częścią mowy. Centralnym i~najczęściej stosowanym apofatycznym przymiotnikiem w~kontekście teologicznego sceptycyzmu jest \textit{unknowable}, ale często towarzyszą mu inne, takie jak \textit{inconceivable}, \textit{incomprehensible} czy \textit{inapprehensible}.}. Na przykład argument na rzecz sprzeczności teologicznego sceptycyzmu w~kontekście pojmowalności Boga wyrażony w~języku potocznym jest mniej-więcej następujący: teoria ta stwierdza, że o~Boga nie można pojąć; zrozumieliśmy więc coś o~Bogu -- mianowicie to, że jest niepojmowalny, co w~konsekwencji pociąga za sobą sprzeczność\footnote{W~pełnej wersji paradoks Niepojmowalnego przyjąłby następującą postać: Zgodnie z~wykładnią teologicznego sceptycyzmu jedynym, co możemy zrozumieć o~Bogu, jest idea, wedle której jest on niepojmowalny. Czy zatem -- zgodnie z~tą interpretacją teologii negatywnej -- Bóg jest czy nie jest niepojmowalny? Jeśli przyjmiemy, że nie jest on niepojmowalny, to nie ma niczego innego, co moglibyśmy o~nim pojąć, a~zatem jest niepojmowalny. Jeśli natomiast założymy, że jest on niepojmowalny, to rozumiemy przynajmniej to, że jest niepojmowalny, a~zatem nie jest niepojmowalny. W~konsekwencji Bóg nie jest niepojmowalny wtedy i~tylko wtedy, gdy jest niepojmowalny.}.

W~tym miejscu należy się kolejna uwaga o~różnorodności procesów umysłowych -- podobna do tej, którą przedstawialiśmy na marginesie omawiania zależności między językiem a~poznaniem\footnote{Zob. rozdz.~\ref{scep-werapo}.}. Przy tamtej okazji przekonywałem, że relacje między myśleniem, rozumieniem, nabywaniem i~utrzymywaniem przekonań, wiedzą oraz językiem i~jego użyciem są bardzo złożone. Tym razem zwracam uwagę, że samo myślenie, pojmowanie oraz nabywanie przekonań i~wiedzy mogą być i~są z~pewnością terminami technicznymi nauk kognitywnych, oznaczającymi nieco odmienne zjawiska i~procesy. W~obrębie niniejszej pracy są one jednak traktowane zbiorczo a~terminy oznaczające te zjawiska używane wymiennie. Jednakże za wyróżnione mentalne zjawisko i~proces w~kontekście teologicznego sceptycyzmu uważać będę posiadanie i~nabywanie wiedzy. Zatem zasadniczo, gdy mówimy, że Boga nie można poznać, pojąć, skoneceptualizować czy zrozumieć, mamy na myśli to, że nie da się posiąść o~nim żadnej wiedzy. Pozwalam sobie na taką idealizację z~co najmniej trzech powodów.
Po pierwsze, dla wygody -- pisanie za każdym razem o~niepojętym, niepoznawalnym, niemożliwym do zrozumienia i~skonceptualizowania Bogu z~pewnością ubogaca apofatyczny dyskurs, lecz na dłuższą metę jest niepraktyczne. Po drugie, z~punktu widzenia nauk kognitywnych teza, wedle której pojmowanie, myślenie czy rozumienie sprowadza się do nabywania lub posiadania wiedzy, być może nie jest pozbawiona kontrowersji, ale z~pewnością możliwa do obrony. Na potrzeby rozważań o~epistemicznym aspekcie teologii negatywnej przedstawianych w~niniejszej części pracy takie przybliżenie zjawisk mentalnych, jakie daje nam przyjęta idealizacja, jest w~zupełności wystarczające i~nie będzie wpływać na wyciągane w~ramach tych rozważań wnioski. Po trzecie i~być może najważniejsze, takie postawienie sprawy otwiera nam drogę do użycia dobrze ugruntowanych rachunków logicznych -- logiki epistemicznej i~(ewentualnie) doksastycznej.





\chapter{Epistemiczne paradoksy teologicznego sceptycyzmu}\label{scep-par}

W~świetle tego, co zostało powiedziane w~poprzednim rozdziale, w~celu formalnej rekonstrukcji sceptycyzmu teologicznego najrozsądniej byłoby skorzystać z~epistemicznej logiki modalnej. Język tego rachunku buduje się dodając do alfabetu symbol operatora $K$~mającego ujmować intuicje stojące za zjawiskiem wiedzy i~poznania. Epistemiczny operator $K$~możemy zatem czytać jako ,,jest znane (przez kogoś w~pewnym czasie), że'', ,,(ktoś) wie, że'', ,,wiadomo, że'' a~nawet, pod pewnymi warunkami, jako ,,jest poznawalne, że'' itp. Dla celów naszych analiz możemy przyjąć, że $K$~oznacza ,,jest znane przez kogokolwiek w~dowolnym czasie, że''. Do zbioru wyrażeń sensownych takiego języka musimy, rzecz jasna, dodać wyrażenia o~postaci $KA$ (o ile $A$~jest już wyrażeniem sensownym języka epistemicznej logiki modalnej).

Wydaje się, że rozsądnie jest przyjąć, iż według wykładni teologicznego sceptycyzmu wszystkie prawdziwe twierdzenia dotyczące Boga są nam nieznane. Niech ponownie $\mathcal{G}$~będzie zbiorem wszystkich prawdziwych sądów o~Bogu, niech $K$~będzie operatorem epistemicznym wyrażającym intuicje towarzyszące pojęciu wiedzy i~poznania, wówczas epistemiczna zasada teologii negatywnej przyjęłaby postać:
\begin{flalign*}
& \forall_{A \in \mathcal{G}}\ \neg K A. &\tag{KNT}\label{scep-par-preKNT}
\end{flalign*}
Zakładając, że $\mathcal{G}$ jest niepuste i $p \in \mathcal{G}$, otrzymujemy
%$p \land \neg K p$.\label{scep-par-postKNT}
\begin{flalign*}
& p \land \neg K p &\tag{KNT'}\label{scep-par-postKNT}
\end{flalign*}
W~sytuacji teologicznego sceptycyzmu wiemy, że nigdy nie poznamy żadnych prawd o~Bogu. Możemy więc jeszcze zapisać
%$K(A \land \neg K A)$.\label{scep-par-KNT-bis}
\begin{flalign*}
&K(A \land \neg K A)&\tag{KNT''}\label{scep-par-KNT-bis}
\end{flalign*}
W~ten sposób wkraczamy na teren dobrze znanych paradoksów epistemicznych.

Po pierwsze, zauważmy, że \ref{scep-par-postKNT} ma postać (odpowiednio zinterpretowanego) tzw. problemu Moore'a. Nie jest on paradoksem \textit{sensu stricto}, lecz raczej pewną łamigłówką epistemiczną. Wymaga narzucenia perspektywy pierwszoosobowej na interpretację operatora K, a~także potencjalnie wymusza wprowadzenie drugiego operatora (mającego wyrażać słabszą postawę propozycjonalną -- przekonanie) oraz użycie logiki multimodalnej. Problem ten w~kontekście teologicznego sceptycyzmu omówię pokrótce w~pierwszej sekcji niniejszego rozdziału.

Po drugie, \ref{scep-par-postKNT} i~\ref{scep-par-KNT-bis} wiążą się z~paradoksem poznawalności Churcha-Fitcha. Z~grubsza rzecz biorąc, paradoks ten mówi, że jeśli wiemy, że p~jest niepoznawalną prawdą, to niepoznawalne jest, że p~jest niepoznawalną prawdą (lub, przez kontrapozycję, że skoro wszystkie prawdy są co do zasady poznawalne, to wszystkie prawdy są znane). Podczas analizy tego paradoksu okaże się, że dosyć skromne założenia dotyczące działania epistemicznego operatora $K$~każą uznać \ref{scep-par-KNT-bis} za zdanie fałszywe.

Po trzecie, samozwrotną naturę powyższych wyrażeń możemy zredukować do tzw. paradoksu wiedzy (zwanego także -- przez analogię do paradoksu kłamcy -- paradoksem znawcy) i~związanego z~nim twierdzenia Montague'a, które uważane jest za generalizację twierdzenia Tarskiego (mówiącego, że każda teoria sformułowana w~semantycznie zamkniętym języku i~zwierająca T-równoważności jest sprzeczna). Zatem, mimo badań biorących za punkt wyjścia odrębne interpretacje teologii negatywnej oraz mimo stosowania odmiennych logicznych narzędzi do przedstawiania formalnej struktury tych interpretacji, w~ostateczności można uznać, że logiczne problemy teologii negatywnej mają tożsamą (a~przynamniej podobną) strukturę, zarówno, gdy rozpatrywane są w~jej aspekcie semantycznym, jak i~epistemicznym.

\section{Problem Moore'a}\label{scep-par-problemmoorea}

Choć opisywany w~tej sekcji problem ma swoje źródło w~filozofii (moralności), szybko znalazł zainteresowanie w~gronie bardziej analitycznie i~formalnie usposobionych badaczy. Jego nazwa została nadana przez Wittgensteina, który mówił o~\textit{paradoksie} Moore'a. Dziś, zagadnienie to nazywa się raczej \textit{problemem} niż paradoksem. Jego sednem jest przyznawanie prawdziwości czemuś, co do czego nie jest się przekonanym a~absurdalny charakter takich wyrażeń zależy silnie od kontekstu ich wypowiedzi. Przy odpowiedniej interpretacji sceptycyzm teologiczny można sprowadzić do tego epistemicznego zagadnienia.

Jego pierwsze znane sformułowanie pochodzi z~krótkiej wzmianki George'a Edwarda Moore'a zawartej w~pracy z~1942 roku z~poświęconego mu tomu z~serii \textit{Library of Living Philosophers}. Brzmi ono następująco:

\begin{quote}
,,W ostatni wtorek poszedłem do kina, ale w~to nie wierzę'' to całkowicie absurdalne stwierdzenie, chociaż sąd wyrażony tym zdaniem, jest w~zupełności możliwy logicznie\footnote{G.E. Moore, \textit{A~reply to my critics}, [w:] \textit{The Philosophy of G.E. Moore}, red. P.A. Schilpp, \textit{The Library of Living} \textit{Philosophers}, vol. 4, Northwestern University, Evanston 1942, ss.~543. Cytat za: H. Van Ditmarsch I~in., \textit{Everything is knowable - How to get to know whether a~proposition is True}, ,,Theoria'', vol. 78 (2012), nr 2, ss.~93-114.}.
\end{quote}

Moore sam nigdy nie opublikował pracy poświęconej pogłębionej dyskusji nad tym problem. Jednakże w~sporządzonych kilkanaście miesięcy później a~wydanych po jego śmierci notatkach -- w~odpowiedzi na komentarze Wittgensteina -- ponownie do tego wraca.

\begin{quote}
[…] zakładam, że absurdalne lub bezsensowne jest mówienie takich rzeczy jak ,,Nie wierzę, że pada deszcz, choć w~rzeczywistości pada'' […]. Chciałbym jednak zauważyć, że nie ma nic bezsensownego w~samym tym wyrażeniu. […] Nie ma nic bezsensownego w~powiedzeniu: ,,Jest całkiem możliwe, że choć nie wierzę, że pada deszcz, to w~rzeczywistości pada'' lub ,,Jeśli nie wierzę, że pada deszcz, choć w~rzeczywistości pada, to mylę się w~swoich przekonaniach''. We wszystkich tych przypadkach mamy do czynienia z~tym samym wyrażeniem, ale jest ono wypowiadane w~kontekście z~innymi słowami, w~taki sposób, że nie ma w~nich nic bezsensownego\footnote{G.E. Moore, \textit{Moore's Paradox}, [w:] \textit{G.E. Moore: Selected Works}, red. T. Baldwin, Routledge, London and New York 1993, s.~207.}.
\end{quote}

Już z~tych krótkich fragmentów możemy wyciągnąć kilka wniosków. Po pierwsze, ,,absurdalne'' czy ,,nonsensowne'' Moore'owi wydają się zdania o~postaci:
p, ale nie wierzę, że p.\label{scep-par-moore}
\begin{flalign}
& p \text{, ale nie wierzę, że } p. &\label{scep-par-moore}
\end{flalign}

Po drugie, zdania te nie stanowią par zdań sprzecznych. Używając jego słów, są ,,zupełnie możliwe logicznie''. Również, bez dodatkowych założeń, do sprzeczności nie będą prowadzić. Jest to jeden z~powodów, dla którego w~ich kontekście mówimy najczęściej o~\textit{problemie}, a~nie o~\textit{paradoksie}, Moore'a (choć należy uznać, że problem ten można nazwać paradoksem w~ogólnym, szerszym sensie\footnote{Por. rozdz.~\ref{sil-int-par}.}). Mimo tego, że wyrażenia o~postaci \eqref{scep-par-moore} nie stanowią ani bezpośrednio nie prowadzą do pary zdań sprzecznych, z~pewnością nazywanie ich ,,absurdalnymi'' czy ,,nonsensownymi'' nie jest pozbawione podstaw. Jest nich coś ,,logicznie osobliwego''\footnote{Por. H. Tennessen, \textit{Logical Oddities and Locutional Scarcities: Another Attack upon Methods of Revelation}, ,,Synthese'', vol. 11 (1959), nr 4, ss.~369-388.}.

Po trzecie, postawa propozycjonalna, o~której mowa w~\eqref{scep-par-moore} różni się od tej, z~którą mamy do czynienia w~\ref{scep-par-postKNT}. Logika epistemiczna, którą chcemy zaprzęgnąć do formalizacji teologicznego sceptycyzmu, próbuje uchwycić intuicje stojące za zjawiskiem wiedzy. W~przypadku \ref{scep-par-moore} należy mówić raczej o~przekonaniach i~użyć logiki przekonań, czyli logiki doksastycznej. W~takiej logice do alfabetu wprowadzamy modalny operator $B$~i dopuszczamy, by wyrażenia o~postaci $BA$ były wyrażeniami sensownymi (pod warunkiem, że A~jest wyrażeniem sensownym języka takiej logiki). Doksastyczny operator $B$, wraz rządzącymi nim aksjomatami, ma ujmować intuicje stojące za posiadaniem przekonań. $Bp$ Czytamy najczęściej jako ,,jest wiarygodne, że'' ,,jest zgodne z~(czyimiś) przekonaniami, że $p$'', ,,(ktoś) jest przekonany, że $p$'', ,,(ktoś) wierzy, że $p$''. Moore'owskie zdanie w~takiej logice zapiszemy:
%p~{\textbackslash}land {\textbackslash}neg B~p.\label{scep-par-moore-form}
\begin{flalign}
& p \land \neg B p. &\label{scep-par-moore-form}
\end{flalign}

W~formalnych rozważaniach nie unika się budowania rachunków multimodalnych, w~których występuje zarówno operator K, jak i~operator $B$. Z~reguły operatory te nie są wzajemnie definiowane (w taki sposób, w~jaki ma to miejsce na przykład w~przypadku operatorów możliwości i~konieczności w~aletycznej logice modalnej), choć zdarzają się takie podejścia. Najczęściej jednak w~multimodalnych rachunkach epistemicznych w~celu uchwycenia zależności miedzy wiedzą a~przekonaniem przedstawia się następujące aksjomaty\footnote{Por. J-J.C.Meyer, \textit{Modal Epistemic and Doxastic Logic}, [w:] \textit{Handbook of Philosophical Logic}, vol. 10, red. D.M. Gabbay, F. Guenthner, Springer, Dordrecht 2003, s.~6 oraz R. Rendsvig, J. Symons, \textit{Epistemic Logic}, [w:] \textit{The Stanford Encyclopedia of Philosophy}, wyd. lato 2021, red. E.N. Zalta, <\url{https://plato.stanford.edu/archives/sum2021/entries/logic-epistemic/}>.}:
%(K -{}-{}-+B) K{\textless}p -{}-{}-+ B{\textless}p oraz \label{scep-par-kdob}
%(KB) B{\textless}p -{}-{}-+ KB{\textless}p,
\begin{flalign}
\quad& Kp \to Bp &\tag{$K{\to}B$}\label{scep-par-kdob}\\
& Bp \to KBp &\tag{KB}
\end{flalign}

co zresztą jest zgodne z~tzw. klasyczną koncepcją wiedzy, wedle której jest ona prawdziwym i~uzasadnionym \textit{przekonaniem}\footnote{Por. E.L. Gettier, \textit{Is Justified True Belief Knowledge}?, ,,Analysis'', vol. 23 (1963), nr 6, ss.~121-123.}
. Oznacza to, że \ref{scep-par-postKNT} nie pociąga za sobą zdania w~stylu \ref{scep-par-moore-form}, a~przynajmniej nie wprost. Gdyby tak było, musielibyśmy uznać, że wiedza jest tożsama z~przekonaniem (albo odrzucić \ref{scep-par-kdob} i~uznać, że sytuacja, w~której coś wiemy jest ,,słabsza'' od tej, w~której jesteśmy co do tego przekonani). Istnieją jednak badacze, którzy problem Moore'a próbują analizować formalnie w~rachunku, który sytuację wiedzy i~przekonania wyraża jednym operatorem\footnote{Zob. H. Van Ditmarsch I~in., \textit{Everything is knowable}…, dz. cyt. Takie podejście uważam za błędne, a~przynajmniej mylące. Autorzy tej pracy redukują problem Moore'a do paradoksu Churcha-Fitcha. Taka redukcja prowadzi do spłycenia problemu I~jego filozoficznych konsekwencji. Por. C. de Almeida, \textit{What Moore's Paradox Is About}, ,,Philosophy and Phenomenological Research'', vol. 62 (2001), nr 1, s.~33-58.} lub rozważają analogon problemu Moore'a dla wiedzy, czyli zdania o~postaci
%p, ale nie wiem, czy p.\label{scep-par-moore-wiedz}
\begin{flalign}
& p \text{, ale nie wiem, czy } p. &\label{scep-par-moore-wiedz}
\end{flalign}

Do tych ostatnich należy Jaakko Hintikka\footnote{Zob. J. Hintikka, \textit{Knowledge and Belief}, Cornell University Press, Ithaka 1962, rozdz.~4.11.}, według którego -- podobnie, jak w~przypadku \eqref{scep-par-moore} -- wypowiadanie zdań o~postaci \eqref{scep-par-moore-wiedz} jest ,,cokolwiek niezręczne'', chociaż samo w~sobie spójne. Wydaje się więc, że mimo iż teologiczny sceptycyzm nie pociąga za sobą wprost problemu Moore'a, przedstawienie na jego przykładzie epistemicznego aspektu teologii apofatycznej jest jak najbardziej wskazane.

Czwartym i~ostatnim wnioskiem z~przytoczonych na początku tej sekcji fragmentów opisujących zdania Moore'a jest fakt, że ich ,,absurdalność'' i~,,bezsens'' są bardzo silnie uzależnione od kontekstu ich wypowiedzi. W~cytowanym fragmencie notatek wydanych \textit{post mortem} Moore zgadza się z~Wittgensteinem, że zdania o~postaci
%Jest całkiem możliwe, że choć nie wierzę, że p, to w~rzeczywistości p
\begin{flalign}
& \text{Jest całkiem możliwe, że choć nie wierzę, że } p \text{, to w~rzeczywistości } p. &
\end{flalign}
oraz
%Jeśli nie wierzę, że p, choć w~rzeczywistości p, to się mylę
\begin{flalign}
& \text{Jeśli nie wierzę, że } p \text{, choć w~rzeczywistości } p \text{, to się mylę.} &
\end{flalign}
są już pozbawione swojego absurdalnego charakteru. Podobnie, bezsensowność zdań Moore'a znika, gdy zrezygnujemy z~pierwszoosobowej perspektyw. Wydaje się, że nie ma nic paradoksalnego, ,,absurdalnego'', ,,dziwnego'' czy ,,osobliwego'' w~stwierdzeniu
%p, choć on nie wierzy, że p.
\begin{flalign}
& p \text{, choć on nie wierzy, że  } p. &
\end{flalign}

W~podobny sposób pozbawiamy się kłopotliwego charakteru zdań Moore'owskich, jeśli zmienimy czas gramatyczny wypowiadanych zdań na przeszły. Znów, wydaje się, że możemy bez żadnego ,,zgrzytu'' wypowiedzieć
%p, choć wtedy nie wierzyłem, że p
\begin{flalign}
& p \text{, choć wtedy nie wierzyłem, że } p. &
\end{flalign}
i~to niezależnie od tego, czy samo $p$~także wyrażone jest przy użyciu czasu przeszłego. Nie dziwi zatem, że w~formalnej rekonstrukcji tych zdań oraz intuicji związanych z~wiedzą i~przekonaniem nie dochodzimy do sprzeczności. Parę zdań sprzecznych możemy jednak otrzymać wprowadzając pewne (skromne) założenia wiążące operator $K$~oraz zakładając, że jesteśmy świadomi, że znaleźliśmy się w~sytuacji, o~której mówi \eqref{scep-par-moore-wiedz}. W~taki sposób dochodzimy do paradoksu Churcha-Fitcha.

\section{Paradoks Churcha-Fitcha}


Mówiąc w~skrócie, paradoks Churcha-Fitcha, zwany także paradoksem lub problemem poznawalności, zakłada, że istnienie nieznanych prawd jest niepoznawalne. Jeśli więc wszystkie prawdy są poznawalne, to w~rzeczywistości wszystkie prawdy są znane. Zagadnienie to posiada ciekawą historię i~choć przez niektórych badaczy zostało okrzyknięte ,,problemem na życzenie''\footnote{Taką etykietę paradoksowi Churcha-Fitcha nadał Piotr Łukowski argumentując, że wynika on z~niepoprawnego zapisu oraz wskazując na konstruktywistyczny przypadek zdań nierozstrzygniętych. Zob. P. Łukowski, \textit{Paradoksy}, Wydawnictwo Uniwersytetu Łódzkiego, Łódź 2006, s.~33. Argumentację Łukowskiego łatwo jest jednak odeprzeć lub odwrócić. Zob. B. Brogaard, J. Salerno, \textit{Fitch's Paradox of Knowability}, [w:] \textit{The Stanford Encyclopedia of Philosophy}, wyd. jesień 2019, red. E.N. Zalta, <\url{https://plato.stanford.edu/archives/fall2019/entries/fitch-paradox/}>. Zob.~także J.L Kvaanvig, \textit{The Knowability Paradox}, Oxford University Press, New York 2006.}, posiada ono zarówno niebagatelne motywacje filozoficzne, jak i~daleko idące konsekwencje -- nie tylko na gruncie teologii negatywnej, lecz przede wszystkim filozofii i~logiki.

Problem poznawalności został po raz pierwszy przedstawiony w~1963 roku w~krótkiej pracy poświęconej logicznymi analizom pojęcia wartości autorstwa Frederica Fitcha\footnote{F.B. Fitch, \textit{A~Logical Analysis of Some Value Concepts}, ,,The Journal of Symbolic Logic'', vol.~28, (1963), nr 2, ss.~135-142.}. Zawarte w~niej Twierdzenie 5., które było pierwszym opublikowanym sformułowaniem paradoksu, zostało dodane do manuskryptu w~celu uniknięcia pewnego rodzaju ,,błędu warunkowego'', który zagrażał definicji wartości opartej na świadomym pragnieniu\footnote{Z~grubsza rzecz biorąc, zgodnie z~analizami Fitcha, $x$~jest wartościowe dla $s$~wtedy i~tylko wtedy, gdy istnieje prawda $p$~taka, że gdyby $p$~było znane $s$, to $s$~pragnąłby $x$.}. W~oryginalnej postaci Twierdzenie 5. brzmi:
%(Fitch) Jeśli istnieje zdanie prawdziwe, którego prawdziwości nikt nie zna (lub nie znał lub nie będzie znał), to istnieje zdanie prawdziwe, którego prawdziwość jest niepoznawalna\footnote{``If there is some true proposition which nobody knows (or has known or will know) to be true, then there is a~true proposition which nobody can know to be true''. Tamże, s.~139.}.\label{scep-par-cf-fitch}
%\begin{flalign*}
%		& \parbox[t]{.87\linewidth}{ 
%		Jeśli istnieje zdanie prawdziwe, którego prawdziwości nikt nie zna (lub nie znał lub nie będzie znał), to istnieje zdanie prawdziwe, którego prawdziwość jest niepoznawalna\footnote{``If there is some true proposition which nobody knows (or has known or will know) to be true, then there is a~true proposition which nobody can know to be true''. Tamże, s.~139.}.} &\tag{Fitch}\label{scep-par-cf-fitch}
%\end{flalign*}
\begin{tw}[Fitch\footnote{``If there is some true proposition which nobody knows (or has known or will know) to be true, then there is a~true proposition which nobody can know to be true''. Tamże, s.~139.}]\label{scep-par-cf-fitch}
Jeśli istnieje zdanie prawdziwe, którego prawdziwości nikt nie zna (lub nie znał lub nie będzie znał), to istnieje zdanie prawdziwe, którego prawdziwość jest niepoznawalna.
\end{tw}
Twierdzenie to wraz z~prototypowym dowodem zostało zaproponowane i~przedstawione Fitchowi przez anonimowego recenzenta\footnote{Według informacji zawartej w~przypisie umieszczonym w~pracy Fitcha, anonimowy recenzent jest autorem poprzedzającego powyższe Twierdzenia 4. Badania Joe Salerno nie tylko ujawniły nazwisko tego recenzenta, lecz także wskazały jego wkład w~sformułowanie i~dowiedzenie Twierdzenia 5. Zob. B. Brogaard, J. Salerno, \textit{Fitch's Paradox of Knowability}, dz. cyt.} wczesnej wersji jego pracy, którą w~1945 roku (\textit{sic}!) zgłosił do tego samego czasopisma (choć wtedy, rzecz jasna, jej nie opublikował). W~2003 roku, dzięki badaniom archiwalnym Joe Salerno, okazało się recenzentem tym był Alonzo Church\footnote{Recenzja Churcha została opublikowana w~2009 roku. Zob. A. Church, \textit{Referee Reports on Fitch's ``A Definition of Value}'', [w:] \textit{New Essays on the Knowability Paradox}, red. J. Salerno, Oxford University Press, New York 2009, ss.~13-20.} a~inspiracją do sformułowania twierdzenia była dla Churcha najprawdopodobniej wspomniana w~poprzedniej sekcji praca Moore'a\footnote{G.E. Moore, \textit{A~reply to my critics}, dz. cyt. Zob. B. Brogaard, J. Salerno, dz. cyt. Por. rozdz.~\ref{scep-par-problemmoorea}. Fitch prawdopodobnie nie uznawał paradoksalności tego wyniku, natomiast Church podawał szereg sposobów na uniknięcie tej paradoksalności.}.

Mimo, iż paradoksu poznawalności po raz pierwszy pojawił się w~badaniach logicznych pod postacią twierdzenia \ref{scep-par-cf-fitch} (i jego dowodu), w~literaturze pod taką etykietą zdecydowanie dużo bardziej znana jest jego kontrapozycja.
%(Paradoks poznawalności) Jeśli wszystkie zdania prawdziwe są co do zasady poznawalne, to wszystkie zdania prawdziwe są w~rzeczywistości znane.\label{scep-par-cf-parpozn}
%\begin{flalign*}
%		& \parbox[t]{.65\linewidth}{ 
%		Jeśli wszystkie zdania prawdziwe są co do zasady poznawalne, to wszystkie zdania prawdziwe są w~rzeczywistości znane.} &\tag{Paradoks poznawalności}\label{scep-par-cf-parpozn}
%\end{flalign*}
\begin{tw}[Paradoks poznawalności]\label{scep-par-cf-parpozn}
Jeśli wszystkie zdania prawdziwe są co do zasady poznawalne, to wszystkie zdania prawdziwe są w~rzeczywistości znane.
\end{tw}

By udowodnić twierdzenie \ref{scep-par-cf-parpozn} należy zadać pewne (bardzo skromne logicznie) warunki na działanie modalnego operatora $K$. Musimy założyć rozdzielność $K$~względem koniunkcji:
%(M) K(A ${\wedge}$ B) {213} (KA ${\wedge}$ KA).
\begin{flalign}
& K(A \land B) \to (KA \land KA). &\tag{\textsf{M}}\label{ruleM}
\end{flalign}
Oznacza to, że ilekroć mamy wiedzę o~koniunkcji dwóch zdań, wiemy o~każdym z~jej członów. Musimy też przyjąć regułę \eqref{ruleT}, według której wiedza dotyczy zawsze zdań prawdziwych:
%$KA \to A$,
%(T) KA -{\textgreater} A,
\begin{flalign}
& KA \to A, &\tag{\textsf{T}}\label{ruleT}
\end{flalign}
co znów odpowiada klasycznej koncepcji wiedzy, wedle której jest ona \textit{prawdziwym} i~uzasadnionym przekonaniem.

Wróćmy na moment do zdania \ref{scep-par-KNT-bis} i~sytuacji teologicznego sceptycyzmu. Oba powyższe aksjomaty wystarczają, by udowodnić, że zdanie to jest fałszywe a~teologiczny sceptycyzm jest teorią sprzeczną.
%Lemat {\textbackslash}neg K(A {\textbackslash}land {\textbackslash}neg K~A)\label{scep-par-cf-lemat}
%Dowód z~belgii lub art.
\begin{lem}\label{scep-par-cf-lemat}
$\neg K(A \land \neg K~A)$
\end{lem}
\begin{proof}
\begin{flalign}
& K (A \land \neg K A) & \text{(założenie, \ref{scep-par-KNT-bis})}\label{lematK1} \\
& K A \land K \neg K A & \text{(MP \ref{lematK1}, \ref{ruleM})}\label{lematK2} \\
& K A & (\land\text{ elim. \ref{lematK2})}\label{lematK3} \\
& K \neg K A & (- | | -)\label{lematK4} \\
& \neg K A & \text{(MP \ref{lematK4}, \ref{ruleT})}\label{lematK5} \\
& \neg K (A \land \neg K A) & \text{(\ref{lematK1}, \ref{lematK3}, \ref{lematK5}, reductio ad absurdum)}\qedhere
\end{flalign}
\end{proof}
Zatem, jak pokazuje powyższy lemat, nie możemy mieć wiedzy o~zdaniach w~stylu Moore'a. Stawiając tę sprawę w~kontekście teorii Niepoznawalnego: nawet jeśli istnieją jakieś nieznane nam prawdy (o Bogu), nie moglibyśmy o~tym wiedzieć. Staje się to jasne, gdy prześledzimy dalsze fragmenty dowodu twierdzenia \ref{scep-par-cf-parpozn}. Najpierw jednak przedstawmy je w~bardziej formalnym zapisie\footnote{Przy takiej notacji twierdzeniu \ref{scep-par-cf-fitch} odpowiada wyrażenie $p \land \neg K p \vdash p \land \neg \Diamond K p$. W~badaniach wokół paradoksu poznawalności najczęściej stosuje się kwantyfikację po zmiennych zdaniowych. W~takim zapisie twierdzenia \ref{scep-par-cf-fitch} oraz \ref{scep-par-cf-parpozn} przyjmują następującą postać: $\exists p (p \land \neg K p) \vdash \exists p (p \land \neg \Diamond K p)$ oraz $\forall p (p \to \Diamond K p) \vdash \forall p (p \to K p)$. Notacja ta wprowadza na raz kwantyfikację, operatory atletyczne oraz epistemiczne, więc -- by nie powodować dodatkowego zamieszania -- korzystam z~tego, że w~obrębie przedstawianych tu rozważań można pominąć kwantyfikatory bez utraty sensu twierdzenia oraz poprawności jego dowodu.}:
%Twierdzenie $A~\to \Diamond K~A \vdash A~\to K~A$.\label{scep-par-cf-unkno}
\begin{tw}\label{scep-par-cf-unkno}
$A \to \Diamond K A \vdash A \to K A$
\end{tw}
W~powyższej formule pojawia się dodatkowy symbol -- $\Diamond$. Stosowany jest tu w~standardowy, znany z~modalnej logiki aletycznej, sposób -- na oznaczenie operatora możliwości. Dualnym do operatora możliwości jest operator konieczności ($\square$). $\Diamond p$~oznacza ,,jest możliwe, że $p$'', $\square p$ czytamy jako ,,konieczne, że $p$''. Wyrażenie o~postaci $\Diamond K p$ odczytywane jako ,,jest możliwe, by wiedzieć, że $p$'' wydaje się dobrze oddawać intuicje towarzyszące stwierdzeniu, że sąd stojący za zdaniem $p$~jest \textit{poznawalny}. Zatem kolejny raz mamy tu do czynienia z~rachunkiem multimodalnym. Tym razem rachunkiem próbującym rekonstruować sytuacje, w~których mamy do czynienia z~wiedzą, oraz możliwością i~koniecznością lub, inaczej mówiąc, rachunek ten aspiruje do oddania głębszej struktury pojęcia poznawalności oraz zależności między zdaniami poznawalnymi a~zdaniami znanymi i~prawdziwymi. Wprowadzając dwa dodatkowe symbole ponownie musimy dodać wyrażenia o~postaci $\Diamond A$~oraz $\square A$~do zbioru wyrażeń sensownych naszego nowego, multimodalnego rachunku (zakładając, że $A$ należy już do tego zbioru). By dowieść twierdzenia \ref{scep-par-cf-unkno} należy przyjąć, że zachowaniem tych dwóch operatorów rządzą znów niepozorne i~dobrze znane reguły logiki aletycznej:
%
%(Reguła wymuszania) reguła oraz
%
%$(\square \neg -\neg \Diamond) \square \neg A~\to \neg \Diamond A.$
\begin{flalign}
& \frac{A}{\square A} & \tag{RW}\label{necessitation-rule} \\[10pt]
\quad& \square \neg A~\to \neg \Diamond A. & \tag{$\square$ elim.}\label{nec-elim-rule}
\end{flalign}

Pierwsza z~wymienionych zasad każe uznać, że twierdzeniem jest wyrażenie $\square A$, o~ile $A$~jest już wcześniej udowodnionym twierdzeniem. Druga zasada mówi, że koniecznie fałszywe sądy są niemożliwe i~wynika wprost z~dualności operatorów $\Diamond$ i~$\square$. Mając te dwie zasady oraz lemat \ref{scep-par-cf-lemat} możemy podać dowód twierdzenia \ref{scep-par-cf-unkno}\footnote{Por. H. Wansing, \textit{Diamonds Are a~Philosopher's Best Friends: The Knowability Paradox and Modal Epistemic Relevance Logic}, ,,Journal of Philosophical Logic'', vol. 31 (2002), p. 593; W.H.~Holliday, \textit{Epistemic Logic and Epistemology}, [w:] \textit{Introduction to Formal Philosophy}, red. S.O.~Hansson, V.F. Hendricks, Springer, Cham 2018, p. 364.}.
%Dowód. Chyba cały na schematach.
\begin{proof}
\begin{flalign}
& A \to \Diamond K A 							& \text{(założenie)}\label{twwierF1} \\
& (A \land \neg K A) \to \Diamond K	(A \land \neg K A)		& \text{(\ref{twwierF1})}\label{twwierF2} \\
& \square \neg K(A \land \neg K~A)				& \text{(\ref{necessitation-rule}  \ref{scep-par-cf-lemat})}\label{twwierF3} \\
& \neg \Diamond  K(A \land \neg K~A)			&\text{(\ref{nec-elim-rule} \ref{twwierF3})}\label{twwierF4} \\
& \neg (A \land \neg K A)								& \text{(p. kontrapozycji \ref{twwierF2}, \ref{twwierF4})}\label{twwierF5}\\
& A \to K A  									& \text{(p. negacji implikacji \ref{twwierF5})}\qedhere
\end{flalign}
\end{proof}

Twierdzenie \ref{scep-par-cf-parpozn} i~(a, co za tym idzie, \ref{scep-par-cf-unkno}) uznawane jest za paradoksalne, ponieważ wychodząc od stosunkowo niepozornego założenia -- że wszystkie prawdy są poznawalne -- oraz wykorzystując naturalne i~logicznie niekontrowersyjne reguły rządzące operatorami $K$~oraz $\Diamond$ dochodzimy do nieoczekiwanie silnego i~trudnego do przyjęcia wniosku -- że wszystkie prawdy są znane\footnote{Podobnie paradoksalne wydaje się przejście od istnienia przypadkowej ignorancji do istnienia koniecznej niepoznawalności w~twierdzeniu \ref{scep-par-cf-fitch}.}. Inaczej mówiąc, twierdzenie to i~jego dowód wydają się paradoksalne dlatego, że zdają się sprowadzać stosunkowo słabe założenie o~poznawalności prawdy, do sytuacji wszechwiedzy.

Mimo, iż praca Fitcha została sporządzona w~celu analizy pojęcia wartości, paradoks poznawalności nigdy potem nie był używany w~podobnym kontekście. Od początku lat 80. wykorzystywany jest on najczęściej jako argument w~sporze realistów epistemicznych z~tzw. antyrealistami, którzy -- mimo iż przyjmują istnienie nieznanych sądów prawdziwych -- odrzucają ich jakąś zasadniczą niepoznawalność. Paradoks Churcha-Fitcha stosowany jest więc w~próbach obalenia tych teorii, które zakładają jakąś wersję zasady poznawalności (zgodnie z~którą wszystkie sądy prawdziwe są poznawalne). Wśród tych teorii wymienić należy choćby semantyczny realizm Michaela Dummetta, realizm wewnętrzny Hilarego Putnama, czy logiczny pozytywizm Koła Wiedeńskiego\footnote{Przegląd argumentów przeciw takim teoriom można znaleźć w: J. Salerno (red.), N\textit{ew Essays on the Knowability Paradox},-Oxford University Press, New York 2009, część II.}. Uogólniona postać paradoksu Churcha-Fitcha nadaje się także do uchwycenia innych samozwrotnych aporii, pozbawionych kontekstu poznawalności -- tworzonych także w~kontekście teologicznym\footnote{Zob. Tenże, \textit{Introduction}, [w:] Tamże, s.~4.}.

\section{Paradoks znawcy (knower's paradox)}

Genezą paradoksu znawcy są rozważania dotyczące paradoksu kata (zwanego także paradoksem niespodziewanego testu\footnote{To dwie najpopularniejsze polskojęzyczne nazwy tego problemu. W~literaturze angielskiej jest ich jeszcze więcej.}).

\subsubsection{Paradoks kata (niespodziewanego testu)}

Wyobraźmy sobie, że sędzia skazuje oskarżonego na karę śmierci, która ma zostać wykonana w~przyszłym tygodniu w~południe w~dniu, w~którym skazany nie będzie się tego spodziewał. Wydaje się, że taki wyrok nie może zostać wykonany. Skazany nie może zostać stracony w~niedzielę, bo jeśli przeżyje do sobotniego popołudnia, będzie musiał spodziewać się wykonania kary w~niedzielę. Nie może zostać stracony w~sobotę, bo jeśli przeżyje do piątkowego popołudnia a~niedzielę należy wykluczyć jako dzień wymierzenia kary, musiałby się spodziewać wykonania wyroku w~sobotę, co znów jest sprzeczne z~literą orzeczenia sądu. Z~tego samego powodu nie może zostać stracony w~piątek, czwartek, środę i~wtorek. W~końcu, wyroku nie można wykonać i~w poniedziałek, bo -- skoro skazany nie może zostać stracony we wszystkie pozostałe dni tygodnia -- musiałby się spodziewać, że karę wymierzą mu właśnie w~ten dzień.

\bigskip

Powyższa historia uważana jest za paradoksalną, ponieważ wniosek z~niej płynący przeczy naszym najogólniejszym intuicjom, które zdają się podpowiadać, że kat powinien móc zaskoczyć skazanego przeprowadzając niespodziewaną egzekucję (na przykład, dajmy na to, w~czwartek). Paradoks ten sam w~sobie jest interesującym zagadnieniem i~doczekał się rozmaitych odmian i~wariacji oraz bogatej literatury. W~przełomowej pracy David Kaplan i~Richard Montague\footnote{D. Kaplan, R. Montague, \textit{A~paradox regained}, ,,Notre Dame Journal of Formal Logic'', vol. 1 (1960), nr 3, ss.~79-90.} manipulują liczbą dni, w~których należy rozpatrywać możliwość wykonania wyroku. Okazuje się, że przy zredukowaniu tej liczby do zera, paradoks kata przeobraża się w~zdanie w~stylu paradoksu kłamcy angażujące już nie semantyczne pojęcie prawdy, lecz epistemiczne pojęcie wiedzy.
%({\textbackslash}lambda)Zdanie (lambda) jest nieznane.\label{scep-klamca}
\begin{flalign*}
		& \text{Zdanie $\lambda$ jest nieznane.} &\tag{$\lambda$}\label{scep-klamca}
\end{flalign*}

Aby pokazać autoreferencyjność tego zdania załóżmy najpierw, że \ref{scep-klamca} jest znane. Następnie -- zakładając, że obowiązuje aksjomat \eqref{ruleT} -- musimy uznać, że \ref{scep-klamca} jest prawdziwe, a~mówi ono, że \ref{scep-klamca} jest nieznane. Zatem \ref{scep-klamca} jest nieznane i~udowodniliśmy, że nasze pierwotne założenie jest fałszywe. Przyjmijmy więc, że \ref{scep-klamca} jest nieznane. Następnie -- zakładając, że o~udowodnionym zdaniu wiemy, że jest prawdziwe -- musimy uznać, że wiemy, że \ref{scep-klamca} jest nieznane, a~zatem \ref{scep-klamca} jest znane. W~ten sposób wracamy do naszego pierwotnego założenia i~domykamy pętlę błędnego koła. \ref{scep-klamca} jest znane wtedy i~tylko wtedy, gdy jest nieznane.

Już ta krótka demonstracja podpowiada, jakie reguły musimy przyjąć, by dowieść sprzeczności wywołanej przez paradoks znawcy. Podobnie, jak z~powodu zdań wyrażających paradoks kłamcy każda semantycznie zamknięta (tj. niewprowadzająca stratyfikacji języków) teoria (będąca rozszerzeniem arytmetyki pierwszego rzędu) zawierająca T-równoważności okazywała się niespójna\footnote{Por. rozdz.~\ref{sil-boch}.}, tak możemy udowodnić analogiczne twierdzenie w~kontekście epistemicznym.

\begin{tw}[Montague'a\footnote{R. Montague, \textit{Syntactical treatments of modality, with corollaries on reflexion principles and finite axiomatizability}, ,,Acta Philosophical Fennica'', vol. 16 (1963), ss.~153-167. Przedruk w: red. R.H. Thomason, \textit{Formal Philosophy: Selected Papers of Richard Montague}, Yale University Press, New Haven -- London 1974, ss.~286-302.}]\label{scep-znaw-montag}
Każda teoria (będąca rozszerzeniem arytmetyki pierwszego rzędu) zawierająca epistemiczną regułę wymuszania
\begin{flalign}
& \frac{A}{K A} & \tag{RW\textsubscript{K}}\label{Knecessitation-rule}
\end{flalign}
oraz aksjomat \eqref{ruleT}\footnote{W~rzeczywistości podobnych twierdzeń można dowodzić przy wykorzystaniu innych, także słabszych założeń. Dobry teorio-dowodowy przegląd paradoksu znawcy przedstawia: Z. Tworak, \textit{Paradoks znawcy (The Knower Paradox}), ,,Filozofia Nauki'', vol. 19 (2011), nr 3(75), ss.~29-47.}
jest sprzeczna.
\end{tw}


Dowód opiera się na tzw. lemacie o~diagonalizacji dopuszczającym możliwość sformułowania w~języku takiej teorii zdań w~stylu \ref{scep-klamca}\footnote{Dowód tego lematu naśladuje pewne fragmenty dowodu twierdzenia Gödla. Jego przedstawienie nie jest konieczne dla zrozumienia toku wywodu i~raczej zaciemniłoby przedstawiany tu obraz. Choć podobieństwo samozwrotności oraz twierdzeń Gödla jest samo w~sobie ciekawym zagadnieniem.} i~właściwie w~sposób formalny odtwarza wykazanie niespójności paradoksu znawcy\footnote{Por. J. Stern, \textit{Montague's Theorem and Modal Logic}, ,,Erkenntnis'', vol. 79 (2014), nr 3, s.~554.}.
\begin{proof}
\begin{flalign}
& \lambda \equiv \neg K \lambda 			& \text{(lemat o~diagonalizacji, paradoks znawcy)}\label{znafca1} \\
& K\lambda \to \lambda		& \text{(\ref{ruleT})}\label{znafca2} \\
& K \lambda \to \neg K \lambda				& \text{($\equiv$ elim., syl. hip., \ref{znafca1},\ref{znafca2})}\label{znafca3a} \\
& \neg K \lambda				& \text{(p. Claviusa, \ref{znafca3a})}\label{znafca3} \\
& \lambda			& \text{($\equiv$ elim., MP \ref{znafca1}, \ref{znafca3})}\label{znafca4} \\
& K \lambda								& \text{(\ref{Knecessitation-rule} \ref{znafca4})}\label{znafca5}\\
& \qquad \text{contr. \ref{znafca3}, \ref{znafca5}} 									& \nonumber\qedhere
\end{flalign}
%$\lambda \equiv \neg K~\lambda$ lemat o~diagonalizacji, paradoks znawcy
%
%$K\lambda \to \lambda$ (T)
%
%$\neg K (\lambda)\equiv$ elem., MP 1,2
%
%$\lambda\equiv$ elem., MP 1,3
%
%$K(\lambda)$RW
%
%Sprz. 3, 5
\end{proof}

Paradoks znawcy może posłużyć do podparcia wielu argumentów wytaczanych w~dyskusjach z~zakresu podstaw i~filozofii matematyki, epistemologii teorii dowodu czy filozofii umysłu\footnote{Zob. R. Sorensen, \textit{Epistemic Paradoxes}, [w:] \textit{The Stanford Encyclopedia of Philosophy}, wyd. wiosna 2022, red. E.N. Zalta, <\url{https://plato.stanford.edu/archives/spr2022/entries/epistemic-paradoxes/}>.}. Dla nas jednak jego najciekawszym zastosowaniem jest wykorzystanie go jako ilustracji teologicznego sceptycyzmu. Okazuje się, że samozwrotny, paradoksalny charakter teorii Niepojmowalnego można zredukować do prostego zdania w~tylu znawcy (\ref{scep-klamca}). W~końcu, gdy mówimy, że żadne prawdziwe zdania o~Bogu nie jest poznawalne, to zdanie wyrażające tę zasadę pośrednio mówi o~sobie, że samo nie powinno być znane.

Powyższa obserwacja wraz z~twierdzeniem \ref{scep-znaw-montag} zdają się sugerować, że do rozwikłania paradoksu Niepoznawalnego można próbować wykorzystać rozwiązania w~stylu Bocheńskiego przedstawione w~poprzedniej części pracy\footnote{Zob. rozdz.~\ref{sil-boch-nonsens}.}. Należy jednak pamiętać, że oprócz dodatkowych trudności wynikających z~faktu, że w~sytuacji epistemicznej próbujemy eksploatować rozwiązanie semantyczne, wszystkie obiekcje dyskutowane poprzednio\footnote{Zob. rozdz.~\ref{sil-boch-dyskusja}.} będą zasadne także dla takiego zastosowania.



