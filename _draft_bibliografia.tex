\clearpage
\begin{thebibliography}{99}
\addtocounter{section}{1}
\addcontentsline{toc}{section}{\arabic{section}\quad  {Literatura cytowana}}
\interliniatexowa{1.6}

\bibitem{} J.M. Bocheński, \textit{Logika religii}, tłum. S. Magala, [w:]: Tenże, \textit{Logika i
filozofia}. Wybór pism, Wydawnictwo Naukowe PWN, Warszawa 1993, s.
325$-$468.

\bibitem{} J.M. Bocheński, \textit{The Logic of Religion}, New York University
Press, New York 1965.

\bibitem{} J.M. Bocheński, \textit{Logika i filozofia. Wybór pism}, Wydawnictwo
Naukowe PWN, Warszawa 1993.

\bibitem{} J.Bowker (red.), \textit{The Oxford Dictonary of World Religions},
Oxford University Press, Oxford 1997.

\bibitem{} B. Brożek, A. Olszewski, M. Hohol (red.), \textit{Logic in Theology},
Copernicus Center Press, Kraków 2013.

\bibitem{} I. N. Bulhof, L. Kate (red.), \textit{Flight of the Gods. Philosophical
Perspectives on Negative Theology}, Fordham University Press, Fordham
2000.

\bibitem{} H. Burkhardt, B. Smith (red.), \textit{Handbook of Metaphysics and
Ontology}, t. 2, PhilosophiaVerlag, München 1991.

\bibitem{} B. Davies, \textit{Aquinas on What God is Not}, [w:] red. B. Davies,
\textit{Thomas Aquinas. Contemporary Philosophical Perspectives},
Oxford University Press Oxford 2002, ss. 227–242.

\bibitem{} B. Davies\textit{, Thomas Aquinas. Contemporary Philosophical
Perspectives}, Oxford University Press Oxford 2002.

\bibitem{} M. Durrant,\textit{ The Meaning of ‘God’ (I)}, [w:] \textit{Religion and
Philosophy}, red.  M. Warner, Cambridge University Press, Cambridge
1992, ss.~71$-$84.

\bibitem{} J.I. Gellman, \textit{The Meta-Philosophy of Religious Language},
„No\^us”, nr 11 (1971), ss.~151-161.

\bibitem{} G.W.F. Hegel, \textit{Logic. Being Part One of the Encyclopaedia of the
Philosophical Sciences}, tłum. W. Wallace. Oxford University Press,
Oxford 1975.

\bibitem{} J. Hick,\textit{ An Interpretation of Religion. Human Responses to the
Transcendent}, Yale University Press, New Haven – Londyn 1989.

\bibitem{} J. Wissink, \textit{Two Forms of Negative Theology Explained Using
Thomas Aquinas}, [w:] red. I. N. Bulhof, L. Kate, \textit{Flight of the
Gods. Philosophical Perspectives on Negative Theology}, Fordham
University Press, Fordham 2000, ss. 100$-$120.

\bibitem{} J.J. Jones, \textit{Sculpting God: The Logic of Dionysian Negative
Theology}, „Harvard Theological Review” nr 89 (1996), ss. 355–371.

\bibitem{} J.A. Lamm (red.), \textit{The Wiley-Blackwell Companion to Christian
Mysticism}, Wiley-Blackwell, Malden 2013.

\bibitem{} G. O'Collins, E.G. Farrugia, \textit{Leksykon pojęć
teologicznych i kościelnych}, tłum. J. Ożóg, B. Żak, Wydawnictwo WAM,
Kraków 2002.

\bibitem{} F. O'Rourke, \textit{Pseudo-Dionysius and the
Metaphysics of Aquinas}, EJ. Brill, Leiden – New York 1992.

\bibitem{} J. Paśniczek, \textit{Predykacja}, Copernicus Center Press, Kraków 2014.

\bibitem{} J. Perzanowski\textit{, Ontological Arguments II: Cartesian and
Leibnizian}, [w:] red. H. Burkhardt, B. Smith, \textit{Handbook of
Metaphysics and Ontology}, t. 2, PhilosophiaVerlag, München 1991,ss.
625–633.

\bibitem{} J. Perzanowski, \textit{The Way of Truth}, [w:] red. R. Poli, P.
Simons\textit{, Formal Ontology}, Kluwer Academic Publishers, Dordrecht
1996, ss. 61$-$130.

\bibitem{} R. Poli, P. Simons (red.), \textit{Formal Ontology}, Kluwer Academic
Publishers, Dordrecht 1996.

\bibitem{} Pseudo-Dionizy Areopagita, \textit{Pisma teologiczne}, tłum. M.
Dzielska, Wydawnictwo Znak, Kraków 1997.

\bibitem{} Pseudo-Dionizy Areopagita, \textit{Teologia Mistyczna}, [w:]
\textit{Pisma teologiczne}, tłum. M. Dzielska, Wydawnictwo Znak, Kraków
1997.

\bibitem{} Pseudo-Dionizy Areopagita, \textit{The Complete Work}, tłum. C.
Luibheid,. Paulist Press, New York 1987.

\bibitem{} P. Rojek, \textit{Logika teologii negatywnej}, „Pressje”, nr 29 (2012),
ss.~216-230.

\bibitem{} P. Rorem, Komentarze [w:] Pseudo-Dionizy Areopagita, \textit{The
Complete Work}, tłum. C. Luibheid,. Paulist Press, New York 1987.

\bibitem{} P. Rorem, \textit{Pseudo-Dionysius. A Commentary on the Texts and an
Introduction to Their Influence}, Oxford University Press, Oxford – New
York 1993.

\bibitem{} S. Ruczaj, \textit{Analogia i apofatyczny pazur}, „Pressje”, nr 30/31
(2012), ss. 280-282.

\bibitem{} P. Sikora, \textit{Logos Niepojęty}, Wydawnictwo Universitas, Kraków
2010.

\bibitem{} J. Słupecki, \textit{Stanisław Leśniewski’s Calculus of Names}, „Studia
Logica”, nr 3 (1955), ss. 7$-$76.

\bibitem{} C.M. Stang, \textit{Negative Theology from Gregory of Nyssa to Dionysius
the Areopagite}, [w:] \textit{The Wiley-Blackwell Companion to
Christian Mysticism}, J.A. Lamm (red.), Wiley-Blackwell, Malden 2013,
ss.~161-176.

\bibitem{} W. Stróżewski, \textit{Istnienie i sens}, Wydawnictw Znak, Kraków 1994.

\bibitem{} W. Stróżewski, \textit{Z problematyki negacji}, [w:] Tenże,
\textit{Istnienie i sens}, Wydawnictw Znak, Kraków 1994, ss.~373$-$395.

\bibitem{} Tomasz z Akwinu, św., \textit{Traktat o Bogu. Summa teologii}, tłum. G.
Kurylewicz, Z. Nerczuk, M. Olszewski, Wydawnictwo Znak, Kraków 2001.

\bibitem{} M. Warner (red.), \textit{Religion and Philosophy}, Cambridge University
Press, Cambridge 1992.

\bibitem{} Z. Wolak, \textit{Naukowa filozofia koła krakowskiego}, „Zagadnienia
Filozoficzne w Nauce”, nr.~36 (2005), ss.~97-122.

\bibitem{} J. Woleński, \textit{Theology and Logic}, [w:] \textit{Logic in
Theology}, red. B. Brożek et. al, Copernicus Center Press, Kraków 2013,
ss.~11-38.

\bibitem{} A. Zinowjew\textit{, Foundations of the Logical Theory of Scientific
Knowledge (Complex Logic)}, D. Reidel Publishing Company, Dordrecht
1967.

\bibitem{} A. Zinowjew, \textit{Logika nauki}, tłum. Z. Simbierowicz, PWN, Warszawa
1976.

\end{thebibliography}