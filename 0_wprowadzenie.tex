\chapter{Wprowadzenie}
%\addcontentsline{toc}{chapter}{Wprowadzenie} 

Logiczna analiza dyskursu teologicznego nie jest może najpopularniejszą spośród metod badawczych teorii teologicznych a~prace jej poświęcone stanowią niewielki odsetek zarówno w~literaturze z~dziedziny logiki i~filozofii języka z~jednej strony, jak i~teologii, filozofii religii czy filozofii Boga z~drugiej. Jednakże charakter badań przedstawionych w~niniejszej pracy nie jest przedsięwzięciem odosobnionym w~historii myśli. Logicznymi aspektami teologii w~sposób systematyczny zajmowali się już w~latach trzydziestych ubiegłego stulecia przedstawiciele tzw. Koła Krakowskiego -- Józef Bocheński, Jan Łukasiewicz, Jan Salamucha, Jan Drewnowski oraz Bolesław Sobociński\footnote{Zob. np. Z. Wolak, \textit{Naukowa filozofia koła krakowskiego}, ,,Zagadnienia Filozoficzne w~Nauce'', nr~36 (2005), ss.~97-122.} . Pierwszy z~nich zaznaczył się szczególnie w~historii tego typu rozważań swoją przełomową pracą \textit{The Logic of Religion}\footnote{J.M. Bocheński, \textit{The Logic of Religion}, New York University Press, New York 1965. Pierwsze polskie wydanie: J.M.~Bocheński, \textit{Logika religii}, tłum. S. Magala, Instytut wydawniczy PAX, Warszawa 1990. W~tej pracy posługuję się tym samym tłumaczeniem, zamieszczonym w~J.M. Bocheński, \textit{Logika i~filozofia}, Wydawnictwo Naukowe PWN, Warszawa 1993.} . Zapoczątkowany przez nich nurt powrócił w~poprzednim dziesięcioleciu do Krakowa jako program badawczy ,,logika w~teologii'' prowadzony w~ramach Centrum Kopernika Badań Interdyscyplinarnych. Zaowocował on szeregiem spotkań, seminariów i~konferencji, których rezultatem jest wydana w~2013 roku praca zbiorowa \textit{Logic in Theology} pod redakcją Adama Olszewskiego, Bartosza Brożka i~Mateusza Hohola\footnote{B. Brożek, A. Olszewski, M. Hohol (red.), \textit{Logic in Theology}, Copernicus Center Press, Kraków 2013. Zob. także P. Urbańczyk, \textit{Logika i~teologia}, ,,Zagadnienia Filozoficzne w~Nauce'', nr 57 (2013), ss.~143-151.}. W~niniejszej pracy\footnote{Niniejsza praca stanowi twórcze rozwinięcie artykułu mojego autorstwa: P. Urbańczyk, \textit{The logical challenge of negative theology}, ,,Studies in Logic, Grammar and Rhetoric'', vol. 54 (2018), nr 1, ss.~149-174.} będę podążał śladami wymienionych wyżej autorów. Skupię się jednak na szczególnej formie teologii zwanej teologią apofatyczną lub negatywną. Jest to bardzo osobliwy rodzaj myślenia teologicznego, co przedstawia poniższy zestaw definicji.


\begin{defin}\label{int-deftn-fi}
Teologia apofatyczna, znana także jako teologia negatywna, \textit{via negativa} albo \textit{via negationis}, jest to teologia próbująca opisać Boga, boskie dobro poprzez negację. Innymi słowy, o~Bogu można wyrażać się tylko w~formach negatywnych - tzn. mówić jaki on nie jest, a~nigdy jaki jest\footnote{Por. N. Bunnin, J. Yu, \textit{Negative theology}, [w:] ciż, \textit{The Blackwell Dictionary of Western Philosophy}, Blackwell Publishing, Malden -- Oxford 2009, ss.~465-466.}.
\end{defin}
\begin{defin}
Teologia apofatyczna to nurt teologii oparty na założeniu, że jakiekolwiek pozytywne poznanie natury Boga przekracza granice możliwości ludzkiego rozumu. Teologia apofatyczna, podkreślając niewspółmierność wszystkich poznawczych wysiłków zmierzających do opisania tajemnicy Boga, odrzuca wszelkie symbole, obrazy i~abstrakcyjne pojęcia jako nieadekwatne do opisu natury Boga i~próbuje przybliżyć Jego tajemnicę za pomocą formuł przeczących, mówiąc, jaki Bóg nie jest\footnote{Por. T. Stępień, \textit{Znany -- nieznany Bóg. Uwagi na temat rozwoju doktryny niepoznawalności Boga u~chrześcijańskich autorów od II do VI wieku}, ,,Internetowy Magazyn Filozoficzny Hybris'', vol. 20 (2013), s.~98 oraz A. Synowiecki, \textit{Od mitu o~nauce do powagi naukowej, cz. I}, ,,Studia Philosophiae Christianae'', vol. 30 (1994), nr 2, s.~257. Teksty te są źródłem definicji teologii apofatycznej umieszczonej w~polskiej edycji Wikipedii.}.
\end{defin}
\begin{defin}
Apofatyczna teologia (gr. \textit{apofatikos} -- przeczący), teologia stosująca w~poznaniu Boga metodę negacji, antynomii, paradoksu, doświadczenia i~kontemplacji mistycznej, wychodząc z~założenia, że natura Boga i~tajemnice wiary są niepoznawalne na drodze czysto rozumowej\footnote{W. Hryniewicz, \textit{Apofatyczna teologia}, [w:] \textit{Encyklopedia katolicka}, t.~1, red. F. Grylewicz i~in., TN KUL, Lublin 1989, kol. 745-748.}.
\end{defin}
\begin{defin}
Teologia apofatyczna to inna nazwa dla ,,teologii poprzez negację'', wedle której Bóg jest poznawany poprzez negowanie pojęć, które mogłyby być mu przypisane. Teologia apofatyczna podkreśla, że ludzkie pojęcia i~język są nieodpowiednim narzędziem do opisania Boga\footnote{J. Bowker, \textit{Apophatic theology}, [w:] \textit{The Concise Oxford Dictionary of World Religions}, red. tenże, Oxford University Press, Oxford 2000.}.
\end{defin}
\begin{defin}
,,Teologia negatywna'' to nazwa nadana teologicznemu\footnote{W~oryginale ,,chrześcijańskiemu''. W~mojej opinii ograniczanie teologii negatywnej do tradycji chrześcijańskiej jest nieporozumieniem.} nurtowi, wedle którego Bóg jest absolutnie transcendentny i~na tyle przekracza nadawane mu nazwy i~pojęcia, że w~rzeczywistości musimy je zanegować, by uwolnić go z~tych ciasnych kategorii\footnote{Por. C.M. Stang, \textit{Negative Theology from Gregory of Nyssa to Dionysius the Areopagite}, [w:] \textit{The Wiley-Blackwell Companion to Christian Mysticism}, red. J.A. Lamm, Blackwell Publishing, Malden -- Oxford 2013, s.~161.}.
\end{defin}
\begin{defin}
Teologia negatywna -- podobnie jak w~teologii apofatycznej podejście do Bożej Tajemnicy, które podkreśla, że raczej potrafimy powiedzieć, czym Bóg nie jest, niż czym jest w~rzeczywistości. Jest to sposób uprawiania teologii, który kładzie większy nacisk na to, co się wyraża po łacinie słowem sapientia (łac. mądrość) niż scientia (łac. wiedza)\footnote{G. O'Collins, E.G. Farrugia, \textit{Leksykon pojęć teologicznych i~kościelnych z~indeksem angielsko-polskim}, tłum. J. Ożóg, B. Żak, Wydawnictwo WAM, Kraków 2002.}.
\end{defin}
\begin{defin}
Teologia apofatyczna -- (gr. negatywny, przeczący) podstawowe pojęcie w~teologii Wschodu, często tłumaczone jako teologia negatywna. Kładzie ono nacisk na niewspółmierność wszystkich naszych wysiłków zmierzających do opisania absolutnej tajemnicy Boga. Każde nasze twierdzenie o~Bogu musi być ograniczone przez odpowiednie zaprzeczenie i~przez uznanie, że Bóg w~sposób nieskończony przekracza wszelkie nasze kategorie\footnote{Tamże.}.
\end{defin}
\begin{defin}\label{int-deftn-last}
Teologia negatywna to doktryna głosząca, że żadne afirmatywne czy pozytywne atrybuty nie mogą być orzekane o~Bogu, że Bóg jest całkowicie nieznany i~niepoznawalny, że możemy znacząco mówić wyłącznie, jaki on nie jest (mówić o~nim w~terminach negatywnych); doktryna, według której najwyższą wiedzą o~Bogu jest to, że nie jesteśmy zdolni posiąść o~nim żadnej wiedzy\footnote{I. Franck, \textit{Maimonides and Aquinas on Man's Knowledge of God: A~Twentieth Century Perspective}, ,,Review of Metaphysics'', vol. 38 (1985), nr 3, s.~593.}.
\end{defin}

Przymiotnik \textit{negativus}, który pojawia się w~\textit{via negativa} jest łacińską wersją greckiego przymiotnika \textgreek{apofatikos}, pochodzącego od \textgreek{apofasis}, które tłumaczone jest jako odwołanie (mowy, swoich słów), zaprzeczanie lub -- najczęściej -- jako negacja.

Mimo, iż teologia apofatyczna nigdy nie należała do głównego nurtu teologicznego i~filozoficznego myślenia o~Bogu, w~toku historii miała całkiem pokaźne grono protagonistów:

\begin{itemize}
\item Prekursorów myślenia apofatycznego należy szukać w~klasycznej filozofii greckiej wśród Platona i~myślicieli neoplatońskich, takich jak Platon, Plotyn, Proklos czy Damascjusz.
\item W~tradycji chrześcijańskiej za reprezentantów teologii negatywnej często uznaje się Orygenesa, Cyryla Jerozolimskiego, Ojców Kapadockich (Bazylego Wielkiego, Grzegorza z~Nazjanzu), Jana Damasceńczyka, Maksyma Wyznawcę, Jana Chryzostoma, św. Jana od Krzyża czy Marcina Lutra, jednakże do ,,pełnokrwistych'' teologów negatywnych z~pewnością należy zaliczyć m.in. Klemensa Aleksandryjskiego, Grzegorza z~Nyssy, Pseoudo-Dionizego Areopagitę, Mistrza Eckharta, Mikołaja z~Kuzy czy Tomasza z~Akwinu\footnote{Umieszczenie Tomasza z~Akwinu wśród ,,pełnokrwistych'' teologów apofatycznych może wydać się kontrowersyjne, jednakże analiza jego pism i~źródeł inspiracji nie pozostawia co do tego żadnych wątpliwości. Por. P. Sikora, \textit{Logos niepojęty}, Wydawnictwo Universitas, Kraków 2012, s.~79-87.}. Tradycja apofatyczna rozpowszechniona była zwłaszcza w~Kościołach wschodnich. Do dziś jest bardzo silna wśród wielu współczesnych teologów prawosławnych\footnote{Zob. np. T. Obolevitch, \textit{Teologia negatywna a~nauka w~ujęciu Siemiona Franka}, ,,Zagadnienia Filozoficzne w~Nauce'', nr 42 (2008), ss.~68-77, Taż, \textit{Katafatyczny wymiar wschodniochrześcijańskiej teologii apofatycznej. Wokół propozycji S. Bułgakowa}, ,,Logos i~Ethos'', vol. 21 (2006) nr 2, ss.~56-63, Taż, \textit{Filozofia rosyjskiego renesansu patrystycznego}, Copernicus Center Press, Kraków 2014.}.
\item Pewne wersje apofatycznej doktryny można odnaleźć bez większych problemów także w~tradycji żydowskiej: Filon z~Aleksandrii, Bahya ibn Paquda, Mojżesz Majmonides; islamskiej i~sufistycznej: Ibn Arabi, Wasil ibn Ata, Avicebron; a~także buddyjskiej i~hinduistycznej i~innych tradycjach religijnych\footnote{Zob. np. A. Kars, \textit{What is ``negative theology''? Lessons from the encounter of two sufis}, ,,Journal of the American Academy of Religion'', vol. 86 (2016), nr 1, ss.~181-211, E.~Altaş, A.~Asadov, \textit{The Power and Limits of Reason: Al-Razı on the Possibility of General and Particular Metaphysical Knowledge}, ,,Nazariyat İslam Felsefe ve Bilim Tarihi Araştırmaları Dergisi (Journal for the History of Islamic Philosophy and Sciences)'', vol. 7 (2021), nr 2, s.~121-155. Bardzo dobry przegląd apofatyzmu w~obrębie różnorodnych tradycji religijnych można odnaleźć w~pracy T.D.~Knepper, L.E.~Kalmanson (red.), \textit{Ineffability: An Exercise in Comparative Philosophy of Religion}, Springer, Cham 2017.}.
\end{itemize}
Teologia apofatyczna może być rozumiana jako przeciwieństwo teologii katafatycznej (pozytywnej, afirmatywnej).


\begin{defin}
Teologia katafatyczna (gr. \textgreek{katafasic}, \textgreek{katafatikoc}, \textgreek{katafatika} -- twierdzący, orzekający w~sposób pozytywny, afirmacja), nazywana też ,,teologią pozytywną''. Nurt teologii oparty na założeniu, że człowiek jest w~stanie pozytywnie wypowiadać się ,,jaki Bóg jest''. Teologia katafatyczna rozumiana jest często jako przeciwieństwo lub uzupełnienie teologii apofatycznej\footnote{Por. W. Hryniewicz, \textit{Katafatyczna teologia}, [w:] \textit{Encyklopedia katolicka}, t.~8, red. F. Grylewicz i~in., TN KUL, Lublin 2000, kol. 976-978.}.
\end{defin}
\begin{defin}
Teologia katafatyczna (gr. twierdzący, pozytywny) -- pojęcie dopełniające do teologii apofatycznej, czyli negatywnej, właśnie dlatego nazywane czasami ,,teologią pozytywną''. Mimo że kategorie poznawcze, którymi się kierujemy, są u~samych swoich podstaw niewystarczające, możemy wypowiadać wiele prawdziwych twierdzeń o~Bogu, który się nam objawił w~sposób niezrównany w~Jezusie Chrystusie i~dał się nam poznać teraz przez Ducha Świętego. Teologia apofatyczna jednak kładzie nacisk na to, że nawet mimo faktu, iż Bóg sam siebie nam objawił i~sam siebie nam darował, pozostaje On dla nas zasadniczą tajemnicą\footnote{G. O'Collins, E.G. Farrugia, \textit{Leksykon pojęć teologicznych}\ldots, dz. cyt.}.
\end{defin}

Między tymi dwoma nurtami myślenia teologicznego istnieje wiele odmienności. Po pierwsze, teologia negatywna dużo częściej (choć nie zawsze) posiada charakter mistyczny, ponieważ jej negacje mają służyć głównie jako droga do spotkania z~transcendentnym Bogiem. Po drugie, neguje ona nie tylko wszystkie twierdzenia dane na gruncie teologii pozytywnej, lecz także -- krótko mówiąc -- neguje swoje własne negacje\footnote{Por. D. Brylla, \textit{Rozważania o~apofatycznej kategorii ,,negacja negacji}'', ,,Seminare'', vol. 38 (2017), nr 1, ss.~65-76.}. Tym samym teologia negatywna wydaje się być na pierwszy rzut oka teorią sprzeczną\footnote{Por. rozdz.~\ref{sil-int-par}.}. By ukazać, jak wymagającym zadaniem jest zajmowane się tym rodzajem teologii z~logicznej perspektywy, posłużmy się następującym cytatem z~Pseudo-Dionizego Areopagity:

\begin{quote}
Bóg nie jest ani duszą, ani intelektem, ani wyobrażeniem, ani mniemaniem, ani pojmowaniem, ani słowem i~pojmowaniem; [\ldots] nie jest bez ruchu ani w~ruchu, ani nie odpoczywa. [\ldots] Nie jest [\ldots] ani Boskością, ani dobrocią [\ldots]. Nie jest też niczym z~niebytu ani czymś z~bytu. [\ldots] Nie istnieje ani słowo, ani imię, ani wiedza o~Nim. [\ldots] Jest ponad wszelkim twierdzeniem i~ponad wszelkim zaprzeczeniem.\footnote{Pseudo-Dionizy Areopagita, \textit{Teologia Mistyczna}, [w:] \textit{Pisma teologiczne}, tłum. M. Dzielska, Wydawnictwo Znak, Kraków 2005, s.~167-168.}
\end{quote}

Jak zauważa Paweł Rojek\footnote{Zob. P. Rojek, \textit{Logika teologii negatywnej}, ,,Pressje'', nr 29 (2012), s.~216.}, wydaje się, ze Dionizy nie tylko narusza pewne fundamentalne zasady teologii katafatycznej, lecz także podstawowe prawa logiczne. Twierdząc, że Bóg nie jest ani niebytem, ani bytem, Dionizy wydaje się ignorować prawa wyłączonego środka. Oczywiście istnieją ugruntowane w~literaturze i~dobrze uzasadnione filozoficznie rachunki logiczne, które odrzucają zasadę wyłączonego środka\footnote{Do najważniejszych z~nich należy logika intuicjonistyczna.}, jednak lista naruszanych przez tego myśliciela praw na tym się nie kończy. Twierdzenie, że Bóg nie jest Boskością każe sądzić, że Dionizy podważa prawo tożsamości. Ponadto, w~sposób oczywisty zaprzecza samemu sobie utrzymując, że ,,nie ma żadnego mówienia o~Bogu'', podczas gdy sam się tego zadania podejmuje.

Rzeczywiście, nie ma wielkiej przesady w~twierdzeniu, że teologia negatywna jest jedną z~najbardziej mistycznych, niespójnych i~niejasnych teorii Boga, jakie kiedykolwiek powstały\footnote{Por. Tamże. }. Z~tego powodu niektórzy filozofowie, filozofowie logiki, lub teologowie mogliby utrzymywać, że ten rodzaj teologicznego myślenia nie posiada żadnej teoretycznej wartości a~zdania takiej teologii powinny być traktowane raczej jako tzw. performatywny: modlitwa, hymn uwielbienia, wyraz czci lub środek prowadzący do zjednoczenia z~Bogiem\footnote{Takie stanowisko jest popularne wśród myślicieli prawosławnych takich, jak Władimir Łosski. Utrzymują je także Andrew Louth i~Paul Rorem. Zob. Tamże, s.~217.}. Mogliby oni stwierdzić, że logika formalna nie będzie w~ogóle użyteczna w~wyjaśnianiu i~precyzowaniu takiej teologii a~każda próba odkrycia jej logicznej struktury i~formalizacji z~góry skazana jest na niepowodzenie. Na przykład Jan Woleński wprost pisze, że

\begin{quote}
Owa teologia [negatywna -- P.U.] utrzymuje, że nie możemy stwierdzić niczego pozytywnego o~Bogu i~jego własnościach. Powinniśmy powstrzymać się od stwierdzeń pozytywnych i~ograniczyć się do takich zdań, jak: ,,Nie wiem, jaki Bóg jest, ani jaki nie jest\ldots''. Według tego rodzaju teologii, luka poznawcza wyłaniająca się z~takich stwierdzeń jest w~wystarczającym stopniu wypełniona poprzez przekonanie rozumiane jako wiara religijna. Jeśli wierzymy, nie musimy przejmować się oczywistymi sprzecznościami w~zbiorze zdań teologicznych. [\ldots] Niewątpliwie, logika nie spełnia zadniej istotnej roli w~teologii negatywnej, która nie jest szczególnie zainteresowana argumentami\footnote{J. Woleński, \textit{Theology and Logic}, [w:] \textit{Logic in Theology}, red. B. Brożek i~in., dz. cyt., ss.~11-12.}.
\end{quote}
W~tej pracy przyjmuję odmienne stanowisko. Uważam, że także teologia negatywna może być badana za pomocą narzędzi logicznych (w tym narzędzi logiki formalnej) a~wynik tego badania może być owocny i~ciekawy filozoficznie.

Przyjmując takie stanowisko wypada w~tym miejscu doprecyzować, co w~niniejszej pracy będzie rozumiane przez termin ,,logika'' i~w jakim sensie wyniki tego typu badań mogą być użyteczne lub chociaż intersujące z~filozoficznego (lub teologicznego) punktu widzenia. Paradoksalnie, zadanie to może okazać się nawet trudniejsze niż podanie definicji teologii negatywnej. Nie dlatego, że definicjom logiki brakuje precyzji, lecz dlatego, że jest ich bardzo wiele -- ,,logika'' jest terminem wieloznacznym i~to wieloznacznym na różnych płaszczyznach. Także w~tej pracy jest on używany w~różnych znaczeniach. Dla równowagi i~w tym przypadku rozpocznę od podania kilku rzeczywistych przykładów użycia definicji tego pojęcia.


\begin{defin}
Logika jest formalną ekspozycją intuicji\footnote{Ten \textit{bon mot} pochodzi od Stanisława Leśniewskiego. Zob. J. Woleński, \textit{Czy Leśniewski był filozofem}?, ,,Filozofia Nauki'', vol. 8 (2000), nr 3-4, s.~66.}.
\end{defin}
\begin{defin}
Logika to dyscyplina zajmująca się analizą zdań czy też sądów i~dowodów biorąc pod uwagę ich formę, natomiast abstrahując od ich przedmiotu\footnote{Por. A. Church, \textit{Introduction to Mathematical Logic}, Princeton University Press, Princeton -- New Jersey 1956, s.~1.}.
\end{defin}
\begin{defin}
Słowa ,,logika'' używa się jako nazwy dla nauki, która analizuje znaczenie pojęć wspólnych dla wszystkich nauk oraz ustala ogólne prawa rządzące tymi pojęciami\footnote{Por. A. Tarski, \textit{Podstawowe pojęcia metodologii nauk dedukcyjnych}, [w:] tenże, \textit{Pisma logiczno-filozoficzne, t.~2 : Metalogika}, tłum. J. Zygmunt, Wydawnictwo Naukowe PWN, Warszawa 2001, s.~XV.}.
\end{defin}
\begin{defin}
Logika zajmuje się badaniem obiektów logicznych (stałych, wartości prawdziwościowych, możliwych światów, sądów, klas, własności, relacji itp.) oraz sposobów, w~jaki obiekty te mogą zostać skonstruowane na podstawie innych takich obiektów\footnote{P. Tichy, \textit{Questions, Answers, and Logic}, ,,American Philosophical Quarterly'', vol. 15 (1978), nr 4, s.~275.}.
\end{defin}
\begin{defin}
Logika to nauka zajmująca się badaniem rozmaitych ogólnych typów dedukcji\footnote{Por. B. Russell, \textit{The Principles of Mathematics}, vol. I, Cambridge University Press, Cambridge 1903, s.~10.}.
\end{defin}


Jak już wspomniałem, w~niniejszej pracy pojęcie logiki może przybrać jedno z~kilku znaczeń:
\begin{itemize}
\item Logika może być rozumiana jako rachunek logiczny, czyli pewna struktura z~dobrze określną syntaktyką i~semantyką: językiem, alfabetem, definicją wyrażenia sensownego, zestawem aksjomatów, interpretacją, modelem i~często dowodami twierdzeń o~trafności i~pełności. W~takim sensie mówimy na przykład o~logice klasycznej i~logikach nieklasycznych: intuicjonistycznej, modalnej, epistemicznej, aletycznej, parakonsystentnej i~innych. Definicję logiki w~tym rozumieniu można doprowadzić do pewnego ekstremum wykorzystując w~definensie tego pojęcia operator konsekwencji\footnote{W~takim (ekstremalnym) ujęciu przez logikę rozumie się parę uporządkowaną $\langle \mathcal{L}, Cn\rangle$, gdzie $\mathcal{L}$~jest językiem (zasadniczo zdaniowym) a~$Cn$ operacją konsekwencji zdefiniowaną na zbiorze potęgowym $\mathcal{P}(\mathcal{L})$ spełniającą określone własności. Por. np. J.~Czelakowski, A.~Olszewski, \textit{Logics of Order and Related} \textit{Notions}, ,,Studia Logica'' 2022, praca przyjęta do publikacji (w opracowaniu wydawniczym).}.
\item W~sensie \textit{ogólnym} lub \textit{szerokim} przez logikę rozumiem tutaj naukę badającą warunki poprawności wnioskowań. Dodać należy, że w~szerokim rozumieniu badania te mogą odbywać się przy użyciu metod nieformalnych i~filozoficznych.
\item Przez logikę formalną\footnote{Logikę formalną nazywa się czasem ,,symboliczną'', ,,matematyczną'' lub ,,teoretyczną''. Jednakże zdarza się również, że przynajmniej pewną część tych terminów definiuje się rozłącznie.} rozumiem w~tej pracy dyscyplinę badającą poprawność wnioskowań ścisłymi i~formalnymi metodami. Logika formalna powstała w~celu zbadania podstaw matematyki. Podejście to charakteryzuje się wykorzystaniem sztucznego i~symbolicznego języka, dążeniem do jak największej precyzji, ostrymi definicjami, ścisłym reżimem matematycznej poprawności oraz bogactwem matematycznych struktur -- na jej gruncie w~ramach modelu jakiegoś zagadnienia bądź teorii można stosować na przykład algebrę (najczęściej boolowską), teorię mnogości, teorię porządku, topologię, mereologię, teorię kategorii i~wiele innych struktur.
\end{itemize}
Mimo iż to teorie matematyczne są pierwotnym przedmiotem badań logiki formalnej, jej zastosowanie bynajmniej nie ogranicza się do tak wąskiego zakresu. Może być ona z~powodzeniem stosowana w~badaniach teorii filozoficznych czy -- jak pokazuje ta praca -- teologicznych. Mówimy wtedy o~analizie logicznej, czyli metodzie polegającej na zastosowaniu środków logicznych do kontroli prawdziwości lub sensowności sądów i~twierdzeń takich teorii a~także do zbadania poprawności obecnych w~niech rozumowań i~dedukcji. Ważnym elementem analizy logicznej jest formalna rekonstrukcja. Polega ona na wyrażeniu analizowanych teorii lub jakichś ich wyróżnionych fragmentów (na przykład jakiegoś rozumowania, kluczowego wywodu lub zbioru najważniejszych tez) w~języku wybranego systemu formalnego. Rekonstrukcję formalną często poprzedza próba wyjaśniania pojęć takich teorii w~terminach zaczerpniętych z~logiki (formalnej lub szeroko pojętej). Cytowany powyżej Woleński w~ten sposób pisze o~analizie logicznej:

\begin{quote}
Czy jest to metoda uniwersalna? Nie jest, a~nawet nie może być, ponieważ każda formalna rekonstrukcja musi być poprzedzona wyeksplikowaniem intuicji, jakie chce się wziąć pod uwagę. Bywa i~tak, że na zebraniu i~uporządkowaniu materiału intuicyjnego trzeba poprzestać\footnote{J. Woleński, \textit{Metateoretyczne problemy epistemologii}, ,,Diametros'', vol. 6 (2005), s.~72.}.
\end{quote}
Oczywiście, można pomyśleć o~różnorodnych przeszkodach i~powodach, dla których logiczna analiza jakiejś teorii zatrzymuje się w~takim miejscu. Jednym z~nich może być sama natura analizowanych w~ten sposób rozważań, co Woleński zdaje się sugerować o~teologii apofatycznej w~poprzednio przywoływanym cytacie.

Mimo, iż zadanie formalnej rekonstrukcji tego typu rozważań na pierwszy rzut oka nie wygląda zbyt obiecująco, istnieją autorzy, którzy uznają, że teologia negatywna posiada pewną teoretyczną (a nie tylko duchową) wartość a~w literaturze można spotkać co najmniej kilka prób zachowania spójności tej teorii. W~niniejszej pracy próby te zostały przedstawione, sklasyfikowane i~poddane dyskusji a~w~wyniku tych badań zdiagnozowany został podstawowy logiczny problem, z~jakim mierzą się autorzy takich propozycji. Problem ten identyfikuję w~samoodniesieniu charakteryzującym myślenie apofatyczne (zwanym także samozwrotnością lub autoreferencją)\footnote{Zob. T. Bolander, \textit{Self-Reference}, [w:] \textit{The Stanford Encyclopedia of Philosophy}, wyd. jesień 2017, red. E.N. Zalta, {\textless}https://plato.stanford.edu/archives/fall2017/entries/self-reference/{\textgreater}. Zob także rozdz.~\ref{sil-int-par}.} -- popularnym sposobie generowania paradoksów, z~których najbardziej znanym jest paradoks kłamcy. Identyfikację tego problemu i~wskazanie samozwrotności w~różnych logicznych aspektach teologii apofatycznej można uznać za dodatkowy wynik tej pracy.

Przedstawienie samozwrotnych paradoksów apofatycznego myślenia o~Bo\-gu (wraz z~analizą prób ich usuwania lub obchodzenia) zostało w~kolejnych rozdziałach uporządkowane według klucza wyznaczonego przez najpopularniejsze interpretacje tej doktryny oraz związane z~nimi aspekty logiczne. W~świetle definicji \ref{int-deftn-fi} -- \ref{int-deftn-last} teologię negatywną można rozumieć m.in. jako doktrynę, według której ludzki język nigdy nie będzie dostatecznie zdolny do tego, by wyrazić transcendentną naturę Boga. Bóg leży ,,poza'' i~,,ponad'' niedoskonałymi pojęciami ludzkiego języka i~ostatecznie żadnego z~nich nie można mu przypisać. Modelowym reprezentantem tego rodzaju myślenia jest Pseudo-Dionizy Areopagita. Taką interpretację teologii apofatycznej (szczegółowo przedstawioną w~rozdziale \ref{sil-general}) nazywam teologią milczenia, ponieważ głosi ona, że o~Bogu nic nie możemy powiedzieć a~jakiekolwiek środki służące do jego opisu są niewspółmierne i~nieadekwatne. W~konsekwencji Bóg pozostaje niewysławialny, niewyrażalny i~nieopisywalny. Teoria ta akcentuje językowy wymiar transcendencji Boga i~pozwala ukazać teologię apofatyczną w~aspekcie semantycznym. To do tej interpretacji najczęściej odwołują się autorzy, których celem jest obrona apofatycznej doktryny przed zarzutami o~bycie teorią logicznie sprzeczną. Teologia milczenia zaangażowana jest bowiem w~paradoks Niewyrażalnego analogiczny do semantycznego paradoksu kłamcy: skoro twierdzimy, że o~Bogu nic nie da się powiedzieć, właśnie o~nim coś powiedzieliśmy, przeczymy zatem sami sobie.

%Szereg autorów próbuje uchronić tę doktrynę przed paradoksem stosując różne metody i środki oraz odwołując się do odmiennych koncepcji i pojęć.
Niektórzy z~autorów -- tacy jak np. John J. Jones oraz John Hick -- próbują obronić spójność teologii apofatycznej nie wykorzystując w~tym celu żadnych zaawansowanych środków formalnych.
%Jones przekonuje, że apofatyczne rozważania Dionizego mają na celu wyraźne wyeksponowanie transcendencji Boga i ostre oddzielenie go od kategorii bytów.
W~filozoficznej analizie tekstów Dionizego Jones stosuje czasem pojęcia zakresu logiki, ale (jak się wydaje) w większości przypadków w~znaczeniach odmiennych od tych znanych z~logicznych rozważań. Jego obrona polega uznaniu orzeczenia „jest” w wypowiedziach Dionizego za wyrażenie metaforyczne. Mimo wstępnych deklaracji, nie pozbywa się paradoksu ciążącego na tej doktrynie, lecz godzi się na niego nazywając go nawet ,,tautologią'' (\textit{sic}!): Bóg jest niepojmowalny dlatego, że nie możemy pojąć jego niepojmowalności.

O~ile teologia apofatyczna w~rekonstrukcji Jonesa jest tylko nieco bardziej klarowna od pierwotnej treści oryginalnych pism teologów negatywnych, o~tyle Hick podchodzi do zagadnienia apofatyzmu w~sposób nieco bardziej metodyczny.
Na paradoksalny charakter tej teorii napotyka w badaniach fenomenu pluralizmu religijnego. W~celu pozbycia się paradoksu stosuje strategię polegającą na podzieleniu zbioru własności na dwie klasy -- klasę własności substancjalnych oraz klasę własności formalnych --  i~ograniczeniu orzekania o Bogu tylko do tej drugiej. Przy takiej strategii własność ,,taki, że nie przysługują mu własności substancjalne'' musi należeć do  dopuszczonej w~mowie o Bogu klasy własności formalnych. Jednakże, jak pokazują krytycy Hicka, zaproponowany przez niego podział jest co najmniej niejasny albo -- w~najlepszym razie -- przebiega w~nieodpowiednich miejscach. Okazuje się także, że taka koncepcja uwikłana jest w~ten sam rodzaj cyrkularności myślenia, którego miała się pozbywać.
%W~ramach rozważań o~pluralizmie religijnym proponuje on tzw. słabą teorię Niewysłowionego, w~obrębie której dzieli on własności na dwie klasy -- klasę własności substancjalnych oraz klasę własności formalnych -- i~postuluje ograniczenie mówienia o~Bogu wyłącznie do tej drugiej.
Propozycjom tym poświęcone są odpowiednio rozdziały \ref{sil-jones} oraz \ref{sil-slabatn}.

Szeroko dyskutowana praca Hicka jest częstym źródłem inspiracji dla innych badaczy zajmujących się teologią negatywną -- także takich, którzy w~obrębie swoich analiz wykorzystują już pewne podstawowe narzędzia logiczne. Jednym z~takich badaczy jest Peter Kügler.
Punktem wyjścia jego badań jest metafora ,,ciemności'', którą uważa za centralną ideę apofatyzmu. W~swoich analizach ostatecznie sprowadza ją do tezy o boskiej niewysławialności, którą próbuje wyrazić w półformalny sposób w~postaci dwóch zasad -- uniwersalnej i egzystencjalnej zasady teologii negatywnej. On sam, z~pewnego osobliwego powodu preferuje tę drugą -- głoszącą, że nie istnieje żadna własność
taka, że Bóg ją posiada lub nie posiada.
%, którą można by było wprost lub w zanegowanej postaci przypisać Bogu.
Obie zasady stoją w sprzeczności z~prawem wyłączonego środka. Logika intuicjonistyczna, która wydaje się naturalnym kandydatem na model dla takich rozważań zostaje jednak wykluczona z powodu posiadania pewnej niepożądanej własności (dopuszcza równoważność obu zasad). W~świetle braku odpowiedniego modelu trudno mówić o~zachowaniu spójności w~ramach w~ten sposób analizowanej teorii. Dodatkowo teologia apofatyczna rozumiana w myśl zarówno uniwersalnej, jak i~egzystencjalnej zasady, nie staje się wcale wolna problemu samozwrotności.
%Tajemniczość Boga, obecną w~pismach teologów negatywnych, wyjaśnia on w~kategoriach metafory ciemności, którą uważa za centralną ideę apofatycznego mistycyzmu. Próbuje on argumentować, że możliwe jest niesprzeczne utrzymanie ,,ciemności'' jako metafory Boga dzięki przyjęciu jednej z~dwóch zasad: uniwersalnej bądź egzystencjalnej zasady teologii negatywnej. On sam preferuje tę drugą -- głoszącą, że nie istnieje żadna własność taka, że Bóg ją posiada lub nie posiada. Postuluje jednak z~zasięgu oddziaływania tej zasady usunąć własność wyrażoną przez nią samą, co nazywa strategią samowykluczenia.
Pomysł Küglera poddany jest krytyce w~rozdziale \ref{sil-kugler}.

Bardzo podobną strategię radzenia sobie z antynomiami teologii apofatycznej proponuje Jerome I.~Gellman, choć jego punkt wyjścia jest nieco inny -- 
%Bardzo podobna próba uniknięcia paradoksu Niewysłowionego dyskutowana jest w~rozdziale \ref{sil-gell}. Jej protagonista, Jerome I. Gellman,
ocenia on adekwatność różnych teorii języka religijnego. W~jego badaniach teologia apofatyczna stanowi jedną z~takich teorii, w~ramach której jakikolwiek predykat języka ,,skończonych bytów'' nie może być prawdziwie orzekany o~Bogu. By uprawomocnić to twierdzenie Gellman wprowadza (semantyczne) pojęcie przedmiotowego zakresu odniesienia predykatu. W~jego rozumieniu główna teza teologii milczenia głosi, że Bóg nie należy do przedmiotowego zakresu odniesienia żadnego z~predykatów ludzkiego języka. Ta teza stanowi przedmiot mojej krytyki. Jednakże Gellman sam ostatecznie odrzuca teologię negatywną jako teorię oferującą nieadekwatną analizę znaczenia języka religijnego.

W~ostatnich latach pewne ożywienie zainteresowania analitycznym podejściem do teologii apofatycznej wywołała praca Jonathana D. Jacobsa, w~której broni on spójności teologii milczenia angażując metametafizyczne pojęcie fundamentalności. Na te potrzeby tworzy on pewien niestandardowy rachunek modalny, w~obrębie którego konstruuje tezę głoszącą, że nie istnieje żaden prawdziwy fundamentalny sąd o~Bogu. Sytuację teologii apofatycznej obrazuje on przy użyciu metafory ,,teologicznego pokoju'' w~którym mielibyśmy się ograniczyć do wypowiadania wyłącznie sądów fundamentalnych. O~tym, że nie da się ,,wejść'' do takiego pokoju przekonuję w~rozdziale \ref{sil-jac}.

Wedle mojej najlepszej wiedzy pierwszym logikiem, który próbował zmierzyć się z~paradoksalnym charakterem doktryny apofatycznej jest Józef Maria Bocheński. Co więcej, należy przyznać, że jego rozważania są najzupełniej trafne. Skoro teoria Niewysławialnego produkuje samozwrotne paradoksy przypominające paradoks kłamcy, w~celu usunięcia tej samozwrotności rozsądnym wydaje się zaangażowanie znanych strategii pozbywania się antynomii semantycznych. Proponuje on zatem rozwiązanie przypominające klasyczną teorię prawdy Alfreda Tarskiego, w~której centralnym pojęciem jest już nie prawda czy też prawdziwość zdań, lecz (nie)wysławialność terminów. Problemy nękające to rozwiązanie sygnalizuję w~rozdziale \ref{sil-boch}. Ostatecznie jednak również sam Bocheński odrzuca teologię milczenia z~innych, pozalogicznych powodów.

Ciekawa próba logicznej rekonstrukcji tego typu myślenia teologicznego została omówiona w~rozdziale \ref{sil-pozytywna}. Jej autor, Paweł Rojek, umieszcza ją w~ramach całej typologii różnych interpretacji doktryny Pseudo-Dionizego Areopagity, do których próbuję dopasować pewne modele formalne. Swojej własnej propozycji Rojek nadaje etykietę ,,pozytywnej teologii negatywnej'' postulując w~jej ramach istnienie wyróżnionego zjawiska językowego, które nazywa ,,pozytywną negacją''. Podaje on także zestaw aksjomatów mających formalnie wyznaczać znaczenie nowego, postulowanego w~ten sposób spójnika, choć -- jak się okazuje -- można podawać w~wątpliwość ich niesprzeczność i niezależność. Praca Rojka ma charakter szkicowy i~przygotowawczy. Jednakże pewne ustalenia w~niej zawarte dały asumpt do wyróżnienia w~niniejszej pracy kolejnej interpretacji i~aspektu teologii negatywnej.

Przytoczony powyżej zestaw definicji \ref{int-deftn-fi} -- \ref{int-deftn-last} dopuszcza także taką interpretację apofatycznej doktryny, w~myśl której Bóg jest nie tyle niewysławialny, niewyrażalny i~nieopisywalny, co niepoznawalny, niepojmowalny i~niemożliwy do zrozumienia. Krótko mówiąc, interpretacja ta przesuwa apofatyczne akcenty z~języka na poznanie i~z mówienia na myślenie. Określam ją mianem teologicznego sceptycyzmu, ponieważ w~jej ramach kwestionuje się możliwość zdobycia i~posiadania jakiejkolwiek wiedzy o~Bogu. Akcentuje ona poznawczy wymiar boskiej transcendencji oraz pozwala ukazać teologię apofatyczną w~aspekcie epistemicznym. Przykładem myśliciela uprawiającego teologię apofatyczną w~oderwaniu od jej językowego komponentu jest Mojżesz Majmonides. Sceptycyzm teologiczny, podobnie jak teologia milczenia, uwikłany jest samozwrotny paradoks Niepoznawalnego: skoro utrzymujemy, że Bóg jest niepoznawalny, ostatecznie coś o~nim wiemy -- mianowicie to, że jest niepoznawalny. Podobieństwa i~różnice między tymi dwoma interpretacjami teologii negatywnej -- także w~szerszym, filozoficznym kontekście -- szczegółowo omawiam w~rozdziale \ref{scep}. Natomiast w~kolejnym, \ref{scep-par} rozdziale argumentuję, że gdy do formalnej rekonstrukcji teologicznego sceptycyzmu zaprzęgniemy logikę epistemiczną, przybierze on postać znanych paradoksów epistemicznych: problemu Moore'a, paradoksu Churcha-Fitcha oraz paradoksu znawcy. Tę obserwację również można uznać za dodatkowy wynik niniejszego opracowania. Pozwalają one sądzić, że logiczne problemy teologii apofatycznej mają tę samą samozwrotną strukturę niezależnie od tego, czy rozpatrywane są w~jej aspekcie semantycznym czy epistemicznym.

Wielu komentatorów podkreśla, że mistycyzm teologii apofatycznej jest spuścizną po neoplatońskiej szkole filozoficznej późnego antyku. Neoplatończycy na szczycie ściśle zhierarchizowanej ontologii umieszczali absolutną, niepodzielną, konieczną i~niewysławialną Jednię. Rzeczywiście, ich metafizyczne rozważania do złudzenia przypinają \textit{mutatis mutandis} apofatyczne przemyślenia teologów, a~pewne fragmenty ich pism byłyby wręcz nieodróżnialne, gdyby zamiast filozoficznego pojęcia Jedni użyć w~nich teologicznego terminu ,,Bóg''. O~neoplatońskich źródłach teologii negatywnej zdaję sprawę w~rozdziale \ref{neopl}, co daje mi pretekst do zreferowania i~poddania krytyce dosyć egzotycznej formalnej rekonstrukcji tej doktryny w~ramach algebry zbioru potęgowego. W~obrębie tej rekonstrukcji zbiór potęgowy uporządkowany jest relacją inkluzji a~Bóg utożsamiony jest z~elementem największym takiej struktury. Wydaje się jednak, że przy pewnym niepozornym z~punktu widzenia intuicji stojących za tą rekonstrukcją założeniu można w jej obrębie wykazać pewien rodzaj sprzeczności przypominającej antynomię Russella.

W~kontekście powyższych wyjaśnień kryterium wyboru logicznych aspektów teologii apofatycznej staje się oczywiste -- wybrane zostały te aspekty, które są najsilniej związane z~najpopularniejszymi interpretacjami tego rodzaju teologicznego myślenia. Zatem zasadniczo wyróżniam trzy takie interpretacje, które nazywam odpowiednio teologią milczenia, teologicznym sceptycyzmem oraz mistycyzmem o~pochodzeniu neoplatońskim a~w ich obrębie przedstawiam teologię negatywną w~aspekcie semantycznym, epistemicznym oraz teorio-porządkowym. Dodatkowo, do szczegółowych analiz wybrane zostały te aspekty apofatycznej doktryny, które mają silniejsze związki z~rozważaniami z~zakresu logiki formalnej oraz takie, które podnoszone są literaturze przedmiotu w~ramach obrony teologii negatywnej przed sprzecznościami (w szczególności w~opracowaniach, których autorzy wprost deklarują chęć lub próbę takiej obrony).

Już na pierwszy rzut oka widać pewną dysproporcję między trzema w~ten sposób wyodrębnionymi częściami pracy. Cześć pierwsza, ukazująca teologię apofatyczną w~aspekcie semantycznym (oraz logicznym \textit{sensu largo}), jest bardziej obszerna niż pozostałe. Dzieje się tak z~co najmniej trzech powodów. Po pierwsze,
teologia milczenia to najpopularniejsza z interpretacji teologii negatywnej.
Ta właśnie interpretacja oraz związany z~nią aspekt semantyczny (i~ogólno-logiczny) są najchętniej i~najczęściej podejmowane i~analizowane w~filozoficznych opracowaniach na temat apofatycznej doktryny. Po drugie, w~części im poświęconej zawarłem także te rozważania, które -- mimo wstępnych bezpośrednich deklaracji -- do obrony spójności apofatyzmu nie angażują środków i~metod logiki formalnej. Po trzecie, pewne fragmenty ostatniego rozdziału tej części poświęconego pozytywnej teologii negatywnej do pewnego stopnia przenoszą już akcenty na epistemiczny aspekt apofatyzmu, któremu poświęcono kolejną część. Oryginalna koncepcja Rojka mieści się jednak w~zakresie interpretacji rozumianej jako teologia milczenia.

Ostateczny wynik tej pracy jest negatywny. Jak przekonuję w~każdym z~jej krytycznych rozdziałów, wszystkie analizowane tu próby zachowania spójności teologii apofatycznej okazują się nieskuteczne, przynajmniej do pewnego stopnia. Zasadniczo autorom poddawanych dyskusji opracowań nie udaje się zaproponować takiej rekonstrukcji (dowolnej interpretacji) teologii apofatycznej, która byłaby jednocześnie logicznie poprawna i~spójna a~teologicznie adekwatna i~trafna. Nie oznacza to jednak, by teorie i~interpretacje teologicznej doktryny apofatycznej nie mogły być badane narzędziami logiki formalnej. Wręcz przeciwnie: uważam, że takie badania mogą być obustronnie korzystne. Z~jednej strony wydaje się, że negatywny wynik niniejszej pracy zostałby przyjęty przez reprezentantów teologii apofatycznej z~ulgą i~radością. Ostatecznie obecność antynomii i~paradoksu tam, gdzie dotykamy granic ludzkiego poznania, nie wydaje się niczym zaskakującym a~i one -- w~obrębie rozumienia wyznaczonego przez apofatyczny mistycyzm -- mogą stanowić drogę do zjednoczenia z~transcendentnym Bogiem. Z~drugiej strony teologia apofatyczna może z~powodzeniem stać się probierzem czy też ,,papierkiem lakmusowym'' pewnych rozwiązań wypracowywanych w~ramach logiki czy filozofii języka, albo chociaż ,,słomianą kukłą'' poddawaną trywialnej krytyce w~ramach kontrprzykładu lub przeciwnie: sztandarowym przykładem służącym do zilustrowania pewnych problemów logicznych i~filozoficzno-językowych -- tak, jak ma to miejsce w~przypadku paradoksu kłamcy. Filozoficzny potencjał teologii apofatycznej jest imponujący, z~czego po krótce jeszcze zdaję sprawę na zakończenie niniejszych rozważań.

W~końcu, należy przyznać, że mimo iż w~niniejszej pracy elementy logiki formalnej używane są do analizy i~rekonstrukcji modelów teologii negatywnej, a~także do przedstawienia pewnych partykularnych dowodów lub wyciągnięcia konsekwencji z~formalnie zrekonstruowanych tez, nie jest ona pracą z~zakresu logiki formalnej -- w~tym sensie, w~jakim mówi się, że formalizacja teorii filozoficznych nie przekształca ich w~logikę\footnote{Por. J. Woleński, \textit{Metateoretyczne problemy epistemologii}, dz. cyt., s.~72.}. Nie zawiera ona żadnego nowego, oryginalnego i~substancjalnego wyniku, który mógłby powiększyć korpus wiedzy logicznej (np. w~postaci dowodu lub uogólnienia jakiegoś zaawansowanego twierdzenia). Można jednak włączyć ją w~szereg zastosowań logicznych lub sklasyfikować jako pracę z~logiki w~szerokim sensie. Z~drugiej strony niniejsze opracowanie bynajmniej nie pretenduje  do bycia uznanym za rozprawę z~teologii, choć jego autor żywi nadzieję, że teolog uzna umieszczone w~nim wywody za warte uwagi. Wreszcie, uważam także, że nie będzie w~błędzie ten, kto ze względu na stosowaną metodę zaliczy zawarte w~tej pracy rozważania w~poczet filozofii analitycznej -- z~uwagi na przedmiot analizowanych w~ten sposób teorii: analitycznej filozofii Boga lub analitycznej filozofii religii.


