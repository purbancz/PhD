
\chapter{Teologia negatywna jako teologia milczenia}


\section{Wprowadzenie}

Jedna z~najczęściej spotykanych interpretacji teologii negatywnej zwraca szczególną uwagę na boską transcendencję. Interpretacja ta ma swoje źródła w~tym, że dla wielu teologów negatywnych zaprzeczenia, w~takim samym stopniu jak potwierdzenia, nie mogą stanowić odpowiednich środków do opisu Boga. Transcendentny Bóg jest do tego stopnia ,,ponad'' niedoskonałymi pojęciami ludzkiego języka, że w~zasadzie żadnego z~nich nie powinniśmy mu przypisywać. Zatem -- według takiej interpretacji -- teologia negatywna głosi przede wszystkim, że Bóg jest zasadniczo niepojmowalny i~niewyrażalny. To podejście ma daleko idące konsekwencje, ponieważ -- skoro nie możemy przypisać Bogu żadnej własności -- powinniśmy zaprzestać mówienia o~nim i~zamilczeć.

***

Warto odnotować, że podkreślanie transcendencji Boga jest popularną strategią wśród teologów i~filozofów religii nie zawsze utożsamianych wprost z~nurtem apofatycznym. Trudno nie dostrzec takiego podejścia w~pismach wpływowych dwudziestowiecznych przedstawicieli tych dziedzin, takich jak Rudolf Oto, Karl Barth czy Karl Rahner.

Pierwszy z~nich najbardziej znany jest z~analizy doświadczenia, które -- w~jego opinii -- leży podstaw jakiejkolwiek religii. To, co jest doświadczane podczas przeżycia religijnego Otto określa terminem \textit{numinosum}. Doświadczeniu numinotycznemu towarzyszą dwa komplementarne uczucia: \textit{misterium tremendum} -- uczucie przerażenia, grozy i~lęku, lecz także mocy i~majestatu oraz \textit{misterium fascinans}\- -- uczucie fascynacji i~zachwytu. Jednakże kluczowym dla niniejszych rozważań jest to, że stanowią one \textit{misterium} -- zawierają element tajemnicy. Dla Otto rzeczywistość sakralna jest czymś całkowicie innym (niem. \textit{ganz Andere}, ang. \textit{wholly Other}) zarówno w~stosunku do świata naturalnego, jak i~człowieka. Radykalna odmienność tej rzeczywistości sprawia, że jest ona niewysłowiona, niedyskursywna i~irracjonalna\footnote{Por. R. A. Rappaport, \textit{Ritual and Religion in the Making of Humanity}, Cambridge University Press, Cambridge 1999, s.~377.}. Można jej doświadczyć, ale nie da się jej wyrazić słowami. Wykracza ona poza możliwości poznawcze i~zdolności językowe człowieka. Mimo tego, że według Otto doświadczenie numinotyczne jest prawdziwym spotkaniem ze świętością, nie może być ono przetłumaczone na mowę. Wiedza o~bóstwie jest ponad pojęciowym rozumieniem i~opisem, nie jest możliwe, by ująć ją w~pojęciowe kategorie ludzkiego języka\footnote{Na temat relacji pomiędzy doświadczeniem a~językiem religijnym w~teorii Rudolfa Otto pojawiło się wiele dyskusji. Zob. np. L. Schlamm, \textit{Numinous Experience and Religious Language}, ``Religious Studies'', vol. 28 (1992), nr 4, ss.~533-551 oraz L. P. Barnes, Rudolf Otto and the Limits of Religious Description, ``Religious Studies'', vol. 30 (1994), nr 2, ss.~219-230.}. Z~tego powodu właściwą reakcją na doświadczenie \textit{numinosum} pozostaje milczenie\footnote{Zob. R~Otto, \textit{The Idea of the Holly. An Inquiry into the Non-Rational Factor in the Idea of the Divine and Its Relation to the Rational}, tłum. John W. Harvey, Oxford University Press, London 1923, ss.~216-220.}.

***

Przyjęcie, że zasadnicza teza teologii negatywnej głosi, że Bóg przekracza wszystko, co możemy o~nim powiedzieć, jest popularną interpretacją pism takich autorów jak Grzegorz z~Nyssy, Pseudo-Dionizy Areopagita, Augustyn, Tomasz z~Akwinu czy Mistrz Eckhart. Wydaje się jednak, że to w~pismach Dionizego można odnaleźć najbardziej dosadny wyraz takiego sposobu myślenia o~Bogu.


\section{Pseudo-Dionizyjskie źródła teologii milczenia}\label{sil-dionizy}

Tożsamość Pseudo-Dionizego Areopagity nie jest do końca znana. Współcześnie najczęściej przyjmuje się, że był on syryjskim mnichem żyjącym na przełomie V~oraz VI wieku, wywodzącym się ze szkoły neoplatońskiej. On sam przedstawia siebie jako św. Dionizego, członka ateńskiej rady sądowniczej -- Areopagu, który jako jeden z~nielicznych Ateńczyków nawrócił się pod wpływem przemówienia św. Pawła\footnote{Zob. K. Corrigan, M. L. Harrington, \textit{Pseudo-Dionysius the Areopagite}, [w:] \textit{The Stanford Encyclopedia of Philosophy}, wyd. zima 2019, red. E. N. Zalta, {\textless}https://plato.stanford.edu/‌archives/‌spr2015/‌entries/‌pseudo-dionysius-areopagite{\textgreater} oraz T. Stępień, \textit{Przedmowa} [w:] Pesudo-Dionizy Areopagita, \textit{Pisma teologiczne}, tom I, Wydawnictwo Znak, Kraków 1997, s.~9. Nawrócenie Dionizego na rynku ateńskim jest wydarzeniem biblijnym, opisanym w~\textit{Dziejach Apostolskich}: Dz 17, 32nn.}. W~swoich pismach niejednokrotnie nazywa św. Pawła swoim nauczycielem\footnote{Trudno nie dostrzec racji, dla których ojciec teologii negatywnej przybrał tożsamość ucznia św. Pawła. Pewne apofatyczne wątki można odnaleźć już w~sformułowaniach biblijnego autora. Paweł w~wielu swoich listach stosuje taką apofatyczną terminologię, jak: $\text{\textgreek{>a}}\text{\textgreek{'o}}\rho \alpha \tau o\varsigma $ -- niewidzialny (Rz 1, 20; Kol 1, 15; 1 Tm 1, 17; Hbr 11, 27); $\text{\textgreek{>'a}}\rho \rho \eta \tau o\varsigma $ -- niewyrażalny, niewysłowiony (2 Kor 12, 4); $\text{\textgreek{>a}}\nu \varepsilon \kappa \delta \iota \text{\textgreek{'h}}\gamma \eta \tau o\varsigma $ -- niewysłowiony, nieopisywalny (2 Kor 9, 15); $\text{\textgreek{>a}}\pi \rho \text{\textgreek{'o}}\sigma \iota \tau o\varsigma $ -- niedostępny (1 Tm 6, 16) itp. Por. G. Rocca, \textit{Speaking the Incomprehensible God. Thomas Aquinas on the Interplay of Positive and Negative Theology}, The Catholic University of America Press, Waszyngton 2004, s.~8. Warto dodać, że motywem, od którego Paweł rozpoczął swoje kazanie na Areopagu, był ateński ołtarz poświęcony \textit{Nieznanemu} Bogu (Dz 17, 23).} a~niektóre spośród listów adresuje do jego towarzyszy, Tymoteusza i~Tytusa, czy nawet Jana Apostoła\footnote{Zob. choćby Pseudo-Dionizy Areopagita, \textit{List IX i~X}, [w:] tenże, \textit{Pisma Teologiczne}, tłum. M. Dzielska, Wydawnictwo Znak, Kraków 1997, s.~197-207.}. Taką tożsamość chrześcijańskiego autora zaczęto podważać dopiero na przełomie XV i~XVI wieku. Fakt, że Pseudo-Dionizy przez nieomal dziesięć wieków cieszył się niezachwianym autorytetem ucznia św. Pawła, sprawił, że to pisma zebrane w~\textit{Corpus Dionysiacum} wywarły największy wpływ na kształtowanie się późniejszej tradycji apofatycznej -- nie tylko w~późnej patrystyce i~Średniowieczu, lecz także w~Renesansie i~czasach współczesnych. To właśnie Dionizego nazywa się ,,ojcem teologii negatywnej'' -- mimo, iż myślenie apofatyczne było obecne w~tradycji chrześcijańskiej niemalże od samego początku\footnote{Zob. P. Sikora, \textit{Logos niepojęty}, Wydawnictwo Universitas, Kraków 2012, s.~58. O~teologii apofatycznej przed Dionizym przeczytać można: Tamże, Rozdziały I-II; C. M. Stang, \textit{Negative Theology from Gregory of Nyssa to Dionysius the Areopagite}, [w:] \textit{The Wiley-Blackwell Companion to Christian Mysticism}, red. J.A. Lamm red., Wiley-Blackwell, Malden 2013, ss.~161-176; G. Rocca, \textit{Speaking the Incomprehensible God}\ldots, dz. cyt. Rozdział I.}.


\subsection{Teologia krytyczna}

Według Johna N. Jonesa\footnote{J. N. Jones, S\textit{culpting God: The Logic of Dionysian Negative Theology}, ,,Harvard Theological Review'', vol. 89 (1996), ss.~355–371.}, teologia dionizyjska jest w~dużej mierze teologią krytyczną. Polemizuje ona z~błędnym sposobem mówienia o~Bogu -- takim, który traktuje Go jak inne byty, czyli rzeczy lub pojęcia. W~\textit{Teologii mistycznej} Dionizy wspomina o~dwóch typach nieporozumień:

Mówię tu o~tych, którzy grzęznąc w~bytach nie są zdolni wyobrazić sobie czegoś, co rzeczywiście nadsubstancjalnie istnieje ponad bytami, I~twierdzą, że w~wiedzy, która jest w~nich, płynie znajomość Tego, który wybrał ``ciemność za swoje schronienie''. Skoro nawet dla tego typu ludzi dostęp do świętych wtajemniczeń nie jest możliwy, to cóż dopiero można powiedzieć o~jeszcze większych profanach, którzy najpośledniejsze spośród bytów poczytują za przekraczającą wszystko najwznioślejszą przyczynę i~zaprzeczają jej wyższości nad ich bezbożnymi idolami o~różnorodnych kształtach.\footnote{Pseudo-Dionizy Areopagita, \textit{Teologia mistyczna}: I, 2, tłum. M. Dzielska [w:] Tenże, \textit{Pisma teologiczne}, tom I, Wydawnictwo Znak, Kraków 1997, ss.~163-164. O~ile nie podano inaczej, wszystkie poniższe cytaty z~Pseudo-Dionizego Areopagity pochodzą z~niniejszego wydania.}

Według Areopagity, bałwochwalcy mylą Boga z~przedmiotami, zaś inni ,,profani'' -- prawdopodobnie ma tu na myśli środkowych platoników -- z~pojęciami. W~innym tekście próbuje przedstawić, jak ci ostatni mogliby krytykować wykorzystywanie materialnych obrazów do przestawienia Boga, preferując raczej utożsamianie Boga z~pojęciem lub pojęciami:

ktoś [\ldots] mógłby dowodzić, że święci autorzy, chcąc uformować cieleśnie te czyste bezcielesności, powinni je wymodelować i~ukazać pod stosownymi dla nich kształtami im pokrewnymi, na ile to możliwe wzorując się na substancjach najbardziej przez nas cenionych [\ldots] Tego rodzaju ujęcia lepiej by przecież służyły anagogicznej drodze naszego intelektu i~nie ściągałyby nadprzyrodzonych objawień w~dół, do poziomu absurdalnych niepodobieństw. Tymczasem to postępowanie zdaje się, w~sposób niedopuszczalny, ubliżać boskim mocom i~równocześnie wypacza nasz intelekt, wpędzając go w~pułapkę bezbożnych alegorii.\footnote{Pseudo-Dionizy Areopagita, \textit{Hierarchia niebiańska}: II, 2.}

Dionizy zgadza się z~teologicznym stanowiskiem, wedle którego materialne obrazy nie mogą przedstawiać boskiej istoty. Jednakże, odrzuca on także takie rozwiązanie, wedle którego lepszym sposobem przedstawiania Boga są pojęcia. Zarówno przedmioty, jak i~pojęcia nie stanowią odpowiednich reprezentacji dionizyjskiego Boga z~tego samego powodu -- ponieważ jest On ponad wszelkim bytem. Wnioskiem, jaki wypływa z~tego obrazu, jest fakt, że język, który służy do opisu bytów, nie może być wykorzystywany do opisu Boga. Skoro Bóg nie należy do kategorii bytów, nie można o~Nim mówić w~taki sposób, w~jaki mówi się o~czymkolwiek innym. Zdaniem Jonesa konsekwencją negatywnego języka teologii Dionizego jest niemożliwość powiedzenia o~Bogu czegokolwiek\footnote{Por. niżej -- rozdz.~{\textbackslash}ref\{sil-jones\}.}.


\subsection{Apofatyzm kompletny}

Podobną interpretację dzieł Dionizego przedstawił jeden z~jego najbardziej znanych komentatorów -- Paul Rorem. Teologię negatywną Areopagity -- w~przeciwieństwie do tej, którą można odnaleźć u~Grzegorza z~Nysy, Maksyma Wyznawcy, czy Bonawentury -- Rorem nazywa apofatyzmem kompletnym\footnote{P. Rorem, \textit{Negative Theologies and the Cross}, ,,Harvard Theological Review'', vol. 101 (2008), ss.~451-464.}. Twierdzi on, że ostatnie dwa rozdziały najbardziej ,,negatywnego'' dzieła areopagity -- \textit{Teologii mistycznej} -- tłumaczą, na czym polega ,,anagogiczna droga przez negację'' i~należy je odczytywać łącznie. Pierwszy z~nich głosi, że najwyższa przyczyna wszystkich rzeczy postrzegalnych sama nie jest postrzegalna, ten drugi natomiast, że najwyższa przyczyna wszystkich pojęć sama nie ma charakteru pojęciowego\footnote{P. Rorem, \textit{Pseudo-Dionysius. A~Commentary on the Texts and an Introduction to Their Influence}, Oxford University Press, New York -- Oxford 1993, ss.~205-213.}.

W~interpretacji Rorema wspomniane dzieło Dionizego ma przede wszystkim wartość duchową a~jego podstawowym celem jest przedstawianie sposobów służących do zjednoczenia z~Bogiem. Droga ku temu prowadzi najpierw poprzez zanegowanie wszystkich rzeczy postrzegalnych, zwłaszcza wszystkich symboli, które mają wskazywać na najwyższa przyczynę. Dzięki temu wstępujemy na poziom pojęć, które są przez te symbole reprezentowane. Kolejny krok, opisany w~rozdziale piątym \textit{Teologii mistycznej}, polega na zanegowaniu także i~tych pojęć:

Wznosząc się coraz wyżej, mówimy, że Bóg nie jest duszą, intelektem, wyobrażeniem, mniemaniem, rozumem i~rozumieniem, słowem i~pojmowaniem; [\ldots] nie jest liczbą, porządkiem, wielkością, małością, równością, nierównością, podobieństwem, niepodobieństwem; nie stoi, nie porusza się, nie odpoczywa, nie posiada mocy i~nie jest ani mocą, ani światłością, nie żyje i~nie jest życiem; nie jest substancją, wiecznością i~czasem; [\ldots] nie jest królem ani mądrością, ani jednią ani jednością, ani Boskością, ani dobrocią, ani duchem (o ile znamy ducha), ani synostwem, ani ojcostwem [\ldots]\footnote{Pseudo-Dionizy Areopagita, \textit{Teologia mistyczna}: V.}.

Jak zauważa Rorem, wiele z~pojęć, które pojawiają się w~niniejszym fragmencie służyło Dionizemu do określenia Boga w~tych traktatach, w~których pozostawał na poziomie teologii pozytywnej\footnote{Por. np. Tenże, \textit{Imiona boskie}: VI-X.}. Co więcej, zanegowane są tutaj także imiona osób Trójcy świętej. One także należą do kategorii niedoskonałych pojęć ludzkiego umysłu. Ich znaczenie jest skończone, a~więc ostatecznie nie można ich przypisać nieskończonej naturze Boga. Na tym jednak droga do zjednoczenia z~Bogiem się nie kończy. Cały ten proces osiąga swój szczyt, by w~końcu go przekroczyć w~ostatnich zdaniach najczęściej komentowanego rozdziału \textit{Teologii mistycznej}:

[\ldots] nie istnieje ani słowo, ani imię, ani wiedza o~Nim; nie jest ani ciemnością, ani światłością, ani błędem, ani prawdą; nie można o~Nim niczego zaprzeczać ani nic pewnego orzekać, bo twierdząc o~Nim lub zaprzeczając rzeczy niższego rzędu, nic o~nim nie stwierdzamy, ani nie zaprzeczamy. Ta najdoskonalsza przyczyna wszystkiego jest bowiem ponad wszelkim potwierdzeniem i~ponad wszelkim zaprzeczeniem: wyższa nad wszystko, całkowicie niezależna od wszystkiego i~przenosząca wszystko\footnote{Tenże, \textit{Teologia mistyczna}: V.}.

Na końcu tej drogi zaprzeczamy nawet samym zaprzeczeniom. Ponieważ negacja także należy do pojęć ludzkiego języka, również przy jej pomocy nie da się uchwycić nieskończonego, transcendentnego Boga. Proces wznoszenia się ku Bogu kończy się w~ciemności niewiedzy. Twierdzenia, a~następnie zaprzeczenia, są jedynie środkami do spotkania Boga, ale ostateczne zjednoczenie z~Nim odbywa się nie tylko ponad wszelkim twierdzeniem, ale i~ponad wszelkim zaprzeczeniem. Na tym etapie język nie odgrywa już żadnej roli. Swoje rozważania na ten temat Rorem kończy w~następujący sposób:

Według ostatnich słów traktatu ,,Bóg jest ponad wszelkim zaprzeczeniem''. Negacja zostaje zanegowana a~zamroczony umysł ludzki popada w~milczenie. Traktat, \textit{corpus}, jego autor a~także niniejszy komentarz nie mają nic więcej do powiedzenia. Pozostaje wyłącznie milczenie\footnote{P. Rorem, \textit{Pseudo-Dionysius. A~Commentary on the Texts}\ldots, dz. cyt., s.~213.}.


\section{Teologia milczenia -- źródła niewysławialności i~kwestia nazewnictwa}\label{sil-int-nazw}

Powyższe paragrafy pokazują, że kluczowa teza omawianej w~tym rozdziale interpretacji teologii negatywnej -- ilustrowanej najchętniej dziełami Pseudo-Dionizego Areopagity -- głosi, że ludzki język jest bezsilny wobec zadania opisu i~wyrażenia transcendentnego Boga. Można próbować wskazać dwie (niewykluczające się) przyczyny takiego stanu rzeczy. Przede wszystkim, może być on spowodowany samymi ograniczeniami ludzkiego języka i~niedostatecznymi zdolnościami poznawczymi człowieka, które czynią go niezdolnym do opisania Boga w~należyty sposób. Stanowisko to jest zdecydowanie mniej popularne. Najpoważniejszym autorem, który reprezentuje taki pogląd, jest John Hick. Twierdzi on, że

boska transkategorialność\footnote{\textit{Transcategriality} -- jest to (dość niezgrabne) określenie, które Hick w~późniejszych pracach stosował zamiennie ze słowem ,,niewysławialność'' (\textit{ineffability}).} nie pociąga za sobą wniosku, że Bóstwo nie posiada żadnej natury, lecz jedynie taki, który mówi, że ta natura nie może zostać ujęta w~ludzkich myślach i~języku, ponieważ niewysławialność odnosi się do zdolności poznawczych poznającego\footnote{J. Hick, \textit{Ineffability}, ,,Religious Studies'', vol. 36 (2000), ss.~41-42.}.

Z~drugiej strony, bezsilność ludzkiego języka w~staraniach o~podanie opisu Boga może być ugruntowana w~samej naturze Boga i~jego transcendencji. Oznaczałoby to, że Bóg jest niewysławialny ze swojej natury, jest to Jego istotna, ,,wewnętrzna'' własność. Wydaje się, że właśnie takie stanowisko jest zdecydowanie częściej reprezentowane zarówno wśród samych myślicieli apofatycznych, jaki i~badaczy zajmujących się tym rodzajem teologii. Peter Kügler, nawiązując bezpośrednio do pracy Hicka, pisze:

Z~pewnością niewysławialność Boga jest związana z~poznawczymi ograniczeniami ludzkiego umysłu, ale to \textit{natura} Boga jest taka, że ludzki język nie może jej uchwycić\footnote{P. Kügler, \textit{The meaning of mystical ‘darkness}', ,,Religious Studies'', vol. 41 (2005), s.~101. Podobne stanowisko zajmuje Christopher Insole -- por. C.J. Insole, \textit{Why John Hick cannot, and should not, stay out of the jam pot}, ,,Religious Studies'', vol. 36 (2000), ss.~28-30, Jonathan Jacobs -- por. J. D. Jacobs, \textit{The Ineffable, Inconceivable, and Incomprehensible God: Fundamentality and Apophatic Theology}, [w:] \textit{Oxford Studies in Philosophy of Religion VI}, red. R. Audi \textit{et al}., Oxford University Press, New York 2015, s.~165 i~wielu innych -- por. dalsze części niniejszego rozdziału.}.

W~podobnym duchu wypowiadają się inni autorzy. Na przykład Jonathan D. Jacobs zakłada, że

Teologia apofatyczna nie polega na twierdzeniu, że Bóg jedynie jest trudny do opisania, że z~ogromnym wysiłkiem moglibyśmy go sobie wyobrazić, albo że istnieją tylko pewne prawdy o~Bogu, których nie jesteśmy w~stanie pojąć. To nie jest zwykły chwyt retoryczny. [\ldots] Bóg jest istotowo niewyrażalny\footnote{J.D. Jacobs, \textit{The Ineffable, Inconceivable, and Incomprehensible God}\ldots, dz. cyt., s.~159.}.

Z~kolei Jan Maria Bocheński pisze o~,,absolutnej'' naturze boskiej niewyrażalności:

Jedną za cech charakterystycznych wszystkich tych teorii [\ldots] jest okoliczność, iż ograniczenia nałożone na znaczenie w~związku z~wykorzystaniem ,,tajemnicy'' i~,,tajemniczości'' są traktowane poniekąd absolutnie, co znaczy, że przypisuje się te ograniczenia samej naturze przedmiotu religii sądząc, iż żaden człowiek nie jest w~stanie ich pokonać [\ldots]\footnote{J. M. Bocheński, \textit{Logika religii}, tłum. S. Magala, Instytut wydawniczy PAX, Warszawa 1990, s.~415.}.

Ponieważ w~ramach tej interpretacji teologii negatywnej największy nacisk kładzie się na to, że Bóg jest zasadniczo, istotowo i~substancjalnie niewysławialny, nieopisywalny i~niewyrażalny a~ludzki język nie jest w~stanie w~żaden sposób powiedzieć o~nim czegokolwiek, często nazywa się ją \textit{teorią Niewyrażalnego}, \textit{teorią Niewysłowionego}\footnote{Taką nazwę zaproponował Józef Maria Bocheński -- zob. tamże, s.~352. Odróżniał on jednak teorię Niewysłowionego od teologii negatywnej \textit{tout court} -- por. tamże, ss.~416-418. Polski tłumacz pracy Bocheńskiego zostawił ten apofatyczny przymiotnik w~wersji dokonanej, używając sformułowania ,,to, co niewysłowione''. W~oryginale brak możliwości wypowiedzenia czegokolwiek o~Bogu jest mocniej zaznaczony -- Bocheński pisze o~,,the Unspeakable'' (zapisując ten przymiotnik wielką literą), a~nie o~,,unspoken''. Por. np. J.M. Bocheński, \textit{The Logic of Religion}, New York University Press, New York 1965, ss.~31-36.} lub nawet \textit{teologią milczenia}\footnote{Autorem tego określenia jest George Englebretsen -- zob. G. Englebretsen, \textit{The Logic of Negative Theology}, ,,New Scholasticism'', vol. 47 (1973), s.~232. W~niniejszej pracy tych i~podobnych im nazw będę używał zamiennie.}.


\section{Paradoksalny charakter teologii milczenia}\label{sil-int-par}

Teologia apofatyczną jest często oskarżana o~sprzeczność. Najczęściej wytykanym problemem tej doktryny jest ciążący na niej pewien rodzaj paradoksu samoodniesienia. Nietrudno postawić taki zarzut także teorii Niewysłowionego -- skoro głosi ona, że o~Bogu nie można nic powiedzieć, tym samym sama mówi coś Bogu, a~zatem jest niespójna i~należy ją odrzucić. Michael Durrant paradoks teologii milczenia ujmuje w~następujący sposób:

w~tej teorii, mówiąc, że natura Boga jest zasadniczo niewyrażalna, opisujemy właśnie naturę Boga -- jest mianowicie zasadniczo niewyrażalna. Innymi słowy, ci, którzy bronią tego stanowiska, nie mogą tego robić nie przecząc sobie\footnote{M. Durrant, \textit{The Meaning of ‘God'– I} [w:] \textit{Religion and Philosophy}, red. M. Warner, Cambridge University Press, Cambridge 1992, s.~74. Cytat w~j. polskim za P. Rojek, \textit{Logika teologii negatywnej}, ,,Pressje'', nr 29 (2012), s.~222-223.}.

Podobnie argumentuje John Hick, który uważa, że nie ma sensu

mówić o~X, że żadne nasze pojęcie się do niego nie stosuje. Jest bowiem w~oczywisty sposób niemożliwe odnosić się do czegoś, co nie posiada nawet własności 'bycia możliwym przedmiotem odniesienia\footnote{J. Hick, \textit{An Interpretation of Religion. Human Responses to the Transcendent}, Yale University Press, New Haven -- Londyn 1989, s.~239. Cytuję za P. Sikora, \textit{Logos Niepojęty}, Wydawnictwo Universitas, Kraków 2010, s.~118.}.

Dodaje on także, że określenie

,,taki, że nasze pojęcia się do niego nie stosują'' nie może, jeśli chcemy uniknąć paradoksu, odnosić się do własności, którą opisuje\footnote{Tamże.}.

To właśnie z~paradoksalnym charakterem teorii Niewysłowionego najczęściej mierzą się ci badacze, którzy próbują rozważać jej logiczno-językową strukturę. Warto więc wyjaśnić, co będziemy rozumieć przez sprzeczność, paradoks i~jakie są ich logiczne konsekwencje.

,,Sprzecznością'' lub ,,antynomią'' będę tu nazywał parę zdań, z~których jedno jest negacją drugiego. Terminem ,,paradoks'' tradycyjnie zwykło się określać twierdzenie, które prowadzi do zaskakujących lub sprzecznych wniosków. Sprzeczność tak rozumianych paradoksów nie musi stanowić antynomii w~powyższym sensie -- może być sprzecznością pozorną, sprzecznością z~tzw. zdrowym rozsądkiem, z~dobrze uzasadnionymi przekonaniami i~wynikającymi z~nich oczekiwaniami\footnote{Zob. P. Łukowski, \textit{Paradoksy}, Wydawnictwo Uniwersytetu Łódzkiego, Łódź 2006, s.~7.} czy z~,,powszechną opinią''\footnote{Zob. A. Cantini. R. Bruni, \textit{Paradoxes and Contemporary Logic}, [w:] \textit{The Stanford Encyclopedia of Philosophy}, wyd. jesień 2021, red. E.N. Zalta, {\textless}https://plato.stanford.edu/archives/fall2021/entries/paradoxes-contemporary-logic/{\textgreater}.}. W~niniejszej pracy termin ,,paradoks'', o~ile nie zostanie wskazane inaczej, zasadniczo będzie używany na określenie sprzeczności nietrywialnej. Mówiąc krótko, w~paradoksie będziemy mieć do czynienia z~sytuacją, w~której w~obrębie danej teorii, rachunku lub sytemu zarówno pewne zdanie, jak i~zdanie z~nim sprzeczne, wydają się być jednakowo dowiedzione lub przynajmniej w~jednakowy sposób ugruntowane czy uprawnione do utrzymywania.

W~historii myśli paradoksy niejednokrotnie zmuszały do intelektualnych zmagań. W~obliczu nietrywialnych sprzeczności należało odnaleźć błąd ukryty w~dowodzie, dokonać rewizji założeń lub zrekonstruować cały system. Znanym przykładem z~zakresu logiki będzie tutaj paradoks kłamcy, który inspiruje logików do dziś, czy też antynomie Russella, które doprowadziły do filozoficznych badań nad podstawami matematyki. Fizyka również zna taki inspirujący wpływ ,,paradoksalnych'' eksperymentów myślowych prowadzących do zaskakujących lub sprzecznych wniosków, jak ma to miejsce na przykład w~przypadku tzw. paradoksu bliźniąt czy paradoksu kota Schrödingera\footnote{Fizyczne ,,paradoksy'' często nie zwierają logicznych sprzeczności, ale bez wątpienia można je uznawać za paradoksy w~tym szerszym, ogólnym sensie.}.

Paradoksy samoodniesienia (zwane także paradoksami samozwrotności\footnote{Por. np. J. Woleński, \textit{Samozwrotność i~odrzucanie}, ,,Filozofia Nauki'', vol. 1 (1993), nr 1, ss.~89-102.}, cyrkularności\footnote{Por, P. Łukowski, dz. cyt, ss.~178-250.} lub, rzadziej, autoreferencji\footnote{Por. np. R. Poczobut, \textit{Paradoksy w~wyjaśnianiu świadomości}, ,,Ethos. Kwartalnik Instytutu Jana Pawła II KUL'', vol. 26 (2013), nr 1(101), ss.~62-80}) to najczęstsza grupa paradoksów badanych narzędziami logicznymi. Do grupy tej należą najchętniej rozważane tzw. paradoksy semantyczne\footnote{Paradoksy semantyczne czasem nazywane są także ,,syntaktycznymi'' lub, rzadziej, ,,epistemicznymi''. Por. P. Łukowski, dz. cyt., s.~185.} (na przykład paradoks kłamcy), paradoksy teoriomnogościowe (na przykład paradoks Russella) czy paradoksy epistemiczne (na przykład paradoks znawcy). Choć te trzy grupy paradoksów tworzone są w~obrębie odmiennych rachunków a~ich konsekwencje istotne są dla innych dyscyplin -- odpowiednio dla teorii prawdy, podstaw matematyki i~epistemologii -- mają one wspólną strukturę i~często bada się je przy użyciu podobnych narzędzi logicznych\footnote{Zob. T. Bolander, \textit{Self-Reference}, [w:] \textit{The Stanford Encyclopedia of Philosophy}, wyd. jesień 2017, red. E.N. Zalta, {\textless}https://plato.stanford.edu/archives/fall2017/entries/self-reference/{\textgreater}.}. By dostrzec tę strukturę prześledźmy kilka przykładów samozwrotnych paradoksów semantycznych.

\textbf{Paradoks Grellinga-Nelsona}\footnote{Por. K. Grelling, \textit{The Logical Paradoxes}, ,,Mind'', vol. 45 (1936), nr 180, ss.~481-486.}


Przymiotnik nazwiemy autologicznym, jeśli posiada wyrażoną przez siebie własność -- na przykład ,,polski'', ,,pięciozgłoskowy'', ,,sześciosylabowy'' itp.

Przymiotnik nazwiemy heterologicznym, jeśli nie posiada wyważanej przez siebie własności -- na przykład ,,chiński'', ,,jednosylabowy'', ,,złożony'' itp.

Czy przymiotnik ,,heterologiczny'' jest autologiczny czy heterologiczny? Jeśli przyjmiemy, że jest on przymiotnikiem autologicznym, to ma własność, którą wyraża, a~więc jest heterologiczny. Jeśli założymy, że jest on przymiotnikiem heterologicznym, to nie posiada on własności, którą wyraża a~wyraża własność heterologiczności, a~zatem jest autologiczny. W~konsekwencji ,,heterologiczny'' jest przymiotnikiem autologicznym wtedy i~tylko wtedy, gdy jest przymiotnikiem heterologicznym.

\textbf{Paradoks liczb Richardowskich}\footnote{Por. J. Richard, \textit{The principles of mathematics and the problem of sets}, [w:] \textit{From Frege to Gödel. A~source book in mathematical logic 1879–1931}, red. J. van Heijenoort, Harvard University Press, Cambridge 1967, ss.~142-144. W~oryginalnym sformułowaniu paradoksu Richard mówi o~liczbach rzeczywistych i~wykorzystuje metodę przekątniową, ale nie ma to większego znaczenia dla zrozumienia idei zamozwrotności ilustrowanej przedstawianymi tu paradoksami.}


Rozważmy skończone ciągi słów języka naturalnego definiujące arytmetyczne własności liczb naturalnych, na przykład ,,liczba naturalna posiadająca dokładnie dwa dzielniki całkowite'', ,,liczba pierwsza taka, że liczba większa od niej o~2 też jest liczbą pierwszą'', ,,liczba podzielna przez 7'' itp. i~uporządkujmy te definicje w~sposób leksykograficzny przyporządkowując każdej z~nich liczbę naturalną. Załóżmy, że na miejscu \textit{n}-tym znalazła się definicja liczby Richardowskiej: ,,liczba naturalna \textit{k}, która nie posiada własności wyrażonej \textit{k}-tą definicją''.

Graf/img/tabela?

Czy \textit{n} jest liczbą Richardowską? Jeśli odpowiemy twierdząco, to \textit{n} nie posiada własności wyrażonej \textit{n}-tą definicją, a~zatem nie jest liczbą Richardowską. Jeśli odpowiemy przecząco, to \textit{n} posiada własność wyrażoną \textit{n}-tą definicją a~jest to definicja liczby Richardowksiej. Zatem \textit{n} jest liczbą Richardowską wtedy i~tylko wtedy, gdy \textit{n} nie jest liczbą Richardowską.

\textbf{Paradoks kłamcy}\footnote{Paradoks kłamcy jest niekwestionowanym ,,celebrytą'' wśród paradoksów, któremu poświęcono niejedną monografię, pracę zbiorową czy artykuł. Przegląd podejść do rozwiązania tego paradoksu można odnaleźć w~J.C. Beall, M. Glanzberg, D. Ripley, \textit{Liar Paradox}, [w:] \textit{The Stanford Encyclopedia of Philosophy}, wyd. jesień 2020, red. E.N. Zalta, {\textless}https://plato.stanford.edu/archives/fall2020/entries/liar-paradox/{\textgreater}. Dobry wgląd z~punktu widzenia teorii prawdy dają także: B. Brożek, \textit{Rola paradoksu kłamcy w~konstrukcji logicznych teorii prawdy}, ,,Zagadnienia Filozoficzne w~Nauce'', nr 30 (2002), ss.~48-88 oraz J. Pruś, \textit{Semantyczna teoria prawdy a~antynomie semantyczne}, ,,Rocznik Filozoficzny Ignatianum'', vol 27(2021), nr 1, ss.~341-363.}


Chyba najbardziej eleganckim sformułowaniem tego paradoksu jest stwierdzenie ,,\textit{Hoc est falsum}'':

(l)Zdanie (l) jest fałszywe.{\textbackslash}label\{sil-klamca\}

Jaka jest wartość logiczna zdania {\textbackslash}ref\{sil-klamca\}? Załóżmy wpierw, że {\textbackslash}ref\{sil-klamca\} jest prawdziwe. Zatem jest tak, jak {\textbackslash}ref\{sil-klamca\} głosi a~mówi ono o~sobie, że jest fałszywe. A~więc {\textbackslash}ref\{sil-klamca\} jest fałszywe. Jeśli natomiast założymy, że {\textbackslash}ref\{sil-klamca\} jest fałszywe, to nie jest tak, jak {\textbackslash}ref\{sil-klamca\} głosi a~mówi ono o~sobie, że jest fałszywe. Sokor tak nie jest, to {\textbackslash}ref\{sil-klamca\} jest prawdziwe. W~konsekwencji {\textbackslash}ref\{sil-klamca\} jest prawdziwe wtedy i~tylko wtedy, gdy jest fałszywe.

\textbf{Paradoks Niewyrażalnego}

Możemy teraz sformatować paradoks teorii Niewysławianego w~sposób przedstawiony w~powyższych ilustracjach:

Zgodnie z~teologią milczenia jedynym sposobem wyrażenia Boga jest stwierdzenie, że jest on niewyrażalny.

Czy zatem -- zgodnie z~tą interpretacją teologii negatywnej -- Bóg jest czy nie jest niewyrażalny? Jeśli przyjmiemy, że nie jest on niewyrażalny, to nie ma innego sposobu, by go wyrazić, a~zatem jest niewyrażalny. Jeśli założymy, że jest on niewyrażalny, to wyraziliśmy, że jest niewyrażalny, a~zatem nie jest niewyrażalny. W~konsekwencji Bóg nie jest niewyrażalny wtedy i~tylko wtedy, gdy jest niewyrażalny.

Oczywiście, powyższy paradoks\footnote{Różne aspekty paradoksu niewyrażalności, także poza kontekstem religijnym, przedstawione zostały w: J. Shaw, \textit{Truth, Paradox, and Ineffable Propositions}, ,,Philosophy and Phenomenological Research'', vol. 86 (2013), nr 1, ss.~64-104.} będzie wciąż generował sprzeczność, gdy użyjemy innych przymiotników z~apofatycznego słownika, które negatywni teologowie chętnie przypisują Bogu, jak na przykład ,,nieopisywalny'' czy ,,niewysławialny''.

Warto w~tym miejscu zatrzymać się na chwilę i~wyjaśnić, dlaczego racjonalny dyskurs nie powinien dopuszczać sprzeczności. Co jest złego w~sprzecznościach, że muszą zostać usunięte? Odpowiedź na to pytanie niekoniecznie musi być oczywista. Wydaje się, że najważniejszym argumentem przeciw dopuszczaniu sprzeczności -- przynajmniej z~punktu widzenia logiki klasycznej i~większości innych użytecznych rachunków logicznych -- jest fakt, że w~systemach, w~których pojawia się para zdań sprzecznych, można dowieść cokolwiek, nawet kompletny nonsens\footnote{Dobry przegląd argumentów za unikaniem sprzeczności -- \textit{notabene} wraz z~próbą ich odrzucenia -- można odnaleźć w: G. Priest, \textit{What so bad about contradictions}?, ,,The Journal of philosophy'', vol. 95 (1998), nr 8, ss.~410-426.}. Mówiąc bardziej ścisłym językiem, systemy takie ulegają przepełnieniu. Wnioskowanie prowadzące do niedorzeczności sformalizowane jest w~postaci prawa nazywanego \textit{ex contradictione quodlibet} (w literaturze anglosaskiej: \textit{law of explosion}, w~literaturze polskiej: prawo przepełnienia, prawo Dunsa Szkota): A, not A~to B.

Przedstawiony powyżej związany bezpośrednio z~samozwrotnością paradoks Niewyrażalnego można nazwać ,,wewnętrznym'' paradoksem niniejszej interpretacji teologii apofatycznej. Wynika on bowiem wprost faktu, że w~teologii milczenia Boga próbuje się opisać jako nieopisywalnego. Jednakże -- w~oczywisty sposób -- teoria ta uwikłana jest jeszcze w~inne rodzaje paradoksów. Niektórzy badacze nazywają je paradoksami ,,pośrednimi'' lub ,,zewnętrznymi'' teorii Niewysłowionego. Mogą one przybrać charakter twierdzeniowy lub nietwierdzeniowy.

\textbf{Zewnętrzny twierdzeniowy paradoks teologii milczenia (paradoks niekonsekwentnego apofatyzmu})

Teologia negatywna powstawała i~była rozwijana w~obrębie wszystkich wielkich tradycji religijnych\footnote{Por. T.D. Knepper, L.E. Kalmanson (red.), \textit{Ineffability: An Exercise in Comparative Philosophy of Religion}, ser. \textit{Comparative Philosophy of Religion}, vol. 1, Springer, Cham 2017.} a~każda z~tych tradycji posiada swoje święte księgi, \textit{credo} lub po prostu zbiór tez czy dogmatów, które przyjmuje się w~obrębie danego dyskursu religijnego, np. ,,Bóg jest jeden w~trzech osobach'', ,,Allah jest jedynym Bogiem a~Mahomet jest jego prorokiem'', ,,Reinkarnacja to cykl życia i~ponownych narodzin kierowany karmą'' itp. Inaczej mówiąc, każdy dyskurs religijny zawiera jakiś niepusty zbiór zdań, które mają przypisywać Bogu przedmiotowo-językowe własności. Teologia apofatyczna nigdy nie rozwijała się w~oderwaniu od teologii katafatycznej, lecz raczej w~obecności i~w głębokim związku ze swoją pozytywną odpowiedniczką. Zasadniczo wydaje się, że myśliciele apofatyczni byli głęboko wierzącymi ludźmi (a przynajmniej nie ma większych powodów, by w~to wątpić\footnote{W~literaturze pojawiają się analizy teologii milczenia w~odniesieniu do ateizmu, lecz raczej w~kontekście obrony przed takim zarzutem. Zob. np. R. Pouivet, \textit{Bocheński on divine ineffability}, ,,Studies in East European Thought'', vol. 65 (2013), nr 1-2, ss.~50-51 lub J. D. Jacobs, \textit{The Ineffable, Inconceivable, and Incomprehensible God. Fundamentality and Apophatic Theology}, [w:] \textit{Oxford Studies in Philosophy of Religion}, vol. 6, red. J. Kvanvig, Oxford University Press, Oxford 2015, ss.~168-171.}) i~starali się zachowywać prawomyślność wiary -- zachowywali i~wyznawali wszystkie tezy i~dogmaty swoich doktryn religijnych. Nawet więcej, utrzymując lub rozwijając apofatyczne stanowiska wchodzili oni w~dyskusje, wyrażali silne przekonania i~wydawali sądy dotyczące katafatycznych ustaleń, często tych najważniejszych -- w~zakresie prawomyślności doktryny. Dobrym przykładem są tutaj zwłaszcza chrześcijańscy przedstawiciele teologii milczenia. Ci sami Ojcowie Kościoła, którzy nalegali, by Boga uznawać za niewyrażalnego, niepojmowalnego i~ponad umysłem, potrafili wchodzić w~zażarte spory o~bardzo precyzyjne sformułowania dogmatyczne, jak np. rozróżnianie między $\text{\textgreek{<o}}\mu oo\text{\textgreek{'u}}\sigma \iota o\nu $ i~$\text{\textgreek{<o}}\mu o\iota o\text{\textgreek{'u}}\sigma \iota o\nu $ czy w~spór o~\textit{Filioque}.

,,Zewnętrzny'' paradoks teologii milczenia w~swojej twierdzeniowej postaci polega więc na utrzymywaniu, że Bóg jest nieopisywalny, niewyrażalny i~niepojmowalny przy jednoczesnym twierdzeniu, że jest \textit{jakiś}, na przykład wszechmogący, jeden w~trzech osobach, jest stworzycielem świata itp.

Mimo iż paradoks ten ani nie zwiera, ani bezpośrednio nie generuje sprzeczności, przy odpowiedniej interpretacji może pociągać za sobą parę zdań sprzecznych.

\textbf{Zewnętrzny nietwierdzeniowy (performatywny) paradoks teologii milczenia}

Dyskurs religijny często zawiera wypowiedzi, które są performatywami: modlitwą, wyrażeniami pochwały, czci itd. W~takich aktach mowy wierni często 1) zwracają się do przedmiotu tych wypowiedzi oraz 2) przypisują przedmiotowi tych wypowiedzi pewną wartość. Wielu badaczy\footnote{Wśród nich np. J.M. Bocheński, \textit{Logika religii}, dz. cyt., ss.~355-356; S. Gäb, \textit{Languages of ineffability: the rediscovery of apophaticism in contemporary analytic philosophy of religion}, [w:] \textit{Negative Knowledge}, red. S. Hüsch i~in., Narr, Tübingen 2020, ss.~191-206; Por. także dyskusję na temat kategorii językowej terminu ,,Bóg'' poniżej -- rozdz.~\ref{sil-kt-jez}.} twierdzi, że zakłada się tu pewne przedmiotowo-językowe własności, co stoi w~sprzeczności z~główną ideą teologii milczenia. Po pierwsze trudno zwracać się do czegoś, o~czym wiemy jedynie, że nic o~tym nie można powiedzieć\footnote{Zob. dyskusję w~sekcji \ref{sil-kt-jez} poniżej.}. Po drugie, trudno czemuś takiemu przypisywać jakąkolwiek wartość -- w~zasadzie nie wiedząc czemu jakaś wartość ma być przypisywana. Jak formułuje to Bocheński:

Niemożliwe byłoby zapewne oddawanie czci, tzn. przypisywanie wartości czemuś, o~czym zakładamy wyłącznie, iż nie da się o~tym nic powiedzieć. Takie coś byłoby dla wypowiadającego się całkowicie pozbawione własności przedmiotowo-językowych. Mógłby to być np. szatan. Nie byłoby absolutnie żadnego powodu, by go wielbić, chwalić itd. Musimy przeto odrzucić teorię tego, co niewysłowione\footnote{J.M. Bocheński, \textit{Logika religii}, dz. cyt., s.~356.}.

,,Zewnętrzny'' nietwierdzeniowy paradoks teologii milczenia wynika zatem z~tego, że niemożliwe wydaje się zwracanie się w~aktach mowy będących np. wyrażeniami czci do czegoś, o~czym nic nie można powiedzieć.

Oczywiście, bez odpowiedniej, zaawansowanej filozoficznie interpretacji i~formalnej rekonstrukcji powyższy paradoks nie będzie prowadził do pary zdań sprzecznych. Można zatem uznać, że jest to paradoks w~ogólnym, szerszym sensie tego słowa. Niektórzy badacze\footnote{Np. P. Rojek, \textit{Logika teologii negatywnej}, dz. cyt., s.~225.} mówią w~jego kontekście o~niezgodności teologii apofatycznej z~praktyką religijną. Z~tego powodu paradoks ten można także nazwać paradoksem performatywnym lub prakseologicznym.

W~poniższych rozdziałach przedstawię rozmaite próby obrony teologii milczenia i~zachowania jej spójności. W~większości przypadków analizowani przeze mnie autorzy mierzą się z~wynikającym z~samoodniesionia ,,wewnętrznym'' paradoksem tej teorii. W~niektórych pracach pojawiają się jednak odwołania także do paradoksów ,,zewnętrznych''. W~pewnych przypadkach w~celu obrony apofatyzmu i~przed tym rodzajem paradoksu\footnote{Zob. rozdz.~{\textbackslash}ref\{sil-jac\}.}. W~innych po to, by argumentować przeciw teologii milczenia, mimo zachowania jej spójności w~zakresie paradoksu Niewyrażalnego\footnote{Zob. rozdz.~{\textbackslash}ref\{sil-boch\}.}.


\section{Uwaga o~kategorii językowej terminu ,,Bóg''}\label{sil-kt-jez}

Do poniższych rozdziałów należy dodać jeszcze jedną uwagę. Używany w~języku naturalnym termin ,,Bóg'' jest dwuznaczny w~sensie jego kategorii językowej. W~analitycznej filozofii Boga nie został jeszcze rozstrzygnięty spór, czy należy rozumieć go jako nazwę własną czy może stanowi on deskrypcję określoną. Jeśli ktoś jest skłonny uważać, że termin ,,Bóg'' jest nazwą, w~języku formalnym będzie przedstawiał go za pomocą stałej indywiduowej (na przykład \textit{g} -- tak jak zostało to zrobione w~powyższym przykładzie). W~przeciwnym wypadku termin ten zostanie wyrażony przy pomocy predykatu G(x) rozumianego jako ,,x jest Bogiem'' (lub -- w~zależności od religijnego kontekstu dyskursu -- ,,x jest bóstwem'', ,,x jest boskie'', ,,x jest absolutem'', ,,x jest ostateczną rzeczywistością'' itp.). W~ostateczności G(x) możemy zdefiniować poprzez iloczyn wszystkich predykatów przypisywanych Bogu przez dane wyznanie wiary (tu oznaczony jako {\textbackslash}phi(x)):

G(x) {\textbackslash}equiv\_\{def\} {\textbackslash}exists x~({\textbackslash}phi(x) {\textbackslash}land {\textbackslash}forall y~({\textbackslash}phi(y) {\textbackslash}equiv x~= y))\footnote{Por. J.M. Bocheński, \textit{Logika religii}, dz. cyt., §21.1. Oczywiście, taki iloczyn mógłby zostać zastąpiony kwantyfikacją po predykatach w~logice drugiego rzędu.}.

Oba te rozumienia terminu ,,Bóg'' muszą, siłą rzeczy, prowadzić do nieco odmiennych zapisów zdań dyskursu religijnego. Przykładowo, przypisywanie Bogu jakiejś własności będzie różnić się w~obu tych rozstrzygnięciach. Załóżmy, że chcemy stwierdzić, że Bóg jest wszechmocny, a~predykat W(x) oznacza ,,x jest wszechmocny''. Jeśli uznajemy, że termin ,,Bóg'' należy do kategorii nazw, zdanie to zapiszemy po prostu jako

W(g).

Jeśli natomiast przypiszemy termin ,,Bóg'' do kategorii deskrypcji określonych, zdanie to przyjmie postać

{\textbackslash}forall x~(G(x) {\textbackslash}to W(x)) lub

{\textbackslash}neg {\textbackslash}exists x~(G(x) {\textbackslash}land {\textbackslash}neg W(x).

Z~pozoru wskazana dwuznaczność może wydawać się trywialnym sporem o~notację i~logicznym ,,dzieleniem włosa na czworo''. W~rzeczywistości jednak rzadko zdarza się, by ustalania o~charakterze formalnym miały tak daleko idące konsekwencje filozoficzne, jak ma to miejsce w~przypadku rozstrzygnięcia niniejszej kwestii. Przyjęcie, że jakiś termin należy do kategorii nazw własnych w~konsekwencji pociąga za sobą przyjęcie istnienia denotowanego przez nazwę obiektu. W~tym wypadku wiązałoby się to z~założeniem istnienia Boga. Drugie z~przedstawionych rozwiązań pozbawione jest już tak silnych zobowiązań o~charakterze ontologicznym\footnote{Co nie oznacza, że pozbawione jest jakichkolwiek wymogów. Jednym z~warunków, jaki powinno ono spełniać, jest warunek niesprzeczności.}. I~choć kwestia ta jest drugorzędna z~punktu widzenia przedmiotu niniejszej pracy -- w~teologii apofatycznej istnienie Boga nie jest podważane, zasadniczo należy zwracać na nią szczególną uwagę przy formalizowaniu dyskursu teologicznego, zwłaszcza w~rozważaniach dotyczących tzw. dowodów za istnieniem Boga. Tak czy owak, jak zauważa Adam Olszewski, kwesta kategorii językowej terminu ,,Bóg'' ,,nie została [\ldots] ostatecznie rozstrzygnięta, a~nawet wyczerpująco rozważona. Zresztą podejście do kwestii nazw i~deskrypcji posiada wiele różnych wersji i~każda z~nich ma swoje \textit{pro} i~\textit{contra}''\footnote{A. Olszewski, \textit{Pewna krytyka teologii naturalnej}, ,,Analecta Cracoviensia'', vol. 46 (2014), s.~214.}.

Skrajne stanowisko w~tej kwestii zajęli Walter Terence Stace oraz Janet Martin Soskice. Co ciekawe, oboje przeprowadzają swoją argumentację w~kontekście teologii negatywnej i~w związku z~teologią milczenia. Stace przekonuje, że termin ,,Bóg'' należy rozumieć jako nazwę własną. Zwraca on uwagę na te fragmenty \textit{Kazań} Mistrza Eckharta, w~których nazywa on Boga bezimiennym, ponieważ ,,wszystkie nazwy, jakie mu nadaje dusza, pochodzą od jej umysłu''\footnote{Mistrz Eckhart, \textit{Kazanie 80}, [w:] Tenże, \textit{Kazania}, tłum. i~oprac. W. Szymona, W~drodze, Poznań 1986, s.~436.} . Stace argumentuje, że w~celu uniknięcia sprzeczności termin ,,Bóg'' należy w~wypowiedziach Eckharta traktować jak nazwę własną, natomiast wszystkie imiona i~nazwy, których przypisywania Bogu Eckhart odmawia, odnoszą się do predykatów języka:

Każdy logik wie, że dowolna nazwa, dowolne słowo w~dowolnym języku, z~wyjątkiem nazw własnych, oznacza pojęcie lub powszechnik [\ldots]. Ani ,,Bóg'', ani ,,Nirwana'' nie oznaczają pojęć. Oba te terminy są nazwami własnymi. Nie ma sprzeczności, gdy Eckhart używa nazwy ,,Bóg'', a~jednocześnie uznaje Go za bezimiennego, ponieważ -- mimo, iż ma On nazwę własną -- nie ma dla Niego nazwy w~sensie słowa oznaczającego pojęcie\footnote{W.T. Stace, \textit{Time and Eternity: An Essay in the Philosophy of Religion}, Princeton University Press, Princeton 1952, s.~24 -- cyt. za W.P. Alston, \textit{Ineffability}, ``The Philosophical Review'', vol 65, nr 4 (1956), s.~511; cudzysłowy moje. Warto zwrócić uwagę na podobieństwo takiej egzegezy Eckharta do ENT. W~kontekście teologii negatywnej problem ten rozważany jest także w: E.Z. Benor, \textit{Meaning and reference in Maimonides' negative theology}, ,,Harvard Theological Review'', vol. 88 (1995), nr 3, ss.~339-360.}.

W~podobnym tonie wypowiada się Soskice\footnote{J.M. Soskice, \textit{Metaphor and Religious Language}, Clarendon Press, Oxford 1985, s.~24.} twierdząc, że Boga możemy tylko nazywać, a~nigdy opisywać. Ona także twierdzi, że kategorią językową terminu ,,Bóg'' jest w~rzeczywistości wyłącznie nazwa własna. Nazwa ta posiada swój desygnat, możemy więc ,,wskazywać na Boga'', ale nie jesteśmy w~stanie podać jego opisu i~przypisać mu żadnej własności.

Takie podejście szybko znalazło krytykę w~pracach m.in., Williama P.~Alstona\footnote{Zob. W.P. Alston, Ineffability, ``The Philosophical Review'', vol. 65 (1956), nr 4, ss.~506-522.} oraz Michaela Durranta\footnote{Zob. M. Durrant, \textit{The Meaning of 'God}'—\textit{I}, [w:] Religion and Philosophy, \textit{Royal Institute of Philosophy Supplement}, vol. 31, red. M. Warner, Cambridge University Press, Cambridge 1992, ss.~71-84. Dyskusji nad słusznością obu sposobów reprezentowania terminu ,,Bóg'' poświęcona jest jego książka: M. Durrant, \textit{The Logical Status of ‘God' and the Function of Theological Sentences}, Macmillan, Edynburg 1973.}. Zwracają oni uwagę na możliwość weryfikacji poprawności danego odniesienia. W~obrębie takiego stanowiska nie ma żadnego sposobu sprawdzenia, czy dwie różne osoby mówiące o~Bogu mówią o~jednym i~tym samym. Boga nie da się wskazać przez ostensję, ani w~żaden inny sposób. Z~tego powodu, jeśli używamy terminu ,,Bóg'', musimy być w~stanie podać przynajmniej jego minimalną deskrypcję -- w~przeciwnym razie nie wiedzielibyśmy nawet, o~czym mówimy. Durrant, odnosząc się wprost do pomysłu Soskice, pyta:

Jeśli jednak nie można w~żaden sposób opisać Boga [\ldots], a~jedynie ,,wskazywać na Niego'' poprzez użycie nazwy własnej ,,Bóg'', to jak można (a) twierdzić w~sposób zrozumiały, że się wskazuje na \textit{Niego} -- aby tak twierdzić w~sposób zrozumiały, Bóg musi być już pomyślany jako osoba lub jako byt analogiczny do osoby; (b) twierdzić, że się w~ogóle na cokolwiek ,,wskazuje''? Mogę twierdzić, że wskazuję na coś -- jeśli już używać tego wyrażenia -- tylko wtedy, gdy mogę zaoferować przynajmniej \textit{jakiś} opis tego, na co wskazuję -- w~przeciwnym razie jak mogę twierdzić, że moje wskazywanie było lub jest \textit{skuteczne}? A~jeśli nie potrafię powiedzieć, co stanowiłoby sukces lub porażkę mojego wskazania, to jak mogę w~ogóle mówić o~,,wskazywaniu''\footnote{M. Durrant, \textit{The Meaning of 'God}'\ldots, dz. cyt., s.~73.}?

W~kontekście powyższego sporu Bocheński rozpatruje dwie przeciwstawne teorie dotyczące sytuacji poznawczej użytkowników dyskurs religijnego. Według pierwszej z~nich, wierny może bezpośrednio spotkać Boga np. w~akcie oddawania czci. Według drugiej teorii wierny nie ma możliwości bezpośredniego kontaktu z~Bogiem -- żyje on ,,wiarą'', w~,,mroku wiary'' a~Bóg znany mu jest tylko dzięki niektórym predykatom obecnym w~pismach lub \textit{credo} danej religii. Zgodnie z~pierwszą teorią, dla wiernych termin ,,Bóg'' byłby nazwą. W~myśl drugiej teorii byłaby to dla nich deskrypcja. Bocheński konkluduje, że to ta druga teoria jest bardziej trafnym opisem współczesnego dyskursu religijnego:

Mimo braku poważniejszych badań empirycznych w~tym zakresie, wydaje się jednak, że większość wiernych, jakich znamy dzisiaj, nie ma żadnego rzeczywistego doświadczenia Boga w~ogóle. Modlą się i~oddają Mu cześć takiemu, jakim Go znają a~nic w~ich wypowiedziach nie wskazuje na to, by w~akcie modlitwy czy innych czynnościach religijnych dowiadywali się czegoś więcej o~Bogu niż ze swego \textit{credo}. Ale \textit{credo} zawsze opisuje Boga i~ze swej natury nie może przekazywać wiedzy o~Nim opartej na osobistej znajomości. Zakładając, że tak jest, mamy prawo stwierdzić, co następuje: termin ,,Bóg'', którym posługuje się dzisiaj większość wiernych, jest deskrypcją\footnote{J.M. Bocheński, \textit{Logika religii}, dz. cyt., s.~381.}.

Przywoływany powyżej Olszewski uważa, że raczej należy

rozumieć ten termin jako deskrypcję określoną (bądź nawet nieokreśloną), gdyż traktowanie go jako nazwy własnej nie pozwala jasno wyjaśnić kwestii z~fundamentalną wieloznacznością, czy też jej wielodenotacyjnością. Natomiast można pojmować termin ,,Bóg'' jako ,,metanazwę'', czyli nazwę deskrypcji określonych, które to deskrypcje denotują Boga, zaś ,,metanazwa'' należy do metajęzyka. [\ldots] Sprawa ta jest ciekawa sama w~sobie i~wymagałaby pogłębionego studium\footnote{A. Olszewski, \textit{Pewna krytyka}\ldots, dz. cyt., ss.~214-215.}.

Niniejsza praca nie jest miejscem na takie pogłębione studium. Oczywiście, w~poniższych rozdziałach dokonuję formalnej rekonstrukcji pewnych zdań teologii negatywnej, a~zdania te często zawierają naturalno-językowy termin ,,Bóg''. W~tych rekonstrukcjach termin ten interpretowany jest w~taki sposób, by jak najlepiej oddać sens formalizowanej teorii. Już powyższe paragrafy sugerują, że użycie predykatu w~celu oddania słowa ,,Bóg'' jest niemożliwe, nieintuicyjne lub prowadzi wprost do sprzeczności w~tych interpretacjach teologii milczenia, które odmawiają przypisywania Bogu jakiegokolwiek predykatu. Z~drugiej strony, w~innych interpretacjach, na przykład w~interpretacji Bocheńskiego, twierdzi się wprost, że ,,Bóg'' jest deskrypcją. Oczywiście, wiele z~analizowanych stanowisk pozostawia dowolność co do wyboru kategorii językowej tego problematycznego terminu. W~takich sytuacjach albo przedstawię obie alternatywy zapisu analizowanych zdań, albo podam argumenty za wyborem danego sposobu rekonstrukcji w~języku formalnym.
