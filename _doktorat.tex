\documentclass[12pt,a4paper,openany,oneside,leqno,titlepage]{report}

\usepackage{amsmath}
\usepackage{amssymb}
\usepackage{amsthm}
%\usepackage[mathcal,mathscr]{euscript}
\usepackage{mathrsfs}
\usepackage{dsfont} % dla \mathds{} - trzeba zainstalować
\usepackage{multirow}
\usepackage[greek,english,polish]{babel}
\usepackage[utf8]{inputenc} % albo cp1250
\usepackage[T1]{fontenc}
\usepackage{purbthesis} % paczka przygotowana przez Piotra Urbańczyka mailto: p.m.urbanczyk@gmail.com
%\usepackage{appendix}
\selectlanguage{polish}

%%%%%%%%%%%%%%%%%%%




%---------hyperrefs-----------
\usepackage[hyphens,spaces,obeyspaces]{url}
\usepackage{hyperref} %to nowsza wersja pakietu url, lepsza m.in. robie linki hyperref
\hypersetup{colorlinks=true, linkcolor=black, citecolor=black, urlcolor=black, breaklinks=true, linktocpage}  %to sprawia że nie widać w pdfie aktywnych linków
\urlstyle{rm}
%---------------------------------------------------


\usepackage{array}
\usepackage{longtable}

\usepackage{footmisc}

\usepackage{geometry}
 \geometry{
 a4paper,
 total={150mm,237mm},
 left=30mm,
 top=30mm,
 }


\usepackage{relsize,etoolbox}% http://ctan.org/pkg/{relsize,etoolbox}
\AtBeginEnvironment{quote}{\small}% Step font down one size relative to current font.

\makeatletter
\renewcommand\small{%
	\@setfontsize\small{11}{15}%
	\abovedisplayskip 10\p@ \@plus2\p@ \@minus5\p@
	\abovedisplayshortskip \z@ \@plus3\p@
	\belowdisplayshortskip 6\p@ \@plus3\p@ \@minus3\p@
	\def\@listi{\leftmargin\leftmargini
		\topsep 6\p@ \@plus2\p@ \@minus2\p@
		\parsep 3\p@ \@plus2\p@ \@minus\p@
		\itemsep \parsep}%
	\belowdisplayskip \abovedisplayskip
}

%\renewcommand\footnotesize{%
%	\@setfontsize\footnotesize{8.3}{11}%
%	\abovedisplayskip 8\p@ \@plus2\p@ \@minus4\p@
%	\abovedisplayshortskip \z@ \@plus\p@
%	\belowdisplayshortskip 4\p@ \@plus2\p@ \@minus2\p@
%	\def\@listi{\leftmargin\leftmargini
%		\topsep 4\p@ \@plus2\p@ \@minus2\p@
%		\parsep 2\p@ \@plus\p@ \@minus\p@
%		\itemsep \parsep}%
%	\belowdisplayskip \abovedisplayskip
%}

\makeatother


\usepackage{tikz}
\usetikzlibrary{positioning,arrows,fit,calc}


\usepackage{chngcntr}
\counterwithout{figure}{chapter}

\usepackage{float}


%\newcommand{\captionfonts}{\small} %wydaje sie, ze nie dziala
\usepackage[font={small},singlelinecheck=false]{caption}


%%%%%%%%%%%%%%%%%%%


\newtheorem{tw}{Twierdzenie}[section]
\newtheorem{lem}[tw]{Lemat}

\theoremstyle{definition}
\newtheorem{defin}{Definicja}[section]

\renewcommand{\normalsize}{\fontsize{13}{18}\selectfont}


\usepackage{graphicx}


\title{Wybrane logiczne aspekty\\teologii apofatycznej}
\author{Piotr Urbańczyk}

\uczelnia{Uniwersytet Papieski Jana Pawła II w~Krakowie}
\wydzial{Wydział Filozoficzny}
%\promotor{ks. dr hab. Adam Olszewski}
\promotordop{ks. dra hab. Adama Olszewskiego} % 'promotor' w genetivie
\praca{doktorska} % licencjacka/magisterska/
\rok{2022}
\miejsce{Kraków}

\draft




\begin{document}

%\interliniatexowa{1.2}
\stronatytulowa

\tableofcontents
\listoffigures

\cleardoublepage
%\part{Aspekt semantyczny}
\part{Teologia apofatyczna jako teologia milczenia}

%\part{Aspekt logiczny \textit{sensu largo}}

\chapter{Teologia apofatyczna jako teologia milczenia}\label{sil-general}
%\chapter{Teologia apofatyczna w~aspekcie semantycznym (i~logicznym \textit{sensu largo})}\label{sil-general}
%\chapter{Teologia apofatyczna w~aspekcie semantycznym}\label{sil-general}
%\chapter{Teologia apofatyczna jako teoria Niewysłowionego}\label{sil-general}

%\section{Wprowadzenie}
%\chapter{Wprowadzenie}


Jedna z~najczęściej spotykanych interpretacji teologii negatywnej zwraca szczególną uwagę na boską transcendencję. Interpretacja ta ma swoje źródła w~tym, że dla wielu teologów negatywnych zaprzeczenia, w~takim samym stopniu jak potwierdzenia, nie mogą stanowić odpowiednich środków do opisu Boga. Transcendentny Bóg jest do tego stopnia ,,ponad'' niedoskonałymi pojęciami ludzkiego języka, że w~zasadzie żadnego z~nich nie powinniśmy mu przypisywać. Zatem -- według takiej interpretacji -- teologia negatywna głosi przede wszystkim, że Bóg jest co do zasady nieopisywalny i~niewyrażalny. To podejście ma daleko idące konsekwencje, ponieważ -- skoro nie możemy przypisać Bogu żadnej własności -- powinniśmy zaprzestać mówienia o~nim i~zamilczeć.

%***

Warto odnotować, że podkreślanie transcendencji Boga jest popularną strategią wśród teologów i~filozofów religii nie zawsze utożsamianych wprost z~nurtem apofatycznym. Trudno nie dostrzec takiego podejścia w~pismach wpływowych dwudziestowiecznych przedstawicieli tych dziedzin, takich jak Rudolf Otto\index[names]{Otto, Rudolf}, Karl Barth\index[names]{Barth, Karl} czy Karl Rahner\index[names]{Rahner, Karl}.

Pierwszy z~nich najbardziej znany jest z~analizy doświadczenia, które -- w~jego opinii -- leży u~podstaw jakiejkolwiek religii. To, co jest doświadczane podczas przeżycia religijnego, Otto\index[names]{Otto, Rudolf} określa terminem \textit{numinosum}. Doświadczeniu numinotycznemu towarzyszą dwa komplementarne uczucia: \textit{misterium tremendum} -- uczucie przerażenia, grozy i~lęku, lecz także mocy i~majestatu oraz \textit{misterium fascinans}\- -- uczucie fascynacji i~zachwytu. Jednakże kluczowym dla niniejszych rozważań jest to, że stanowią one \textit{misterium} -- zawierają element tajemnicy. Dla Otto\index[names]{Otto, Rudolf} rzeczywistość sakralna jest czymś całkowicie innym (niem. \textit{ganz Andere}, ang. \textit{wholly Other}) zarówno w~stosunku do świata naturalnego, jak i~człowieka. Radykalna odmienność tej rzeczywistości sprawia, że jest ona niewysłowiona, niedyskursywna i~irracjonalna\footnote{Por. R.A. Rappaport, \textit{Ritual and Religion in the Making of Humanity}, Cambridge University Press, Cambridge 1999, s.~377.}. Można jej doświadczyć, ale nie da się jej wyrazić słowami. Wykracza ona poza możliwości poznawcze i~zdolności językowe człowieka. Mimo tego, że według Otto\index[names]{Otto, Rudolf} doświadczenie numinotyczne jest prawdziwym spotkaniem ze świętością, nie może być ono przetłumaczone na mowę. Wiedza o~bóstwie jest ponad pojęciowym rozumieniem i~opisem, nie jest możliwe, by ująć ją w~pojęciowe kategorie ludzkiego języka\footnote{Na temat relacji pomiędzy doświadczeniem a~językiem religijnym w~teorii Rudolfa Otto\index[names]{Otto, Rudolf} pojawiło się wiele dyskusji. Zob. np. L. Schlamm, \textit{Numinous Experience and Religious Language}, ,,Religious Studies'', vol. 28 (1992), nr 4, ss.~533-551 oraz L.P. Barnes, \textit{Rudolf Otto and the Limits of Religious Description}, ,,Religious Studies'', vol. 30 (1994), nr 2, ss.~219-230.}. Z~tego powodu właściwą reakcją na doświadczenie \textit{numinosum} pozostaje milczenie\footnote{Zob. R.~Otto, \textit{The Idea of the Holy. An Inquiry into the Non-Rational Factor in the Idea of the Divine and Its Relation to the Rational}, tłum. J.W. Harvey, Oxford University Press, London 1923, ss.~216-220.}.

%***

Przyjęcie, że zasadnicza teza teologii negatywnej głosi, że Bóg przekracza wszystko, co możemy o~nim powiedzieć, jest popularną interpretacją pism takich autorów, jak Grzegorz z~Nyssy\index[names]{Grzegorz z~Nyssy, \textit{św}.}, Pseudo-Dionizy Areopagita\index[names]{Pseudo-Dionizy Areopagita}, Augustyn\index[names]{Augustyn@Augustyn, \textit{św}.}, Tomasz z~Akwinu\index[names]{Tomasz z~Akwinu@Tomasz z~Akwinu, \textit{św}.} czy Mistrz Eck\-hart\index[names]{Eckhart@Mistrz Eckhart}. Wydaje się jednak, że to w~pismach Dionizego\index[names]{Pseudo-Dionizy Areopagita} można odnaleźć najbardziej dosadny wyraz takiego sposobu myślenia o~Bogu.


\section{Pseudo-Dionizyjskie korzenie teologii milczenia}\label{sil-dionizy}

Tożsamość Pseudo-Dionizego Areopagity\index[names]{Pseudo-Dionizy Areopagita} nie jest do końca znana. Współcześnie najczęściej przyjmuje się, że był on syryjskim mnichem żyjącym na przełomie V~oraz VI wieku, wywodzącym się ze szkoły neoplatońskiej. On sam przedstawia siebie jako św.~Dionizego, członka ateńskiej rady sądowniczej -- Areopagu, który jako jeden z~nielicznych Ateńczyków nawrócił się pod wpływem przemówienia św. Pawła\footnote{Zob. K. Corrigan, M.L. Harrington, \textit{Pseudo-Dionysius the Areopagite}, [w:] \textit{The Stanford Encyclopedia of Philosophy}, wyd. zima 2019, red. E.N. Zalta, {\textless}\url{https://plato.stanford.edu/archives/spr2015/entries/pseudo-dionysius-areopagite/}{\textgreater} oraz T. Stępień, \textit{Przedmowa}, [w:] Pesudo-Dionizy Areopagita, \textit{Pisma teologiczne}, tom I, Wydawnictwo Znak, Kraków 1997, s.~9. Nawrócenie Dionizego\index[names]{Pseudo-Dionizy Areopagita} na rynku ateńskim jest wydarzeniem biblijnym, opisanym w~\textit{Dziejach Apostolskich}: Dz~17,~32nn.}. W~swoich pismach niejednokrotnie nazywa św. Pawła\index[names]{Paweł@Paweł z~Tarsu, \textit{św}.} swoim nauczycielem\footnote{Trudno nie dostrzec racji, dla których ,,ojciec teologii negatywnej'' przybrał tożsamość ucznia św.~Pawła. Pewne apofatyczne wątki można odnaleźć już w~sformułowaniach biblijnego autora. Paweł w~wielu swoich listach stosuje taką apofatyczną terminologię, jak:
%$\text{\textgreek{>a}}\text{\textgreek{'o}}\rho \alpha \tau o\varsigma $
\textgreek{>a'oratos}
-- niewidzialny (Rz 1, 20; Kol 1, 15; 1~Tm~1,~17; Hbr 11, 27);
%$\text{\textgreek{>'a}}\rho \rho \eta \tau o\varsigma $
\textgreek{>'arrhtos}
-- niewyrażalny, niewysłowiony (2~Kor~12,~4);
%$\text{\textgreek{>a}}\nu \varepsilon \kappa \delta \iota \text{\textgreek{'h}}\gamma \eta \tau o\varsigma $
\textgreek{>anekdi'hghtos}
-- niewysłowiony, nieopisywalny (2 Kor 9, 15);
%$\text{\textgreek{>a}}\pi \rho \text{\textgreek{'o}}\sigma \iota \tau o\varsigma $
\textgreek{>apr'ositos}
-- niedostępny (1~Tm~6,~16) itp. Por. G. Rocca, \textit{Speaking the Incomprehensible God. Thomas Aquinas on the Interplay of Positive and Negative Theology}, The Catholic University of America Press, Washington 2004, s.~8. Warto dodać, że motywem, od którego Paweł rozpoczął swoje kazanie na Areopagu, był ateński ołtarz poświęcony \textit{Nieznanemu} Bogu (Dz 17, 23).},
a~niektóre spośród listów adresuje do jego towarzyszy, Tymoteusza\index[names]{Tymoteusz@Tymoteusz Apostoł, \textit{św}.} i~Tytusa\index[names]{Tytus@Tytus z~Krety, \textit{św}.}, czy nawet Jana Apostoła\index[names]{Jan Apostoł@Jan Apostoł, \textit{św}.}\footnote{Zob. choćby Pseudo-Dionizy Areopagita, \textit{Listy}, [w:] tenże, \textit{Pisma teologiczne}, tłum. M. Dzielska, Wydawnictwo Znak, Kraków 1997, s.~197-207.}. Taką tożsamość chrześcijańskiego autora zaczęto podważać dopiero na przełomie XV i~XVI wieku. Fakt, że Pseudo-Dionizy\index[names]{Pseudo-Dionizy Areopagita} przez nieomal dziesięć wieków cieszył się niezachwianym autorytetem ucznia św. Pawła\index[names]{Paweł@Paweł z~Tarsu, \textit{św}.} sprawił, że to pisma zebrane w~\textit{Corpus Dionysiacum} wywarły największy wpływ na kształtowanie się późniejszej tradycji apofatycznej -- nie tylko w~późnej patrystyce i~średniowieczu, lecz także w~renesansie i~czasach współczesnych. To właśnie Dionizego\index[names]{Pseudo-Dionizy Areopagita} nazywa się ,,ojcem teologii negatywnej'' -- mimo iż myślenie apofatyczne było obecne w~tradycji chrześcijańskiej niemalże od samego początku\footnote{Zob. P. Sikora, \textit{Logos niepojęty}, Wydawnictwo Universitas, Kraków 2010, s.~58. O~teologii apofatycznej przed Dionizym\index[names]{Pseudo-Dionizy Areopagita} przeczytać można: tamże, Rozdziały I-II; C.M. Stang, \textit{Negative Theology from Gregory of Nyssa to Dionysius the Areopagite}, [w:] \textit{The Wiley-Blackwell Companion to Christian Mysticism}, red. J.A. Lamm, Wiley-Blackwell, Malden 2013, ss.~161-176; G. Rocca, \textit{Speaking the Incomprehensible God}\ldots, dz. cyt., rozdział I.}.


\subsection*{Teologia krytyczna}

Według Johna N. Jonesa\index[names]{Jones, John J.}\footnote{J.N. Jones, \textit{Sculpting God: The Logic of Dionysian Negative Theology}, ,,Harvard Theological Review'', vol. 89 (1996), ss.~355–371.}, teologia Areopagity jest w~dużej mierze teologią krytyczną. Polemizuje ona z~błędnym sposobem mówienia o~Bogu -- takim, który traktuje go jak inne byty, czyli rzeczy lub pojęcia. W~\textit{Teologii mistycznej} Dionizy\index[names]{Pseudo-Dionizy Areopagita} wspomina o~dwóch typach nieporozumień:

\begin{quote}
Mówię tu o~tych, którzy grzęznąc w~bytach nie są zdolni wyobrazić sobie czegoś, co rzeczywiście nadsubstancjalnie istnieje ponad bytami, i~twierdzą, że w~wiedzy, która jest w~nich, płynie znajomość Tego, który wybrał ,,ciemność za swoje schronienie''. Skoro nawet dla tego typu ludzi dostęp do świętych wtajemniczeń nie jest możliwy, to cóż dopiero można powiedzieć o~jeszcze większych profanach, którzy najpośledniejsze spośród bytów poczytują za przekraczającą wszystko najwznioślejszą przyczynę i~zaprzeczają jej wyższości nad ich bezbożnymi idolami o~różnorodnych kształtach\footnote{Pseudo-Dionizy Areopagita, \textit{Teologia mistyczna}, I, 2, [w:] tenże, \textit{Pisma teologiczne}, tłum. M. Dzielska, Wydawnictwo Znak, Kraków 1997, ss.~163-164. O~ile nie podano inaczej, wszystkie poniższe cytaty z~Pseudo-Dionizego Areopagity\index[names]{Pseudo-Dionizy Areopagita} pochodzą z~niniejszego, dwutomowego wydania.}.
\end{quote}

Według Areopagity\index[names]{Pseudo-Dionizy Areopagita}, bałwochwalcy mylą Boga z~przedmiotami, zaś inni ,,profani'' -- prawdopodobnie ma tu na myśli środkowych platoników -- z~pojęciami. W~innym tekście próbuje przedstawić, jak ci ostatni mogliby krytykować wykorzystywanie materialnych obrazów do przestawienia Boga, preferując raczej utożsamianie Boga z~pojęciem lub pojęciami:

\begin{quote}
ktoś [\ldots] mógłby dowodzić, że święci autorzy, chcąc uformować cieleśnie te czyste bezcielesności, powinni je wymodelować i~ukazać pod stosownymi dla nich kształtami im pokrewnymi, na ile to możliwe wzorując się na substancjach najbardziej przez nas cenionych. [\ldots] Tego rodzaju ujęcia lepiej by przecież służyły anagogicznej drodze naszego intelektu i~nie ściągałyby nadprzyrodzonych objawień w~dół, do poziomu absurdalnych niepodobieństw. Tymczasem to postępowanie zdaje się, w~sposób niedopuszczalny, ubliżać boskim mocom i~równocześnie wypacza nasz intelekt, wpędzając go w~pułapkę bezbożnych alegorii\footnote{Pseudo-Dionizy Areopagita, \textit{Hierarchia niebiańska}, II, 2, [w:] tenże  \textit{Pisma teologiczne II}, tłum. M.~Dzielska, Wydawnictwo Znak, Kraków 1999.}.
\end{quote}

Dionizy\index[names]{Pseudo-Dionizy Areopagita} zgadza się z~teologicznym stanowiskiem, wedle którego materialne obrazy nie mogą przedstawiać boskiej istoty. Jednakże odrzuca on także takie rozwiązanie, wedle którego lepszym sposobem przedstawiania Boga są pojęcia. Zarówno przedmioty, jak i~pojęcia nie stanowią odpowiednich reprezentacji Boga z~tego samego powodu -- ponieważ jest on ponad wszelkim bytem. Wnioskiem, jaki wypływa z~tego obrazu, jest fakt, że język, który służy do opisu bytów, nie może być wykorzystywany do opisu Boga. Skoro Bóg nie należy do kategorii bytów, nie można o~nim mówić w~taki sposób, w~jaki mówi się o~czymkolwiek innym. Zdaniem Jonesa\index[names]{Jones, John J.} konsekwencją negatywnego języka teologii Dionizego\index[names]{Pseudo-Dionizy Areopagita} jest niemożliwość powiedzenia o~Bogu czegokolwiek\footnote{Por. niżej -- rozdz.~\ref{sil-jones}.}.


\subsection*{Apofatyzm kompletny}

Podobną interpretację dzieł Dionizego\index[names]{Pseudo-Dionizy Areopagita} przedstawił jeden z~jego najbardziej znanych komentatorów -- Paul Rorem\index[names]{Rorem, Paul}. Teologię negatywną Areopagity\index[names]{Pseudo-Dionizy Areopagita} -- w~przeciwieństwie do tej, którą można odnaleźć u~Grzegorza z~Nyssy\index[names]{Grzegorz z~Nyssy, \textit{św}.}, Maksyma Wyznawcy\index[names]{Maksym Wyznawca, \textit{św}.} czy Bonawentury\index[names]{Bonawentura, \textit{św}.} -- Rorem\index[names]{Rorem, Paul} nazywa apofatyzmem kompletnym\footnote{P. Rorem, \textit{Negative Theologies and the Cross}, ,,Harvard Theological Review'', vol. 101 (2008), ss.~451-464.}. Twierdzi on, że ostatnie dwa rozdziały najbardziej ,,negatywnego'' dzieła Areopagity\index[names]{Pseudo-Dionizy Areopagita} -- \textit{Teologii mistycznej} -- tłumaczą, na czym polega ,,anagogiczna droga przez negację'' i~należy je odczytywać łącznie. Pierwszy z~nich głosi, że najwyższa przyczyna wszystkich rzeczy postrzegalnych sama nie jest postrzegalna, ten drugi natomiast, że najwyższa przyczyna wszystkich pojęć sama nie ma charakteru pojęciowego\footnote{P. Rorem, \textit{Pseudo-Dionysius. A~Commentary on the Texts and an Introduction to Their Influence}, Oxford University Press, New York -- Oxford 1993, ss.~205-213.}.

W~interpretacji Rorema\index[names]{Rorem, Paul} wspomniane dzieło Dionizego\index[names]{Pseudo-Dionizy Areopagita} ma przede wszystkim wartość duchową, a~jego podstawowym celem jest przedstawianie sposobów służących do zjednoczenia z~Bogiem. Droga ku temu prowadzi najpierw poprzez zanegowanie wszystkich rzeczy postrzegalnych, zwłaszcza wszystkich symboli, które mają wskazywać na najwyższa przyczynę. Dzięki temu wstępujemy na poziom pojęć, które są przez te symbole reprezentowane. Kolejny krok, opisany w~rozdziale piątym \textit{Teologii mistycznej}, polega na zanegowaniu także i~tych pojęć:

\begin{quote}
Wznosząc się coraz wyżej, mówimy, że Bóg nie jest duszą, intelektem, wyobrażeniem, mniemaniem, rozumem i~rozumieniem, słowem i~pojmowaniem; [\ldots] nie jest liczbą, porządkiem, wielkością, małością, równością, nierównością, podobieństwem, niepodobieństwem; nie stoi, nie porusza się, nie odpoczywa, nie posiada mocy i~nie jest ani mocą, ani światłością, nie żyje i~nie jest życiem; nie jest substancją, wiecznością i~czasem; [\ldots] nie jest królem ani mądrością, ani jednią ani jednością, ani Boskością, ani dobrocią, ani duchem (o ile znamy ducha), ani synostwem, ani ojcostwem [\ldots]\footnote{Pseudo-Dionizy Areopagita, \textit{Teologia mistyczna}, V.}.
\end{quote}

Jak zauważa Rorem\index[names]{Rorem, Paul}, wiele z~pojęć, które pojawiają się w~niniejszym fragmencie, służyło Dionizemu\index[names]{Pseudo-Dionizy Areopagita} do określenia Boga w~tych traktatach, w~których pozostawał na poziomie teologii pozytywnej\footnote{Por. np. tenże, \textit{Imiona Boskie}: VI-X.}. Co więcej, zanegowane są tutaj także imiona osób Trójcy Świętej. One także należą do kategorii niedoskonałych pojęć ludzkiego umysłu. Ich znaczenie jest skończone, a~więc ostatecznie nie można ich przypisać nieskończonej naturze Boga. Na tym jednak droga do zjednoczenia z~Bogiem się nie kończy. Cały ten proces osiąga swój szczyt, by w~końcu go przekroczyć w~ostatnich zdaniach najczęściej komentowanego rozdziału \textit{Teologii mistycznej}:

\begin{quote}
[\ldots] nie istnieje ani słowo, ani imię, ani wiedza o~Nim; nie jest ani ciemnością, ani światłością, ani błędem, ani prawdą; nie można o~Nim niczego zaprzeczać ani nic pewnego orzekać, bo twierdząc o~Nim lub zaprzeczając rzeczy niższego rzędu, nic o~Nim nie stwierdzamy, ani nie zaprzeczamy. Ta najdoskonalsza przyczyna wszystkiego jest bowiem ponad wszelkim potwierdzeniem i~ponad wszelkim zaprzeczeniem: wyższa nad wszystko, całkowicie niezależna od wszystkiego i~przenosząca wszystko\footnote{Tenże, \textit{Teologia mistyczna}: V.}.
\end{quote}

Na końcu tej drogi zaprzeczamy nawet samym zaprzeczeniom. Ponieważ negacja także należy do pojęć ludzkiego języka, również przy jej pomocy nie da się uchwycić nieskończonego, transcendentnego Boga. Proces wznoszenia się ku Bogu kończy się w~ciemności niewiedzy. Twierdzenia, a~następnie zaprzeczenia, są jedynie środkami do spotkania Boga, ale ostateczne zjednoczenie z~nim odbywa się nie tylko ponad wszelkim twierdzeniem, ale i~ponad wszelkim zaprzeczeniem. Na tym etapie język nie odgrywa już żadnej roli. Swoje rozważania na ten temat Rorem\index[names]{Rorem, Paul} kończy w~następujący sposób:

\begin{quote}
Według ostatnich słów traktatu ,,Bóg jest ponad wszelkim zaprzeczeniem''. Negacja zostaje zanegowana, a~zamroczony umysł ludzki popada w~milczenie. Traktat, \textit{corpus}, jego autor, a~także niniejszy komentarz nie mają nic więcej do powiedzenia. Pozostaje wyłącznie milczenie\footnote{P. Rorem, \textit{Pseudo-Dionysius. A~Commentary on the Texts}\ldots, dz. cyt., s.~213.}.
\end{quote}


\section{Teologia milczenia -- źródła niewysławialności i~kwestia nazewnictwa}\label{sil-int-nazw}

Powyższe paragrafy pokazują, że kluczowa teza omawianej w~tym rozdziale interpretacji teologii negatywnej -- ilustrowanej najchętniej dziełami Pseudo-Dionizego Areopagity\index[names]{Pseudo-Dionizy Areopagita} -- głosi, że ludzki język jest bezsilny wobec zadania opisu i~wyrażenia transcendentnego Boga. Można próbować wskazać dwie (niewykluczające się) przyczyny takiego stanu rzeczy. Przede wszystkim, może być on spowodowany samymi ograniczeniami ludzkiego języka i~niedostatecznymi zdolnościami poznawczymi człowieka, które czynią go niezdolnym do opisania Boga w~należyty sposób. Stanowisko to jest zdecydowanie mniej popularne. Najpoważniejszym autorem, który reprezentuje taki pogląd, jest John Hick\index[names]{Hick, John}. Twierdzi on, że:

\begin{quote}
boska transkategorialność\footnote{\textit{Transcategriality} -- jest to (dość niezgrabne) określenie, które Hick\index[names]{Hick, John} w~późniejszych pracach stosował zamiennie ze słowem ,,niewysławialność'' (\textit{ineffability}).} nie pociąga za sobą wniosku, że Bóstwo nie posiada żadnej natury, lecz jedynie taki, który mówi, że ta natura nie może zostać ujęta w~ludzkich myślach i~języku, ponieważ niewysławialność odnosi się do zdolności poznawczych poznającego\footnote{J. Hick, \textit{Ineffability}, ,,Religious Studies'', vol. 36 (2000), ss.~41-42.}.
\end{quote}

Z~drugiej strony, bezsilność ludzkiego języka w~staraniach o~podanie opisu Boga może być ugruntowana w~samej naturze Boga i~jego transcendencji. Oznaczałoby to, że Bóg jest niewysławialny ze swojej natury, jest to Jego istotna, ,,wewnętrzna'' własność. Wydaje się, że właśnie takie stanowisko jest zdecydowanie częściej reprezentowane zarówno wśród samych myślicieli apofatycznych, jak i~badaczy zajmujących się tym rodzajem teologii\footnote{Por. T. Dzidek, \textit{Teologia apofatyczna. Uznana bezradność rozumu}, [w:] tenże, \textit{Granice rozumu w~teologicznym poznaniu Boga}, Wydawnictwo M, Kraków 2001, ss.~275-314.}. Peter Kügler\index[names]{Kügler, Peter} nawiązując bezpośrednio do pracy Hicka\index[names]{Hick, John} wyraża przekonanie, że:

\begin{quote}
Z~pewnością niewysławialność Boga jest związana z~poznawczymi ograniczeniami ludzkiego umysłu, ale to \textit{natura} Boga jest taka, że ludzki język nie może jej uchwycić\footnote{P. Kügler, \textit{The meaning of mystical ‘darkness'}, ,,Religious Studies'', vol. 41 (2005), s.~101. Podobne stanowisko zajmuje Christopher Insole\index[names]{Insole, Christopher} -- por. C.J. Insole, \textit{Why John Hick cannot, and should not, stay out of the jam pot}, ,,Religious Studies'', vol. 36 (2000), ss.~28-30, Jonathan Jacobs\index[names]{Jacobs, Jonathan D.} -- por. J.D. Jacobs, \textit{The Ineffable, Inconceivable, and Incomprehensible God: Fundamentality and Apophatic Theology}, [w:] \textit{Oxford Studies in Philosophy of Religion VI}, red. R. Audi i~in., Oxford University Press, New York 2015, s.~165 i~wielu innych -- por. dalsze części niniejszego rozdziału.}.
\end{quote}

W~podobnym duchu wypowiadają się inni autorzy. Na przykład Jonathan D.~Jacobs\index[names]{Jacobs, Jonathan D.} zakłada, że:

\begin{quote}
Teologia apofatyczna nie polega na twierdzeniu, że Bóg jedynie jest trudny do opisania, że z~ogromnym wysiłkiem moglibyśmy go sobie wyobrazić, albo że istnieją tylko pewne prawdy o~Bogu, których nie jesteśmy w~stanie pojąć. To nie jest zwykły chwyt retoryczny. [\ldots] Bóg jest istotowo niewyrażalny\footnote{J.D. Jacobs, \textit{The Ineffable, Inconceivable, and Incomprehensible God}\ldots, dz. cyt., s.~159.}.
\end{quote}
Z~kolei Józef Maria Bocheński\index[names]{Bocheński, Józef Maria} pisze o~,,absolutnej'' naturze boskiej niewyrażalności:

\begin{quote}
Jedną z cech charakterystycznych wszystkich tych teorii [\ldots] jest okoliczność, iż ograniczenia nałożone na znaczenie w~związku z~wykorzystaniem ,,tajemnicy'' i~,,tajemniczości'' są traktowane poniekąd absolutnie, co znaczy, że przypisuje się te ograniczenia samej naturze przedmiotu religii sądząc, iż żaden człowiek nie jest w~stanie ich pokonać [\ldots]\footnote{J.M. Bocheński, \textit{Logika religii}, tłum. S. Magala, Wydawnictwo Naukowe PWN, Warszawa 1993, s.~415.}.
\end{quote}

Ponieważ w~ramach tej interpretacji teologii negatywnej największy nacisk kładzie się na to, że Bóg jest zasadniczo, istotowo i~substancjalnie niewysławialny, nieopisywalny i~niewyrażalny, a~ludzki język nie jest w~stanie w~żaden sposób powiedzieć o~nim czegokolwiek, często nazywa się ją
%\textit{teorią Niewyrażalnego}, \textit{teorią Niewysłowionego}%
,,teorią Niewyrażalnego'', ,,teorią tego, co niewysłowione'', ,,teorią Niewysławialnego''%
\footnote{Taką nazwę zaproponował Józef Maria Bocheński\index[names]{Bocheński, Józef Maria} -- zob. tamże, s.~352. Odróżniał on jednak teorię Niewysławialnego od teologii negatywnej \textit{tout court} -- por. tamże, ss.~416-418. Polski tłumacz pracy Bocheńskiego\index[names]{Bocheński, Józef Maria} zostawił ten apofatyczny przymiotnik w~wersji dokonanej, używając sformułowania ,,to, co niewysłowione''. W~oryginale brak możliwości wypowiedzenia czegokolwiek o~Bogu jest mocniej zaznaczony -- Bocheński\index[names]{Bocheński, Józef Maria} pisze o~\textit{the Unspeakable} (zapisując ten przymiotnik wielką literą), a~nie o~\textit{unspoken}. Por. np. J.M. Bocheński, \textit{The Logic of Religion}, New York University Press, New York 1965, ss.~31-36.} lub nawet
%\textit{teologią milczenia}%
,,teologią milczenia''%
\footnote{Autorem tego określenia jest George Englebretsen\index[names]{Englebretsen, George} -- zob. G. Englebretsen, \textit{The Logic of Negative Theology}, ,,New Scholasticism'', vol. 47 (1973), s.~232. W~niniejszej pracy
tych i~podobnych im
nazw będę używał zamiennie.
%tę interpretację teologii apofatycznej.
%tych i~podobnych im nazw będę używał zamiennie.
}.


\section{Paradoksalny charakter teologii milczenia}\label{sil-int-par}

Teologia apofatyczna jest często oskarżana o~sprzeczność. Najczęściej wytykanym problemem tej doktryny jest ciążący na niej pewien rodzaj paradoksu samoodniesienia. Nietrudno postawić taki zarzut także teorii Niewysłowionego -- skoro głosi ona, że o~Bogu nie można nic powiedzieć, tym samym sama mówi coś Bogu, a~zatem jest niespójna i~należy ją odrzucić. Michael Durrant\index[names]{Durrant, Michael} paradoks teologii milczenia ujmuje w~następujący sposób:

\begin{quote}
w~tej teorii, mówiąc, że natura Boga jest zasadniczo niewyrażalna, opisujemy właśnie naturę Boga -- jest mianowicie zasadniczo niewyrażalna. Innymi słowy, ci, którzy bronią tego stanowiska, nie mogą tego robić, nie przecząc sobie\footnote{M. Durrant, \textit{The Meaning of ‘God'– I}, [w:] \textit{Religion and Philosophy}, red. M. Warner, Cambridge University Press, Cambridge 1992, s.~74. Cytat w~j. polskim za P. Rojek, \textit{Logika teologii negatywnej}, ,,Pressje'', nr 29 (2012), ss.~222-223.}.

\end{quote}
Podobnie argumentuje John Hick\index[names]{Hick, John}, który uważa, że nie ma sensu

\begin{quote}
mówić o~$X$, że żadne nasze pojęcie się do niego nie stosuje. Jest bowiem w~oczywisty sposób niemożliwe odnosić się do czegoś, co nie posiada nawet własności bycia możliwym przedmiotem odniesienia\footnote{J. Hick, \textit{An Interpretation of Religion. Human Responses to the Transcendent}, Yale University Press, New Haven -- London 1989, s.~239. Cytuję za P. Sikora, \textit{Logos Niepojęty}, Wydawnictwo Universitas, Kraków 2010, s.~118.}.

\end{quote}
Dodaje on także, że określenie

\begin{quote}
,,taki, że nasze pojęcia się do niego nie stosują'' nie może, jeśli chcemy uniknąć paradoksu, odnosić się do własności, którą opisuje\footnote{Tamże.}.
\end{quote}

To właśnie z~paradoksalnym charakterem teorii Niewysłowionego najczęściej mierzą się ci badacze, którzy próbują rozważać jej logiczno-językową strukturę. Warto więc wyjaśnić, co będziemy rozumieć przez sprzeczność, paradoks i~jakie są ich logiczne konsekwencje.

%\begin{defin}[sprzeczność]
%,,Sprzecznością'' lub ,,antynomią'' będę tu nazywał parę zdań, z~których jedno jest negacją drugiego.
%\end{defin}
%\begin{defin}[paradoks]
%Terminem ,,paradoks'' tradycyjnie zwykło się określać twierdzenie, które prowadzi do zaskakujących lub sprzecznych wniosków. Sprzeczność tak rozumianych paradoksów nie musi stanowić antynomii w~powyższym sensie -- może być sprzecznością pozorną, sprzecznością z~tzw. zdrowym rozsądkiem, z~dobrze uzasadnionymi przekonaniami i~wynikającymi z~nich oczekiwaniami\footnote{Zob. P. Łukowski, \textit{Paradoksy}, Wydawnictwo Uniwersytetu Łódzkiego, Łódź 2006, s.~7.} czy z~,,powszechną opinią''\footnote{Zob. A. Cantini. R. Bruni, \textit{Paradoxes and Contemporary Logic}, [w:] \textit{The Stanford Encyclopedia of Philosophy}, wyd. jesień 2021, red. E.N. Zalta, {\textless}\url{https://plato.stanford.edu/archives/fall2021/entries/paradoxes-contemporary-logic/}{\textgreater}.}. W~niniejszej pracy termin ,,paradoks'', o~ile nie zostanie wskazane inaczej, zasadniczo będzie używany na określenie sprzeczności nietrywialnej. Mówiąc krótko, w~paradoksie będziemy mieć do czynienia z~sytuacją, w~której w~obrębie danej teorii, rachunku lub sytemu zarówno pewne zdanie, jak i~zdanie z~nim sprzeczne, wydają się być jednakowo dowiedzione lub przynajmniej w~jednakowy sposób ugruntowane czy uprawnione do utrzymywania.
%\end{defin}
,,Sprzecznością'' lub ,,antynomią'' zwykło się nazywać koniunkcję lub równoważność zdań, z~których jedno jest negacją drugiego, będącą wnioskiem jakiegoś rozumowania; ewentualnie sytuację, w której dwa zdania sprzeczne występują w jednym ciągu dowodowym.
Terminem ,,paradoks'' tradycyjnie zwykło się określać twierdzenie, które prowadzi do zaskakujących lub sprzecznych wniosków.
W kontekście tego wyjaśnienia można mówić o paradoksach w wąskim i szerokim sensie.
,,Zaskakujące wnioski'' paradoksów rozumianych w sensie szerokim nie muszą stanowić antynomii -- mogą być sprzecznością pozorną,
sprzeciwiać się tzw. zdrowemu rozsądkowi,
dobrze uzasadnionym przekonaniom i~wynikającym z~nich oczekiwaniom\footnote{Zob. P. Łukowski, \textit{Paradoksy}, Wydawnictwo Uniwersytetu Łódzkiego, Łódź 2006, s.~7.}
czy ,,powszechnej opinii''\footnote{Zob. A. Cantini. R. Bruni, \textit{Paradoxes and Contemporary Logic}, [w:] \textit{The Stanford Encyclopedia of Philosophy}, wyd. jesień 2021, red. E.N. Zalta, {\textless}\url{https://plato.stanford.edu/archives/fall2021/entries/paradoxes-contemporary-logic/}{\textgreater}.}.
W~niniejszej pracy termin ,,paradoks'', o~ile nie zostanie wskazane inaczej, zasadniczo będzie używany w~sensie wąskim -- na określenie twierdzenia prowadzącego do sprzeczności.
%w~obrębie jednego systemu lub teorii.
Mówiąc krótko, w~paradoksie będziemy mieć do czynienia z~nietrywialną sytuacją, w~której w~obrębie jednej teorii, rachunku lub sytemu zarówno pewne zdanie, jak i~zdanie z~nim sprzeczne, wydają się być jednakowo dowiedzione lub przynajmniej w~podobny sposób ugruntowane czy uprawnione do utrzymywania.
%\enlargethispage{-5\baselineskip}

W~historii myśli paradoksy stanowiły wyzwanie dla logików i filozofów i niejednokrotnie zmuszały do intelektualnych zmagań. W~obliczu nietrywialnych sprzeczności należało odnaleźć błąd ukryty w~dowodzie, dokonać rewizji założeń lub zrekonstruować cały system. Znanym przykładem z~zakresu logiki będzie tutaj paradoks kłamcy, który inspiruje logików do dziś, czy też antynomia Russella\index[names]{Russell, Bertrand}, które doprowadziła do filozoficznych badań nad podstawami matematyki. Fizyka również zna taki inspirujący wpływ ,,paradoksalnych'' eksperymentów myślowych prowadzących do zaskakujących lub sprzecznych wniosków, jak ma to miejsce na przykład w~przypadku tzw. paradoksu bliźniąt czy paradoksu kota Schrödingera\index[names]{Schrödinger, Erwin}\footnote{Fizyczne ,,paradoksy'' często nie zawierają logicznych sprzeczności, ale bez wątpienia można je uznawać za paradoksy w~tym szerszym, ogólnym sensie.}.

Paradoksy samoodniesienia (zwane także paradoksami samoodnoszenia\footnote{Por. Z.~Tworak, \textit{Kłamstwo kłamcy i~zbiór zbiorów. O~problemie antynomii}, ser. \textit{Filozofia i logika}, nr~89, Wydawnictwo Naukowe UAM, Poznań 2004, rozdz. 4.}, samozwrotności\footnote{Por. np. J. Woleński, \textit{Samozwrotność i~odrzucanie}, ,,Filozofia Nauki'', vol. 1 (1993), nr 1, ss.~89-102.}, cyrkularności\footnote{Por, P. Łukowski, dz. cyt., ss.~178-250.} lub, rzadziej, autoreferencji\footnote{Por. np. R. Poczobut, \textit{Paradoksy w~wyjaśnianiu świadomości}, ,,Ethos. Kwartalnik Instytutu Jana Pawła II KUL'', vol. 26 (2013), nr 1(101), ss.~62-80.}) to najczęstsza grupa paradoksów badanych narzędziami logicznymi. Do grupy tej należą najchętniej rozważane tzw. paradoksy semantyczne\footnote{Paradoksy semantyczne czasem nazywane są także ,,syntaktycznymi'' lub, rzadziej, ,,epistemicznymi''. Por. P. Łukowski, dz. cyt., s.~185. W~niniejszej pracy termin ten rozumiem w standardowy, ugruntowany w logice sposób wyznaczony m.in. klasyczną pracą Ramseya\index[names]{Ramsey, Frank P.}, jako paradoksy ,,językowe'' -- takie, w których istotną rolę odgrywają pojęcia oznaczania i odnoszenia się. Zob. F.P. Ramsey, \textit{The Foundations of Mathematics}, ,,Proceedings of the London Mathematical Society'', vol. s2-25 (1926), nr 1, ss. 338-384.} (na przykład paradoks kłamcy), paradoksy teoriomnogościowe\footnote{Ramsey\index[names]{Ramsey, Frank P.} nazywał takie paradoksy ,,logicznymi''. Zob. tamże.} (na przykład paradoks Russella\index[names]{Russell, Bertrand}) czy paradoksy epistemiczne\footnote{Związane z pojęciem wiedzy. Zob. R. Sorensen, \textit{Epistemic Paradoxes}, [w:] \textit{The Stanford Encyclopedia of Philosophy}, wyd. wiosna 2022, red. E.N. Zalta, <\url{https://plato.stanford.edu/archives/spr2022/entries/epistemic-paradoxes/}>.} (na przykład paradoks znawcy). Choć te trzy grupy paradoksów analizowane są w~obrębie odmiennych rachunków, a~ich konsekwencje istotne są dla innych dyscyplin -- odpowiednio dla teorii prawdy, podstaw matematyki i~epistemologii -- mają one wspólną strukturę i~często bada się je przy użyciu podobnych narzędzi logicznych\footnote{Zob. G. Priest, \textit{The Structure of the Paradoxes of Self-Reference}, ,,Mind'', vol. 103 (1994), nr 409, ss.~25-34. Zob. także T. Bolander, \textit{Self-Reference}, [w:] \textit{The Stanford Encyclopedia of Philosophy}, wyd. jesień 2017, red. E.N. Zalta, {\textless}\url{https://plato.stanford.edu/archives/fall2017/entries/self-reference/}{\textgreater}.}. By dostrzec tę strukturę prześledźmy kilka przykładów samozwrotnych paradoksów semantycznych.

%\bigskip
%\noindent
\subsubsection[Paradoks Grellinga-Nelsona]{Paradoks Grellinga-Nelsona\footnote{Zob. K. Grelling, \textit{The Logical Paradoxes}, ,,Mind'', vol. 45 (1936), nr 180, ss.~481-486.}}\index[names]{Grelling, Kurt}\index[names]{Nelson, Leonard}


%\begin{quotation}
Przymiotnik nazwiemy autologicznym, jeśli posiada wyrażoną przez siebie własność -- na przykład ,,polski'', ,,pięciozgłoskowy'', ,,sześciosylabowy'' itp.

Przymiotnik nazwiemy heterologicznym, jeśli nie posiada wyrażanej przez siebie własności -- na przykład ,,chiński'', ,,jednosylabowy'', ,,złożony'' itp.
%\end{quotation}

Czy przymiotnik ,,heterologiczny'' jest autologiczny czy heterologiczny? Jeśli przyjmiemy, że jest on przymiotnikiem autologicznym, to ma własność, którą wyraża, a~więc jest heterologiczny. Jeśli założymy, że jest on przymiotnikiem heterologicznym, to nie posiada on własności, którą wyraża, a~wyraża własność heterologiczności, a~zatem jest autologiczny. W~konsekwencji ,,heterologiczny'' jest przymiotnikiem autologicznym wtedy i~tylko wtedy, gdy jest przymiotnikiem heterologicznym.


%\bigskip
%\noindent
\subsubsection[Paradoks liczb Richardowskich]{Paradoks liczb Richardowskich\footnote{Zob. J. Richard, \textit{The principles of mathematics and the problem of sets}, [w:] \textit{From Frege to Gödel. A~source book in mathematical logic 1879–1931}, red. J. van Heijenoort, Harvard University Press, Cambridge 1967, ss.~142-144. W~oryginalnym sformułowaniu paradoksu Richard\index[names]{Richard, Jules} mówi o~liczbach rzeczywistych i~wykorzystuje metodę przekątniową, ale nie ma to większego znaczenia dla zrozumienia idei zamozwrotności ilustrowanej przedstawianymi tu paradoksami.}}\index[names]{Richard, Jules}


Rozważmy skończone ciągi słów języka naturalnego definiujące arytmetyczne własności liczb naturalnych, na przykład ,,liczba naturalna posiadająca dokładnie dwa dzielniki całkowite'', ,,liczba pierwsza taka, że liczba większa od niej o~2 też jest liczbą pierwszą'', ,,liczba podzielna przez 7'' itp. i~uporządkujmy te definicje w~sposób leksykograficzny przyporządkowując każdej z~nich liczbę naturalną. Załóżmy, że na miejscu $n$-tym znalazła się definicja liczby Richardowskiej\index[names]{Richard, Jules}: ,,liczba naturalna $k$, która nie posiada własności wyrażonej $k$-tą definicją''.

%\begin{quotation}
\smallskip
\bgroup
\def\arraystretch{1.3}%
\noindent
\begin{tabular*}{.9\linewidth}{@{\extracolsep{\fill}}m{0.2\linewidth}m{0.7\linewidth}@{}}
	\centering 1 &  liczba naturalna posiadająca dokładnie dwa dzielniki całkowite\\
	\centering 2 &  liczba pierwsza taka, że liczba większa od niej o~2 też jest liczbą pierwszą\\
	\centering 3 &  liczba naturalna podzielna przez 7 co najmniej siedmiokrotnie\\
	\centering 4 &  liczba naturalna równa sumie swoich dzielników\\
	\centering $\vdots$ & $\vdots$ \\
	\centering $n$ &  liczba naturalna $k$, która nie posiada własności wyrażonej $k$-tą definicją\\
	\centering $\vdots$ & $\vdots$ \\
\end{tabular*}
\egroup
\smallskip
%\end{quotation}

Czy $n$ jest liczbą Richardowską\index[names]{Richard, Jules}? Jeśli odpowiemy twierdząco, to $n$ nie posiada własności wyrażonej $n$-tą definicją, a~zatem nie jest liczbą Richardowską\index[names]{Richard, Jules}. Jeśli odpowiemy przecząco, to $n$ posiada własność wyrażoną $n$-tą definicją, a~jest to definicja liczby Richardowskiej\index[names]{Richard, Jules}. Zatem $n$ jest liczbą Richardowską\index[names]{Richard, Jules} wtedy i~tylko wtedy, gdy $n$ nie jest liczbą Richardowską\index[names]{Richard, Jules}.


%\bigskip
%\noindent
\subsubsection[Paradoks kłamcy]{Paradoks kłamcy\footnote{Paradoks kłamcy jest niekwestionowanym ,,celebrytą'' wśród paradoksów, któremu poświęcono niejedną monografię, pracę zbiorową czy artykuł. Przegląd podejść do rozwiązania tego paradoksu można odnaleźć w~J.C. Beall, M. Glanzberg, D. Ripley, \textit{Liar Paradox}, [w:] \textit{The Stanford Encyclopedia of Philosophy}, wyd. jesień 2020, red. E.N. Zalta, {\textless}\url{https://plato.stanford.edu/archives/fall2020/entries/liar-paradox/}{\textgreater}. Dobry wgląd z~punktu widzenia teorii prawdy dają także: B. Brożek, \textit{Rola paradoksu kłamcy w~konstrukcji logicznych teorii prawdy}, ,,Zagadnienia Filozoficzne w~Nauce'', nr 30 (2002), ss.~48-88 oraz J. Pruś, \textit{Semantyczna teoria prawdy a~antynomie semantyczne}, ,,Rocznik Filozoficzny Ignatianum'', vol 27 (2021), nr 1, ss.~341-363.}}
%\nopagebreak[2]

Chyba najbardziej eleganckim sformułowaniem tego paradoksu jest stwierdzenie \textit{Hoc est falsum}:
\begin{flalign}
& \text{Zdanie $l$ jest fałszywe.} &&\tag{$l$}\label{sil-klamca}
\end{flalign}


%(l) Zdanie (l) jest fałszywe.\label{sil-klamca}

Jaka jest wartość logiczna zdania \ref{sil-klamca}? Załóżmy wpierw, że \ref{sil-klamca} jest prawdziwe. Zatem jest tak, jak \ref{sil-klamca} głosi, a~mówi ono o~sobie, że jest fałszywe. A~więc \ref{sil-klamca} jest fałszywe. Jeśli natomiast założymy, że \ref{sil-klamca} jest fałszywe, to nie jest tak, jak \ref{sil-klamca} głosi, a~mówi ono o~sobie, że jest fałszywe. Skoro tak nie jest, to \ref{sil-klamca} jest prawdziwe. W~konsekwencji \ref{sil-klamca} jest prawdziwe wtedy i~tylko wtedy, gdy jest fałszywe.


%\bigskip
%\noindent
\subsubsection{Paradoks Niewyrażalnego}

Możemy teraz sformatować paradoks teorii Niewysławianego w~sposób przedstawiony w~powyższych ilustracjach:

%\begin{quotation}
W~myśl teologii milczenia jedynym sposobem wyrażenia Boga jest stwierdzenie, że jest on niewyrażalny.
%\end{quotation}

Czy zatem -- zgodnie z~tą interpretacją teologii negatywnej -- Bóg jest, czy nie jest niewyrażalny? Jeśli przyjmiemy, że nie jest on niewyrażalny, to nie ma innego sposobu, by go wyrazić, a~zatem jest niewyrażalny. Jeśli założymy, że jest on niewyrażalny, to wyraziliśmy, że jest niewyrażalny, a~zatem nie jest niewyrażalny. W~konsekwencji Bóg nie jest niewyrażalny wtedy i~tylko wtedy, gdy jest niewyrażalny.

Oczywiście, powyższy paradoks\footnote{Różne aspekty paradoksu niewyrażalności, także poza kontekstem religijnym, przedstawione zostały w: J. Shaw, \textit{Truth, Paradox, and Ineffable Propositions}, ,,Philosophy and Phenomenological Research'', vol. 86 (2013), nr 1, ss.~64-104. Dla niektórych pragmatycznie nastawionych badaczy samozwrotność tego paradoksu ędzie miała silnie performatywny charakter. W końcu jego sprzeczność polega na \textit{wyrażaniu} tego, co powinno być niewyrażalne. Jednakże według mojej opinii ten sam ,,performatywny'' charakter muszą mieć wszystkie paradoksy samoodniesienia wyrażone w~języku.
%Dodatkowo, inne paradoksy wymagają dodatkowych ,,performatywnych'' założeń. Na przykład paradoks kłamcy wymaga, by założyć, że to, co mówimy, może być prawdziwe lub fałszywe.
} będzie wciąż generował sprzeczność, gdy użyjemy innych przymiotników z~apofatycznego słownika, które negatywni teologowie chętnie przypisują Bogu, jak na przykład ,,nieopisywalny'' czy ,,niewysławialny''.

Warto w~tym miejscu zatrzymać się na chwilę i~wyjaśnić, dlaczego racjonalny dyskurs nie powinien dopuszczać sprzeczności. Co jest złego w~sprzecznościach, że muszą zostać usunięte? Odpowiedź na to pytanie niekoniecznie musi być oczywista. Wydaje się, że najważniejszym argumentem przeciw dopuszczaniu sprzeczności~-- przynajmniej z~punktu widzenia logiki klasycznej i~większości innych użytecznych rachunków logicznych -- jest fakt, że w~systemach, w~których pojawia się para zdań sprzecznych, można dowieść cokolwiek, dowolne poprawnie sformułowane wyrażenie\footnote{Dobry przegląd argumentów za unikaniem sprzeczności -- \textit{notabene} wraz z~próbą ich odrzucenia -- można odnaleźć w: G. Priest, \textit{What so bad about contradictions}?, ,,The Journal of Philosophy'', vol. 95 (1998), nr 8, ss.~410-426.}. Mówiąc bardziej ścisłym językiem, systemy takie ulegają przepełnieniu. Wnioskowanie prowadzące do niedorzeczności sformalizowane jest w~postaci prawa nazywanego \textit{ex contradictione quodlibet} (w literaturze anglosaskiej: \textit{law of explosion}, w~literaturze polskiej: prawo przepełnienia, prawo Dunsa Szkota\index[names]{Duns Szkot, Jan}):
\begin{flalign*}
&  A \to (\neg A \to B). & 
\end{flalign*}

Przedstawiony powyżej związany bezpośrednio z~samozwrotnością paradoks Niewyrażalnego można nazwać ,,wewnętrznym'' paradoksem niniejszej interpretacji teologii apofatycznej. Wynika on bowiem wprost faktu, że w~teologii milczenia Boga próbuje się opisać jako nieopisywalnego. Jednakże -- w~oczywisty sposób -- teoria ta uwikłana jest jeszcze w~inne rodzaje paradoksów. Niektórzy badacze nazywają je paradoksami ,,pośrednimi'' lub ,,zewnętrznymi'' teorii Niewysłowionego. Mogą one przybrać charakter twierdzeniowy lub nietwierdzeniowy.


%\bigskip
%\noindent
\subsubsection{Zewnętrzny twierdzeniowy paradoks teologii milczenia (paradoks niekonsekwentnego apofatyzmu)}

Teologia negatywna powstawała i~była rozwijana w~obrębie wszystkich wielkich tradycji religijnych\footnote{Por. T.D. Knepper, L.E. Kalmanson (red.), \textit{Ineffability: An Exercise in Comparative Philosophy of Religion}, ser. \textit{Comparative Philosophy of Religion}, vol. 1, Springer, Cham 2017.}, a~każda z~tych tradycji posiada swoje święte księgi, \textit{credo} lub po prostu zbiór tez czy dogmatów, które przyjmuje się w~ramach danego dyskursu religijnego, np. ,,Bóg jest jeden w~trzech osobach'', ,,Allah jest jedynym Bogiem, a~Mahomet jest jego prorokiem'', ,,Reinkarnacja to cykl życia i~ponownych narodzin kierowany karmą'' itp. Inaczej mówiąc, każdy dyskurs religijny zawiera jakiś niepusty zbiór zdań, które mają przypisywać Bogu
przedmiotowo-językowe
%jakieś
własności. Teologia apofatyczna nigdy nie rozwijała się w~oderwaniu od teologii katafatycznej, lecz raczej w~obecności i~w głębokim związku ze swoją pozytywną odpowiedniczką. Zasadniczo wydaje się, że myśliciele apofatyczni byli głęboko wierzącymi ludźmi (a przynajmniej nie ma większych powodów, by w~to wątpić\footnote{W~literaturze pojawiają się analizy teologii milczenia w~odniesieniu do ateizmu, lecz raczej w~kontekście obrony przed takim zarzutem. Zob. np. R. Pouivet, \textit{Bocheński on divine ineffability}, ,,Studies in East European Thought'', vol. 65 (2013), nr 1-2, ss.~50-51 lub J.D. Jacobs, \textit{The Ineffable, Inconceivable, and Incomprehensible God...}, dz. cyt., ss.~168-171.}) i~starali się zachowywać prawomyślność wiary -- zachowywali i~wyznawali wszystkie tezy i~dogmaty swoich doktryn religijnych. Nawet więcej, utrzymując lub rozwijając apofatyczne stanowiska, wchodzili w~dyskusje, wyrażali silne przekonania i~wydawali sądy dotyczące katafatycznych ustaleń, często tych najważniejszych -- w~zakresie prawomyślności doktryny. Dobrym przykładem są tutaj zwłaszcza chrześcijańscy przedstawiciele teologii milczenia. Ci sami Ojcowie Kościoła, którzy nalegali, by Boga uznawać za niewyrażalnego, niepojmowalnego i~ponad umysłem, potrafili wchodzić w~zażarte spory o~bardzo precyzyjne sformułowania dogmatyczne, jak np. rozróżnianie między
%$\text{\textgreek{<o}}\mu oo\text{\textgreek{'u}}\sigma \iota o\nu $
\textgreek{<omoo'usioc}
i~\textgreek{<omoio'usioc}
%$\text{\textgreek{<o}}\mu o\iota o\text{\textgreek{'u}}\sigma \iota o\nu $
czy w~spór o~\textit{Filioque}\footnote{Obie kontrowersje dotyczą sformułowań dogmatycznych w obrębie doktryny o~Trójcy Świętej. Termin ,,\textgreek{<omoo'usioc}'' oznacza współistotność, identyczność istoty lub wspólność natury i jest przyjętą i~zatwierdzoną na soborze w Nicei relacją miedzy osobami Trójcy Świętej. Został on wprowadzony do wyznania wiary w odpowiedzi na pewne koncepcje (uznane za herezje), według których relację między osobami Trójcy wyznacza przymiotnik ,,\textgreek{<omoio'usioc}'' oznaczający jedynie podobieństwo istoty. To od tej kontrowersji pochodzą związki frazeologiczne mówiące na przykład o braku zgody ,,ani na jotę'' lub sprawdzaniu czegoś ,,co do joty''. Łac. \textit{Filioque} (,,i Syna'') jest frazą dodawaną w \textit{credo} Kościoła rzymskiego dotyczącą pochodzenia Ducha Świętego. Do dziś ten fragment wyznania wiary jest pomijany w liturgii prawosławnej i~uznawany w~Kościele wschodnim za błąd w~wierze.}.

,,Zewnętrzny'' paradoks teologii milczenia w~swojej twierdzeniowej postaci polega więc na utrzymywaniu, że Bóg jest nieopisywalny, niewyrażalny i~niepojmowalny przy jednoczesnym twierdzeniu, że jest \textit{jakiś}, na przykład wszechmogący, jeden w~trzech osobach, jest stworzycielem świata itp.

Mimo iż paradoks ten ani nie zawiera, ani bezpośrednio nie prowadzi do sprzeczności, przy odpowiedniej interpretacji może pociągać za sobą parę zdań sprzecznych.


%\bigskip
%\noindent
\subsubsection{Zewnętrzny nietwierdzeniowy (performatywny) paradoks teologii milczenia}

Dyskurs religijny często zawiera wypowiedzi, które są performatywami: modlitwą, wyrażeniami pochwały, czci itd. W~takich aktach mowy wierni często 1)~zwracają się do przedmiotu tych wypowiedzi oraz 2)~przypisują przedmiotowi tych wypowiedzi pewną wartość. Wielu badaczy\footnote{Wśród nich np. J.M. Bocheński, \textit{Logika religii}, dz. cyt., ss.~355-356; S. Gäb, \textit{Languages of ineffability: the rediscovery of apophaticism in contemporary analytic philosophy of religion}, [w:] \textit{Negative Knowledge}, red. S. Hüsch i~in., Narr, Tübingen 2020, ss.~191-206; Por. także dyskusję na temat kategorii językowej terminu ,,Bóg'' poniżej -- rozdz.~\ref{sil-kt-jez}.} twierdzi, że zakłada się tu pewne przedmiotowo-językowe własności, co stoi w~sprzeczności z~główną ideą teologii milczenia. Po pierwsze, trudno zwracać się do czegoś, o~czym wiemy jedynie, że nic o~tym nie można powiedzieć\footnote{Zob. dyskusję w~sekcji \ref{sil-kt-jez} poniżej.}. Po drugie, trudno czemuś takiemu przypisywać jakąkolwiek wartość -- w~zasadzie nie wiedząc, czemu jakaś wartość ma być przypisywana. Jak formułuje to Bocheński\index[names]{Bocheński, Józef Maria}:

\begin{quote}
Niemożliwe byłoby zapewne oddawanie czci, tzn. przypisywanie wartości czemuś, o~czym zakładamy wyłącznie, iż nie da się o~tym nic powiedzieć. Takie coś byłoby dla wypowiadającego się całkowicie pozbawione własności przedmiotowo-językowych. Mógłby to być np. szatan. Nie byłoby absolutnie żadnego powodu, by go wielbić, chwalić itd. Musimy przeto odrzucić teorię tego, co niewysłowione\footnote{J.M. Bocheński, \textit{Logika religii}, dz. cyt., s.~356.}.
\end{quote}
,,Zewnętrzny'' nietwierdzeniowy paradoks teologii milczenia wynika zatem z~tego, że niemożliwe wydaje się zwracanie się w~aktach mowy będących np. wyrażeniami czci do czegoś, o~czym nic nie można powiedzieć.

Oczywiście, bez odpowiedniej, zaawansowanej filozoficznie interpretacji i~formalnej rekonstrukcji powyższy paradoks nie będzie prowadził do pary zdań sprzecznych. Można zatem uznać, że jest to paradoks w~ogólnym, szerszym sensie tego słowa. Niektórzy badacze\footnote{Np. P. Rojek, \textit{Logika teologii negatywnej}, dz. cyt., s.~225.} mówią w~jego kontekście o~niezgodności teologii apofatycznej z~praktyką religijną. Z~tego powodu paradoks ten można także nazwać paradoksem performatywnym, praktycznym lub prakseologicznym.

W~poniższych rozdziałach przedstawię rozmaite próby obrony teologii milczenia i~zachowania jej spójności. W~większości przypadków analizowani przeze mnie autorzy mierzą się z~wynikającym z~samoodniesienia ,,wewnętrznym'' paradoksem tej teorii. W~niektórych pracach pojawiają się jednak także bezpośrednie odwołania  do paradoksów ,,zewnętrznych'' -- w~pewnych przypadkach w~celu obrony apofatyzmu i~przed tym rodzajem paradoksu\footnote{Zob. rozdz.~\ref{sil-jac}.}, w~innych po to, by argumentować przeciw teologii milczenia, mimo przekonania o zachowaniu jej spójności w~zakresie paradoksu Niewyrażalnego\footnote{Zob. rozdz.~\ref{sil-boch}.}.


\section{Uwaga o~kategorii językowej terminu ,,Bóg''}\label{sil-kt-jez}

Do poniższych rozdziałów należy dodać jeszcze jedną uwagę. Używany w~języku naturalnym termin ,,Bóg'' jest dwuznaczny w~sensie jego kategorii językowej. W~analitycznej filozofii Boga nie został jeszcze rozstrzygnięty spór, czy należy rozumieć go jako nazwę własną, czy może stanowi on deskrypcję określoną. Jeśli ktoś jest skłonny uważać, że termin ,,Bóg'' jest nazwą, w~języku formalnym będzie przedstawiał go za pomocą stałej indywiduowej (na przykład $g$). W~przeciwnym wypadku termin ten zostanie wyrażony przy pomocy predykatu $G(x)$ rozumianego jako ,,$x$ jest Bogiem'' (lub -- w~zależności od religijnego kontekstu dyskursu -- ,,$x$ jest bóstwem'', ,,$x$ jest boskie'', ,,$x$ jest absolutem'', ,,$x$ jest ostateczną rzeczywistością'' itp.). W~ostateczności $G(x)$ możemy zdefiniować poprzez iloczyn wszystkich predykatów przypisywanych Bogu przez dane wyznanie wiary (tu oznaczony jako $\phi(x)$):
%$$G(x) \equiv_{\text{def}} \exists x (\phi(x) \land \forall y (\phi(y) \to x = y))\footnote{Por. J.M. Bocheński, \textit{Logika religii}, dz. cyt., §21.1. Oczywiście, taki iloczyn mógłby zostać zastąpiony kwantyfikacją po predykatach w~logice drugiego rzędu.}.$$
\begin{flalign*}
&G(x) \equiv_{\text{def}} \exists x (\phi(x) \land \forall y (\phi(y) \to x = y))\footnotemark. &
\end{flalign*}\footnotetext{Por. J.M. Bocheński, \textit{Logika religii}, dz. cyt., §21.1. Oczywiście, taki iloczyn mógłby zostać zastąpiony kwantyfikacją po predykatach w~logice drugiego rzędu.}
\indent Oba te rozumienia terminu ,,Bóg'' muszą, siłą rzeczy, prowadzić do nieco odmiennych zapisów zdań dyskursu religijnego. Przykładowo, przypisywanie Bogu jakiejś własności będzie różnić się w~obu tych rozstrzygnięciach. Załóżmy, że chcemy stwierdzić, że Bóg jest wszechmocny, a~predykat $W(x)$ oznacza ,,$x$ jest wszechmocny''. Jeśli uznajemy, że termin ,,Bóg'' należy do kategorii nazw, zdanie to zapiszemy po prostu jako
%$$W(g).$$
\begin{flalign*}
		& W(g). &
\end{flalign*}
Jeśli natomiast przypiszemy termin ,,Bóg'' do kategorii deskrypcji określonych, zdanie to przyjmie postać
%$$
%\forall x (G(x) \to W(x)) \text{ lub}$$
%$$\neg \exists x (G(x) \land \neg W(x).$$
\begin{flalign*}
		& \forall x (G(x) \to W(x)) \text{ lub} & \\
		& \neg \exists x (G(x) \land \neg W(x)). &
\end{flalign*}


Z~pozoru wskazana dwuznaczność może wydawać się trywialnym sporem o~notację i~logicznym ,,dzieleniem włosa na czworo''. W~rzeczywistości jednak rzadko zdarza się, by ustalania o~charakterze formalnym miały tak daleko idące konsekwencje filozoficzne, jak ma to miejsce w~przypadku rozstrzygnięcia niniejszej kwestii. Przyjęcie, że jakiś termin należy do kategorii nazw własnych, w~konsekwencji pociąga za sobą przyjęcie istnienia denotowanego przez nazwę obiektu. W~tym wypadku wiązałoby się to z~założeniem istnienia Boga. Drugie z~przedstawionych rozwiązań pozbawione jest już tak silnych zobowiązań o~charakterze ontologicznym\footnote{Co nie oznacza, że pozbawione jest jakichkolwiek wymogów. Jednym z~warunków, jaki powinno ono spełniać, jest warunek niesprzeczności.}. I~choć kwestia ta jest drugorzędna z~punktu widzenia przedmiotu niniejszej pracy -- w~teologii apofatycznej istnienie Boga nie jest podważane, zasadniczo należy zwracać na nią szczególną uwagę przy formalizowaniu dyskursu teologicznego, zwłaszcza w~rozważaniach dotyczących tzw. dowodów za istnieniem Boga. Tak czy owak, jak zauważa Adam Olszewski\index[names]{Olszewski, Adam}, kwesta kategorii językowej terminu ,,Bóg'' ,,nie została [\ldots] ostatecznie rozstrzygnięta, a~nawet wyczerpująco rozważona. Zresztą podejście do kwestii nazw i~deskrypcji posiada wiele różnych wersji i~każda z~nich ma swoje \textit{pro} i~\textit{contra}''\footnote{A. Olszewski, \textit{Pewna krytyka teologii naturalnej}, ,,Analecta Cracoviensia'', vol. 46 (2014), s.~214.}.

Skrajne stanowisko w~tej kwestii zajęli Walter Terence Stace\index[names]{Stace, Terence S.} oraz Janet Martin Soskice\index[names]{Soskice, Janet M.}. Co ciekawe, oboje przeprowadzają swoją argumentację w~kontekście teologii negatywnej i~w związku z~teologią milczenia. Stace\index[names]{Stace, Terence S.} przekonuje, że termin ,,Bóg'' należy rozumieć jako nazwę własną. Zwraca on uwagę na te fragmenty \textit{Kazań} Mistrza Eckharta\index[names]{Eckhart@Mistrz Eckhart}, w~których nazywa on Boga bezimiennym, ponieważ ,,wszystkie nazwy, jakie mu nadaje dusza, pochodzą od jej umysłu''\footnote{Mistrz Eckhart, \textit{Kazanie 80}, [w:] tenże, \textit{Kazania}, tłum. i~oprac. W. Szymona, W~drodze, Poznań 1986, s.~436.} . Stace\index[names]{Stace, Terence S.} argumentuje, że w~celu uniknięcia sprzeczności termin ,,Bóg'' należy w~wypowiedziach Eckharta\index[names]{Eckhart@Mistrz Eckhart} traktować jak nazwę własną, natomiast wszystkie imiona i~nazwy, których przypisywania Bogu Eckhart\index[names]{Eckhart@Mistrz Eckhart} nie dopuszcza, odnoszą się do predykatów języka:

\begin{quote}
Każdy logik wie, że dowolna nazwa, dowolne słowo w~dowolnym języku, z~wyjątkiem nazw własnych, oznacza pojęcie lub powszechnik [\ldots]. Ani ,,Bóg'', ani ,,Nirwana'' nie oznaczają pojęć. Oba te terminy są nazwami własnymi. Nie ma sprzeczności, gdy Eckhart\index[names]{Eckhart@Mistrz Eckhart} używa nazwy ,,Bóg'', a~jednocześnie uznaje Go za bezimiennego, ponieważ -- mimo, iż ma On nazwę własną -- nie ma dla Niego nazwy w~sensie słowa oznaczającego pojęcie\footnote{W.T. Stace, \textit{Time and Eternity: An Essay in the Philosophy of Religion}, Princeton University Press, Princeton 1952, s.~24 -- cyt. za W.P. Alston, \textit{Ineffability}, ,,The Philosophical Review'', vol 65, nr 4 (1956), s.~511; cudzysłowy moje. Warto zwrócić uwagę na podobieństwo takiej egzegezy Eckharta\index[names]{Eckhart@Mistrz Eckhart} do \ref{sil-kug-ent} -- zob.~rozdz.~\ref{sil-kugler}. W~kontekście teologii negatywnej problem ten rozważany jest także w: E.Z. Benor, \textit{Meaning and reference in Maimonides' negative theology}, ,,Harvard Theological Review'', vol. 88 (1995), nr 3, ss.~339-360.}.
\end{quote}

W~podobnym tonie wypowiada się Soskice\index[names]{Soskice, Janet M.}\footnote{J.M. Soskice, \textit{Metaphor and Religious Language}, Clarendon Press, Oxford 1985, s.~24.} twierdząc, że Boga możemy tylko nazywać, a~nigdy opisywać. Ona także twierdzi, że kategorią językową terminu ,,Bóg'' jest w~rzeczywistości wyłącznie nazwa własna. Nazwa ta posiada swój desygnat, możemy więc ,,wskazywać
%na
Boga'', ale nie jesteśmy w~stanie podać jego opisu i~przypisać mu żadnej własności.

Takie podejście szybko znalazło krytykę w~pracach m.in. Williama P.~Alstona\index[names]{Alston, William P.}\footnote{Zob. W.P. Alston, \textit{Ineffability}, dz. cyt., ss.~506-522.} oraz Michaela Durranta\index[names]{Durrant, Michael}\footnote{Zob. M. Durrant, \textit{The Meaning of 'God' -- I},
dz. cyt.,
%[w:] Religion and Philosophy, \textit{Royal Institute of Philosophy Supplement}, vol. 31, red. M. Warner, Cambridge University Press, Cambridge 1992,
ss.~71-84. Dyskusji nad słusznością obu sposobów reprezentowania terminu ,,Bóg'' poświęcona jest jego książka: M. Durrant, \textit{The Logical Status of ‘God' and the Function of Theological Sentences}, Macmillan, Edinburgh 1973.}. Zwracają oni uwagę na możliwość weryfikacji poprawności danego odniesienia. W~obrębie takiego stanowiska nie ma żadnego sposobu sprawdzenia, czy dwie różne osoby mówiące o~Bogu mówią o~jednym i~tym samym. Boga nie da się wskazać przez ostensję, ani w~żaden inny sposób. Z~tego powodu, jeśli używamy terminu ,,Bóg'', musimy być w~stanie podać przynajmniej jego minimalną deskrypcję -- w~przeciwnym razie nie wiedzielibyśmy nawet, o~czym mówimy. Durrant\index[names]{Durrant, Michael}, odnosząc się wprost do pomysłu Soskice\index[names]{Soskice, Janet M.}, pyta:

\begin{quote}
Jeśli jednak nie można w~żaden sposób opisać Boga [\ldots], a~jedynie ,,wskazywać na Niego'' poprzez użycie nazwy własnej ,,Bóg'', to jak można (a) twierdzić w~sposób zrozumiały, że się wskazuje na \textit{Niego} -- aby tak twierdzić w~sposób zrozumiały, Bóg musi być już pomyślany jako osoba lub jako byt analogiczny do osoby; (b) twierdzić, że się w~ogóle na cokolwiek ,,wskazuje''? Mogę twierdzić, że wskazuję na coś -- jeśli już używać tego wyrażenia -- tylko wtedy, gdy mogę zaoferować przynajmniej \textit{jakiś} opis tego, na co wskazuję -- w~przeciwnym razie jak mogę twierdzić, że moje wskazywanie było lub jest \textit{skuteczne}? A~jeśli nie potrafię powiedzieć, co stanowiłoby sukces lub porażkę mojego wskazania, to jak mogę w~ogóle mówić o~,,wskazywaniu''\footnote{M. Durrant, \textit{The Meaning of 'God'}\ldots, dz. cyt., s.~73.}?
\end{quote}

W~kontekście powyższego sporu Bocheński\index[names]{Bocheński, Józef Maria} rozpatruje dwie przeciwstawne teorie dotyczące sytuacji poznawczej użytkowników dyskursu religijnego. Według pierwszej z~nich, wierny może bezpośrednio spotkać Boga, na przykład w~akcie oddawania czci. Według drugiej teorii wierny nie ma możliwości bezpośredniego kontaktu z~Bogiem -- żyje on ,,wiarą'', w~,,mroku wiary'', a~Bóg znany mu jest tylko dzięki niektórym predykatom obecnym w~pismach lub \textit{credo} danej religii. Zgodnie z~pierwszą teorią, dla wiernych termin ,,Bóg'' byłby nazwą. W~myśl drugiej teorii byłaby to dla nich deskrypcja. Bocheński\index[names]{Bocheński, Józef Maria} konkluduje, że to ta druga teoria jest bardziej trafnym opisem współczesnego dyskursu religijnego:

\begin{quote}
Mimo braku poważniejszych badań empirycznych w~tym zakresie, wydaje się jednak, że większość wiernych, jakich znamy dzisiaj, nie ma żadnego rzeczywistego doświadczenia Boga w~ogóle. Modlą się i~oddają Mu cześć takiemu, jakim Go znają, a~nic w~ich wypowiedziach nie wskazuje na to, by w~akcie modlitwy czy innych czynnościach religijnych dowiadywali się czegoś więcej o~Bogu niż ze swego \textit{credo}. Ale \textit{credo} zawsze opisuje Boga i~ze swej natury nie może przekazywać wiedzy o~Nim opartej na osobistej znajomości. Zakładając, że tak jest, mamy prawo stwierdzić, co następuje: termin ,,Bóg'', którym posługuje się dzisiaj większość wiernych, jest deskrypcją\footnote{J.M. Bocheński, \textit{Logika religii}, dz. cyt., s.~381.}.
\end{quote}
Przywoływany powyżej Olszewski\index[names]{Olszewski, Adam} uważa, że raczej należy

\begin{quote}
rozumieć ten termin jako deskrypcję określoną (bądź nawet nieokreśloną), gdyż traktowanie go jako nazwy własnej nie pozwala jasno wyjaśnić kwestii z~fundamentalną wieloznacznością, czy też jej wielodenotacyjnością. Natomiast można pojmować termin ,,Bóg'' jako ,,metanazwę'', czyli nazwę deskrypcji określonych, które to deskrypcje denotują Boga, zaś ,,metanazwa'' należy do metajęzyka. [\ldots] Sprawa ta jest ciekawa sama w~sobie i~wymagałaby pogłębionego studium\footnote{A. Olszewski, \textit{Pewna krytyka}\ldots, dz. cyt., ss.~214-215.}.
\end{quote}

Niniejsza praca nie jest miejscem na takie pogłębione studium. Oczywiście, w~poniższych rozdziałach dokonuję formalnej rekonstrukcji pewnych zdań teologii negatywnej, a~zdania te często zawierają naturalno-językowy termin ,,Bóg''. W~tych rekonstrukcjach termin ten interpretowany jest w~taki sposób, by jak najlepiej oddać sens formalizowanej teorii. Jak się okaże, użycie predykatu w~celu oddania słowa ,,Bóg'' jest czasem niemożliwe, nieintuicyjne lub prowadzi wprost do sprzeczności – zwłaszcza w~tych interpretacjach teologii milczenia, które odmawiają przypisywania Bogu jakiegokolwiek predykatu. Z~drugiej strony, w~innych interpretacjach, na przykład w~interpretacji Bocheńskiego\index[names]{Bocheński, Józef Maria}, twierdzi się wprost, że ,,Bóg'' jest deskrypcją. Oczywiście, wiele z~analizowanych stanowisk pozostawia dowolność co do wyboru kategorii językowej tego problematycznego terminu. W~takich sytuacjach albo przedstawię obie alternatywy zapisu analizowanych zdań, albo podam argumenty za wyborem danego sposobu rekonstrukcji w~języku formalnym.


\chapter{Negacja jako ,,zaprzeczenie wszystkich bytów''}\label{sil-jones}

Deklarację podjęcia próby uchronienia teologii apofatycznej przed zarzutami o~logiczną sprzeczność składa między innymi John N. Jones\index[names]{Jones, John J.} w~cytowanym wyżej artykule \textit{Sculpting God: The Logic of Dionysian Negative Theology}\footnote{J.N. Jones, \textit{Sculpting God: The Logic of Dionysian Negative Theology}, ,,Harvard Theological Review'', nr 89 (1996), ss.~355-371.}. Oprócz 1) stawienia czoła sprzecznościom uwikłanym w~doktrynę Pseudo-Dionizego Areopagity\index[names]{Pseudo-Dionizy Areopagita}, Jones\index[names]{Jones, John J.} analizuje fragmenty różnych jego prac po to, by 2) właściwie zinterpretować niejasne wyrażenia zawarte w~\textit{Teologii mistycznej} i~3) odsłonić logiczną strukturę jego apofatycznej wykładni. Sugeruje on, że głównym celem Dionizego\index[names]{Pseudo-Dionizy Areopagita} było zaprzeczenie, że Bóg należy do kategorii bytów. Jones\index[names]{Jones, John J.} uważa, że jego odczytanie doktryny Dionizego\index[names]{Pseudo-Dionizy Areopagita} pozwala w~efekcie utrzymywać sąd o~tym, że -- wbrew powszechnemu mniemaniu -- teologia negatywna jest teorią logicznie spójną.

Fakt, że Bóg przekracza wszelki byt, nadaje strukturę językowi dyskursu teologicznego, bowiem w~takim wypadku nadawanie Bogu jakichkolwiek przymiotów przysługujących bytom jest z~gruntu błędne. Jak wskazuje Jones\index[names]{Jones, John J.}, w~języku naturalnym, gdy ktoś powie ,,$x$ jest białe'', odbiorca tej wiadomości zrozumie także, że $x$~nie jest czerwone. Przypisywanie jakiemuś obiektowi danych własności jest jednocześnie zaprzeczeniem, że posiada on pewne inne własności. Podobnie, w~drugą stronę, gdy ktoś powie, że ,,$x$ nie jest czerwone'', odbiorca może zakładać, że $x$~posiada inne własności -- jest białe, przeźroczyste lub niewidzialne (lecz np. słyszalne). Odbiorca takiej informacji ma prawo zakładać, że istnieje pewna charakterystyka tego obiektu, można mu przypisać pewne własności, mimo że nie będzie wiedział, jakie własności mu rzeczywiście przysługują. Każdy przedmiot posiada jakąś -- taką a~nie inną -- charakterystykę.

Zwykle, gdy mówimy o~rzeczach, twierdzenia i~przeczenia sprzeciwiają się sobie. Według Dionizego\index[names]{Pseudo-Dionizy Areopagita} nie dzieje się tak w~przypadku Boga. Bóg nie jest jednym z~bytów, zatem język służący do opisu bytów nie jest dla Niego właściwy. W~\textit{Imionach Boskich} Dionizy\index[names]{Pseudo-Dionizy Areopagita} przekonuje, że:

\begin{quote}
nie jest tak, że On jest tym, a~nie tamtym, że istnieje w~jakiś jeden sposób, a~w~inny nie, lecz jest wszystkim jako przyczyna wszystkiego, współposiadając i~przedposiadając w~sobie wszelki początek i~kres wszystkich bytów, i~jest ponad wszystkim, istniejąc ponad bytem, wcześniej niż wszystko, co jest. Dlatego też wszystko naraz można o~Nim twierdzić, choć On nie jest żadną rzeczą z~tego wszystkiego: jest wszechkształtny i~wszechpiękny, i~bezkształtny, i~pozbawiony piękna\footnote{Pseudo-Dionizy Areopagita, \textit{Imiona Boskie}, V, 8.}.
\end{quote}

Według interpretacji Jonesa\index[names]{Jones, John J.}, w~języku teologicznym Areopagity\index[names]{Pseudo-Dionizy Areopagita} twierdzenia i~zaprzeczenia tworzą dwie odmienne grupy zdań, i~stanowią odmienne sposoby mówienia o~Bogu. Ponieważ funkcjonują one w~odmienny sposób, nie należy ich ze sobą mieszać. Te pierwsze przedstawiają Boga jako przyczynę wszystkiego, te drugie wyrażają jego transcendencję. Oba sposoby mówienia można stosować naraz zarówno do opisu Boga, jak i~opisu przedmiotów, jednakże w~ten sposób nie zdołamy wyrazić unikalności Boga -- tego, że jest czymś odrębnym od wszystkich bytów.


\section{Twierdzenia}

W~celu uniknięcia stosowania języka, który nie odzwierciedla wyjątkowości Boga, można próbować każde twierdzenie o~Bogu interpretować w~taki sposób, by nie wynikało z~niego żadne zaprzeczenie. Na przykład, można zestawić kilka twierdzeń, które w~języku naturalnym nie mogą służyć do opisania żadnego z~bytów. Jak przekonuje Jones\index[names]{Jones, John J.}, ten sposób mówienia charakteryzują różnorodne imiona nadawane Bogu w~\textit{Imionach Boskich}, takie jak ,,moc sama w~sobie'' czy ,,prawda''. Skoro połączenie tych określeń w~sposób oczywisty nie może odnosić się do żadnego z~bytów, nadaje się ono do wyróżnienia Boga spośród bytów. Dionizy nazywa to teologią pozytywną. Według Jonesa\index[names]{Jones, John J.}, w~orzeczeniach tego typu twierdzeń -- na przykład w~twierdzeniu ,,Bóg jest prawdą'' -- słowo ,,jest'' występuje w~sensie metaforycznym. Bóg jednocześnie posiada, jak i~nie posiada przypisywanych mu w~orzeczniku własności, w~zależności od kontekstu, w~którym to zdanie jest użyte. Jak wskazuje Jones\index[names]{Jones, John J.}, ten podwójny sens -- tożsamość i~odmienność -- wynika z~roli, jaką twierdzenia odgrywają w~wyrażaniu boskiej przyczynowości. Dla Dionizego\index[names]{Pseudo-Dionizy Areopagita}, tak samo jak dla greckich neoplatoników, przyczyna jest jednocześnie immanentna, jak i~odrębna względem swojego skutku. Z~tego powodu twierdził on, że twierdzenia zawarte w~teologii pozytywnej są metaforycznym sposobem wyrażania odmienności Boga od wszelkich bytów.


\section{Zaprzeczenia ,,jednostkowe'' oraz ,,zaprzeczenie wszystkich bytów''}

Według Jonesa\index[names]{Jones, John J.}, z~logicznego punktu widzenia w~doktrynie Dionizyjskiej\index[names]{Pseudo-Dionizy Areopagita} zaprzeczenia są bardziej wymagające od twierdzeń. Gdy ktoś stwierdzi, że Bóg jest mocą i~prawdą, unika w~ten sposób pomylenia go z~przedmiotami i~pojęciami, ponieważ żadne z~pojęć i~przedmiotów nie jest jednocześnie mocą i~prawdą. Gdy jednak ktoś oznajmi, że Bóg nie jest ani mocą, ani prawdą, tak naprawdę nie wykluczy w~ten sposób wiele: Bóg wciąż może być ,,lwem'' albo ,,pijakiem'' lub też wieloma innymi rzeczami\footnote{Przykłady te pochodzą od Dionizego\index[names]{Pseudo-Dionizy Areopagita}. Zob. tenże, \textit{Imiona Boskie}, III, 1.}. Zdaniem Jonesa\index[names]{Jones, John J.}, strategią, jaką przyjął Dionizy\index[names]{Pseudo-Dionizy Areopagita}, by odróżnić Boga od bytów, jest używanie wzajemnie sprzecznych zaprzeczeń o~Bogu -- takich, które nie mogą być jednocześnie prawdziwe, gdy orzekamy je o~jakimkolwiek bycie. Z~tego powodu w~ostatnim rozdziale \textit{Teologii mistycznej} Areopagita pisze, że Bóg ani nie jest żywy, ani nie pozostaje bez życia itp. To właśnie taki nietypowy sposób mówienia nadaje teologii Dionizego\index[names]{Pseudo-Dionizy Areopagita} jej paradoksalny charakter i~w~sposób oczywisty łamie prawo wyłączonego środka.

Dionizy\index[names]{Pseudo-Dionizy Areopagita} często powtarza, że Bóg jest zaprzeczeniem wszelkich bytów. Dlaczego zatem pod koniec \textit{Teologii mistycznej} stwierdza, że jest on także ponad wszelkim zaprzeczeniem\footnote{Tenże, \textit{Teologia mistyczna}, V.}? W~przeciwieństwie do wielu autorów, którzy chętnie widzieliby w~tym kolejną sprzeczność apofatycznej doktryny Dionizego\index[names]{Pseudo-Dionizy Areopagita}, Jones\index[names]{Jones, John J.} próbuje odpowiedzieć na to pytanie wyraźnie odróżniając \textit{zaprzeczenia jednostkowe} od \textit{zaprzeczenia wszystkich bytów}\footnote{J.N. Jones, \textit{Sculpting God}\ldots, dz. cyt., ss.~360-363.}. Uważa on, że to rozróżnienie jest kluczowe dla zrozumienia teologii Dionizego\index[names]{Pseudo-Dionizy Areopagita}. Według jego interpretacji w~teologii Dionizyjskiej\index[names]{Pseudo-Dionizy Areopagita} Bóg przekracza wszystko, co można wyrazić jakimkolwiek jednostkowym stwierdzeniem lub zaprzeczeniem, jest ponad każdym z~jednostkowych zaprzeczeń. Natomiast transcendencja Boga polega właśnie na tym, że jest on zaprzeczeniem wszelkich bytów. Zaprzeczenie wszelkich bytów Jones\index[names]{Jones, John J.} nazywa \textit{negacją} i~-- w~przeciwieństwie do twierdzeń i~zaprzeczeń jednostkowych -- uznaje ją za uprzywilejowany sposób mówienia o~Bogu, powołując się na fragment \textit{Hierarchii niebiańskiej}:

\begin{quote}
[\ldots] ale równocześnie w~tych samych Pismach wielbi się w~sposób pozaświatowy Boską Zwierzchność w~objawiających Ją, zupełnie niepodobnych do Niej, przedstawieniach. Opisuje się ją jako Niewidzialną i~Nieskończoną, i~Niepojętą, i~jeszcze innymi imionami, które nie oddają tego, czym Ona jest, a~raczej to, czym Ona nie jest. Dlatego wydaje mi się, że ten sposób -- mówienia o~Niej przez negację -- bardziej odpowiada jej dostojności, bo przecież, zgodnie z~pouczeniami naszej tajemnicy i~świętej tradycji, zwykliśmy twierdzić, że nie istnieje Ona podobnie jak inne byty i~że zupełnie nic nie wiemy o~Jej, nie dającej się pojąć intelektem i~wyrazić żadnym słowem, bezgranicznej nadsubstancjalności\footnote{Pseudo-Dionizy Areopagita, \textit{Hierarchia niebiańska}, II, 3.}.
\end{quote}

Dla Jonesa\index[names]{Jones, John J.} przedstawiona w~ostatnim zdaniu niniejszego fragmentu ,,negacja'' jest wręcz ,,tautologią'' -- Bóg jest niepojmowalny dlatego, że nie możemy pojąć Jego niepojmowalności. Własność przypisywana tutaj Bogu, jak i~uzasadnienie takiego przypisywania są tożsame. Wszystkie zaprzeczenia jednostkowe są ,,logicznymi operacjami'', które wynikają z~tego typu negacji. Na tym polega różnica między tymi dwoma sposobami mówienia o~Bogu -- w~przeciwieństwie do zaprzeczeń jednostkowych, ,,negacja odnosi się do (nie)możliwości poznania i~powiedzenia czegokolwiek o~Bogu. Jest to, jeśli można tak powiedzieć, reguła drugiego rzędu posługiwania się nazwami pierwszego rzędu''\footnote{J.N. Jones, \textit{Sculpting God}\ldots, dz. cyt., s.~368. Cytat w~j. polskim za P. Rojek, \textit{Logika teologii negatywnej}, ,,Pressje'', nr 29 (2012), s.~222.}.


\section{Dyskusja}

Wydaje się, że z~deklarowanych na początku swojej pracy celów Jonesowi\index[names]{Jones, John J.} udało się jedynie dokonać reinterpretacji teologii negatywnej Pseudo-Dionizego Areopagity\index[names]{Pseudo-Dionizy Areopagita}. Według niej teologia Dionizyjska\index[names]{Pseudo-Dionizy Areopagita} polega na radykalnym oddzieleniu Boga od kategorii bytów. Sposób, w~jaki Jones\index[names]{Jones, John J.} odczytuje doktrynę Dionizego\index[names]{Pseudo-Dionizy Areopagita} można streścić w~następującym sformułowaniu: ,,Bóg nie jest bytem, a~zatem nie może zostać poznany i~wyrażony w~taki sposób, w~jaki poznaje i~mówi się o~bytach''\footnote{Tamże, s.~369.}. Jego argumentację za takim odczytywaniem Dionizego\index[names]{Pseudo-Dionizy Areopagita} uznaję za przekonującą i~dostatecznie podpartą źródłami.

W~mojej opinii Jones\index[names]{Jones, John J.} nie był jednak wystarczająco skuteczny w~kwestii próby uchronienia teologii apofatycznej przed zarzutami o~sprzeczność. Można jeszcze bronić tezy, że w~interpretacji Jonesa\index[names]{Jones, John J.} nie ma sprzeczności w~tych sformułowaniach Dionizego\index[names]{Pseudo-Dionizy Areopagita}, w~których twierdzi, że Bóg jest jednocześnie zaprzeczeniem wszelkich bytów oraz jest ponad wszelkim zaprzeczeniem. Spójność jest tutaj zachowana zarówno dzięki odpowiedniemu rozumieniu słowa ,,wszelkie'', jak i~na oddzieleniu dwóch pojęć: zaprzeczeń jednostkowych oraz zaprzeczeń wszelkich bytów (negacji). Jednakże za niewystarczające uważam argumenty, które mają służyć do wykazania spójności tego, co Jones\index[names]{Jones, John J.} nazywa zaprzeczeniami jednostkowymi. Ostatni rozdział \textit{Teologii mistycznej} przepełniony jest takimi sformułowaniami Dionizego\index[names]{Pseudo-Dionizy Areopagita}, w~których twierdzi, że Bóg nie jest ani podobieństwem, ani niepodobieństwem, nie znajduje się w~ruchu, ani w~bezruchu itp. Według Jonesa\index[names]{Jones, John J.} łącznik ,,jest'' występujący w~tego typu sformułowaniach nie został użyty w~sensie literalnym, lecz metaforycznym, a~zatem każdy, kto chce oskarżyć Dionizego\index[names]{Pseudo-Dionizy Areopagita} o~sprzeczność, powinien wyjaśnić metaforyczny sens tego wyrażenia i~dopiero wtedy wykazać sprzeczność\footnote{Tamże, s.~365.}. Pomijając fakt, że takie przerzucenie ciężaru dowodu (mimo deklaracji zawartych we wstępie) jest cokolwiek nieuczciwe, trudno wskazać na tyle słabe i~,,metaforyczne'' rozumienie słowa ,,jest'', by po przypisaniu dwóch przeciwnych własności temu samemu obiektowi nie produkowało ono sprzeczności. Można jeszcze próbować obronić stanowisko Jonesa\index[names]{Jones, John J.} powołując się na fakt, że w~jego interpretacji język zaprzeczeń jednostkowych jest podporządkowanym sposobem mówienia o~Bogu, który może zostać właściwe określony dopiero w~języku ,,negacji''. Ale to właśnie owe negacje produkują najbardziej typowe dla teologii negatywnej paradoksy. Skoro negacja jest właściwym sposobem mówienia o~Bogu, jak może polegać na niemożliwości powiedzenia o~nim czegokolwiek? Nieprzekonujące są też tłumaczenia Jonesa ,,tautologiczności'' negacji Dionizego\index[names]{Pseudo-Dionizy Areopagita} -- skoro nie pojmujemy niepojmowalności Boga, skąd wiemy, że jest niepojmowalny? W~końcu, skoro negacja dotyczy niemożliwości powiedzenia czegokolwiek o~Bogu, co nas uprawnia do tego, by nazywać Go niepojmowalnym? Tego typu pytania można mnożyć dla każdej ,,negatywnej'' własności przypisanej Bogu w~\textit{Corpus Dionysiacum}. Zresztą Jones\index[names]{Jones, John J.} sam przyznaje, że wielu myślicieli jego odczytanie teologii negatywnej uzna za twierdzenie czegoś o~Bogu -- pewien rodzaj wiedzy i~mówienia o~nim\footnote{Tamże, s.~369.}, wskazując tym samym paradoksalny charakter takiej interpretacji.

Jednakże, z~punktu widzenia niniejszej pracy, artykuł Jonesa\index[names]{Jones, John J.} najbardziej traci na tym, że -- mimo początkowych deklaracji -- nie dokonał on logicznej rekonstrukcji apofatycznej wykładni Areopagity\index[names]{Pseudo-Dionizy Areopagita}. Niestety, w~swoich rozważaniach Jones\index[names]{Jones, John J.} nawet nie próbuje korzystać ze środków logiki formalnej. Co prawda, w~tekście można odnaleźć nieliczne pojęcia właściwe dla rozważań tego typu -- takie jak ,,tautologia'', ,,negacja'' czy ,,reguła''. Nietrudno jednak zauważyć, że są one używane w~sensie
oderwanym
%często bardzo dalekim
od tego znanego z~roztrząsań logicznych. Pomimo to żaden z~tych terminów nie został należycie zdefiniowany. Ktoś, kto chciałby odsłonić logiczną strukturę teologii apofatycznej bazując na tej interpretacji, mógłby ostatecznie skorzystać z~podziału na trzy sposoby mówienia o~Bogu, który Jones\index[names]{Jones, John J.} wyróżnił w~pismach Dionizego\index[names]{Pseudo-Dionizy Areopagita} (i nazwał twierdzeniami, zaprzeczeniami jednostkowymi oraz zaprzeczeniem wszelkich bytów, czyli ,,negacją''). Problem jednak w~tym, że wywód Jonesa\index[names]{Jones, John J.} jest tylko nieco bardziej klarowny od pism Dionizego\index[names]{Pseudo-Dionizy Areopagita}, które są przedmiotem jego interpretacji, i~taka praca wymagałaby wpierw odpowiedniej interpretacji jego interpretacji (\textit{sic}!).


\chapter{Słaba teoria Niewysłowionego}\label{sil-slabatn}

Próbę obrony teorii Niewysłowionego przed zarzutami o~sprzeczność podjął także jeden z~najbardziej wpływowych współczesnych myślicieli odwołujących się do nurtu apofatycznego -- John Hick\index[names]{Hick, John}\footnote{Por. P. Sikora, \textit{Logos Niepojęty}, Wydawnictwo Universitas, Kraków 2010, s.~118.}. Hick\index[names]{Hick, John} co prawda nawiązuje wprost do doktryny Pseudo-Dionizego Areopagity\index[names]{Pseudo-Dionizy Areopagita}\footnote{J. Hick, \textit{Ineffability}, ,,Religious Studies'', vol. 36 (2000), ss.~37-40.}, ale jego punktem wyjścia bynajmniej nie jest interpretacja dzieł średniowiecznego filozofa, lecz studium religii jako takiej\footnote{Zob. J. Hick, \textit{An Interpretation of Religion. Human Responses to the Transcendent}, Yale University Press, New Haven -- London 1989.}. Celem, jaki przyświeca jego badaniom, jest uzasadnienie fenomenu religijnego pluralizmu\footnote{Zob. tamże, rozdział 14.}.

Przyjmując taki punkt wyjścia, Hick\index[names]{Hick, John} na określenie rzeczywistości transcendentnej nie używa terminu ,,Bóg'', czy ,,bóstwo'', lecz ,,Rzeczywiste'' (\textit{the Real})\footnote{Polskie tłumaczenie terminologii Hicka\index[names]{Hick, John} podaję za P. Sikora, \textit{Logos Niepojęty}, dz. cyt.}. W~swojej teorii religijnego pluralizmu posługuje się kantowskim rozróżnieniem na Rzeczywiste samo w~sobie, \textit{an sich} (\textit{noumenon}) oraz Rzeczywiste, jakie jest doświadczane i~pojmowane w~rozmaitych religiach (\textit{phenomenon}). Boska rzeczywistość doświadczana przez liczne wspólnoty religijne może zostać ujęta w~pojęcia i~kategorie ludzkiego języka. Może być na przykład osobowym Bogiem, stwórcą świata lub bezosobowym absolutem, jaki można spotkać w~niektórych tradycjach religijnych. Natomiast Rzeczywiste samo w~sobie jest bezwzględnie niewysławialne.

\begin{quote}
Przez ,,niewysławialne'' rozumiem [\ldots] posiadanie takiej natury, która jest poza zasięgiem siatki ludzkich pojęć. Dlatego o~Rzeczywistym samym w~sobie nie można w~sposób uprawniony twierdzić, że jest osobowe lub bezosobowe, celowe lub niecelowe, dobre lub złe, substancją lub procesem, czy nawet że jest jedno lub jest ich wiele. Jednakże zaprzeczając na przykład temu, że Rzeczywiste jest osobowe, nie twierdzi się tym samym, że jest bezosobowe, lecz raczej, że taka polaryzacja pojęciowa czy dualizm nie ma do niego zastosowania\footnote{J. Hick, \textit{A~Christian Theology of Religions: The Rainbow of Faiths}, Westminster John Knox Press, Louisville 1995, ss.~27-28.}.
\end{quote}

Jednakże, jak wykazałem powyżej\footnote{Por. rozdz.~\ref{sil-int-par}.}, Hick\index[names]{Hick, John} zdaje sobie sprawę z~paradoksu, do jakiego prowadzi takie podejście. Ma świadomość, że próba opisu czegoś, co z~zasady jest niewysławialne lub inaczej: odnoszenie się do czegoś, co z~zasady nie może być przedmiotem odniesienia, jest pozbawione sensu. Takie podejście Hick\index[names]{Hick, John} nazywa silną teorią Niewysłowionego (\textit{strong ineffability}). W~zamian za nią proponuje jej słabszą wersję.


\section{Własności substancjalne i~formalne}

Kluczowym elementem słabej teorii Niewysłowionego Hicka\index[names]{Hick, John} jest rozróżnienie pomiędzy dwoma rodzajami własności: 1) własnościami \textit{substancjalnymi} (treściowymi)\footnote{W~\textit{A Christian Theology of Religions}\ldots Hick\index[names]{Hick, John} nazywa te własności wewnętrznymi (\textit{intrinsic}).\label{przypis-hick-wew}} oraz 2) ,,generowanymi logicznie'' własnościami czysto \textit{formalnymi}. Słaba teoria Niewysłowionego, którą -- zdaniem Hicka\index[names]{Hick, John} -- reprezentowali teologowie apofatyczni wszystkich większych religii, polega na stwierdzeniu, że Bogu (Rzeczywistemu) nie można przypisać żadnej własności substancjalnej. Dopuszcza ona jednak możliwość przypisywania mu własności formalnych. Ponieważ ,,żaden konkretny opis, który ma zastosowanie w~dziedzinie ludzkiego doświadczenia nie może zostać zastosowany w~sposób dosłowny do jego niedoświadczalnej podstawy'', o~Rzeczywistym samym w~sobie możemy jedynie formułować ,,pewne czysto formalne wypowiedzi''\footnote{J. Hick, \textit{An Interpretation of Religion}\ldots, dz. cyt., s.~246. Interesujące jest podobieństwo tego sformułowania do rozumienia analogii relacji zaproponowanego przez Józefa Marię Bocheńskiego\index[names]{Bocheński, Józef Maria}. Bocheński  postuluje, że dwie relacje można uznać za analogiczne wtedy, gdy ich własności formalne są wspólne. Zob. J.M. Bocheński, \textit{Logika religii}, tłum. S. Magala, [w:] tenże, \textit{Logika i~filozofia}, Wydawnictwo Naukowe PWN, Warszawa 1993, ss. 452-456. Jak się jednak okaże, nie do końca wiadomo, jak Hick rozumie pojęcie własności formalnych.}.
%WNT W~przeciwieństwie do własności formalnych, Bogu (Rzeczywistemu) nie można przypisać własności substancjalnych.
\begin{flalign*}
		& \parbox[t]{.87\linewidth}{ 
		Bogu (Rzeczywistemu) nie można przypisać własności substancjalnych,
		(w~przeciwieństwie do własności formalnych, które mu przysługują).} &\tag{WNT}\label{sil-hick-wnt}
	\end{flalign*}

Niestety, Hick\index[names]{Hick, John} nie podaje żadnej definicji, która pozwalałaby ostro odróżnić jedne własności od drugich, a~czytelnika pozostawia wyłącznie z~krótką listą obu rodzajów takich predykatów:

\begin{itemize}
\item własności substancjalne: ,,jest dobry'', ,,jest potężny'', ,,posiada wiedzę''.
\item własności formalne: ,,jest przedmiotem odniesienia jakiegoś terminu'', ,,jest taki, że nasze substancjalne pojęcia się do niego nie stosują''\footnote{J. Hick, \textit{An Interpretation of Religion}\ldots, dz. cyt., s.~239. Por. P. Sikora, \textit{Logos Niepojęty}, dz. cyt., s.~119-120.}.
\end{itemize}
Takie podejście nie tylko otworzyło drogę dla różnych interpretacji jego teorii, lecz także bezpośrednio naraziło ją na rozmaitą krytykę\footnote{Według Hicka\index[names]{Hick, John} w~ciągu piętnastu lat od opublikowania \textit{An Interpretation of Religion}\ldots powstało ponad sto trzydzieści artykułów i~około stu książek zawierających głosy krytyczne na temat jego pracy -- por. J. Hick, \textit{Introduction to the Second Edition}, [w:] tenże, \textit{An Interpretation of Religion}\ldots, dz. cyt., s.~xvii.}.

William Rowe\index[names]{Rowe, William}\footnote{W.L. Rowe, \textit{Religious pluralism}, ,,Religious Studies'', vol. 35 (1999), ss.~139-150.} -- tak jak większość głosów krytycznych -- podkreśla brak rzetelnego zdefiniowania tak ważnych dla tej koncepcji pojęć. Bazując na pracy Hicka\index[names]{Hick, John} próbuje on podać swoje własne definicje. Według niego ,,własność formalna Rzeczywistego jest pewną \textit{abstrakcyjną} charakterystyką posiadaną przez Rzeczywiste, która jest warunkiem możliwości naszego odnoszenia się do niego lub postulowania go jako tego, co spotykane za pośrednictwem osobowych bóstw lub nieosobowych absolutów wielkich religii świata''. Natomiast własność substancjalna Rzeczywistego to ,,istotna (istotowa) własność należąca do jego natury''\footnote{Tamże, s.~145. Pierwszy cytat za: P. Sikora, \textit{Logos Niepojęty}, dz. cyt., s.~120. Warto zauważyć, że w~niektórych swoich pracach Hick\index[names]{Hick, John} własności substancjalne nazywa własnościami wewnętrznymi -- zob. przypis \ref{przypis-hick-wew}.}.

Podobny zarzut przeciwko słabej teorii Niewysłowionego Hicka\index[names]{Hick, John} wysuwa Christopher Insole\index[names]{Insole, Christopher}. On także uważa, że pojęcia własności formalnych i~substancjalnych nie zostały dostatecznie dobrze określone. Odrzuca on jednak koncepcję Rowe'a\index[names]{Rowe, William} twierdząc, że terminy użyte w~jego definicjach -- \textit{abstrakcyjny} oraz \textit{istotowy} -- nie są wcale mniej tajemnicze, niż terminy \textit{formalny i~substancjalny} obecne w~oryginalnej teorii Hicka\index[names]{Hick, John}. W~zamian za nie proponuje swoje definicje. Według niego własność formalna to taka własność, która ,,wyłącznie i~bezpośrednio określa, jakie \textit{inne} własności mogą (lub nie) być przypisane przedmiotowi''\footnote{C.J. Insole, \textit{Why John Hick cannot, and should not, stay out of the jam pot}, ,,Religious Studies'', vol. 36 (2000), s.~28.}. Natomiast własności substancjalne nie zawierają bezpośrednio takiej informacji.
%, choć niektóre z~nich -- takie, jak ,,wszechmocny'', ,,wszechmogący'' czy ,,wszechwiedzący'' -- mogą wymagać współwystępowania.

Jeszcze inaczej pojęcie własności formalnej rozumie Alvin Plantinga\index[names]{Plantinga, Alvin}. W~jego odczytaniu są to takie własności, które posiada \textit{wszystko} i~-- ponadto -- wszystko posiada je \textit{z~konieczności}. Aby własność została uznana za formalną, powinna ona spełniać oba te warunki. Przykładem będzie tutaj: ,,jest tożsamy z~samym sobą'', ,,posiada własności'', ,,posiada własności istotowe'', ,,jest taki, że 7+5=12''\footnote{Zob. A. Plantinga, \textit{Warranted Christian Belief}, Oxford University Press, New York 2000, s.~47. Plantinga nie definiuje osobno własności substancjalnych.}.

Ciekawą interpretacją rozwiązania Hicka byłoby przyjęcie, że własności substancjalne stanowią własności przedmiotowo-językowe, natomiast własności formalne należą do zbioru własności metajęzykowych. Takie odczytanie sprowadziłoby teorię Hicka do klasycznej teorii niewysławialności\footnote{Zob. rozdz. \ref{sil-boch}}.


\section{Dyskusja}

%Jednakże to nie na niezdefiniowaniu kluczowych pojęć polega główny zarzut, jaki wyżej wymieni autorzy wysuwają przeciw słabej teorii Niewysłowionego.
Główny zarzut, jaki wyżej wymienieni autorzy wysuwają przeciw słabej teorii Niewysłowionego, nie polega jednak na wskazaniu braku definicji kluczowych pojęć.
Hick\index[names]{Hick, John} twierdzi, że z~niewysławialności Rzeczywistego wynika, że nie można o~nim orzekać pewnych par własności, np. nie można powiedzieć o~nim, że jest dobry lub zły, czy że jest osobowy lub nieosobowy. Zdaniem Rowe'a\index[names]{Rowe, William}, Hick\index[names]{Hick, John} eksplikując w~taki sposób działanie swojej teorii używa dwóch rodzajów par własności:

\begin{itemize}
\item własności przeciwnych (np. dobry -- zły, czerwony -- zielony);
\item własności sprzecznych (np. osobowy -- nieosobowy, formalny -- nieformalny).
\end{itemize}
Tak jak zdania przeciwne nie mogą być jednocześnie prawdziwe -- tak własności przeciwnych nie można przypisać jednocześnie temu samemu obiektowi. Nie wyklucza to jednak sytuacji, w~której ani jedna, ani druga własność z~pary własności przeciwnych nie przysługuje danemu obiektowi, np. woda w~jeziorze może nie być gorąca ani zimna (lecz letnia), albo cytryna może nie być czerwona ani zielona (lecz żółta). Natomiast własności sprzeczne obarczone są silniejszym kryterium. Tak jak zdania sprzeczne nie tylko nie mogą być jednocześnie prawdziwe, lecz także nie mogą być jednocześnie fałszywe -- podobnie zawsze jedna z~pary własności sprzecznych musi danemu obiektowi przysługiwać. A~zatem główny zarzut Rowe'a\index[names]{Rowe, William} przeciw słabej teorii Niewysłowionego polega na tym, że nie widzi on sposobu, w~jaki Rzeczywiste może uniknąć posiadania jednej z~pary własności sprzecznych\footnote{W.L. Rowe, \textit{Religious pluralism}, dz. cyt., s.~146.} lub -- inaczej mówiąc -- że teoria ta nie pokazuje, w~jaki sposób i~dlaczego ogranicza się w~niej działanie prawa wyłączonego środka.


\begin{figure}[H]
\begin{center}

 \begin{tikzpicture}[node distance=5cm]

    \node (A) {\begin{tabular}{c} $P(a)$ \\ \end{tabular}};
    \node (E) [right=of A, xshift=2cm] {\begin{tabular}{c} $Q(a)$ \end{tabular}};
    \node (I) [below=of A] {\begin{tabular}{c}  $\neg Q(a)$ \end{tabular}};
    \node (O) at (I-|E) {\begin{tabular}{c}  $\neg P(a)$ \end{tabular}};

    \coordinate (CENTER) at ($(A)!0.5!(O)$);

    \node (contra) at (CENTER) {\begin{tabular}{c}Własności sprzeczne \end{tabular}};
    \path[-] (A) edge node[] {\begin{tabular}{c}Własności przeciwne\\ \ \end{tabular}} (E);
    \path[-] (I) edge node[] {\begin{tabular}{c}\ \\Własności podprzeciwne \end{tabular}} (O);
    \path[->] (A) edge node[rotate=90] {\begin{tabular}{c}Założenie\\ \ \end{tabular}} (I);
    \path[->] (E) edge node[rotate=-90] {\begin{tabular}{c}\ \\ \ \end{tabular}} (O);

    \path[-] (contra) edge (A);
    \path[-] (contra) edge (E);
    \path[-] (contra) edge (I);
    \path[-] (contra) edge (O);

\end{tikzpicture}

\caption[,,Kwadrat logiczny'' własności]{,,Kwadrat logiczny'' własności. Zależności zachodzą dla dwóch własności z~jednej domeny (np.~domeny koloru) lub po prostu dla dowolnych własności $P$ oraz $Q$, dla których zachodzi (lub zakłada się) $\forall x (P(x) \to \neg Q(x))$.}\label{sil-hic-kwadrat}
\end{center}
\end{figure}

By zobrazować swoje wątpliwości, Rowe\index[names]{Rowe, William} podaje przykład pary własności sprzecznych: zielony -- nie-zielony i~liczby dwa. Czy liczba dwa może być zielona lub nie-zielona? Jeśli mielibyśmy pozostać w~obrębie koncepcji Hicka\index[names]{Hick, John}, należałoby stwierdzić, że -- ponieważ liczbom nie można przypisać własności koloru -- liczba dwa nie jest ani zielona, ani nie-zielona. Rowe\index[names]{Rowe, William} stanowczo odrzuca taki pogląd. Uważa on, że -- ponieważ nie jest możliwe, by liczba dwa była zielona -- jest ona z~konieczności nie-zielona, a~takie stwierdzenie wcale nie pociąga za sobą tezy, że liczbom można przypisywać własności koloru\footnote{Zob. tamże, ss.~147-149.}.

Podobny zarzut przeciw pomysłowi Hicka\index[names]{Hick, John} wysuwa Alvin Plantinga\index[names]{Plantinga, Alvin}:

\begin{quote}
Jeśli Hick\index[names]{Hick, John} uważa, że żaden z~naszych terminów nie może zostać użyty dosłownie na określenie Rzeczywistego, to nie jest możliwe, by to, co mówi, miało jakikolwiek sens. Zakładam, że termin trzykołowy nie przysługuje Rzeczywistemu; Rzeczywiste nie jest trzykołowe. Ale jeśli Rzeczywiste nie jest trzykołowe, to własność ,,nie jest trzykołowe'' przysługuje mu w~sposób dosłowny; jest nie-trzykołowe. Trudno byłoby nie być ani trzykołowym, ani nie nie-trzykołowym, ani nie uważam, że Hick\index[names]{Hick, John} chciałby sugerować, że jest to możliwe\footnote{A. Plantinga, \textit{Warranted Christian Belief}, dz. cyt., s.~45. Ocena apofatyzmu Hicka\index[names]{Hick, John} dokonana przez Plantingę\index[names]{Plantinga, Alvin} jest tak samo rozbudowana, co surowa -- por. tamże, ss.~43-63. Należy jednak stwierdzić, że uczynił on z~koncepcji Hicka\index[names]{Hick, John} pewien rodzaj słomianej kukły, w~wielu miejscach krytykując go za poglądy, których ten nigdy nie utrzymywał.}.
\end{quote}

Ponadto nie jest pewne, czy sama para drugorzędowych własności ,,substancjalny -- formalny'' jest parą własności przeciwnych, sprzecznych lub zupełnie od siebie niezależnych (z którą mamy do czynienia, gdy na przykład dzielimy kwiatki na czerwone i~pachnące). Pozostawienie czytelnika z~prostym wyliczeniem kilku własności substancjalnych i~formalnych sprawia, że można niesprzecznie założyć, że istnieją własności, które nie są ani substancjalne, ani formalne. Podobnie, nietrudno wyobrazić sobie własność, która jest jednocześnie substancjalna, jak i~formalna -- choćby, nie szukając daleko, ,,jest sprzeczny'' oraz ,,jest taki, że można orzec o~nim dwa prawdziwe sądy, z~których jeden jest negacją drugiego''\footnote{Nawet jeśli komuś ten przykład nie przypadnie do gustu, to można niesprzecznie założyć, że istnieją takie własności.}. Inaczej mówiąc, nie jest jasne, czy podział zaproponowany przez Hicka\index[names]{Hick, John} jest pełny i~wyczerpujący, tj. dzieli zbiór własności na podzbiór właściwy i~jego dopełnienie. Jeśli istnieje jakaś własność substancjalna i~formalna zarazem, utrzymywanie WNT traci sens.

Zupełnie odmienny zarzut wobec pomysłu Hicka\index[names]{Hick, John} przedstawił Insole\index[names]{Insole, Christopher}. Uważa on, że ograniczenie mówienia o~Bogu wyłącznie do wypowiedzi o~charakterze czysto formalnym jest bezzasadne, ponieważ aby przypisać Rzeczywistemu samemu w~sobie jakąkolwiek własność formalną, musimy wpierw posiadać wiedzę na temat jego własności substancjalnych. W~rzeczywistości, przypisywanie jakiemukolwiek przedmiotowi własności formalnych wymaga od podmiotu posiadania dużo większej wiedzy niż przypisywanie mu wyłącznie własności substancjalnych. Skoro własność formalna mówi o~(nie)możliwości przypisania danemu przedmiotowi innych własności, aby móc ją w~sposób uprawniony orzec o~tym przedmiocie, musimy wpierw ustalić:

\begin{enumerate}[label = (\arabic*)]
\item jego ontologiczny typ (obiekt fizyczny, fikcyjny, rzeczywistość boska itp.),
\item jego ontologiczną naturę (prosta, złożona, osobowa, transcendentna, immanentna itp.),
\item nasz dostęp poznawczy do przedmiotów tego typu oraz
\item jakie typy własności można danemu przedmiotowi przypisywać w~sposób uprawniony na podstawie tego, co wiemy o~(1), (2) i~(3).
\end{enumerate}
Tymczasem do orzekania o~danym przedmiocie własności substancjalnych nie potrzeba posiadać tego rodzaju wiedzy. Wystarczy wiedzieć, że np. ,,ta sukienka jest czerwona'' lub że ,,Sherlock Holmes był bystry''\footnote{Zob. C.J. Insole, \textit{Why John Hick cannot}\ldots, dz. cyt., ss.~19-20. Por. P. Sikora, \textit{Logos Niepojęty}, dz. cyt., s.~121.}.

Wszyscy wymienieni wyżej autorzy podnoszą przeciw Hickowi\index[names]{Hick, John} również zarzut niekonsekwencji wskazując, że w~wielu miejscach na określenie Rzeczywistego \textit{an sich} używa on pojęć, które -- wedle jego własnego pomysłu -- należałoby zaliczyć do kategorii własności substancjalnych, a~nie formalnych. W~jednym ze swych dzieł Hick\index[names]{Hick, John} określa Rzeczywiste samo w~sobie jako ,,takie, że gdyby nie było prawdziwe, cała dziedzina doświadczeń religijnych, w~swojej różnorodności, byłaby czystą projekcją wyobraźni''\footnote{J. Hick, \textit{A~Christian Theology of Religions}\ldots, dz. cyt., ss.~59-60.}. Można byłoby uparcie twierdzić, że jest to własność formalna, ale by przysługiwała ona Rzeczywistemu, należy wpierw przyznać mu (substancjalną) własność ,,jest prawdziwe''. Oczywiście zakładając, że doświadczenia religijne nie są czystą projekcją wyobraźni\footnote{Por. W.L. Rowe, \textit{Religious pluralism}, dz. cyt., s.~145.}. W~innych miejscach Hick\index[names]{Hick, John} przypisuje Rzeczywistemu własności ,,bycia źródłem i~podstawą wszystkiego''\footnote{J. Hick, \textit{A~Christian Theology of Religions}\ldots, dz. cyt., s.~27.}, ,,bycia autentycznie doświadczanym zarówno jako fenomen teistyczny, jak i~nieteistyczny''\footnote{Tenże, \textit{An Interpretation of Religion}\ldots, dz. cyt., s.~242, 246-247.}, czy też ,,posiadania bogatej natury''\footnote{Tenże, \textit{A~Christian Theology of Religions}\ldots, dz. cyt., s.~62.} i~wiele innych określeń, które sprzeciwiają się głównej zasadzie słabej teorii Niewysłowionego, którą sam nałożył na mowę o~Bogu\footnote{Por. W.L. Rowe, \textit{Religious pluralism}, dz. cyt., s.~146; C.J. Insole, \textit{Why John Hick cannot}\ldots, dz. cyt., s.~26 oraz A. Plantinga, \textit{Warranted Christian Belief}, dz. cyt., ss.~44-45.}.

Oprócz wspomnianej wyżej niekonsekwencji, koncepcja Hicka\index[names]{Hick, John} oskarżana jest także o~zwykłą sprzeczność. Według Hicka\index[names]{Hick, John} niewysławialność Rzeczywistego samego w~sobie polega na tym, że jego natura ,,znajduje się poza zasięgiem siatki ludzkich pojęć''. Wszystko wskazuje na to, że własności formalne również należą do takiej kategorii. Gdyby trzymać się ściśle tego sformułowania, także one nie powinny zostać przypisane Rzeczywistemu \textit{an sich}.

\begin{quote}
Rzeczywiste w~sobie nie może być tym, za co ma je Hick\index[names]{Hick, John} -- rzeczywistością całkowicie przekraczającą sieć ludzkich pojęć. Stwierdzić powyższe, to zaprzeczyć, że Rzeczywiste przekracza wszystkie pojęcia, bo ,,przekracza'' też jest ludzkim pojęciem\footnote{W.L. Rowe, \textit{Religious pluralism}, dz. cyt., s.~145-146. Cytat w~j. polskim za P. Sikora, \textit{Logos Niepojęty}, dz. cyt., s.~120.}.
\end{quote}

Argument ten można próbować odeprzeć biorąc za dobrą monetę interpretację własności formalnych i~substancjalnych zaproponowaną przez Rowe'a\index[names]{Rowe, William}. Według niej własności substancjalne Rzeczywistego to własności istotowe, należące do jego natury. Ponieważ ta przekracza sieć ludzkich pojęć, nie mogą one zostać przypisane Rzeczywistemu. Należałoby jednak wtedy uznać, że własności formalne -- np. chętnie przez Hicka\index[names]{Hick, John} przypisywana Rzeczywistemu własność ,,taki, że nasze substancjalne pojęcia się do niego nie stosują'' -- nie należą do \textit{natury} Rzeczywistego. Próba ta ma szanse powodzenia. Wymagałaby jednak najpierw wyeksplikowania pojęcia \textit{natury} i~przedstawienia możliwości epistemologicznego dostępu do niej. Nietrudno zauważyć, że taka filozoficzna teoria miałaby kilka niepożądanych własności. Na przykład taką, że dopuszczałaby istnienie obiektów posiadających istotowe, należące do natury własności, które w~zasadzie im nie przysługują, lub których z~pewnych powodów nie można im przypisać. Ponadto, nie jest do końca jasne, dlaczego proste, bezpośrednie własności substancjalne Rzeczywistego \textit{an sich} mają znajdować się ,,poza zasięgiem siatki ludzkich pojęć'', a~bardziej złożone formalne własności mają już w~tej siatce się mieścić. Należy także odnotować, że podejście Rowe'a\index[names]{Rowe, William} jest niezgodne z~tą interpretacją teorii Niewysłowionego, zgodnie z~którą niewysławialność nie wynika z~samych tylko ograniczeń podmiotu poznającego, lecz jest istotną częścią transcendentnej natury Boga.

W~podobnym duchu wypowiada się Sebastian Gäb\index[names]{Gäb, Sebastian}\footnote{Zob. S. Gäb, \textit{Languages of ineffability: the rediscovery of apophaticism in contemporary} \textit{analytic philosophy of religion}, [w:] \textit{Negative Knowledge}, red. S. Hüsch i~in., Narr Francke Attempto Verlag, Tübingen 2020, ss.~191-206.} konstatując, że samą niewysławialność Rzeczywistego należałoby właściwie zaliczyć do jego własności substancjalnych. Jego zdaniem, gdy Hick\index[names]{Hick, John} nazywa Rzeczywiste niewysławialnym, jego celem jest nie tyle określenie zasad używania tego terminu, lecz raczej wyjaśnienie swojej koncepcji Rzeczywistego, a~bycie niewysławialnym jest istotną częścią tego pojęcia. Niewysławialność Rzeczywistego należy zatem do jego natury i~jest dokładnie tym, co odróżnia Rzeczywiste od innych obiektów. Z~tego powodu należy uznać, że niewysławialność jest jego istotową i~substancjalną własnością, lecz wtedy -- zgodnie z~samą słabą teorią Niewysłowionego -- nie może ona zostać przypisana Rzeczywistemu.

Kolejna grupa argumentów przeciwko koncepcji Hicka\index[names]{Hick, John} uderza w~nią jako teorię pluralizmu religijnego. Insole\index[names]{Insole, Christopher} podaje dwa rudymentarne warunki, jakie musi spełniać dobra hipoteza -- sprawiać, że dane zostają wyjaśnione 1) lepiej, niż gdyby to miało miejsce bez tej hipotezy oraz 2) lepiej niż w~świetle innych hipotez. W~takim wypadku słaba teoria Niewysłowionego Hicka\index[names]{Hick, John} byłaby dobrą hipotezą religijnego pluralizmu, gdyby wyjaśniała różnorodność religijnych doświadczeń 1) lepiej, niż gdyby to miało miejsce bez postulowania Rzeczywistego samego w~sobie oraz 2) lepiej niż inne teorie wyjaśniające ten fenomen. Według Insole'a\index[names]{Insole, Christopher}, postulowanie istnienia takiego $X$, o~którym możemy stwierdzić jedynie tyle, że nic o~nim nie może zostać stwierdzone, nie może stanowić wyjaśnienia czegokolwiek. A~zatem teoria Hicka\index[names]{Hick, John} nie spełnia żadnego z~tych warunków\footnote{Por. C.J. Insole, \textit{Why John Hick cannot}\ldots, dz. cyt., ss.~31-32.}.

Plantinga\index[names]{Plantinga, Alvin} idzie o~krok dalej i~oskarża Hicka\index[names]{Hick, John} o~intelektualny imperializm. Skoro wyjaśnieniem religijnego pluralizmu w~słabej teorii Niewysłowionego jest postulowanie Rzeczywistego samego w~sobie, któremu w~sposób dosłowny nie możemy przypisywać żadnej własności substancjalnej, Hick\index[names]{Hick, John} konkluduje, że standardowy język religijny, wszelkie dogmaty, tezy i~doktryny wszystkich religii należy uznać za fałszywe -- o~ile tylko traktujemy je dosłownie a~nie ,,mitologicznie''\footnote{Por. J. Hick, \textit{An Interpretation of Religion}\ldots, dz. cyt., s.~348.}. Nietrudno zauważyć, że motywacją stojącą za takim stwierdzeniem jest chęć zachowania równorzędności wszystkich wyznań. Jednakże, zdaniem Plantingi\index[names]{Plantinga, Alvin}, taka konstatacja przynosi zupełnie odmienny skutek:

\begin{quote}
Teraz deklarujemy, że wszyscy się mylą -- wszyscy, oprócz nas i~kilku oświeconych dusz. [\ldots] Wspaniałomyślnie uważamy, że reszta ludzkości błądzi; nie ulega wątpliwości, że chęci mają dobre, ale niestety mylą się co do tego, co uważają za najważniejsze i~najcenniejsze. Trudno mi uznać taką postawę za manifestację tolerancji i~intelektualnej pokory, wygląda to raczej na wyraz protekcjonalności\footnote{A. Plantinga, \textit{Warranted Christian Belief}, dz. cyt., s.~62.}.
\end{quote}

W~końcu, tak jak praca Jonesa\index[names]{Jones, John J.}, tak i~słaba teoria Niewysłowionego Hicka\index[names]{Hick, John} pozbawiona jest rozważań o~charakterze logicznym. Jak zauważa Insole\index[names]{Insole, Christopher}:

\begin{quote}
Mimo iż Hick\index[names]{Hick, John} nazywa własności formalne ,,generowanymi logicznie'', nie ma w~nich nic specjalnie ,,logicznego''. Stwierdzenie, że Bóg jest taki, że ,,żadne substancjalne pojęcia się do niego nie stosują'' nie ma charakteru analitycznego w~żadnym z~niekontrowersyjnych sensów intensji pojęcia ,,Bóg''. Jeśli istnieją jakieś ,,logiczne'' podstawy, które generowałyby takie formalne własności, Hick\index[names]{Hick, John} powinien je wyraźnie określić. Jednakże perspektywa wskazania takich podstaw nie wygląda obiecująco\footnote{C.J. Insole, \textit{Why John Hick cannot}\ldots, dz. cyt., s.~28.}.
\end{quote}
Okazuje się jednak, że praca Hicka\index[names]{Hick, John} potrafiła zainspirować innych autorów do przeprowadzenia rozważań, w~których teologia apofatyczna ujęta jest już w~bardziej formalne struktury. Przykładem takich rozważań jest Petera Küglera\index[names]{Kügler, Peter} propozycja sformułowania uniwersalnej i~egzystencjalnej zasady teologii negatywnej.




%%%%%%%%%%%%%%%%%%%%%%%%%%%%%%%%%%%%%%%%%%%%%%%%%%%%%%%%%%%%%%%%%%%%%%

%\part{Aspekt semantyczny}




\chapter{Uniwersalna i~egzystencjalna zasada teologii negatywnej}\label{sil-kugler}


\section{Metafora ciemności}

Punktem wyjścia rozważań Petera Küglera\index[names]{Kügler, Peter}\footnote{P. Kügler, \textit{The meaning of mystical ‘darkness}', ,,Religious Studies'', vol. 41 (2005), ss.~95-105.} są pisma Pseudo-Dionizego Areopagity\index[names]{Pseudo-Dionizy Areopagita}. Uważa on, że kluczową zarówno dla rozważań Dionizego\index[names]{Pseudo-Dionizy Areopagita}, jak i~dla całego nurtu religijnego mistycyzmu, jest metafora ciemności\footnote{,,Boska ciemność'' to tytuł pierwszego rozdziału \textit{Teologii mistycznej}.}. W~szczególności sądzi, że w~języku Areopagity\index[names]{Pseudo-Dionizy Areopagita} słowo ,,ciemność'' związane jest z~poznawczymi deficytami, które nazywa on ,,prawdziwie mistyczną ciemnością niepoznawalności''\footnote{Pseudo-Dionizy Areopagita, \textit{Teologia mistyczna}, I, 3.}. Skoro Bóg przekracza wszelkie pojmowanie i~wszelką wiedzę, nasza sytuacja poznawcza może zostać porównana do ciemności nocy, w~której nie możemy uzyskać wiedzy o~otaczającym nas świecie bazując wyłącznie na zmyśle wzroku. Zdaniem Küglera\index[names]{Kügler, Peter} podobieństwo tych przypadków jest na tyle wystarczające, by mówić sensownie o~,,ciemności'' jako metaforze Boga. Pod pewnym względem sytuacje te są jednak niepodobne. W~ciemności mimo wszystko możemy tworzyć pewne przekonania na temat otaczającego nas świata, jakkolwiek nie możemy uzasadnić tych przekonań poprzez odwołanie się do tego, co widzimy. Z~powodu tego ograniczenia, takie przekonania nigdy nie staną się wiedzą. Natomiast w~apofatycznej teologii Dionizego\index[names]{Pseudo-Dionizy Areopagita} nie tylko formułowanie wiedzy o~Bogu jest poza zasięgiem ludzkich możliwości, lecz nawet samo nabywanie przekonań o~nim wydaje się niemożliwe. Bóg jest niewysławialny i~niepojmowalny, jest ponad wszelkim stwierdzeniem i~ponad wszelkim zaprzeczeniem.

Przyjęcie takiej metafory Boga ma jednak pewne niepożądane konsekwencje. We współczesnej filozofii Boga uważa się, że metaforyczny dyskurs o~Bogu presuponuje posiadanie o~nim pewnych dosłownych przekonań\footnote{Na to, że za religijnym językiem metaforycznym powinny stać jakieś dosłowne przekonania, wskazuje wielu myślicieli. Na przykład Walter Stace\index[names]{Stace, Terence S.} twierdzi, że każda sensowna metafora powinna być oparta na \textit{podobieństwie} oraz przynajmniej teoretycznie powinna być ona \textit{przekładalna} na język dosłowny. Te dwa warunki odpowiadają dwóm popularnym teoriom metafor. Według pierwszej z~nich metafory to zakamuflowane porównania, według drugiej posiadają one to samo znaczenie, co odpowiadający im opis dosłowny -- por. W.T. Stace, \textit{Mysticism and Philosophy}, Macmillan \& Co Ltd, London 1961, ss.~284-306. Cytowany w~poprzedniej sekcji Christopher Insole\index[names]{Insole, Christopher} podaje nieco bardziej ogólną teorię (religijnych) metafor. W~jego koncepcji sensowna wypowiedź metaforyczna powinna spełniać trzy warunki: 1) jej odbiorca powinien pojąć, że dosłowna interpretacja nie jest odpowiednia; 2) powinien potrafić rozpoznać wszystkie metaforyczne znaczenia tego wyrażenia oraz 3) powinien móc wybrać właściwe znaczenie metaforyczne użyte w~danym kontekście -- por. C.J. Insole, \textit{Metaphor and the Impossibility of Failing to Speak about God}, ,,International Journal for Philosophy of Religion'', vol. 52 (2002), ss.~35-43. Podobne podejście do językowych metafor prezentuje John Searl\index[names]{Searl, John} -- por. J. Searl, \textit{Metaphor}, [w:] \textit{Metaphor and Thought}, red. A.~Ortony, Cambridge University Press, Cambridge 1993, ss.~83-111.}. Jeśli przyjmiemy taką tezę, nietrudno wskazać paradoksalny charakter metafory ciemności. Mówi ona, że nabywanie i~utrzymywanie jakichkolwiek dosłownych przekonań o~Bogu nie jest możliwe, a~jednocześnie -- jak każda metafora -- zakłada posiadanie takich przekonań. Celem rozważań Küglera\index[names]{Kügler, Peter} jest uniknięcie tego paradoksu. Mówiąc dokładniej, chce on pokazać, że możliwe jest utrzymanie ,,ciemności'' jako metafory Boga przy jednoczesnym zachowaniu tezy, że religijny język metaforyczny presuponuje pewne dosłowne przekonania o~Bogu\footnote{Zob. P. Kügler, \textit{The meaning of mystical ‘darkness}', dz. cyt., s.~96.}.


\section{Dwie zasady teologii negatywnej i~ich paradoksalne konsekwencje}\label{sil-kug-zasady}

Zdaniem Küglera\index[names]{Kügler, Peter} dosłowną podstawą dla metafory ciemności jest centralna idea teologii apofatycznej rozumianej jako teologia milczenia -- mianowicie taka, że Bóg jest niewysławialny, nie można o~nim nic powiedzieć. Z~punktu widzenia niniejszej pracy interesujący jest fakt, że Kügler\index[names]{Kügler, Peter} stara się wyrazić ją we względnie formalny\footnote{On sam deklaruje, że jest to zapis półformalny (\textit{semiformal}) -- por. tamże, s.~98.} sposób. Pierwszą iteracją jego starań jest ogólna (uniwersalna) zasada teologii negatywnej (\ref{sil-kug-unt}):
%(UNT) Dla każdej własności Q, Bóg ani posiada, ani nie posiada Q.
\begin{flalign*}
		& \parbox[t]{.87\linewidth}{ 
		Dla każdej własności $Q$, Bóg ani posiada, ani nie posiada $Q$.} &\tag{UNT}\label{sil-kug-unt}
\end{flalign*}

Niestety, sąd wyrażony przez \ref{sil-kug-unt} dzieli z~teorią Niewysłowionego jej dobrze znane problemy z~zachowaniem spójności. Skoro \ref{sil-kug-unt} jest wyrażaniem wykorzystującym ogólną kwantyfikację przebiegającą po zbiorze własności, możemy ją egzemplifikować używając jakiejkolwiek własności. Rozważmy w~takim razie własność $B$~oznaczającą ,,jest niebieski''. Możemy następnie sformułować (bardziej złożoną) własność ,,ani jest, ani nie jest niebieski'' i~oznaczyć ją przez $B^*$. Teraz, powołując się na \ref{sil-kug-unt}, zmuszeni jesteśmy przyznać, że Bóg posiada własność $B^*$ i~jednocześnie -- zgodnie z~tą samą zasadą -- ani posiada, ani nie posiada $B^*$. Nietrudno zauważyć, że w~rzeczywistości powyższy argument zupełnie nie zależy od własności wyrażonej przez $B$~i przy jego pomocy możemy wyprodukować tyle sprzeczności, ile tylko istnieje własności. Z~tego powodu Kügler\index[names]{Kügler, Peter} proponuje zastąpić \ref{sil-kug-unt} szczegółową (egzystencjalną) zasadą teologii negatywnej (\ref{sil-kug-ent}).
%(\ref{sil-kug-ent}) Nie istnieje własność Q~taka, że Bóg posiada lub nie posiada Q.
\begin{flalign*}
		& \parbox[t]{.87\linewidth}{ 
		Nie istnieje własność $Q$~taka, że Bóg posiada lub nie posiada $Q$.} &\tag{ENT}\label{sil-kug-ent}
\end{flalign*}

Zanim posuniemy się dalej w~naszych rozważaniach, przedstawmy powyższe zasady w~nieco bardziej formalny sposób. Skoro \ref{sil-kug-unt} wykorzystuje kwantyfikację po własnościach, na potrzeby jej formalizacji najrozsądniej będzie zaadaptować logikę drugiego rzędu. W~takim ujęciu \ref{sil-kug-unt} przybierze postać
%(UNT'),
\begin{flalign*}
		& \forall_Q\ \neg (Q(g) \lor \neg Q(g)), &\tag{UNT'}\label{sil-kug-untprim}
\end{flalign*}
gdzie $Q$~jest symbolem predykatowym, a~$g$ stałą indywiduową oznaczającą Boga. W~takim sformułowaniu sprzeczność jest jeszcze bardziej oczywista:
\begin{flalign}
& \forall_Q\ \neg (Q(g) \lor \neg Q(g)) &  \eqref{sil-kug-untprim}\label{untprim1} \\
& \neg (B(g) \lor \neg B(g)) &  (\ref{untprim1},\ \forall \text{ elim.})\label{untprim2}  \\
& \neg B(g) \land B(g) & \qquad (\text{\ref{untprim2}, p. De Morgana\index[names]{De Morgan, Augustus}, }\neg\neg\text{ elim.})\label{untprim3}  \\
& B(g) & (\ref{untprim3},\ \land\text{ elim.})\label{untprim4}  \\
& \neg B(g) & ( -||- )\label{untprim5}  \\
& \qquad \text{contr. \ref{untprim4}, \ref{untprim5}} & \nonumber
\end{flalign}

Zinterpretowanie terminu ,,Bóg'' w~kategoriach deskrypcji określonej prowadziłoby do przeformułowania \ref{sil-kug-unt} w~następujący sposób:
%(UNT'')
%\begin{flalign*}
%		& G(x) \to \forall_Q\ \neg (Q(x) \lor \neg Q(x)). &\tag{UNT''}\label{sil-kug-untbis}
%\end{flalign*}
\begin{flalign*}
		& G(x) \equiv \forall_Q\ \neg (Q(x) \lor \neg Q(x)). &\tag{UNT''}\label{sil-kug-untbis}
\end{flalign*}
Jednakże w~takim wypadku oddanie naturalno-językowego terminu ,,Bóg'' przy pomocy predykatu prowadzi do dodatkowych niepożądanych konsekwencji. Jedną z~nich jest zdanie, że coś jest Bogiem wtedy i tylko wtedy, gdy nim nie jest:
%\begin{flalign}
%& G(x) to \forall_Q\ \neg (Q(x) \lor \neg Q(x)) & \eqref{sil-kug-untbis}\label{untprim6} \\
%& G(x) \to \neg (G(x) \lor \neg G(x)) & (\ref{untprim6},\ \forall \text{ elim., syl. hip.})\label{untprim7} \\
%& G(x) \to \neg G(x) \land G(x) & (\text{\ref{untprim7}, p. De Morgana, }\neg\neg\text{ elim., syl. hip.})\label{untprim8} \\
%& G(x) \to \neg G(x) & (\ref{untprim8},\ \land\text{ elim.}\label{untprim9})
%\end{flalign}
\begin{flalign}
& G(x) \equiv \forall_Q\ \neg (Q(x) \lor \neg Q(x)) & \eqref{sil-kug-untbis}\label{untprim6} \\
& G(x) \equiv \neg (G(x) \lor \neg G(x)) & (\ref{untprim6},\ \forall \text{ elim., syl. hip.})\label{untprim7} \\
& G(x) \equiv \neg G(x) \land G(x) & (\text{\ref{untprim7}, p. De Morgana\index[names]{De Morgan, Augustus}, }\neg\neg\text{ elim., syl. hip.})\label{untprim8} \\
& G(x) \equiv \neg G(x) & (\ref{untprim8},\ \land\text{ elim.})\label{untprim9}\\
& G(x)  & (\equiv \text{ elim., p. Claviusa }\ref{untprim9})\label{untprim10}\\
& \neg G(x) & ( -||- )\label{untprim11}\\
& \qquad \text{contr. \ref{untprim10}, \ref{untprim11}} & \nonumber
\end{flalign}

Z~powodu powyższych problemów w~tejże sekcji będę traktował termin ,,Bóg'' jako nazwę własną. Za takim ujęciem stoją jeszcze dwa argumenty. Po pierwsze, ułatwi to porównanie zasad Küglera\index[names]{Kügler, Peter} z~tymi formalnymi ujęciami teorii Niewysłowionego, w~których użycie predykatu w~celu oddania słowa ,,Bóg'' jest niemożliwe, nieintuicyjne lub prowadzi wprost do sprzeczności. Po drugie, skoro kwestia (nie)istnienia Boga nie jest centralnym zagadnieniem teologii apofatycznej -- przeciwnie, myśliciele apofatyczni raczej uznawali jego istnienie za pewnik -- przyjęcie tej kategorii językowej nie będzie kwestią kontrowersyjną\footnote{Por rozdz.~\ref{sil-kt-jez}.}. Odpowiadające takiemu ujęciu sformułowanie \ref{sil-kug-ent} przybierze następującą postać:
%(ENT')
\begin{flalign*}
		& \neg \exists_Q\ (Q(g) \lor \neg Q(g)). &\tag{ENT'}\label{sil-kug-entprim}
\end{flalign*}

Powodem, dla którego \ref{sil-kug-unt} zostało zastąpione przez \ref{sil-kug-ent} był fakt, że pierwsza z~tych zasad prowadziła do potencjalnie nieskończonej liczby paradoksów -- przy jej pomocy można było wygenerować paradoks samoodniesienia dla każdej własności obecnej w~języku. Nie oznacza to jednak, że \ref{sil-kug-ent} pozwala ostatecznie uniknąć tego problemu. Szczegółowa zasada teologii negatywnej także jest dosłowną wypowiedzią o~Bogu wyrażoną w~języku. Przypisuje ona Bogu pewną (złożoną) własność -- ,,taki, że nie ma żadnej własności, którą On posiada lub nie posiada''. Nazwijmy tę własność $G^*$. Ponownie jesteśmy zmuszeni stwierdzić, że z~\ref{sil-kug-ent} wynika, że Bóg posiada własność $G^*$, co jednocześnie jest sprzeczne z~\ref{sil-kug-ent}, ponieważ zgodnie z~jej treścią nie istnieje żadna własność, którą Bóg posiada (ani taka, której nie posiada). Jednakże, zdaniem Küglera\index[names]{Kügler, Peter}, logiczny problem \ref{sil-kug-ent} nie jest tak wielki jak trudności nękające \ref{sil-kug-unt}. Jest on jednak na tyle poważny, że w~ostatniej części pracy\footnote{Zob. P.~Kügler, \textit{The meaning of mystical ‘darkness}', dz. cyt., s.~99-103.} próbuje sobie z~nim poradzić. Rozważa on trzy strategie: ,,samopodważenie'', ,,ograniczenie'' i~,,samowykluczenie''\footnote{Org. odpowiednio: \textit{self-subversion}, \textit{restriction} i~\textit{self-exclusion}. Ten środkowy termin Piotr Sikora\index[names]{Sikora, Piotr} tłumaczy dosłownie, jako ,,restrykcję'', por. P. Sikora, \textit{Logos Niepojęty}, Wydawnictwo Universitas, Kraków 2010, ss.~121-122.}.


%\section{Strategie uniknięcia sprzeczności \ref{sil-kug-ent}}
\section{Strategie Küglera}
%. Samowykluczenie jako sposób na uniknięcie sprzeczności \ref{sil-kug-ent}}

Strategia samopodważenia polegałaby właściwe na rezygnacji z~metafory ciemności. Warto podkreślić, że jest ona bliska treści samej \textit{Teologii mistycznej} jak i~wielu interpretacji tego tekstu, wedle których doktryna apofatyczna Dionizego\index[names]{Pseudo-Dionizy Areopagita} w~takim samym stopniu kwestionuje zasadność afirmatywnego, jak i~negatywnego dyskursu o~Bogu. Wielu autorów nazywa to podejście ,,negacją negacji''\footnote{D. Turner, \textit{The Darkness of God: Negativity in Christian Mysticism}, Cambridge University Press, Cambridge 1995, s.~45; P. Rorem, \textit{Pseudo-Dionysius. A~Commentary on the Texts and an Introduction to Their Influence}, Oxford University Press, New York -- Oxford 1993, s.~213; D.~Brylla, \textit{Rozważania o~apofatycznej kategorii ,,negacja negacji''}, ,,Seminare'', vol. 38 (2017), nr 1, ss.~65-76.} i~wskazuje je jako najważniejszy powód, dla którego teologia negatywna jest teologią milczenia. Oferuje ona pewną drogę ku Bogu, lecz im dalej na niej się znajdujemy, tym większe trudności napotyka nasz język i~umysł. Na końcu tej drogi także i~ona sama musi zostać porzucona, ponieważ i~ona stanowi niewłaściwy sposób mówienia i~myślenia o~nim -- wobec nieopisywalnego Boga pozostaje wyłącznie milczenie. Zdaniem Küglera\index[names]{Kügler, Peter} jego \ref{sil-kug-ent} może przyczynić się do lepszego zrozumienia takiego stanowiska. Jego szczegółowa zasada teologii negatywnej pełniłaby tutaj rolę eksplanacyjną -- miałaby bowiem tłumaczyć, na czym dokładnie polegają trudności, które język ludzki napotyka przy próbie opisu transcendencji. Jednakże sam Kügler\index[names]{Kügler, Peter} nie jest przekonany co do słuszności takiego podejścia, ponieważ oznacza ono pogodzenie się z~paradoksalnym charakterem teorii Niewysłowionego, a~nie usunięcie go:

\begin{quote}
[\ldots] to, czy dojście do sprzeczności (i wywołane nim milczenie) zostanie uznane za najwspanialsze osiągnięcie czy też niepożądane niepowodzenie teologii jest kwestią filozoficznego temperamentu. Nie każdy będzie zachwycony strategią samopodważenia\footnote{P~Kügler, \textit{The meaning of mystical ‘darkness}', dz. cyt., s.~100.}.
\end{quote}
Najwyraźniej filozoficzny temperament Küglera\index[names]{Kügler, Peter} sprawia, że on sam zachwycony nią nie jest.

Druga z~rozważanych strategii polegałaby na podzieleniu zbioru własności na dwie klasy i~ograniczeniu stosowania \ref{sil-kug-ent} tylko do jednej z~nich. Przykładem takiego rozwiązania jest podział własności przypisywanych Bogu na substancjalne i~formalne, którego dokonał Hick\index[names]{Hick, John}\footnote{Zob. rozdz.~\ref{sil-slabatn}.}. Próbując obronić \ref{sil-kug-ent} przed paradoksem samoodniesienia można by założyć, że dotyczy ona tylko własności substancjalnych. Przy takim założeniu szczegółowa zasada teologii negatywnej Küglera\index[names]{Kügler, Peter} głosiłaby, że nie istnieje \textit{substancjalna} własność $Q$~taka, że Bóg posiada lub nie posiada $Q$. Natomiast ta własność, która wyrażona jest przez samą treść tej zasady (a którą poprzednio oznaczyliśmy jako $G^*$), należy do kategorii własności formalnych. W~ten sposób unikamy sytuacji, w~której \ref{sil-kug-ent} odnosi się do samej siebie. Jednakże Kügler\index[names]{Kügler, Peter} nie jest przekonany także do takiego stanowiska. Uważa on, że podział na własności formalne i~substancjalne zaproponowany przez Hicka\index[names]{Hick, John}, a~także próby doprecyzowania tego podziału przez jego krytyków są albo beznadziejnie niejasne, albo -- w~najlepszym razie -- przebiegają w~nieodpowiednich miejscach.

W~zamian za to proponuje on trzecią strategię, którą nazywa samowykluczeniem. W~myśl tego rozwiązania szczegółowa zasada teologii negatywnej dotyczy wszystkich własności przypisywanych Bogu poza tą, którą sama ta zasada wyraża. Innymi słowy, z~zasięgu obowiązywania \ref{sil-kug-ent} wykluczamy własność $G^*$ (,,taki, że nie ma żadnej własności, którą On posiada lub nie posiada''), czyli tę właśnie, która jest przypisywana Bogu przez \ref{sil-kug-ent}. Zdaniem Küglera\index[names]{Kügler, Peter} taki zabieg najlepiej służy przedstawieniu \ref{sil-kug-ent} jako dosłownej podstawy dla metafory ciemności.


\section{Dyskusja}\label{sil-kug-dyskusja}

Warto zauważyć, że \ref{sil-kug-entprim} (a także naturalno-językowe sformułowanie zawarte w~\ref{sil-kug-ent}) przyjmuje pewną postać negacji prawa wyłączonego środka. Mając to na uwadze, przy próbie ujęcia rozważań Küglera\index[names]{Kügler, Peter} w~formalne struktury powinniśmy zrezygnować ze stosowania logiki klasycznej z~co najmniej dwóch powodów. Po pierwsze, w~klasycznych przypadkach zasada \textit{tertium non datur} musi obowiązywać. Po drugie, zastępowania (\ref{sil-kug-untprim}) przez (\ref{sil-kug-entprim}) na gruncie logiki klasycznej traci jakikolwiek sens, ponieważ wyrażenia te są równoważne (nietrudno zauważyć, że równoważność obu zasad konstytuowana jest przez prawo De Morgana\index[names]{De Morgan, Augustus})\footnote{Warto dodać, że nie ma znaczenia tutaj fakt, że do formalnej rekonstrukcji zasad Küglera\index[names]{Kügler, Peter} zastosowaliśmy logikę drugiego rzędu. Prawa rządzące tą równoważnością zostają zachowane.}. Kügler\index[names]{Kügler, Peter} jest świadomy tego faktu i~sam proponuje, by do przedstawienia apofatycznego dyskursu zastosować jedną z~logik nieklasycznych\footnote{Zob. P~Kügler, \textit{The meaning of mystical ‘darkness}', dz. cyt., s.~98-99.}. Niestety, nie podaje on wprost żadnego konkretnego rachunku, który mógłby służyć jako odpowiednia rama dla jego rozważań. Spróbujmy jednak przyjrzeć się bliżej tej kwestii.

Na pierwszy rzut oka wydawałoby się, że najlepszą kandydatką do oddania formalnej struktury poglądów Küglera\index[names]{Kügler, Peter} jest logika intuicjonistyczna. Powstała ona jako rezultat formalizacji pewnych poglądów dotyczących podstaw matematyki (zwanych intuicjonizmem). Poglądy te były rozwijane w~pierwszych dekadach ubiegłego stulecia i~podobnie jak większość nurtów filozoficznych badań nad podstawami matematyki tamtego okresu, powstały w~reakcji na pewne antynomie dostrzeżone w~teorii mnogości\footnote{Historię i~okoliczności powstania logiki intuicjonistycznej opisuję w~P. Urbańczyk, \textit{Geneza intuicjonistycznego rachunku zdań i~Twierdzenie Gliwienki}, ,,Zagadnienia Filozoficzne w~Nauce'', nr 56 (2014), ss.~33-56.}. Od samego początku oczywistym było, że logika intuicjonistyczna, jako próba sprecyzowania intuicjonistycznych sposobów wnioskowań pochodzących z~odmiennej, zrekonstruowanej matematyki, nie może akceptować wszystkich praw stosowanych na gruncie rachunku klasycznego. Do jej najbardziej znanych własności należy fakt, że odrzuca ona prawo wyłączonego środka. Dziś można śmiało powiedzieć, że logika intuicjonistyczna jest najlepiej ugruntowaną i~najszerzej zbadaną logiką nieklasyczną posiadającą tę własność -- co dla celów formalnej rekonstrukcji \ref{sil-kug-ent} byłoby pożądaną cechą.

Powiedzieliśmy jednak, że rachunek logiczny, który by miał leżeć u~podstaw rekonstrukcji \ref{sil-kug-ent} powinien także blokować równoważność \ref{sil-kug-untprim} i~\ref{sil-kug-entprim}. Problem w~tym, że logika intuicjonistyczna nie spełnia tego kryterium. W~rzeczywistości odrzuca ona ,,jedną czwartą'' praw De Morgana\index[names]{De Morgan, Augustus}, to znaczy jedną implikację jednego z~tych praw, mianowicie:
%Not in Int (chyba nie to, co w~artykule).
\begin{flalign*}
&\nvdash_{\text{INT}} \neg \forall x A \to \exists x \neg A.&
\end{flalign*}
Implikacja w~drugą stronę oraz drugie z~praw De Morgana\index[names]{De Morgan, Augustus} stanowiące o~tym, że \ref{sil-kug-untprim} pociąga za sobą \ref{sil-kug-entprim} i~\textit{vice versa}, w~logice intuicjonistycznej zachodzi:
%In int x3.
\begin{flalign*}
&\vdash_{\text{INT}} \exists x \neg A \to \neg \forall x A.&\\
&\vdash_{\text{INT}}  \forall  x \neg A \to \neg \exists x  A.&\\
&\vdash_{\text{INT}}  \neg \exists x  A \to  \forall x \neg A.&
\end{flalign*}

Z~tego powodu także logika intuicjonistyczna nie może stać się właściwym narzędziem do oddania struktury wnioskowań Küglera\index[names]{Kügler, Peter}. Można jeszcze próbować osłabiać intuicjonistyczną negację na tyle, by w~ten sposób utworzony rachunek zachował własność $\nvdash A \lor \neg A$~i, dodatkowo, posiadał własność $\nvdash \neg (A \land B) \to \neg A \lor \neg B$. Okazuje się jednak, że te same prawa De Morgana\index[names]{De Morgan, Augustus} zostają zachowane w~słabszej od logiki intuicjonistycznej logice minimalnej Johanssona\index[names]{Johansson, Ingebrigt}, a~nawet w~logice subminimalnej z~najsłabszą intuicjonistyczną negacją\footnote{Por. J.M. Dunn, \textit{Generalized Ortho Negation}, [w:] \textit{Negation: A~Notion in Focus}, red. H.~Wansing, Walter de Gruyter, Berlin -- New York 1996, ss.~3-26; S.P. Odintsov, \textit{Constructive Negations and Paraconsistency}, Trends in Logic, vol. 26, Springer-Verlag, New York 2008; S.P. Odintsov, \textit{On the structure of paraconsistent extensions of Johansson's logic}, ,,Journal of Applied Logic'', vol. 3 (2005), ss. 43-65; Y.~Shramko, \textit{Dual Intuitionistic Logic and a~Variety of Negations: The Logic of Scientific Research}, ,,Studia Logica: An International Journal for Symbolic Logic'', vol. 80 (2005), ss.~347-367; A. Colacito i~in., \textit{Subminimal negation}, ,,Soft Computing'', vol. 21 (2016), ss.~165-174; N. Bezhanishvili i in., \textit{A~Study of Subminimal Logics of Negation and Their Modal Companions}, [w:] \textit{Language, Logic, and Computation}, red. A. Silva i~in., Springer, Berlin -- Heidelberg 2019, ss.~21-41.}. Filozof badający teorie zawierające sprzeczności powinien także ucieszyć się z~faktu, że obie te logiki posiadają parakonsystentne własności. Być może istnieje jakaś odmiennie budowana logika parakonsystentna, która blokuje zarówno prawo wyłączonego środka, jak i~to prawo De Morgana\index[names]{De Morgan, Augustus}, które prowadzi do niechcianych przez Küglera\index[names]{Kügler, Peter} równoważności \ref{sil-kug-unt} i~\ref{sil-kug-ent}. Niestety, Kügler\index[names]{Kügler, Peter} -- poza uwagą o~nieklasyczności rachunku, w~ramach którego \ref{sil-kug-ent} stanowiłoby formalną teorię teologii milczenia -- nie dokłada starań, by taki rachunek wskazać lub skonstruować. Mimo tego utrzymuje, że taka właśnie nieklasyczna (ale nieokreślona) logika jest \textit{implicite} wykorzystywana przez Pseudo-Dionizego Areopagitę\index[names]{Pseudo-Dionizy Areopagita} w~ostatnim rozdziale \textit{Teologii mistycznej}\footnote{Zob. P~Kügler, \textit{The meaning of mystical ‘darkness}', dz. cyt., s.~98.}.

Skoro w~dyskusji pojawił się już wątek logik parakosystentnych, warto wspomnieć, że motywacją dla wprowadzenia \ref{sil-kug-ent} był dla Küglera\index[names]{Kügler, Peter} fakt, że \ref{sil-kug-unt} generowało potencjalnie nieskończenie wiele zdań sprzecznych -- dokładnie tyle, ile własności (lub predykatów) znajduje się w~języku. Po zastąpieniu \ref{sil-kug-unt} przez \ref{sil-kug-ent} sprzeczność pojawiała się tylko w~przypadku użycia tej własności, którą \ref{sil-kug-ent} wyraża. Nie jest jednak do końca jasne, dlaczego nieskończenie wiele sprzeczności miałoby być gorsze niż jedna. Jak wspomnieliśmy w~rozdziale \ref{sil-int-par}, głównym powodem, dla którego w~teoriach unika się sprzeczności, jest fakt, że pociągają one za sobą wszystko -- w~systemach, w~których pojawia się para zdań sprzecznych, można dowieść dowolne zdanie\footnote{Zob. rozdz.~\ref{sil-int-par} a~także G. Priest, \textit{What so bad about contradictions}?, ,,The Journal of Philosophy'', vol. 95 (1998), nr 8, ss.~410-426.}. Mówiąc bardziej ścisłym językiem, systemy takie ulegają przepełnieniu. O~ile nie poruszamy się w~sferze logik parakonsystentnych, do przepełnienia systemu wystarczy pojawienie się jednej pary zdań sprzecznych. Nie potrzeba do tego ich nieskończonej liczby.

Warto także zwrócić uwagę na sposób, w~jaki Kügler\index[names]{Kügler, Peter} próbuje poradzić sobie ze sprzecznością. Strategia ,,samowykluczenia'', którą proponuje, nakazuje wyłączyć z~zasięgu oddziaływania \ref{sil-kug-ent} własność, którą sama ta zasada wyraża. Kügler\index[names]{Kügler, Peter} usiłuje uzasadnić zastosowanie takiej strategii uzusem takich naturalno-językowych wyrażeń, jak np. ,,nie ufaj nikomu'' (w domyśle, używający takiego zwrotu miałby chcieć przekazać: ,,nie ufaj nikomu prócz mnie mówiącemu Ci teraz, byś nie ufał nikomu'') lub pojawianiem się podobnych strategii w~pewnych interpretacjach klasycznego paradoksu kłamcy. Nietrudno jednak uznać samowykluczenie za zwykłe rozwiązanie \textit{ad hoc}, które -- zamiast rozwiązywać -- ucieka od paradoksu samoodniesienia generowanego przez szczegółową zasadę teologii negatywnej i~jako takie nie może zostać uznane za satysfakcjonującą strategię radzenia sobie ze sprzecznościami teologii milczenia.


\chapter{Semantyczna zasada teologii negatywnej}\label{sil-gell}

Obiekcje podobne do tych podniesionych w~poprzednim rozdziale mogą równie dobrze dotyczyć interpretacji teologii negatywnej, którą przedstawił Jerome I.~Gellman\index[names]{Gellman, Jerome I.}\footnote{Zob. J.I. Gellman, \textit{The Meta-Philosophy of Religious Language}, ,,Nous'', vol. 11 (1971), ss.~151-161.}. W~jego rozważaniach teologia apofatyczna jest teorią, w~ramach której jakikolwiek predykat $P$~języka ,,skończonych bytów'' nie może być prawdziwie orzekany o~Bogu. Główną tezą tej teorii jest, że Bóg nie należy do przedmiotowego zakresu odniesienia żadnego z~predykatów naszego języka. Jeśli teolog apofatyczny neguje posiadanie przez Boga jakiejś własności, ma on na myśli raczej negację \textit{wykluczającą} niż negację \textit{wyboru}. Jego zamiarem jest wyłącznie stwierdzenie, że to nieprawda, że predykat $P$~przysługuje Bogu, niekoniecznie sugerując, że można o~nim orzec dopełnienie tego predykatu, czyli nie-$P$. Rozważania te doprowadziły Gellmana\index[names]{Gellman, Jerome I.} do sformułowania pewnego wariantu zasady teologii negatywnej, który (przynajmniej na poziomie syntaktycznym) tożsamy jest z~\ref{sil-kug-unt} przedstawioną w~poprzednim rozdziale. Mimo podobnych zarzutów wysuwanych w~stronę obu autorów, punkt wyjścia teorii Gellmana\index[names]{Gellman, Jerome I.} jest zgoła odmienny od tego, który przyświecał Küglerowi\index[names]{Kügler, Peter}.


\section{Wewnętrzne i~zewnętrzne problemy języka religijnego}

Gellman\index[names]{Gellman, Jerome I.} ocenia adekwatność różnych analiz znaczenia języka religijnego. Wedle tej oceny, takie analizy powinny być klasyfikowane i~porządkowane nie ze względu na to, jak wiele z~języka religijnego mogą objąć, lecz ze względu na to, których jego elementów dotyczą. Gellman\index[names]{Gellman, Jerome I.} zauważa, że zdania języka religijnego nie mają równego statusu. Jedne z~nich są bardziej istotne od drugich. Np. w~przypadku tradycji chrześcijańskiej, najważniejszym fragmentem języka religijnego są teksty biblijne i~pisma Ojców Kościoła, a~-- dajmy na to -- teksty publikowane we współczesnych poradnikach kaznodziejskich czy nawet czasopismach teologicznych stanowią jego drugorzędną część. Podobnych podziałów można dokonać w~języku religijnym większości zachodnich religii. Na tej podstawie Gellman\index[names]{Gellman, Jerome I.} wyróżnia dwa rodzaje problemów religijnych: \textit{zewnętrzne} i~\textit{wewnętrzne}.

Problemem religijnym jest dowolny problem sformułowany w~języku religijnym, który na pierwszy rzut oka przedstawia wierzeniom religijnym jakiekolwiek trudności, np. kontestuje je -- najczęściej zawierając parę zdań przeciwnych lub sprzecznych, ale jednakowo ugruntowanych w~języku religijnym\footnote{Gellman\index[names]{Gellman, Jerome I.} używa określenia ,,niekompatybilnych''. Czy jest to ,,niekompatybilność'' o~charakterze logicznym (rozumiana na przykład jako sprzeczność lub przeciwieństwo), czy nie, pozostawia kwestią otwartą, zob. tamże, s.~153.}. \textit{Wewnętrzny} problem religijny musi być rozpoznany w~tej części języka religijnego, która stanowi jego podstawę i~najistotniejszy fragment. Wszystkie inne problemy religijne są \textit{zewnętrzne}. Przykładem \textit{zewnętrznego} problemu religijnego jest niezgodność twierdzenia o~wszechwiedzy Boga (w szczególności wiedzy dotyczącej przyszłości) z~tezą o~wolnej woli. Zagadnienie to nie pojawia się w~podstawowych tekstach chrześcijańskich czy judaistycznych -- zostało wypracowane dopiero przez filozofów scholastycznych, którzy te teksty komentowali. Przykładem religijnego problemu \textit{wewnętrznego} jest twierdzenie o~wszechmocy, wszechwiedzy i~dobroci Boga w~obliczu twierdzenia, że na świecie istnieje zło i~niesprawiedliwość.

\begin{figure}[H]
\begin{center}

 \begin{tikzpicture}[node distance=2cm]
 
% \clip [rounded corners=.5cm] (0,0) rectangle (4,8);

    \node (A) {\begin{tabular}{c} $D$ \\ \end{tabular}};
    \node (E) [right=of A, xshift=5cm] {\begin{tabular}{c} $C$ \end{tabular}};
    \node (I) [below=of A] {\begin{tabular}{c}  $A$ \end{tabular}};
    \node (O) at (I-|E) {\begin{tabular}{c}  $B$ \end{tabular}};
    \node (Z) [below=of I, yshift=2cm, xshift=6cm] {\begin{tabular}{c} \textbf{Twardy rdzeń języka religijnego} \end{tabular}};

%    \coordinate (CENTER) at ($(A)!0.5!(O)$);

%    \node (contra) at (CENTER) {\begin{tabular}{c}Własności sprzeczne \end{tabular}};
    \path[-] (A) edge node[] {\begin{tabular}{c}Problem zewnętrzny\\ \ \end{tabular}} (E);
    \path[-] (I) edge node[] {\begin{tabular}{c}Problem wewnętrzny\\ \  \end{tabular}} (O);
%    \path[->] (A) edge node[rotate=90] {\begin{tabular}{c}Założenie\\ \ \end{tabular}} (I);
%    \path[->] (E) edge node[rotate=-90] {\begin{tabular}{c}Założenie\\ \ \end{tabular}} (O);

%    \path[-] (O) edge (A);
    \path[-] (I) edge node[rotate=20] {\begin{tabular}{c}\qquad\quad Problem zewnętrzny\\ \ \end{tabular}} (E);
%    \path[-] () edge (I);
%    \path[-] () edge (O);
    
%    \draw [red,line width=1pt,rounded corners=.3cm]
%            ([shift={(0.5\pgflinewidth,0.5\pgflinewidth)}]0,0) rectangle
%            ([shift={(-0.5\pgflinewidth,-0.5\pgflinewidth)}]4,8);

	\draw[thick,rounded corners=.3cm]     ($(I)+(-1,1)$) rectangle ($(O)+(1.5,-1.3)$);

\end{tikzpicture}

\caption[Wewnętrzne i~zewnętrzne problemy religijne według Gellmana]{Wewnętrzne i~zewnętrzne problemy religijne. W~koncepcji Gellmana\index[names]{Gellman, Jerome I.} wewnętrzny problem religijny może wystąpić tylko w przypadku ,,konfliktu'' pomiędzy dwoma tezami należącymi do twardego rdzenia języka religijnego. Na powyższym schemacie \textit{wewnętrzny} problem religijny wystąpiłby tylko w przypadku ,,niezgodności'' między zdaniami $A$ oraz $B$. W~przypadku potencjalnych sprzeczności między zdaniami $A$ i~$C$, $D$ i~$C$ (a~także, ewentualnie, między $A$ i~$D$, $C$ i~$B$ oraz $B$ i~$D$) mówimy o~\textit{zewnętrznych} problemach religijnych.}\label{sil-gell-prob}
\end{center}
\end{figure}
%Gellmaan-pic z~opisem

Według Gellmana\index[names]{Gellman, Jerome I.} każda adekwatna analiza języka religijnego nie powinna usuwać wewnętrznych problemów danej religii. Innymi słowy, jeśli przy analizie znaczenia języka religijnego jakiś rozpoznany \textit{wewnętrzny} problem religijny nie wystąpi lub nie da o~sobie znać, taka analiza będzie musiała zostać uznana za nieadekwatną (przynajmniej w~tym zakresie).


\section{Bóg jako obiekt poza zakresem przedmiotowym predykatów ludzkiego języka}

Gellman\index[names]{Gellman, Jerome I.} rozważa krótko kazus teologii negatywnej jako jeden z~przykładów teorii języka religijnego (obok formalnej analizy języka religijnego Józefa Marii Bocheńskiego\index[names]{Bocheński, Józef Maria} i~Tomaszowej\index[names]{Tomasz z~Akwinu@Tomasz z~Akwinu, \textit{św}.} teorii analogii). Kojarzy ją jednak wyłącznie z~teologami żydowskimi i~arabskimi\footnote{Wśród podanych przez niego przykładów teologów negatywnych pojawia się Mojżesz Majmonides\index[names]{Majmonides, Mojżesz}, Abraham ibn Daud\index[names]{Daud@ibn Daud, Abraham} oraz Bahya ibn Paquda\index[names]{Paquda@ibn Paquda, Bahya}. Powiązanie teologii negatywnej wyłącznie z~myślicielami żydowskimi i~arabskimi wydaje się jednak niewłaściwe.}. Podobnie jak Kügler\index[names]{Kügler, Peter} uważa on, że przypisywanie Bogu jakiegokolwiek atrybutu na gruncie teologii negatywnej wiąże się z~popełnieniem błędu (przesunięcia) kategorialnego. Innymi słowy, w~rozważaniach Gellmana\index[names]{Gellman, Jerome I.} teologia negatywna jest teorią, w~myśl której jakikolwiek predykat $P$~języka ,,skończonych bytów'' nie może być prawdziwie orzekany o~Bogu. Nie oznacza to jednak, że możemy o~nim orzec negacje wszystkich predykatów. By wykluczyć taką możliwość, Gellman\index[names]{Gellman, Jerome I.} wprowadza definicję zakresu przedmiotowego predykatu\footnote{Tę własność można nazwać także ,,zakresem rodzajowym'' lub ,,zakresem odniesienia'' predykatu (org. \textit{sortal range}).}.

%Definicja SR. Zakres przedmiotowy predykatu P~to dziedzina obiektów, o~których można znacząco orzec P~lub jego dopełnienie.
\begin{defin}[Zakres przedmiotowy predykatu]\label{sil-gell-srdef}
Zakres przedmiotowy predykatu $P$~to dziedzina obiektów, o~których można znacząco orzec $P$~lub jego dopełnienie.
\end{defin}
\noindent
W~świetle tej definicji, główna teza teologii negatywnej w~interpretacji Gellmana\index[names]{Gellman, Jerome I.} głosi, że
%(SNT) Bóg nie należy do zakresu przedmiotowego żadnego z~predykatów naszego języka.
\begin{flalign*}
		& \parbox[t]{.87\linewidth}{ 
		Bóg nie należy do zakresu przedmiotowego żadnego z~predykatów naszego języka.} &\tag{SNT}\label{sil-gell-snt}
\end{flalign*}

Motywacje stojące za wprowadzeniem takiej zasady są oczywiste -- pozwala ona na zaprzeczanie, że Bóg posiada jakąś własność $P$~bez jednoczesnego uznawania, że posiada on własność nie-$P$ i~\textit{vice versa}. Według Gellmana\index[names]{Gellman, Jerome I.}, gdy mówimy na przykład, że Bóg nie jest mądry, mamy na myśli raczej negację \textit{wykluczającą}, niż negację \textit{wyboru}\footnote{Potrzebę wprowadzenia podziału na negację \textit{wykluczającą} i~negację \textit{wyboru} na gruncie języka naturalnego zauważył Gerrit Mannoury w~G. Mannoury, \textit{Les fondements psycho-linguistiques des mathématiques}, Éditions du Griffon, Neuchâtel 1947. Pewne próby rozwinięcia tego tematu pojawiały się kilkukrotnie w~logice i~jej filozofii, np. pewną ciekawą propozycję podaje Fred Sommers, zob. F. Sommers, \textit{Predicability}, [w:] \textit{Philosophy in America}, red. M. Black, Routledge, London 2002, ss.~262-281. Mimo niezgodności dat, istnieje duże prawdopodobieństwo, że Gellman\index[names]{Gellman, Jerome I.} opierał swoją koncepcję na tekście: R.H. Thomason, \textit{A~semantic theory of sortal incorrectness}, ,,Journal of Philosophical Logic'', vol. 1 (1972), ss.~209-258, który stanowi próbę formalnego podejścia zarówno do kwestii negacji \textit{wykluczającej} i~negacji \textit{wyboru}, jak również zagadnienia zakresu przedmiotowego oddziaływania predykatów. Gellman\index[names]{Gellman, Jerome I.} jednak bezpośrednio nie odwołuje się do tej pracy. Nie podaje też żadnych definicji (formalnych bądź nieformalnych) tak określonych negacji.}. Naszym zamiarem jest stwierdzenie tylko, że to nieprawda, że predykat $P$~przysługuje Bogu, niekoniecznie sugerując jednocześnie, że można o~nim orzec dopełnienie tego predykatu\footnote{J.I. Gellman, \textit{The Meta-Philosophy of Religious Language}, dz. cyt., s.~158.}. Bóg jest poza jego gatunkowym zakresem, to znaczy, że nie należy do zbioru obiektów, o~których można orzec $P$ lub nie-$P$.

I~w~drugą stronę -- zdanie ,,Bóg jest potężny'' myśliciel apofatyczny zrozumie jako negację dopełnienia predykatu ,,jest potężny''. W~innym kontekście oznaczałoby to przypisanie obiektowi, o~którym mowa, tej właśnie własności. Jednakże w~przypadku Boga negowanie dopełnienia predykatu $P$~nie oznacza przypisywania mu $P$, ponieważ dopełnienie dane jest innym rodzajem negacji -- negacją \textit{wykluczającą}. Mówiąc ogólnie, zdania języka religijnego negują dopełnienia wszystkich wymienianych przez nie własności Boga, który jest poza zakresem wszystkich naszych predykatów. One, z~kolei -- będąc predykatami ,,języka skończonych bytów'' -- z~konieczności muszą oznaczać niedoskonałe własności\footnote{Zob. tamże.}.

W~końcu, Gellman\index[names]{Gellman, Jerome I.} odrzuca teologię negatywną -- nie jako teorię niespójną i~wewnętrznie sprzeczną, lecz jako teorię nieadekwatną. Po pierwsze, ze względu na wynikającą z~niej niepoznawalność Boga. Po drugie, dlatego, że w~ramach tej teorii nie można postawić wewnętrznych problemów teologicznych. Na przykład mówienie o~wszechmocy, wszechwiedzy i~dobroci Boga -- w~Gellmana\index[names]{Gellman, Jerome I.} interpretacji teologii apofatycznej -- oznacza jedynie, iż nie przypisujemy mu takich własności, jak słabość, głupota czy moralna niedoskonałość. Nie twierdzimy przy tym, że można o~nim orzekać takie predykaty, jak moc, wiedza czy dobroć. W~takim wypadku nie ma mowy o~\textit{wewnętrznym} problemie religijnym wynikającym z~uznania boskiej wszechmocy w~obliczu obecnej w~świecie niesprawiedliwości. Skreśla to, w~oczach Gellmana\index[names]{Gellman, Jerome I.}, teologię apofatyczną z~listy adekwatnych teorii służących do analizy znaczenia języka religijnego.


\section{Dyskusja}

Trudno nie zwrócić uwagi na to, że w~swoim naturalno-językowym sformułowaniu \ref{sil-gell-snt} przyjmuje pewną postać zasady o~treści zbliżonej do tej wyrażonej przez \ref{sil-kug-unt}. Sam Gellman\index[names]{Gellman, Jerome I.} zauważa, że jeśli traktujemy istnienie jako kwantyfikator a~nie jako predykat (co bynajmniej nie jest kontrowersyjnym podejściem), stwierdzenie istnienia Boga polegałoby w~gruncie rzeczy na stwierdzeniu istnienia obiektu poza rodzajowym zakresem wszystkich naszych predykatów\footnote{Tamże.}. Sens tak przedstawionej zasady można próbować zapisać syntaktycznie w~logice drugiego rzędu w~następujący sposób:
%(SNT')${\exists}$\textit{x}${\forall}$\textit{Q}${\neq}$(Q(\textit{x})${\vee}$${\neq}$Q(\textit{x})),
\begin{flalign*}
		&  \exists x \forall Q\ \neg(Q(x) \lor \neg Q(x)). &\tag{SNT'}\label{sil-gell-sntprim}
\end{flalign*}
gdzie $Q$~jest symbolem predykatowym. Nietrudno zaobserwować, że na gruncie klasycznej logiki drugiego rzędu wyrażenie to jest konsekwencją \ref{sil-kug-untprim}%\footnote{Przy założeniu, że Bóg istnieje i jest jeden można mówić nawet o równoważności.}
\begin{flalign}
&\text{\ref{sil-kug-untprim}} \to \text{\ref{sil-gell-sntprim}}.&
\end{flalign}

Ta prosta obserwacja pozwala stwierdzić, że większość zarzutów przedstawionych w~poprzednim rozdziale\footnote{Zob. rozdz.~\ref{sil-kug-dyskusja}.} jest zasadna i~trafna także wtedy, gdy podniesie się je przeciw rozważaniom Gellmana\index[names]{Gellman, Jerome I.}. W~szczególności, jego ujęcie teologii negatywnej nie może być wolne od zarzutu sprzeczności.

Jak wspominałem powyżej, naturalno-językowy termin ,,Bóg'' nie ma jednoznacznie określonej kategorii językowej i~można go przedstawiać zarówno za pomocą stałej indywiduowej, jak i~predykatu $G(x)$ rozumianego jako ,,$x$ jest Bogiem''\footnote{Por. rozdz.~\ref{sil-kt-jez}.}. Podobnie jak w~przypadku poprzednich rozważań, by uniknąć dodatkowych niepożądanych konsekwencji przy formalizacji \ref{sil-gell-snt}, nie możemy reprezentować terminu ,,Bóg'' przy użyciu predykatu, ponieważ -- jak głosi \ref{sil-gell-snt} -- Bóg nie należy do przedmiotowego zakresu żadnego z~predykatów naszego języka, w~szczególności do zakresu odniesienia predykatu $G$. Taka próba doprowadziłaby do natychmiastowej sprzeczności\footnote{Skoro \ref{sil-gell-sntprim} $\equiv$ \ref{sil-kug-untprim}, dowód tego faktu został przedstawiony w~rozdziale \ref{sil-kug-zasady}.}.

Jednakże, by otrzymać parę zdań sprzecznych, wcale nie trzeba decydować się na taki sposób reprezentowania terminu ,,Bóg''. Pozostając wciąż na stosunkowo nieformalnym poziomie możemy z~łatwością uzyskać dobrze nam już znany paradoks samoodniesienia. Ostatecznie \ref{sil-gell-snt}, podobnie jak \ref{sil-kug-unt}, jest pewnym dosłownym wyrażeniem o~Bogu sformułowanym w~języku naturalnym. Głosi ono, że Bóg nie może znaleźć się w~zakresie przedmiotowym żadnej własności. Niech $B$~będzie taką własnością. Rozważmy teraz własność bycia poza zakresem przedmiotowym własności $B$~i oznaczmy ją przez $B^{**}$. Przyjmując \ref{sil-gell-snt} zmuszeni jesteśmy przyznać, że Bóg posiada własność $B^{**}$, podczas gdy -- wedle tej samej zasady -- nie może on znajdować się w~jej zakresie.

Powyższe argumenty, jakkolwiek przemyślne, sformułowane zostały w~dość nieformalnym języku, a~próba bardziej precyzyjnego ujęcia rozważań Gellmana\index[names]{Gellman, Jerome I.} w~postaci \ref{sil-gell-sntprim} może nie wydać się dla wszystkich satysfakcjonująca. \ref{sil-gell-sntprim} zdaje się nie wyrażać dostatecznie dobrze sensu \ref{sil-gell-snt}, ponieważ definicja \ref{sil-gell-srdef} przedmiotowego zakresu predykatu, na którym ta zasada została oparta, posiada wyraźny komponent semantyczny. Można zatem pokusić się o~rekonstrukcję \ref{sil-gell-snt} przy użyciu pojęć semantycznych. Jednakże i~w~takim ujęciu teologia apofatyczna w~rozumieniu Gellmana\index[names]{Gellman, Jerome I.} nie zdoła uchronić się przed sprzecznością.

Niech $M = \langle U, \Delta\rangle$ będzie interpretacją języka taką, że $U$~jest zbiorem niepustym (naszym universum), a~$\Delta$ jest funkcją określoną na zbiorze poszczególnych stałych indywiduowych. Na potrzeby naszej analizy wspomnę tylko, że w~ramach interpretacji utożsamiamy własności ze zbiorami obiektów w~naszym uniwersum, które spełniają odpowiednią relację jednoargumentową, a~dla dowolnej stałej indywiduowej $a$, $\Delta (a) \in U$. Niech $\Delta(B)$ będzie zbiorem obiektów posiadających własność $B$. Oczywiście, $\Delta(B)$ jest podzbiorem $U$ ($\Delta(B)
\subset U$, w~ekstremalnych przypadkach $\Delta(B)$ może być zbiorem pustym). Zauważmy teraz, że zbiór obiektów posiadających własność $\neg B$~identyfikuje się z~dopełnieniem zbioru $\Delta(B)$
%($\overline{\Delta(B)} = U - \Delta(B)$).
($-\Delta(B) = U \setminus \Delta(B)$).
Niech $SR_B$ oznacza zakres przedmiotowy predykatu $B$. Zgodnie z~definicją \ref{sil-gell-srdef}:
\begin{flalign}
%&SR_B = \Delta(B) \cup \overline{\Delta(B)}.&
&SR_B = \Delta(B) \cup -\Delta(B).&
\end{flalign}

Posługując się tak zapisaną definicją można sformułować semantyczną zasadę teologii negatywnej (w rozumieniu Gellmana\index[names]{Gellman, Jerome I.}):
%(SNT'') xxx,
\begin{flalign*}
%		& \Delta(g) \notin SR_B = \Delta(B) \cup \overline{\Delta(B)} = U &\tag{SNT''}\label{sil-gell-sntbis}
		& \Delta(g) \notin SR_B = \Delta(B) \cup -\Delta(B) = U, &\tag{SNT''}\label{sil-gell-sntbis}
\end{flalign*}
gdzie przez stałą indywiduową $g$~oznaczyliśmy wyróżniony obiekt z~naszego uniwersum -- Boga. \ref{sil-gell-sntbis} stoi jednak w~sprzeczności z~samą definicją interpretacji semantycznej, zgodnie z~którą $\Delta(g) \in U$. Jeśli więc jesteśmy skłonni podtrzymywać takie rozumienie teologii negatywnej, musimy ponownie wyjść poza ramy logiki klasycznej. Niemniej jednak, w~tym przypadku nie mamy do dyspozycji wielu alternatyw, ponieważ metalogika jakiejkolwiek logiki, także logik nieklasycznych, jest zasadniczo logiką klasyczną\footnote{Por. np. B. Czernecka-Rej, \textit{On Four Types of Argumentation For Classical Logic}, ,,Roczniki Filozoficzne'', vol. 68, nr 4 (2020), ss.~284-285.}.


\chapter{Elementy logiki modalnej w~formalnej rekonstrukcji tezy o~niewysławialności}\label{sil-jac}

Zadanie obrony teologii milczenia przed zarzutami o~bycie teorią logicznie sprzeczną postawił przed sobą Jonathan D. Jacobs\index[names]{Jacobs, Jonathan D.} w~wyróżnionej nagrodą Sandersa\index[names]{Sanders, Marc} pracy The \textit{Ineffable, Inconceivable, and Incomprehensible God. Fundamentality and Apophatic Theology}\footnote{J.D. Jacobs, \textit{The Ineffable, Inconceivable, and Incomprehensible God. Fundamentality and Apophatic Theology}, [w:] \textit{Oxford Studies in Philosophy of Religion}, vol.~6, red. J.~Kvanvig, Oxford University Press, Oxford 2015, ss.~158- 176. W~ostatnich latach ta właśnie praca zwróciła uwagę analitycznych metafizyków i~filozofów religii na teologię apofatyczną.}.

Warto nadmienić, że wedle zapewnień Jacobsa\index[names]{Jacobs, Jonathan D.} jego praca nie stanowi próby interpretacji jakiejkolwiek doktryny apofatycznej żadnego konkretnego teologia negatywnego -- czy to współczesnego, czy historycznego. Nie jest też teoretycznym rozwinięciem żadnego ze stanowisk teorii Niewysłowionego, lecz właśnie próbą \textit{obrony} takiej teorii. Co ciekawe, jest to próba dokonana z~dwoma dodatkowymi założeniami. Po pierwsze, teza o~niewysławialności Boga wzięta jest najzupełniej poważnie i~broniona w~swojej ,,pełnokrwistej'', silnej wersji. Jacobs\index[names]{Jacobs, Jonathan D.} nie uważa, że niewysławialność Boga wynika z~ułomności czy niedoskonałości ludzkiego systemu poznawczego. Bóg jest niewysławialny ze swej natury, jest to jego wewnętrzna własność. Po drugie, próba ta dokonywana jest w~kontekście doktryny chrześcijańskiej i~przy założeniu, że cały zbiór tez tej doktryny stanowi zbiór tez prawdziwych. Oznacza to, że obrony przed zarzutem sprzeczności wymaga nie tylko sama teoria Niewysłowionego. Przed paradoksem należy uchronić także jednoczesne utrzymywanie, że Bóg jest niewysławialny i~jest ,,jakiś'', że nic o~nim nie można powiedzieć przy jednoczesnym twierdzeniu, że -- na przykład -- jest jeden w~trzech osobach itp.


\section{Prawdy fundamentalne i~przygodne}

Do obrony teorii Niewysłowionego Jacobs\index[names]{Jacobs, Jonathan D.} angażuje popularne w~ostatnich latach w~metametafizyce\footnote{\textit{Sic}! Dyscyplina ta nazywana jest również metaontologią.} pojęcie \textit{fundamentalności}\footnote{Por. R. Bliss, \textit{Fundamentality}, [w:] \textit{The Routledge Handbook of Metametaphysics}, red. tenże, J.T.M. Miller, Routledge, Abingdon -- New York 2020, ss.~211-221; R. Bliss, G. Priest (red.), \textit{Reality and its Structure: Essays in Fundamentality}, Oxford University Press, Oxford 2018; a~także T.E. Tahko, \textit{Fundamentality}, [w:] \textit{The Stanford Encyclopedia of Philosophy}, wyd. jesień 2018, red. E.N. Zalta, {\textless}\url{https://plato.stanford.edu/archives/fall2018/entries/fundamentality/}{\textgreater}.}.
Choć samo w~sobie pojęcie to ma na celu uchwycenie idei, wedle której w~świecie istnieją rzeczy podstawowe i~pierwotne, Jacobs\index[names]{Jacobs, Jonathan D.} koncentruje się raczej na tych rozumieniach fundamentalności, które dotyczą sposobów, w~jaki nasze reprezentacje próbują oddać strukturę świata. Otóż, zdaniem wielu współczesnych metafizyków analitycznych\footnote{Jacobs\index[names]{Jacobs, Jonathan D.} powołuje się wprost na trzech autorów: Kita Fine'a\index[names]{Fine, Kit}, Rossa Camerona\index[names]{Cameron, Ross} oraz Theodore'a Sidera\index[names]{Sider, Theodore}. Jak się jeszcze okaże, to ten ostatni wywarł największy wpływ na przedstawianą tu obronę teorii Niewysłowionego. Por. K. Fine, \textit{The Question of Ontology}, [w:] \textit{Metametaphysics. New Essays on the Foundations of Ontology}, red. D. Chalmers i~in., Oxford University Press, Oxford -- New York 2009, ss.~157-177; R.P. Cameron, \textit{Truthmakers and ontological commitment: or how to deal with complex objects and mathematical ontology without getting into trouble}, ,,Philosophical Studies: An International Journal for Philosophy in the Analytic Tradition'', vol. 140 (2008), nr 1, Selected Papers from the 2007 Bellingham Summer Philosophy Conference, ss.~1-18; T. Sider, \textit{Writing the Book of the World}, Oxford University Press, Oxford 2012.}, o~ile wszystkie prawdziwe reprezentacje odwzorowują świat, niektóre robią to w~specjalny sposób -- przedstawiają, jaki świat jest w~rzeczywistości, jak jest fundamentalnie (na fundamentalnym poziomie), jaka jest prawdziwa natura rzeczy lub też, posługując się platońską metaforą, ,,dzielą rzeczywistość na części pierwsze''\footnote{Dosłownie ,,kroją członki rzeczywistości w~stawach'', org. \textit{carves nature at its joints}. Ta pochodząca z~\textit{Fajdrosa} (265e) metafora przyjęła się idiomatycznie się w~j. angielskim, gdzie oznacza poprawną klasyfikację czy taksonomię lub podział względem właściwych i~trafnych \textit{fundamenta divisionis}.}. Pozwala to Jacobsowi\index[names]{Jacobs, Jonathan D.} wprowadzić podział na prawdy \textit{fundamentalne}, z~którymi mamy do czynienia, gdy prawdziwy sąd odwzorowuje świat w~taki właśnie fundamentalny sposób, oraz -- gdy tak się nie dzieje -- prawdy \textit{niefundamentalne} bądź \textit{przygodne}.

W~celu zilustrowania różnicy pomiędzy prawdami fundamentalnymi i~przygodnymi, Jacobs\index[names]{Jacobs, Jonathan D.} podpiera się przykładem zaczerpniętym od Theodore'a Sidera\footnote{T. Sider, \textit{Writing the Book of the World}, dz. cyt., ss.~1-2. Por. także J.D. Jacobs, \textit{The Ineffable}\ldots, dz. cyt., ss.~161-162.}: wyobraźmy sobie prostokąt utworzony poprzez połączenie dwóch kwadratów, z~których ten po lewej stronie jest biały, a~ten po prawej czarny (rys. \ref{sil-jac-siderpic}). Możemy zgodnie z~prawdą stwierdzić, że połowa prostokąta jest biała, a~połowa czarna oraz że fragment pola powierzchni prostokąta zamalowany kolorem białym jest równy polu powierzchni zamalowanej kolorem czarnym.
%Sider-pic
\begin{figure}[H]
\begin{center}

 \begin{tikzpicture}[every node/.style={inner sep=0,outer sep=0}, line cap=rect,node distance=4cm]

    \node (tl) {};
    \node (tm) [right=of tl] {};
    \node (tr) [right=of tm] {};
    \node (bl) [below=of tl] {};
    \node (bm) [below=of tm] {};
    \node (br) at (bl-|tr) {};

    \coordinate (CENTER) at ($(tl)!0.5!(br)$);
    \node (contra) at (CENTER) {};
    
    \draw[line width=1pt]     (tl) rectangle (br);
    \draw[fill]     (tm) rectangle (br);

    \path[-,color=white,dashed, line width=1pt,inner sep=0,outer sep=0] (contra) edge (tr);
    \path[-,dashed,line width=1pt,inner sep=0,outer sep=0]  (contra) edge (bl);
%    \draw [inner sep=0,outer sep=0](contra) --++ (bl);

\end{tikzpicture}

\caption[Eksperyment myślowy Sidera]{Eksperyment myślowy Sidera\index[names]{Sider, Theodore}. Dwukolorowy prostokąt, którego części zamalowane każdym z kolorów mają równe pola powierzchni.}\label{sil-jac-siderpic}
\end{center}
\end{figure}

Jednakże, rozważmy teraz pewną hipotetyczną społeczność językową, która nie posiada pojęcia ,,czarny'' oraz ,,biały''. Ma ona odmienne pojęcia kolorów -- i~tak, na przykład, zamiast rozpatrywać powyższy prostokąt jako podzielony \textit{pod względem koloru} na dwie równe części wzdłuż osi przechodzącej przez środki jego dłuższych boków, uznaje ona, że prostokąt podzielony jest \textit{kolorystycznie} wzdłuż osi przechodzącej przez jego dwa przeciwstawne wierzchołki. Według tej hipotetycznej społeczności językowej powstałe w~wyniku takiego podziału części prostokąta posiadają odmienne kolory. Mianowicie kolor w~ten sposób wydzielonej lewej górnej części prostokąta nazywają ,,biarnym'', a~kolor prawej dolnej części określają jako ,,czały''. Załóżmy teraz, że jeden z~członków takiej społeczności stwierdza, że ,,pole powierzchni prostokąta zamalowane kolorem czałym jest równe polu powierzchni prostokąta zamalowanemu kolorem biarnym''. Czy twierdzenie takie jest prawdziwe? Zdaniem Sidera\index[names]{Sider, Theodore} należy uznać, że tak. Uwzględniając znaczenia pojęć, jakich używa wypowiadająca taki sąd osoba, sąd ten jest prawdziwy -- biarna część prostokąta nie jest ani większa, ani mniejsza od czałej, ich pola powierzchni są równe. Jednakże, według Sidera\index[names]{Sider, Theodore}, ,,ciężko oprzeć się pokusie stwierdzenia, że ludzie ci się mylą''\footnote{Por. tamże.}. Mimo iż używając swoich specyficznych pojęć kolorów wypowiadają oni sądy prawdziwe, cały czas zdają się być niezdolni do uchwycenia obiektywnej struktury rzeczywistości.

Podążając za tymi intuicjami, Jacobs\index[names]{Jacobs, Jonathan D.} proponuje wprowadzenie do języka logiki klasycznej (modalnego) operatora zdaniowego $\mathscr{F}$, który zestawiony z~dowolnym zdaniem oznacza, że sąd wyrażony przez to zdanie jest sądem \textit{fundamentalnym}. Innymi słowy, wyrażenie $\mathscr{F}A$ powinniśmy czytać, jako ,,Fundamentalnie $A$'', ,,W rzeczywistości $A$'', ,,Sąd wyrażony przez $A$~odwzorowuje obiektywną wewnętrzną strukturę rzeczywistości'', ,,$A$ oddaje prawdziwą naturę rzeczy'' lub nawet ,,$A$ kroi rzeczywistość na części pierwsze''\footnote{Pomysł na wprowadzenie takiego operatora został zaczerpnięty częściowo od Kita Fine'a\index[names]{Fine, Kit} -- zob. K.~Fine, \textit{The Question of Ontology}, dz. cyt., s.~26 -- bezpośrednio zaś od Theodore'a Sidera\index[names]{Sider, Theodore}. Operator proponowany przez Jacobsa\index[names]{Jacobs, Jonathan D.} został jednak znacząco zmodyfikowany. Sider\index[names]{Sider, Theodore} sugeruje, by taki symbol mógł stać przy dowolnej ,,kategorii gramatycznej'' dowolnego wyrażenia językowego. W~szczególności dopuszcza on, by operator ten w~danym wyrażeniu wiązał jakiś funktor, na przykład koniunkcję -- zob. T. Sider, dz. cyt., ss.~216-238. W~mojej opinii, ograniczenie funktora $\mathscr{F}$~do pełnienia roli wyłącznie operatora zdaniowego jest słuszne. Jacobs\index[names]{Jacobs, Jonathan D.} wykazuje się tu względną ogładą logiczną i~chroni tę propozycję przed zarzutami o~brak większego logicznego sensu -- por. J. Jacobs, \textit{The Ineffable}\ldots, dz. cyt., s.~162.}.

Niestety, Jacobs\index[names]{Jacobs, Jonathan D.} nie rozwodzi się wystarczająco wnikliwie na temat precyzyjnego logicznego znaczenia tego operatora, jakie nadane by mu mogło zostać poprzez podanie określonych rządzących nim reguł lub aksjomatów w~obrębie któregoś z~modalnych systemów. Podaje jedynie, jak powinna zachowywać się fundamentalność ($\mathscr{F}$) względem negacji (brak ,,rozdzielności''):
%z~artykułu,\label{sil-jac-fundneg}
\begin{flalign}
& \nvdash \neg \mathscr{F} A \to \mathscr{F} \neg A & \label{sil-jac-fundneg}
\end{flalign}
oraz że zachowana jest tzw. reguła (\textsf{T}), tzn. sądy fundamentalne muszą być prawdzie:
%druga reg. z~art.\label{sil-jac-modalK}
\begin{flalign}
& \vdash \mathscr{F}  A \to A. & \label{sil-jac-modalK}
\end{flalign}

Prawdy fundamentalne, czyli prawdziwe sądy, których ,,struktura idealnie odzwierciedla strukturę rzeczywistości'' możemy teraz zapisywać jako $\mathscr{F}A$\footnote{Jacobs\index[names]{Jacobs, Jonathan D.} prawdy fundamentalne zapisuje używając koniunkcji: $A \land \mathscr{F}A$. Jest to oczywista redundancja -- $A$~jest bezpośrednio wyprowadzalne z~$\mathscr{F}A$ na mocy prawa \ref{sil-jac-modalK}.}. Zdania wyrażające prawdziwe sądy, których struktura nie odzwierciedla rzeczywistości lub odzwierciedla ją w~sposób niedostateczny, nazywane przez Jacobsa\index[names]{Jacobs, Jonathan D.} prawdami niefundamentalnymi lub przygodnymi, będziemy zapisywać jako $A \land \neg \mathscr{F}A$.


\section{Fundamentalna zasada teologii negatywnej i~milczenie w~,,teologicznym pokoju''}

Mając zdefiniowane pojęcia prawd fundamentalnych i~prawd przygodnych, Jacobs\index[names]{Jacobs, Jonathan D.} powołuje do życia kolejną wersję zasady teologii negatywnej. Niech $\mathcal{G}$~będzie zbiorem wszystkich prawdziwych zdań o~tym, jaki Bóg jest wewnętrznie, jaki jest ze swojej natury.
%FNT
\begin{flalign*}
	& \forall_{A \in \mathcal{G}}\ \neg \mathscr{F} A \land \neg \mathscr{F} \neg A. &\tag{FNT}\label{sil-jac-fnt}
\end{flalign*}
W~konsekwencji, nie istnieje żaden prawdziwy \textit{fundamentalny} sąd o~Bogu i~jego naturze -- każdy prawdziwy sąd na ten temat może być co najwyżej prawdą \textit{przygodną}.

Jak powyższa zasada wiąże się z~teologią milczenia? Otóż, jeśli ograniczymy się do wypowiadania o~Bogu wyłącznie prawd fundamentalnych, to nic o~nim nie będziemy mogli powiedzieć. Jacobs\index[names]{Jacobs, Jonathan D.} wyjaśnia ten związek poprzez wprowadzenie metafory ,,pokoju teologicznego''\footnote{Lub ,,pokoju teologii'' (org. \textit{theology room}). Ta metafora także została pożyczona od Sidera\index[names]{Sider, Theodore}, który do zilustrowania swoich przemyśleń o~fundamentalności i~ugruntowaniu używa przenośni ,,pokoju metafizyki''. Por. T. Sider, \textit{Writing the Book of the World}, dz. cyt., ss.~74-77.}. Wchodzimy do pokoju teologicznego zastrzegając, że to, co mówimy o~Bogu, wyraża wyłącznie sądy fundamentalne i~nic ponadto.

\begin{quote}
Jeśli istnieje jakaś fundamentalna prawda wystarczająco bliska temu, co mamy na myśli, możemy ją wypowiedzieć. Jeśli istnieje fundamentalny sąd, ale jest on fundamentalnie fałszywy, możemy go wypowiedzieć i~coś stwierdzić, choć to, co stwierdzimy, będzie fałszywe. Jeśli natomiast nie ma żadnego fundamentalnego sądu wystarczająco bliskiego temu, co mamy na myśli, [\ldots] nie możemy nic stwierdzić. [\ldots]

Jeśli Teza o~Niewyrażalności [\ref{sil-jac-fnt} -- P.U.] jest prawdziwa, a~my wchodzimy do pokoju teologicznego, nie pozostaje nam nic innego, jak zachować milczenie. Nie możemy nic powiedzieć. Jeśli chcielibyśmy opisać Boga w~jakikolwiek sposób -- jako kochającego, miłosiernego czy cierpiącego -- musielibyśmy opuścić pokój teologiczny\footnote{J. Jacobs, \textit{The Ineffable}\ldots, dz. cyt., s.~166.}.
\end{quote}

Teologia apofatyczna w~rozumieniu Jacobsa\index[names]{Jacobs, Jonathan D.} polega więc na tym, że nie możemy konstruować, utrzymywać i~wypowiadać żadnych fundamentalnych sądów o~Bogu i~jego naturze. Możemy konstruować (i konstruujemy) o~nim wyłącznie sądy przygodne. Mogą one być prawdziwe bądź nie, ale każdy, nawet prawdziwy sąd o~Bogu, nie będzie nigdy sądem fundamentalnie prawdziwym. Wchodząc do ,,teologicznego pokoju'' musimy zachować milczenie.


\section{Dyskusja}

Przypomnijmy, że głównym celem pracy Jacobsa\index[names]{Jacobs, Jonathan D.} jest obrona teologii milczenia przed dwoma rodzajami sprzeczności: 1)~,,wewnętrzną'' sprzecznością tej teorii, polegającą na przywoływanym już wielokrotnie paradoksie samoodniesienia oraz 2)~,,zewnętrzną'' sprzecznością, czyli absurdem polegającym na twierdzeniu, że o~Bogu nic nie można powiedzieć, przy jednoczesnym twierdzeniu, że jest \textit{jakiś}, na przykład miłosierny, jeden w~trzech osobach itp.

Zdaniem Jacobsa\index[names]{Jacobs, Jonathan D.} opisana powyższej strategia ukrycia par zdań sprzecznych za (zanegowanym) operatorem fundamentalności $\mathscr{F}$ w~obrębie \ref{sil-jac-fnt} wraz z~ograniczeniem rozdzielności negacji względem fundamentalności \eqref{sil-jac-fundneg} pozwala zachować spójność teorii Niewysłowionego w~węższym, ,,wewnętrznym'' sensie. Zacznijmy jednak od próby obrony teorii niewysłowionego przed sprzecznościami w~tym drugim, szerszym, ,,zewnętrznym'' sensie, ponieważ próba ta, w~mojej opinii, wydaje się bardziej skuteczna. Podział prawd na fundamentalne i~przygodne sprawia, że niesprzeczne wydaje się utrzymywanie i~wypowiadanie dowolnego sądu o~naturze Boga z~jednoczesnym zachowaniem teorii Niewysłowionego, to znaczy z~równoczesnym twierdzeniem, że Bóg jest niewysławialny. Zgodnie z~pierwotnym założeniem Jacobsa\index[names]{Jacobs, Jonathan D.}, można niesprzecznie z~\ref{sil-jac-fnt} uznać, że cały zbiór tez doktryny chrześcijańskiej jest zbiorem zdań prawdziwych. Zdania te są prawdziwe, choć wyrażają jedynie sądy niefundamentalne. Są prawdziwe, ale prawdziwe przygodnie.

Mimo iż Jacobs\index[names]{Jacobs, Jonathan D.} zastrzegał, że jego rozważania nie stanowią interpretacji żadnej z~doktryn teologii negatywnej któregokolwiek z~jej protagonistów -- czy to historycznych, czy współczesnych -- sam wskazuje, że jego pomysł może służyć do modelowania teologii Pseudo-Dionizego Areopagity\index[names]{Pseudo-Dionizy Areopagita}. W~ramach tej propozycji możemy reprezentować trzystopniowe ,,wspinanie się po apofatycznej drabinie''. Zaczynamy od stwierdzenia jakiejś (przygodnej) prawdy w~ramach teologii katafatycznej, na przykład, że Bóg jest jeden w~trzech osobach ($p$). Następnie przechodzimy do pierwszego stopnia zaprzeczenia -- stwierdzamy, że nie jest prawdą, że fundamentalnie Bóg jest jeden w~trzech osobach ($\neg \mathscr{F}p$). W~ostatnim kroku akceptujemy zaprzeczenie zaprzeczenia -- uznajemy, że nieprawda, że fundamentalnie nie jest tak, że Bóg jest jeden w~trzech osobach ($\neg \mathscr{F} \neg p$). Wedle propozycji Jacobsa\index[names]{Jacobs, Jonathan D.} możemy więc przyjmować dowolną doktrynę religijną -- na przykład chrześcijańską doktrynę o~trzech hipostazach jednej boskiej substancji -- pod warunkiem, że tezy takiej doktryny uznamy wyłącznie za prawdy przygodne. Fundamentalnej prawdy o~Bogu nigdy nie poznamy. Bóg Jacobsa\index[names]{Jacobs, Jonathan D.} jest wysławialny, choć przygodnie. Fundamentalnie zaś pozostaje niewysławialny.

Warto zauważyć, że w~konsekwencji w~obronę teorii Niewysłowionego przed paradoksem Jacobs\index[names]{Jacobs, Jonathan D.} angażuje pewien rodzaj teorii podwójnej prawdy. W~obrębie jego rozważań taka obrona jest możliwa tylko, gdy przyjmie się, że \textit{w~pewnym sensie} tezy doktryny religijnej (np. teza o~Trójcy Świętej) są prawdziwe, w~\textit{innym} są fałszywe. Uwaga ta jest o~tyle interesująca, o~ile koncepcja podwójnej prawdy pojawiła się w~historii myśli po raz pierwszy w~kontekście relacji i~związków rozumu i~wiary, a~konkretniej przy potępieniu\footnote{Zob. H. Thijssen, \textit{Condemnation of 1277}, [w:] \textit{The Stanford Encyclopedia of Philosophy}, wyd. zima 2018, red. E.N. Zalta, {\textless}\url{https://plato.stanford.edu/archives/win2018/entries/condemnation/}{\textgreater}.} prób pogodzenia z~doktryną chrześcijańską nowej -- przynajmniej dla ówczesnych, średniowiecznych myślicieli -- filozofii arystotelesowskiej\index[names]{Arystoteles}. Jak pokazuje Bartosz Brożek\index[names]{Brożek, Bartosz}\footnote{Zob. B. Brożek, \textit{The Double Truth Controversy: An Analytical Essay}, Copernicus Center Press, Kraków 2010.}, sama teoria podwójnej prawdy generuje swoje filozoficzne problemy i~jest ciekawym i~wdzięcznym do logicznych analiz zagadnieniem. W~tym miejscu jednak nie będę roztrząsał tych filozoficznych problemów i~logicznych zagadnień. Mogę natomiast przyznać rację Jacobsowi\index[names]{Jacobs, Jonathan D.}\index[names]{Jacobs, Jonathan D.}, że zastosowanie takiego ,,wytrychu'' rozwiązuje problemy z~,,zewnętrznym'' rodzajem sprzeczności teologii milczenia. Trudno jednak oprzeć się wrażeniu, że rozwiązuje ten problem w~sposób trywialny -- podobne konstatacje mogą wyrugować każdą sprzeczność z~dowolnej teorii. Dodatkowo, robi to w~sposób, który nie może zadowolić żadnej ze stron teologicznego spektrum -- ani myśliciela apofatycznego, ani teologa katafatycznego.

Teolog katafatyczny (zwłaszcza apologeta lub taki, którego celem jest obrona prawomyślności wiary) z~pewnością nie będzie doceniał faktu, że w~doktrynie Jacobsa\index[names]{Jacobs, Jonathan D.} wszystkie prawdy religijne uznaje się za niefundamentalne, a~do wyrażenia natury Boga wystarczyć muszą sądy o~charakterze przygodnym. Jeśliby przyjąć takie rozumienie prawd fundamentalnych i~przygodnych, jakie jest tu proponowane, należy stwierdzić, że żadna z~ksiąg teologicznych nie mówi o~niczym, co odpowiadałoby rzeczywistości -- cokolwiek jakikolwiek teolog ustali na temat Boga, jakkolwiek go nie opisze, trzeba sądzić, że \textit{w~rzeczywistości} tak nie jest. Z~drugiej strony, przedstawiona w~powyższy sposób teologia negatywna traci swój ,,apofatyczny pazur''. Bóg przestaje być niewysławialny, niewyrażalny i~nieopisywalny. W~gruncie rzeczy możemy -- nie naruszając ducha tak sformatowanego apofatyzmu -- twierdzić o~nim cokolwiek, godząc się jedynie na to, że mówiąc o~Bogu wyrażamy co najwyżej sądy niefundamentalne.

Można jednak spróbować wziąć za dobrą monetę takie rozumienie apofatyzmu powołując się na fakt, że rozważania Jacobsa\index[names]{Jacobs, Jonathan D.} są dobrze ugruntowane w~filozoficznych ustaleniach Sidera\index[names]{Sider, Theodore}. Problem w~tym, że ustalenia te Jacobs\index[names]{Jacobs, Jonathan D.} przyjmuje właściwie bezkrytycznie i~bezrefleksyjnie -- do pewnego stopnia w~celowy i~świadomy sposób. Tymczasem nietrudno przedstawić argumenty przeciw tej metametafizycznej wizji. Jednym z~takich argumentów może być zwrócenie uwagi na fakt, że podział na prawdy fundamentalne i~przygodne sam nie wydaje się fundamentalny. Trudno oprzeć się wrażeniu, że to dychotomia fundamentalne–przygodne odpowiada parze przymiotników ,,czałe''–,,biarne'' z~eksperymentu myślowego Sidera\index[names]{Sider, Theodore} (rys. \ref{sil-jac-siderpic}), a~dużo bliższy ,,struktury rzeczywistości'' wydaje się być, na przykład, tradycyjny podział na sądy prawdziwe i~fałszywe. Nawet jeśli tak nie jest, to nie do końca wiadomo, jakie są kryteria oceny fundamentalności pojęć, prawd i~sądów. Dlaczego i~na jakiej podstawie jedne sądy mamy uważać za fundamentalne, a~innym odmawiać odzwierciedlania struktury rzeczywistości? Od pytania o~kryterium fundamentalności niedaleko już do oskarżenia metametafizycznej teorii Sidera\index[names]{Sider, Theodore} o~epistemiczną arogancję i~metafizyczny imperializm -- dlaczego mamy uważać, że to akurat nasze pojęcia oddają rzeczywistość taką, jaka jest, a~inne pojęcia (na przykład pojęcia jakiejś hipotetycznej odmiennej społeczności językowej) są niedoskonałe i~,,czegoś im brakuje''\footnote{T. Sider, \textit{Writing the Book of the World}, dz. cyt., s.~2.}. Ewidentnie, Siderowi\index[names]{Sider, Theodore} nie wystarczyło tej epistemicznej pokory, którą wykazał się Jacobs\index[names]{Jacobs, Jonathan D.} zakładając, że może istnieć rzeczywistość, do której nigdy nie będziemy mieli ,,fundamentalnego'' dostępu. Jednakże, przedstawiając swoje rozważania na temat prawd o~charakterze fundamentalnym i~przygodnym, Jacobs\index[names]{Jacobs, Jonathan D.} odcina się od takich oraz szeregu innych filozoficznych dyskusji. Na przykład, świadomie nie rozstrzyga kwestii ontologicznego charakteru samych reprezentacji, sądów czy postaw propozycjonalnych. Równie dobrze mogą to być obiekty abstrakcyjne, językowe czy intencjonalne stany mentalne. Za pierwotne wzięte tu zostało pojęcie struktury rzeczywistości, zatem po prostu ,,mówimy, że \guillemotleft$\mathscr{F}A$\guillemotright, gdy $A$~idealnie odwzorowuje [\ldots] strukturę rzeczywistości''\footnote{J. Jacobs, \textit{The Ineffable}\ldots, dz. cyt., s.~163.}. Podobnie, nie argumentuje on za słusznością przyjętej przez niego wersji metametafizycznej koncepcji fundamentalności oraz jej konsekwencji. Jak sam zauważa, mogą z~niej wynikać pewne istotne, choć wysoce dyskusyjne z~filozoficznego punktu widzenia, twierdzenia -- na przykład takie, że istnieją sądy jednocześnie prawdziwe i~niefundamentalne. Mimo iż jest przekonany, że taki sposób myślenia o~tym, jak nasze reprezentacje odwzorowują świat, jest ,,prawdopodobnie prawdziwy''\footnote{Tamże, s.~164.}, celowo powstrzymuje się on przed argumentowaniem za tym, że jest
%to pogląd słuszny.
tak w~istocie.
Do osiągnięcia zamierzonego celu -- czyli obrony teologii milczenia przed zarzutami o~bycie logicznie sprzeczną -- nie potrzebuje on wykazywania prawdziwości czy słuszności podziału na prawdy fundamentalne i~niefundamentalne czy teorii, jaka za takim podziałem stoi. Wystarczy mu uznanie, że podział ten i~teoria są logicznie spójne. Jednakże nie wszyscy autorzy zgadzają się z~nim w~tej kwestii.

Michael Rea\index[names]{Rea, Michael C.} i~Samuel Lebens\index[names]{Lebens, Samuel R.} sugerują, że w~rozważaniach Jacobsa\index[names]{Jacobs, Jonathan D.} kryje się sprzeczność\footnote{Zob. S.R. Lebens, \textit{Why so negative about negative theology? The search for a~Plantinga-proof apophaticism}, ,,International Journal for Philosophy of Religion'', vol. 76 (2014), nr 3, przypis 1; M.C. Rea, \textit{Essays in Analytic Theology}, vol.~1, ser. \textit{Oxford Studies in Analytic Theology}, Oxford University Press,  Oxford 2021, ss. 120-138.}, choć sami wskazują ją w~niewłaściwym miejscu. Zauważają oni, że -- choć Jacobs\index[names]{Jacobs, Jonathan D.} akceptuje wprost prawa logiki klasycznej -- konstrukcja \ref{sil-jac-fnt} sprawia, że w~obrębie prawd fundamentalnych naruszone zostaje prawo wyłączonego środka\footnote{Oczywiście, ściśle rzecz biorąc \eqref{sil-jac-fundtertium} nie jest prawem wyłączonego środka, ale stanowi pewną jego postać, w której oba człony alternatywy poprzedzone są operatorem fundamentalności. Rea\index[names]{Rea, Michael C.} i Lebens\index[names]{Lebens, Samuel R.} sami nazywają je prawem ,,dwuwartościowości'', a~w tym kontekście nawet ,,fundamentalnym prawem dwuwartościowości''. Por. tamże.}
\begin{flalign}
&\mathscr{F} A \lor \mathscr{F} \neg A.&\label{sil-jac-fundtertium}
\end{flalign}
Następnie Lebens\index[names]{Lebens, Samuel R.} próbuje wykazać, że
\begin{flalign}
&\neg (\mathscr{F} A \lor \mathscr{F} \neg A) \vdash \mathscr{F} A \land \mathscr{F} \neg A,&\label{sil-jac-leb}
\end{flalign}
ale przedstawione przez niego wnioskowanie zawiera szkolny błąd\footnote{Lebens\index[names]{Lebens, Samuel R.} sam przyznaje się do tego błędu w~pracy opublikowanej trzy lata później. Por. S.R.~Lebens, \textit{Negative Theology as Illuminating and/or Therapeutic Falsehood}, [w:] \textit{Negative Theology as Jewish Modernity}, red. M. Fagenblat, Indiana University Press, Bloomington 2017, przypis 29. Co ciekawe, mimo przyznania, że przedstawione przez niego wcześniej wnioskowanie zawierało błąd, Lebens\index[names]{Lebens, Samuel R.} wciąż utrzymuje, że naruszenie zasady wyłączonego środka w~obrębie prawd fundamentalnych musi doprowadzić do sprzeczności.}. Nie da się wykazać takiej konsekwencji, jeśli ograniczymy się -- jak robi to Lebens\index[names]{Lebens, Samuel R.} -- wyłącznie do praw logiki klasycznej. Może jednak ono zajść przy odpowiedniej interpretacji operatora $\mathscr{F}$. Kwestia ta zostanie jeszcze poruszona w~dalszej części niniejszego paragrafu. Na razie zauważmy, że -- choć dowód formuły \eqref{sil-jac-leb} podany przez Lebensa\index[names]{Lebens, Samuel R.} nie jest poprawny -- rzeczywiście, ,,fundamentalne'' prawo wyłączonego środka \eqref{sil-jac-fundtertium} nie tyle zostaje naruszone, co jego negacja jest prostą logiczną konsekwencją \ref{sil-jac-fnt} -- i~można tego dowieść (tym razem skutecznie) na gruncie wyłącznie praw logiki klasycznej.
\begin{flalign}
& \neg \mathscr{F} A \land \neg \mathscr{F} \neg A &  \eqref{sil-jac-fnt}\label{negf3nd1} \\
& (\neg \mathscr{F} A \land \neg \mathscr{F} \neg A) \to \neg (\mathscr{F} A \lor \mathscr{F} \neg A) & \qquad\qquad \text{(p. De Morgana\index[names]{De Morgan, Augustus})}\label{negf3nd2}  \\
& \neg (\mathscr{F} A \lor \mathscr{F} \neg A) & (\text{MP }\ref{negf3nd1},\ref{negf3nd2})\label{negf3nd3}
\end{flalign}

Odnotujmy w~takim razie, że po raz kolejny próba podania sformalizowanej teorii teologii negatywnej prowadzi do przyjęcia pewnej formy zanegowanego prawa wyłączonego środka. W~konsekwencji, stosowne uwagi krytyczne sformułowane w~poprzednich rozdziałach będą miały zastosowanie także do teorii Jacobsa\index[names]{Jacobs, Jonathan D.}. By jednak wykazać, że anonsowana przez Jacobsa\index[names]{Jacobs, Jonathan D.} obrona teologii apofatycznej jest nieefektywna i~sama prowadzi do sprzeczności, spróbujmy wpierw dokonać rekonstrukcji (modalnego) rachunku logicznego, będącego podstawą dla właściwej interpretacji operatora fundamentalności.

Oddajmy Jacobsowi\index[names]{Jacobs, Jonathan D.} to, że podczas oznaczania sądów fundamentalnych i~przygodnych oraz wyrażania tezy o~niewysławialności (\ref{sil-jac-fnt}) posługuje się względnie formalną notacją. Wprowadza do języka operator zdaniowy ($\mathscr{F}$) o~wyraźnie modalnych cechach. Nie podaje on jednak dokładnej charakterystyki rachunku, w~ramach którego jego teoria jest konstruowana. Mimo tego, fakt, że w~pracy zawarto dwie uwagi dotyczące działania operatora $\mathscr{F}$ -- prawa \eqref{sil-jac-fundneg} oraz \eqref{sil-jac-modalK} -- sprawia, że podjęcie próby
%rekonstrukcji
identyfikacji
takiego rachunku nie musi pozostać bezowocne. Zauważmy najpierw, że -- znów na gruncie samej logiki klasycznej -- \ref{sil-jac-fnt} pociąga za sobą negację prawa dystrybucji negacji względem $\mathscr{F}$. %(\ref{sil-jac-fundneg}).
\begin{flalign}
& \neg \mathscr{F} A \land \neg \mathscr{F} \neg A &  \eqref{sil-jac-fnt}\label{negax1} \\
& (\neg \mathscr{F} A \land \neg \mathscr{F} \neg A) \to \neg(\neg \mathscr{F} A \to \mathscr{F} \neg A) & \qquad \text{(p. negacji implikacji)}\label{negax2}  \\
& \neg(\neg \mathscr{F} A \to \mathscr{F} \neg A) & (\text{MP }\ref{negax1},\ref{negax2})\label{negax3}
\end{flalign}
Przyjmując \ref{sil-jac-fnt} możemy więc powstrzymać się od bezpośredniego stosowania \eqref{sil-jac-fundneg} w~celu scharakteryzowania operatora $\mathscr{F}$, przynajmniej w~obrębie zbioru zdań o~boskiej naturze. Zatem która z~logik modalnych będzie właściwa dla oddania charakterystyki $\mathscr{F}$? Należy lojalnie przyznać, że sprawa ta nie jest ostatecznie rozstrzygnięta nawet w~przypadku drobiazgowo analizowanej aletycznej logiki modalnej, w~której operatorów modalnych używa się na oznaczenie pojęć możliwości i~konieczności. Samą konieczność, dużo lepiej ugruntowaną w~filozoficznych rozważaniach i~uważaną za pierwotną motywację do powstania logik modalnych, można logiczne wyrażać w odmiennych rachunkach i~na różne sposoby. Spróbujmy zatem pozostać uczciwi w~stosunku do autora krytykowanej tu pracy i~powstrzymać się od takiego sposobu rekonstrukcji, z~którym on sam by się nie mógł utożsamiać. Na podstawie wszystkich powyższych uwag można przyjąć, że najlepszym kandydatem na rachunek, w~ramach którego powinno się dokonywać rekonstrukcji rozważań Jacobsa\index[names]{Jacobs, Jonathan D.}, jest najsłabsza z~normalnych logik modalnych zawierająca aksjomat \eqref{sil-jac-modalK}, czyli system K\footnote{System K~to system modalny, który posiada prawo dystrybucji, regułę wymuszania oraz \ref{sil-jac-modalK}. Por. J. Garson, \textit{Modal Logic}, [w:] \textit{The Stanford Encyclopedia of Philosophy}, wyd. lato 2021, red. E.N. Zalta, {\textless}\url{https://plato.stanford.edu/archives/sum2021/entries/logic-modal/}{\textgreater}. Oczywiście, wybór systemu K~nie jest niekontrowersyjny, ponieważ nie jest do końca jasne, jaki system modalny życzyłby sobie stosować Jacobs\index[names]{Jacobs, Jonathan D.}. W~ostatnich paragrafach, w~których odpierane są hipotetyczne zarzuty, zdaje się on wyrażać chęć zablokowania także $ \mathscr{F} p \to \mathscr{F} \neg \neg p$~lub nawet $ \mathscr{F} p \to \mathscr{F} (p \lor q)$. W~kontekście rozważań Jacobsa\index[names]{Jacobs, Jonathan D.} kontrowersyjna może wydać się także, obecna we wszystkich systemach normalnej logiki modalnej, reguła wymuszania.} (wzbogacony, rzecz jasna, o~\ref{sil-jac-fnt}).

W~celu zbadania spójności ,,fundamentalnej'' wersji teologii milczenia, odpowiedzmy wpierw na pytanie, czy samo \ref{sil-jac-fnt}, tak sformułowane, jest sądem fundamentalnym, czy przygodnym. Rozważania Jacobsa\index[names]{Jacobs, Jonathan D.} zostawiają pole do interpretacji ciągnących w~stronę obu tych rozwiązań. Z~jednej strony uważa on, że ,,Bóg jest niewysławialny fundamentalnie'' (a ,,wysławialny przygodnie''\footnote{J. Jacobs, \textit{The Ineffable}\ldots, dz. cyt., s.~167.}), zatem
\begin{flalign}
&\mathscr{F} \eqref{sil-jac-fnt}. \label{sil-jac-funFNT}&
\end{flalign}
Ostatecznie, przecież \ref{sil-jac-fnt} jest jakimś sądem na temat boskiej natury. Jak twierdzi Jacobs\index[names]{Jacobs, Jonathan D.}, ,,Bóg jest niewysławialny, niemożliwy do zrozumienia i~niepojmowalny \textit{sam w~sobie}, jest taki wewnętrznie''\footnote{Tamże, s.~165. Omar Fakhri próbuje rozwinąć pomysł Jacobsa\index[names]{Jacobs, Jonathan D.} w~stronę, która sugerowałaby, że niewysławialność Boga jest ugruntowana raczej w~niedoskonałościach ludzkiego języka: O. Fakhri, \textit{The ineffability of God}, ,,International Journal for Philosophy of Religion'', vol. 89 (2021), ss.~25-41.}. O~ile zakładamy, że \ref{sil-jac-fnt} jest prawdziwe (a wynika to bezpośrednio z~\ref{sil-jac-funFNT}\ oraz \ref{sil-jac-modalK}), to $\text{\ref{sil-jac-fnt}} \in \mathcal{G}$. Zatem, na mocy samego \ref{sil-jac-fnt},
\begin{flalign}
&\neg \mathscr{F} \eqref{sil-jac-fnt}.&
\end{flalign}


Można próbować jeszcze bronić teorię Jacobsa\index[names]{Jacobs, Jonathan D.} przyjmując, że \ref{sil-jac-fnt} jest prawdą niefundamentalną. Niemniej, jeśli poruszamy się w~obrębie systemu K, na mocy reguły wymuszania otrzymamy $\mathscr{F} \eqref{sil-jac-fnt}$ lądując w~bezpośredniej sprzeczności. Oczywiście, obrońca doktryny Jacobsa\index[names]{Jacobs, Jonathan D.} może podnieść zarzut, że reguła wymuszania, obecna w~systemie K (i~wszystkich normalnych logikach modalnych), blokuje pożądaną możliwość istnienia prawd prawdziwych, choć niefundamentalnych. Problem w~tym, że taki obrońca niespecjalnie miałby co zaoferować w~zamian. Alternatywy, jakie miałby w zanadrzu, nie są atrakcyjne z~logicznego punktu widzenia. Są tu dwie drogi wyjścia. Można 1) zrezygnować z~pełności rachunku, który przyjmujemy za ramę naszych rozważań. Taką alternatywę odrzucam ze względu na jej nieracjonalność. Gdybyśmy do formalnej rekonstrukcji twierdzeń teologii apofatycznej dopuszczali rachunki, co do których nie zachodzi twierdzenie o~pełności, nasze rozwiązania zostałyby strywializowane i~pozbawione sensu. Można też 2) zaproponować w~zamian jakiś rachunek modalny nieposiadający reguły wymuszania i~prawa dystrybucji. Postępujący tą drogą obrony zwolennik doktryny Jacobsa\index[names]{Jacobs, Jonathan D.} miałby do dyspozycji tzw. nienormalną logikę modalną, która posiada taką właśnie charakterystykę. Problem w~tym, że ten wybór zablokuje z~kolei inne pożądane cechy systemu, np. podawane \textit{explicite} prawo \eqref{sil-jac-modalK}\footnote{Zresztą, do nienormalnej logiki modalnej można podnosić też pewne wątpliwości natury filozoficznej, zob. M. Klonowski, K. Krawczyk, \textit{Problem wszechwiedzy logicznej. Krytyka światów nienormalnych i~propozycja nowego rozwiązania}, ,,Filozofia Nauki'', vol. 27 (2019), nr 1, ss.~27-48.}. Zatem, nawet gdy założymy, że \ref{sil-jac-fnt} jest tylko prawdą przygodną, problem samozwrotności generujący sprzeczności wciąż pozostanie aktualny. Skoro twierdzimy, że apofatyzm polega na ograniczeniu się do wypowiadania wyłącznie fundamentalnych prawd i~sądów, musimy także zrezygnować z~utrzymania \ref{sil-jac-fnt}. Mówiąc krótko, przy takim założeniu musimy przemilczeć także ten fakt, że Bóg jest niewyrażalny. Używając metafory Jacobsa\index[names]{Jacobs, Jonathan D.}, aby wejść do ,,teologicznego pokoju'', musimy powstrzymać się od wchodzenia do środka i~pozostać na zewnątrz.


\chapter{Klasyczna teoria niewysławialności}\label{sil-boch}

Swoje rozważania dotyczące teologii apofatycznej Józef Maria Bocheński\index[names]{Bocheński, Józef Maria} zawarł w~dziele \textit{Logika religii}\footnote{J.M. Bocheński, \textit{Logika religii}, tłum. S. Magala, [w:] tenże, \textit{Logika i~filozofia}, Wydawnictwo Naukowe PWN, Warszawa 1993. Wyd. org.: J.M. Bocheński, \textit{The Logic of Religion}, New York University Press, New York 1965.}. Przeprowadza je w~kontekście analizy dyskursu religijnego -- jego języka, logiki oraz znaczenia, jakie przez taki dyskurs jest przekazywane. W~tym kontekście Bocheński\index[names]{Bocheński, Józef Maria} odróżnia teologię negatywną od teorii tego, co niewysłowione. Ta pierwsza przyznaje dyskursowi religijnemu jakieś znaczenie, choć ogranicza je do ,,czystych negacji''. Ta druga prowadzi do stwierdzenia, że dyskurs religijny nie posiada żadnego znaczenia. W~obu przypadkach Bocheński\index[names]{Bocheński, Józef Maria} próbuje wykazać, że teorie te wcale nie muszą zawierać bezpośredniej sprzeczności i~pokazuje warunki, przy których należy je uznać za spójne. Jednakże ostatecznie odrzuca je -- przyznając, że mimo wszystko nie mogą one stanowić właściwych teorii dyskursu religijnego -- z~innych, pozalogicznych powodów.


\section{Typologia znaczeń i~możliwych teorii religii}\label{sil-boch-znaczteol}

W przeciwieństwie do filozofii, w której metoda analizy logicznej jest powszechnie akceptowana, stosowalność logiki na gruncie badań dyskursu religijnego bywała często podważana -- także z~powodu pewnych antylogicznych tendencji obecnych właściwie we wszystkich religiach.
%O~ile posługiwanie się logiką w~celu modelowania i~formalnych badań rozmaitych problemów z~zakresu filozofii czy nauki jest generalnie akceptowalną metodą badawczą, o~tyle stosowanie jej na gruncie religii bywało często kwestionowane -- także z~powodu pewnych antylogicznych tendencji obecnych właściwie we wszystkich religiach.
Bocheński\index[names]{Bocheński, Józef Maria}, badając prawomocność logiki religii, podaje typologię znaczeń (rys. \ref{sil-boch-typ}) niesionych przez dyskurs (także dyskurs religijny) oraz generowane i~porządkowane przez tę typologię możliwe\footnote{Bocheński\index[names]{Bocheński, Józef Maria} mówi o~wszystkich możliwych \textit{a~priori} teoriach religii zakładając, że mogą znaleźć się wśród nich także takie teorie, co do których sam twierdzi, że nikt ich \textit{explicite} nie utrzymywał. Por. J.M.~Bocheński, \textit{Logika religii}, dz. cyt., ss.~347-352.} teorie religii. Wedle tej typologii wypowiedź może posiadać lub nie posiadać znaczenia. Znaczenie niesione przez wypowiedź może być subiektywne lub obiektywne oraz komunikowalne lub niekomunikowalne. Komunikowalne znaczenie obiektywne możne być pełne albo niepełne. Pełne znaczenia mogą przybrać postać twierdzenia lub inną (performatywy, modlitwy, nakazu, reguły itd.).
%Bochenski-typologia-znaczenia-pic \label{sil-boch-typ}
\begin{figure}[H]
\begin{center}
 \begin{tikzpicture}[node distance=1cm]

    \node (1) {znaczenie};
    \node (21) [below=of 1, xshift=-2cm] {subiektywne};
    \node (22) [below=of 1, xshift=2cm] {obiektywne};
    \node (31b) [below=of 22, xshift=-2cm] {niekomunikowalne};
    \node (32b) [below=of 22, xshift=2cm] {komunikowalne};
    \node (41) [below=of 32b, xshift=-2cm] {niepełne};
    \node (42) [below=of 32b, xshift=2cm] {pełne};
    \node (51) [below=of 42, xshift=-2cm] {twierdzenia};
    \node (52) [below=of 42, xshift=2cm] {inne};

    \path[-] (1) edge (21);
    \path[-] (1) edge (22);
    \path[-] (22) edge (31b);
    \path[-] (22) edge (32b);
    \path[-] (32b) edge (41);
    \path[-] (32b) edge (42);
    \path[-] (42) edge (51);
    \path[-] (42) edge (52);


\end{tikzpicture}

\caption[Typologia znaczeń według Bocheńskiego]{Typologia znaczeń według Bocheńskiego\index[names]{Bocheński, Józef Maria}\footnotemark.}\label{sil-boch-typ}
\end{center}
\end{figure}
\footnotetext{Por. tamże, s.~348.}


Powyższa typologia pozwala Bocheńskiemu\index[names]{Bocheński, Józef Maria} podzielić i~uporządkować teorie religii na:

\begin{enumerate}
\item Teorie nonsensu głoszące, że dyskurs religijny nie posiada żadnego znaczenia;
\item Teorie emocjonalistyczne głoszące, że znaczenie dyskursu religijnego jest subiektywne i~czysto uczuciowe;
\item Teorie niekomunikowalności, według których dyskurs religijny posiada obiektywne znaczenie, choć jest ono niekomunikowalne;
\item Teorie znaczeń częściowych, według których obiektywne znaczenie komunikowalne w~obrębie dyskursu religijnego jest niepełne;
\item Teorie komunikowalności nietwierdzeniowej głoszące, że dyskurs religijny posiada obiektywne komunikowalne znaczenie, ale nie wyraża się ono w~twierdzeniach;
\item Teorie twierdzeniowe zakładające, że przynamniej pewna część dyskursu religijnego składa się ze zdań mających znaczenie.
\end{enumerate}
Warto dodać, że powyższa lista stanowi pewien porządek. Z~perspektywy teorii $n$, cokolwiek twierdzi się w~ramach teorii $m\leq n$ może być uznane za prawdziwe w~stosunku do pewnych fragmentów dyskursu religijnego. Teoria $n$~głosi ponadto, że istnieją jakieś partie dyskursu religijnego cechujące się odmiennym rodzajem znaczenia. Na przykład, z~punktu widzenia teorii twierdzeniowej można uznać, że pewne fragmenty dyskursu religijnego zawierają wyłącznie reguły, inne są niekomunikowalne, w~jeszcze innych liczą się wyłącznie uczucia i~nie ma tam żadnego znaczenia ponad to subiektywne i~związane z~emocjami. Co więcej, z~perspektywy teorii twierdzeniowej można również przyznać, że pewne fragmenty dyskursu religijnego w~ogóle nie niosą żadnego znaczenia. Zatem teorie te są uporządkowane względem wzrastającej mocy. Co ciekawe, w~dociekaniach Bocheńskiego\index[names]{Bocheński, Józef Maria} teoria Niewysłowionego leży u~podstaw tej hierarchii, natomiast teologia negatywna należy do teorii znajdujących się u~jej szczytu -- Bocheński\index[names]{Bocheński, Józef Maria} umieszcza ją wśród teorii znaczeń niepełnych.


\section{Tajemnica częściowego znaczenia}\label{sil-boch-tajem}

W~rozważaniach Bocheńskiego\index[names]{Bocheński, Józef Maria} teologia negatywna jest zatem alternatywnym dla teologii milczenia sposobem przedstawiania i~badania dyskursu religijnego. Terminem, który często pojawia się w~kontekście obu tych doktryn, jest ,,tajemnica''. Bocheński\index[names]{Bocheński, Józef Maria} twierdzi, że choć to o~przedmiocie religijnym twierdzi się, że jest tajemniczy (i~to tajemniczy w~sposób szczególny, ,,wyższy''), własność ,,tajemniczości'' nie przysługuje przedmiotom, lecz zdaniom. Z~tego powodu mówi się na przykład o~,,tajemnicach wiary''\footnote{Według Bocheńskiego\index[names]{Bocheński, Józef Maria} mówienie o~tajemnicach wiary to wypowiedzi metajęzykowe. Por. tamże, s.~413.}. Bocheński\index[names]{Bocheński, Józef Maria} pojęcie tajemnicy opiera na pojęciu rozumienia.

\begin{defin}[Tajemnica]
,,Tajemniczy'' w~sensie ogólnym oznacza stan rzeczy, którego nie daje się w~pełni uchwycić. Mówiąc ściślej, zdanie $p$ jest tajemnicze dla podmiotu $x$ wtedy, i~tylko wtedy, jeżeli $x$ nie w~pełni rozumie $p$\footnote{Ta i~poniższe definicje tajemniczości pochodzą bezpośrednio od Bocheńskiego\index[names]{Bocheński, Józef Maria}. Zob. tamże, ss.~413-414.}. Zatem, w~zależności od znaczenia ,,rozumienia'':

\begin{enumerate}[label = (\arabic*)]
\item $p$ jest tajemnicze dla $x$ wtedy, i~tylko wtedy, jeżeli $x$~dobrze (nawet w~pełni) rozumie znaczenie $p$, ale nie potrafi o~własnych siłach (intuicja lub wnioskowanie) przesądzić prawdziwości $p$.
\item $p$ jest tajemnicze dla $x$ wtedy, i~tylko wtedy, gdy $x$ rozumie znaczenie $p$ jako takie, ale nie zna jego aksjomatycznych powiązań z~innymi zdaniami, jakie akceptuje.
\item $p$ jest tajemnicze dla $x$ wtedy, i~tylko wtedy, gdy $p$ zawiera przynamniej jeden termin \textit{a} taki, że \textit{a} stosowane jest w~$p$ w~sposób tylko częściowo pokrywający się z~zastosowaniem \textit{a} w~dyskursie świeckim.
\item $p$ jest tajemnicze dla $x$ wtedy, i~tylko wtedy, gdy w~$p$ istnieje przynajmniej jeden taki termin \textit{a}, który jest całkowicie pozbawiony znaczenia dla $x$.
\end{enumerate}
\end{defin}

Powyższa definicja wyznacza cztery klasy teorii dotyczących tajemnicy\footnote{Por. typologia znaczeń i~możliwych teorii religii w~sekcji \ref{sil-boch-znaczteol}.}. Według pierwszej z~nich (1) prawd tajemnic religijnych nie da się poznać -- można je przyjąć wyłącznie na drodze wiary. Mogą one zostać przyjęte i~zaakceptowane wyłącznie na mocy autorytetu, a~niezależnie od niego nie da się ich uzasadnić lub określić ich wartości prawdziwościowych. Bocheński\index[names]{Bocheński, Józef Maria} ten rodzaj tajemnicy nazywa ,,tajemnicą prawdziwości'' i~uważa go w~gruncie rzeczy za tezę dotyczącą struktury dyskursu religijnego. Według drugiej z~nich (2) wierny nie rozumie prawd wiary, ponieważ każdorazowo rozumienie ich zależy od kontekstu aksjomatycznego. Twierdzi się tutaj, że słowa, w~których wyrażono prawdy wiary, nie wyczerpują nieskończonego bogactwa Boga, ponieważ istnieją powiązania aksjomatyczne zdań wyrażających prawdy wiary z~innymi zdaniami, których wierny nie zna. Tę odmianę tajemnicy Bocheński\index[names]{Bocheński, Józef Maria} nazywa ,,tajemnicą aksjomatyczną'' i~przekonuje, że nie jest to zjawisko odosobnione i~wyjątkowe dla dyskursu religijnego. W~końcu, w~trzeciej klasie teorii dotyczących tajemnicy (3) twierdzi się, że terminy, których używa się do wyrażenia prawd wiary, w~obrębie dyskursu religijnego nie przekazują pełni znaczenia, jakie jest im przypisywane na gruncie dyskursu świeckiego. W~myśl tej teorii prawdy wiary mają wciąż jeszcze jakieś znaczenie, choć już tylko częściowe. Z~tego powodu Bocheński\index[names]{Bocheński, Józef Maria} nazywa to ,,tajemnicą częściowego znaczenia'', a~wśród przykładów tej klasy teorii wymienia teologię negatywną (i teorię analogii)\footnote{Zob. J.M. Bocheński, \textit{Logika religii}, dz. cyt., ss.~414-415.}. Do czwartej klasy teorii (4) wyznaczanej przez ,,tajemnicę nonsensu'' należy teoria Niewysłowionego, zgodnie z~którą wszystkie terminy dyskursu religijnego są pozbawione znaczenia. Teoria ta zostanie omówiona w~kolejnej sekcji (\ref{sil-boch-nonsens}).

W~obrębie teologii negatywnej uważa się więc, że istnieje tylko częściowa tożsamość między znaczeniem danego terminu na gruncie dyskursu świeckiego, a~znaczeniem tego samego terminu użytego w~dyskursie religijnym. Zatem, w~przeciwieństwie do teorii nonsensu, według teologii negatywnej nie jest tak, że dyskurs religijny nic nie znaczy, jednakże jakiekolwiek ma on znaczenie, ma je na drodze ,,czystej negacji''. Jak zauważa Bocheński\index[names]{Bocheński, Józef Maria}, żaden ze zwolenników teologii negatywnej nie starał się sformułować takiej teorii w~dostatecznie precyzyjny sposób, a~przez swoją niejasną postać prowokuje ona do krytyki. Większość uwag krytycznych przedstawionych przez Bocheńskiego\index[names]{Bocheński, Józef Maria} wskazuje na jej paradoksalny charakter.

Po pierwsze, mając wyrażenie ,,$x$ jest niebiałe'' i~orzekając negację tego wyrażenia o~przedmiocie religii, otrzymamy negację niebiałości. Będzie to oznaczać, że przedmiot religii jest biały -- w~konsekwencji otrzymamy własność całkowicie pozytywną. Po drugie, jeśli dopuścimy do tego, by o~przedmiocie religii orzekać wszystkie negacje, popadniemy w~sprzeczność. Możemy bowiem mu przypisać własność bycia niebiałym (czyli negację własności bycia białym) oraz własność bycia nie-niebiałym (czyli negację bycia niebiałym), a~zatem także własność bycia białym, o~ile utrzymujemy silne prawo podwójnej negacji.

Według Bocheńskiego\index[names]{Bocheński, Józef Maria} można uniknąć tych sprzeczności, gdy ograniczy się zakres teorii do klasy własności pozytywnych. Wpierw jednak należałoby tę klasę zdefiniować. Bocheński\index[names]{Bocheński, Józef Maria} nie podaje jednak odpowiedniej definicji tego rodzaju własności. By nadać swoim rozważaniom nieco więcej precyzji, przedstawia następującą propozycję indukcyjnej definicji własności pozytywnych:
\begin{defin}[Własność pozytywna\footnote{Tamże, s.~416.}]\label{sil-boch-pozytywna}\hfill\ 
\begin{enumerate}%[label = (\arabic*)]
\item Własność postrzegana bezpośrednio jest własnością pozytywną.
\item Własność definiowana za pomocą formuły zawierającej wyłącznie symbole własności pozytywnych i~terminów logiki pozytywnej\footnote{Nie jest do końca jasne, co Bocheński\index[names]{Bocheński, Józef Maria} rozumie przez ,,logikę pozytywną''.} jest własnością pozytywną.
\end{enumerate}
\end{defin}
Od razu jednak dodaje, że nie jest ona zadawalająca z~co najmniej dwóch powodów. Po pierwsze, definicja ta jest za wąska -- klasa własności pozytywnych ograniczona jest dużo surowiej niż wymagaliby tego zwolennicy teologii apofatycznej. Po drugie, pojęcie własności postrzeganej bezpośrednio jest bardzo nieścisłe. Obrazując to jego własnym przykładem -- dlaczego nie można by postrzegać bezpośrednio, że krowa nie jest niebieska? Bocheński\index[names]{Bocheński, Józef Maria} dodaje jednak, że problemy z~podaniem dostatecznie precyzyjnej i~trafnej definicji własności pozytywnych nie są najpoważniejszymi z~tych, które nękają teologię negatywną. Dlatego, na potrzeby dalszych rozważań zakłada on, że pojęcie własności pozytywnej zostało zdefiniowane poprawnie.

Przyjmując to założenie Bocheński\index[names]{Bocheński, Józef Maria} próbuje formalnie zrekonstruować znaczenie teologii negatywnej w~taki sposób, który zadowoliłby jej potencjalnych zwolenników. Niech $t$~będzie terminem występującym w~dyskursie świeckim w~jakimś znaczeniu. W~obrębie tak sformatowanej teorii mamy prawo utrzymywać, że znaczenie $t$~jest inne, gdy przypisywane jest przedmiotowi religii, i~używając $t$~w dyskursie religijnym mamy na myśli coś innego. Niech zapis $M(t,\pi ,\phi)$ oznacza ,,$t$ jest terminem występującym w~dyskursie świeckim $\pi$ w~znaczeniu $\phi $''. $\alpha $ natomiast oznacza klasę własności pozytywnych (jakkolwiek zdefiniowanych). Treść teologii negatywnej można zatem wyrazić w~następujący sposób postać:
\begin{flalign*}
		& \forall t  M(t, \pi, \phi) \land
		\phi \in \alpha
		\to \neg \phi (g)\footnotemark, &\tag{NNT}\label{sil-boch-NNT}
\end{flalign*}
%\begin{align}
%\forall t\footnote{}  M(t, \pi, \phi) \land
%\phi \in \alpha
%\to \neg \phi (g),\label{sil-boch-NNT}
%\end{align}
gdzie $g$~oznacza przedmiot religii, czyli Boga. W~takim wypadku dyskurs religijny może zawierać wyłącznie twierdzenia, które orzekają o~Bogu negację własności pozytywnych wyrażonych w~terminach świeckich. Deskrypcja określona skojarzona z~predykatem ,,jest Bogiem'' przedstawiać się będzie następująco:\footnotetext{Zarówno w~oryginalnej pracy, jak i~w jej polskim tłumaczeniu kwantyfikator ten wiąże zmienną(?) oznaczoną symbolem $F$, która nie ma żadnych wystąpień w~niniejszej formule. Za dobrą monetę przyjmuję, że jest to oczywista pomyłka drukarska. Wskazuje na to także poprawny, jak się wydaje, zapis zawarty w~\ref{sil-boch-NNTbis} poniżej.}
\begin{flalign*}
		\begin{split}
		 G(x) \equiv_{\text{def}}\ &\exists x \big((
		\forall t  M(t, \pi, \phi) \land
		\phi \in \alpha
		\to \neg \phi (x))\ 
		\land\\
		&\forall y ((\forall t
		M(t, \pi, \phi) \land
		\phi \in \alpha
		\to \neg \phi (y))
%		\to
		\equiv
		y=x)\big)\footnotemark.
		\end{split}\tag{NNT'}\label{sil-boch-NNTbis}
\end{flalign*}
%\begin{align}
%\begin{split}
%G(x) \equiv_{\text{def}}\ &\exists x \big((
%\forall t  M(t, \pi, \phi) \land
%\phi \in \alpha
%\to \neg \phi (x))\\
%\land\ &\forall y ((\forall t
%M(t, \pi, \phi) \land
%\phi \in \alpha
%\to \neg \phi (y))
%\to y=x)\big)\footnote{}.\label{sil-boch-NNTbis}
%\end{split}
%\end{align}
\footnotetext{Nie da się ukryć, że zapis $M(t,\pi ,\phi)$ niewiele wnosi do tak skonstruowanej zasady. Jakkolwiek istnieją pewne motywacje za rozróżnianiem znaczeń terminów występujących w~dyskursie religijnym i~świeckim, wszystko wskazuje na to, że w~rozważaniach dotyczących teologii negatywnej można z~tego podziału zrezygnować, mówiąc o~znaczeniach terminów bez osadzania ich w~konkretnych dyskursach. Tak czy owak, rezygnacja z~tego podziału i~uwspólnienie dyskursu doprowadziłyby do znacznego uproszczenia tej formuły. Stopień skomplikowania tego zapisu wynika również z~tego, że Bocheński\index[names]{Bocheński, Józef Maria} korzysta z~notacji używanej w~\textit{Principia Mathematica}, a~prezentowana tu formuła stanowi jej tłumaczenie na bardziej współczesną notację. Por. A.N. Whitehead, B. Russell, \textit{Principia Mathematica}, vol. 1, Cambridge University Press, Cambridge 1910 oraz B. Linsky, \textit{The Notation in Principia Mathematica}, [w:] \textit{The Stanford Encyclopedia of Philosophy}, wyd. zima 2021, red. E.N. Zalta, <\url{https://plato.stanford.edu/archives/win2021/entries/pm-notation/}>.} 

Dzięki ograniczeniu teorii do klasy własności pozytywnych, tak sformułowana teologia negatywna może zostać pozbawiona sprzeczności. Zdaniem Bocheńskiego\index[names]{Bocheński, Józef Maria} własność przypisywana Bogu w~teorii wyrażonej zasadą \ref{sil-boch-NNT} nie jest własnością metajęzykową (mimo użycia metajęzykowych terminów), lecz pewną własnością prze\-dmio\-to\-wo-językową drugiego rzędu -- taką, że nie można przypisać mu żadnej prze\-dmio\-to\-wo-językowej własności pierwszego rzędu. Przy odpowiedniej definicji klasy własności pozytywnych ($\alpha$) z~teorii tej można wyrugować sprzeczności. Mimo tego Bocheński\index[names]{Bocheński, Józef Maria} nie jest usatysfakcjonowany tym osiągnięciem. Fakt, że w~teologii negatywnej nie można przypisać Bogu żadnej przedmiotowo-językowej własności pierwszego rzędu wyklucza ją z~roli adekwatnej teorii dyskursu religijnego z~co najmniej dwóch powodów. Po pierwsze dlatego, że istnieje niepusty (a nawet zawierający więcej niż jeden element) zbiór własności przedmiotowo-językowych pierwszego stopnia przypisywanych Bogu na gruncie dyskursu religijnego. Po drugie wydaje się, że przedmiot, któremu możemy przypisać wyłącznie własność przedmio\-to\-wo-językową drugiego stopnia -- taką, że nie można przypisać mu żadnej przedmiotowo-językowej własności pierwszego stopnia -- nie może być obiektem czci i~chwały. Mówiąc wprost, Bocheński\index[names]{Bocheński, Józef Maria} odrzuca teologię negatywną z~powodu jej uwikłania nie w~,,wewnętrzny'', lecz oba ,,zewnętrzne'' paradoksy\footnote{Por. rozdz.~\ref{sil-int-par}.}.


\section{Teoria nonsensu}\label{sil-boch-nonsens}

W~typologii Bocheńskiego\index[names]{Bocheński, Józef Maria} teoria tego, co niewysłowione, jest teorią \textit{nonsensu}, zgodnie z~którą dyskurs religijny jest pozbawiony jakiegokolwiek znaczenia. Zwraca on uwagę, że według wielu komentatorów teoria ta jest wewnętrznie sprzeczna. Z~reguły argumentacja za taką tezą polega na wskazaniu paradoksu Niewyrażalnego -- twierdząc, że nie da się niczego powiedzieć o~Bogu, teoria ta sama coś o~nim mówi, a~zatem jest sprzeczna i~należy ją odrzucić. Bocheński\index[names]{Bocheński, Józef Maria} występuje przeciwko takiemu przedstawianiu teologii milczenia. Twierdzi on, że da się ją uratować od sprzeczności, lecz nawet mimo tego, nie odpowiada ona potrzebom dyskursu religijnego\footnote{Zob. J.M. Bocheński, \textit{Logika religii}, dz. cyt., ss.~353-356.}.

Bocheński\index[names]{Bocheński, Józef Maria} uważa, że jeśli przestrzega się pewnych obowiązujących w~logice konwencji, zarzut sprzeczności stawiany teorii Niewysłowionego przestanie obowiązywać. Należałoby wpierw dowieść, że w~danym ,,układzie odniesienia'' teoria ta prowadzi do sprzeczności, tymczasem nikt takiego dowodu nie przedstawił. Według Bocheńskiego\index[names]{Bocheński, Józef Maria} jest zupełnie przeciwnie -- nietrudno wykazać, że teoria tego, co niewysłowione, jest spójna. Poniżej przedstawię jego argumentację.

Załóżmy, że dwuargumentowy predykat ${N_w}(x,\mathcal{L})$ oznacza ,,$x$ jest niewyrażalne w~języku $\mathcal{L}$''. Zapiszmy teraz formułę zawierającą ten predykat
\begin{flalign}
&\exists x\exists \mathcal{L}\ N_w(x,\mathcal{L}).&\label{sil-boch-prenw}
\end{flalign}


Wydaje się, że nie tylko można ją wypowiedzieć nie popadając w~sprzeczność, lecz także jest ona prawdziwa -- nietrudno znaleźć taki obiekt $x$ i~taki język $\mathcal{L}$, które spełniałyby zapisany wyżej warunek. (Bocheński\index[names]{Bocheński, Józef Maria} podaje przykład krowy i~języka szachów: nie da się opisać krowy w~języku szachów).

Możemy powyższy przykład uogólnić i~sformułować metajęzykową definicję Boga o~następującej postaci:
\begin{flalign*}
&\forall \mathcal{L}\ N_w(g,\mathcal{L}),\tag{MNT}\label{sil-boch-MNT}&
\end{flalign*}
%${\forall}$lNw(g,l),\label{sil-boch-MNT}
gdzie $g$ jest stałą oznaczającą Boga, lub zgodnie z~podejściem Bocheńskiego\index[names]{Bocheński, Józef Maria} do wyboru kategorii językowej terminu Bóg:
\begin{flalign*}
&G(x) \equiv_{\text{def}} \exists x \forall \mathcal{L} \big(N_w(x,\mathcal{L}) \land \forall y (N_w(y,\mathcal{L})
%\to
\equiv
x = y)\big),\tag{MNT'}\label{sil-boch-MNTbis}&
\end{flalign*}
%G(x) {\textbackslash}equiv\_\{def\} {\textbackslash}exists x~{\textbackslash}forall l~(Nw(x,l) {\textbackslash}land {\textbackslash}forall y~(Nw(y,l) {\textbackslash}equiv x~= y)), \label{sil-boch-MNTbis}
gdzie $G(x)$ jest predykatem oznaczającym ,,$x$ jest Bogiem''. Na pierwszy rzut oka wydaje się, że te formuły są bardziej problematyczne -- twierdzenie, że $x$~jest niewysłowione w~żadnym języku zdaje się prowadzić do sprzeczności. Bocheński\index[names]{Bocheński, Józef Maria} próbuje jednak uniknąć tego problemu, stosując zwykłe konwencje wykorzystywane do pozbywania się antynomii semantycznych. Należy założyć, że żadne zdanie traktujące o~pewnej klasie języków, nie jest formułowane w~żadnym z~tych języków. Aby było pozbawione sprzeczności, musi zostać sformułowane w~innym języku, czyli odpowiednim metajęzyku. Możemy więc założyć, że klasa języków wspominana w~\ref{sil-boch-MNTbis} i~\ref{sil-boch-MNT} jest klasą języków przedmiotowych. W~takim wypadku zasady te stają się zdaniami metajęzyka pierwszego stopnia. Po takim zabiegu, sformułowana definicja jest znacząca i~pozbawiona sprzeczności. Nie ma bowiem niespójności w~twierdzeniu, że coś nie daje się wysłowić w~jakimś języku, lub nawet w~klasie języków, o~ile twierdzenie to jest wyrażone w~języku nienależącym do tej klasy. Według Bocheńskiego, przy takim założeniu -- standardowym z~punktu widzenia logiki ogólnej -- teoria tego, co niewysłowione, pozostaje relewantna i~spójna, a~zarzut sprzeczności zostaje oddalony.

Bocheński\index[names]{Bocheński, Józef Maria} odrzuca jednak teorię niewysłowionego z~tych samych powodów, dla których odrzucał teologię negatywną. Po pierwsze, na mocy \ref{sil-boch-MNT} nie można przypisać Bogu jakiejkolwiek własności przedmiotowo-językowej. Jedyną własnością, jaką możemy mu przypisać, jest metajęzykowa własność bycia niewysłowionym w~żadnym z~języków przedmiotowych. W~takim wypadku wierny nie mógłby akceptować żadnego zdania dyskursu religijnego, które przypisałoby Bogu jakąkolwiek własność przedmiotowo-językową. Wydaje się to niespójne z~faktycznym dyskursem religijnym. Po drugie, niemożliwe byłoby oddawanie czci obiektowi, o~którym wiemy tylko i~wyłącznie to, że nie można o~nim nic powiedzieć. Jeśli wierny miałby czcić obiekt pozbawiony własności przedmiotowo-językowych, równie dobrze tym obiektem mógłby być nie Bóg a~szatan, Ludwik XIV\index[names]{Ludwik XIV} lub Homer\index[names]{Homer}\footnote{Przykłady te pochodzą odpowiednio z~tamże, s.~356 oraz S. Gäb, \textit{Languages of ineffability: the rediscovery of apophaticism in contemporary analytic philosophy of religion}, [w:] \textit{Negative Knowledge}, red. S. Hüsch i~in., Narr, Tübingen 2020, ss.~191-206.}. Mówiąc krótko, zdaniem Bocheńskiego\index[names]{Bocheński, Józef Maria} opisany powyżej metajęzykowy zabieg zdaje się chronić teologię milczenia przed ,,wewnętrznym'' paradoksem Niewyrażalnego. Pozostaje on jednak nieskuteczny w~likwidowaniu jej ,,zewnętrznych'' paradoksów -- zarówno w~wersji twierdzeniowej, jak i~nietwierdzeniowej. Z~tego powodu Bocheński\index[names]{Bocheński, Józef Maria} postuluje odrzucenie tej teorii.


\section{Dyskusja}\label{sil-boch-dyskusja}

W~\textit{Logice religii} Bocheński\index[names]{Bocheński, Józef Maria} przedstawia w~gruncie rzeczy dwie niezależne interpretacje teologii apofatycznej, które nazywa teologią negatywną oraz teorią tego, co niewysłowione. Ta pierwsza dopuszcza, by język religii posiadał jakieś znaczenie, choć jest ono ograniczane do przypisywania Bogu negacji pozytywnych własności przedmiotowo-językowych pierwszego rzędu. Uniknięcie sprzeczności teologii negatywnej dokonuje się poprzez kolejne ograniczenie -- ograniczenie nałożone na klasę tychże
%właśnie
własności pozytywnych. Teoria tego, co niewysłowione, w~typologii Bocheńskiego\index[names]{Bocheński, Józef Maria} pozbawia dyskurs religijny jakiegokolwiek znaczenia. W~tej teorii Bóg jest niewysławialny i~nie da się przypisać mu żadnej przedmiotowo-językowej własności. Uniknięcie jej paradoksalnego charakteru dokonuje się przez zastosowanie metajęzykowego zabiegu (w stylu Tarskiego\index[names]{Tarski, Alfred}). Uznaje się, że ,,niewysławialność'' jest własnością metajęzykową mówiącą o~tym, że Boga nie da się wyrazić w~żadnym z~języków przedmiotowych.

Olbrzymią słabością teologii negatywnej w~rozumieniu Bocheńskiego\index[names]{Bocheński, Józef Maria} jest fakt, że trudno jest znaleźć w~historii myśli jakiegokolwiek jej reprezentanta. On sam zauważa, że nikt tej teorii nie starał się ,,sformułować w~terminach dostatecznie precyzyjnych''\footnote{J.M. Bocheński, \textit{Logika religii}, dz. cyt., s.~416.}. Wydaje się jednak, że w~rzeczywistości po prostu nikt jej nie utrzymywał. Gdy teolog apofatyczny mówi, że do Boga nie można stosować żadnych określeń językowych, nie twierdzi, że trzeba zaprzeczyć każdemu pozytywnemu predykatowi, lecz każdemu predykatowi w~ogóle. Oczywiście, we współczesnej literaturze można znaleźć pewne próby zreinterpretowania wybranych tez autentycznych przedstawicieli teologii apofatycznej w~taki sposób, by tworzyły one coś na kształt proponowanego przez Bocheńskiego\index[names]{Bocheński, Józef Maria} rozwiązania\footnote{Por. P. Rojek, \textit{Logika teologii negatywnej}, ,,Pressje'', nr 29 (2012), a~także rozdz.~\ref{rojek-bochenski}.}. Ostatecznie jednak, są to próby spreparowania sztucznej teorii, a~nie rekonstrukcja rzeczywistej doktryny apofatycznej.

Spróbujmy jednak ustalić charakter własności przypisywanej Bogu w~\ref{sil-boch-NNT}. Mimo występujących w~niej terminów metajęzykowych Bocheński\index[names]{Bocheński, Józef Maria} uważa, że -- w~przeciwieństwie do formalizmu użytego w~teorii tego, co niewysłowione -- nie jest to własność metajęzykowa, lecz własność przedmiotowo-językowa drugiego stopnia. Skoro tak, obecność w~tej formule terminów metajęzykowych budzi pewne zastrzeżenia. W~istocie trudno oprzeć się wrażeniu, że zapis $\lceil \forall t M(t,\pi,\varphi) \rceil$  niewiele wnosi do skonstruowanej na potrzeby teologii negatywnej zasady. Rozróżnianie dyskursu świeckiego i~religijnego ma pewne znaczenie dla rozważań Bocheńskiego\index[names]{Bocheński, Józef Maria} w~ogólności, ale w~kontekście samej teologii negatywnej można z~niego zrezygnować bez straty sensu proponowanych rozstrzygnięć -- mówiąc po prostu o~znaczeniu terminów (bez osadzania ich w~konkretnych dyskursach). Tak czy owak, rezygnacja z~tego podziału i~uwspólnienie dyskursu nie tylko znacznie uprościłoby \ref{sil-boch-NNT} (a zwłaszcza \ref{sil-boch-NNTbis}), lecz także uczyniłoby te zasady bliższymi jakiejkolwiek rzeczywistej doktryny apofatycznej.

Dość nieoczekiwaną konsekwencją użycia $\lceil \forall t M(t,\pi,\varphi) \rceil$ w~formułowaniu zasad teologii negatywnej jest fakt, że w~dyskursie religijnym musimy ograniczyć się wyłącznie do terminów znanych z~dyskursu świeckiego. Inaczej mówiąc, w~obrębie języka religii nie powinniśmy używać jakichkolwiek nowych terminów, niewystępujących w~obrębie dyskursu świeckiego, na przykład jakichś własnych i~wewnętrznych terminów dyskursu religijnego. Gdy przyjdzie nam przypisać Bogu jakąś własność wyrażoną terminem niewystępującym w~jakimś znaczeniu w~języku dyskursu świeckiego, nie będziemy mogli mu przepisać ani tej własności, ani jej negacji, nawet gdyby należała do klasy własności pozytywnych.

Jednakże największą bolączką teorii nazywanej przez Bocheńskiego\index[names]{Bocheński, Józef Maria} teologią negatywną jest brak adekwatnej definicji własności pozytywnych. Skoro w~ostateczności teologia negatywna ma polegać na przypisywaniu Bogu negacji własności pozytywnych, a~na definicji tychże własności ma opierać się obrona tej teorii przed sprzecznościami, mamy prawo oczekiwać, by zostały one dostatecznie trafnie i~precyzyjnie zdefiniowane. Tymczasem epistemiczna definicja podana przez Bocheńskiego\index[names]{Bocheński, Józef Maria} opierająca się na ,,bezpośrednim postrzeganiu'' własności pozytywnych jest przez niego samego uważana za zbyt wąską i~nieścisłą. By móc skutecznie chronić teologię negatywną przed sprzecznościami, definicja ta przede wszystkim nie może dopuszczać wzajemnego określania jednej pozytywnej własności przez negację drugiej (oraz -- w~zależności od konkretnej postaci ewentualnej definicji -- ograniczać w~obrębie stosowania własności pozytywnych działanie silnego prawa podwójnej negacji). Rzeczywiście wydaje się, że dokonana przez Bocheńskiego\index[names]{Bocheński, Józef Maria} próba nadania tej teorii większej precyzji jest nieskuteczna we wskazanym tu zakresie. Dodatkowo, by uniknąć oczywistej sprzeczności w~\ref{sil-boch-NNTbis} z~klasy własności pozytywnych trzeba usunąć jeszcze dwie konkretne własności. Po pierwsze, należy z~tego grona wykluczyć własność wyrażoną predykatem $G(x)$ -- ,,$x$ jest Bogiem''. Po drugie, należy uznać, że pewna własność przedmiotowo-językowa drugiego rzędu -- ,,taki, że nie można przypisać mu żadnej przedmiotowo-językowej własności pierwszego rzędu'' -- także nie należy do własności pozytywnych.

Z~punku widzenia przedstawionych w~tej części pracy rozważań, teologia negatywna w~rozumieniu Bocheńskiego\index[names]{Bocheński, Józef Maria} jest jednak jedynie kompromisowym rozwiązaniem. Ogranicza ona możliwość mówienia o~Bogu, ale w~sposób niepełny -- redukuje mowę o~Bogu do przypisywania mu wyłącznie negacji własności pozytywnych. Bardziej konsekwentna w~tym zakresie jest opisywana przez niego teoria Niewysłowionego, która zabrania mówienia o~Bogu czegokolwiek -- przynajmniej w~obrębie języków przedmiotowych -- czyniąc, zdaniem Bocheńskiego\index[names]{Bocheński, Józef Maria}, dyskurs religijny pozbawionym jakiekolwiek bezpośredniego znaczenia.

Na pierwszy rzut oka wydaje się, że ,,metajęzykowy zabieg'', jaki Bocheński\index[names]{Bocheński, Józef Maria} zastosował w~modelu teorii tego, co niewysłowione, oddala zarzut niespójności tej doktryny i~jest to akceptowalne logicznie rozwiązanie. Należy jednak przyznać, że nie jest to rozwiązanie bez skazy. Pomysłem Bocheńskiego\index[names]{Bocheński, Józef Maria} na usunięcie problematycznego samoodniesieniowego paradoksu Niewyrażalnego jest uznanie, że ,,niewysławialność'' jest metajęzykowym dwuargumentowym predykatem orzekanym o~jakimś obiekcie i~pewnym języku przedmiotowym. Oznacza to, że gdy mówimy o~Bogu, że jest niewysławialny w~żadnym z~języków (przedmiotowych), nie wypowiadamy tego sądu w~żadnym z~języków, o~których mówimy, ale w~odpowiednim metajęzyku lub, inaczej mówiąc, wyłączamy metajęzyk z~zasięgu kwantyfikacji obecnej w wypowiadanym w~ten sposób sądzie. Inspiracji Bocheńskiego dla takiego rozumienia teologii milczenia należy szukać w~logicznych ustaleniach z~pierwszej połowy XX wieku, zwłaszcza w~pracach Alfreda Tarskiego\index[names]{Tarski, Alfred}\footnote{A. Tarski, \textit{Pojęcie prawdy w~językach nauk dedukcyjnych}, Prace Towarzystwa Naukowego Warszawskiego, Wydział III Nauk Matematyczno-Fizycznych, vol. 34, Warszawa 1933.} poświęconych klasycznej teorii prawdy, które powstały w~wyniku zmagań z~paradoksami semantycznymi, w~szczególności z~paradoksem kłamcy\footnote{Por. rozdz.~\ref{sil-int-par}.}.

Według klasycznej teorii prawdy zdania, w~których formułowane są semantyczne paradoksy samoodniesienia -- takie, jak analizowany w~tej części pracy paradoks Niewyrażalnego czy paradoks kłamcy -- są błędnie skonstruowane. Miesza się w~nich dwa poziomy językowe: język przedmiotowy i~metajęzyk. W~języku przedmiotowym mówimy o~świecie i~obiektach znajdujących się w~świecie, metajęzyk jest bogatszy -- możemy w~nim wyrażać także sądy o~języku przedmiotowym. Zatem, by móc mówić o~prawdziwości zdań (lub, w~przypadku teologii milczenia w~ujęciu Bocheńskiego\index[names]{Bocheński, Józef Maria}, niewysławialności obiektów w~danym języku), narzuca się stratyfikację języków i~zabrania ich mieszania. Dla przykładu załóżmy, że mamy zdanie \ref{sil-boch-T1} o~treści ,,Śnieg jest biały''.
%T1Śnieg jest biały.
\begin{flalign}
& \text{Śnieg jest biały.} &&\tag{T1}\label{sil-boch-T1}
\end{flalign}
Możemy teraz o~nim wypowiedzieć się w~metajęzyku, na przykład przy pomocy następującego zdania \ref{sil-boch-T2}:
%T2Zdanie T1 jest prawdziwe.
\begin{flalign}
& \text{Zdanie \ref{sil-boch-T1} jest prawdziwe.} &&\tag{T2}\label{sil-boch-T2}
\end{flalign}
Zdanie T2 sformułowane jest w~metajęzyku nadbudowanym nad językiem przedmiotowym, w~którym wyrażono \ref{sil-boch-T1}. Poziomy języka nie są więc naruszane i~zdanie jest poprawnie skonstruowane. Zgodnie z~koncepcją Tarskiego\index[names]{Tarski, Alfred}, warunek prawdziwości zdań (a~zatem i~\ref{sil-boch-T1}) można określić za pomocą tzw. cząstkowych definicji prawdy zwanych także \mbox{T-równoważnościami}. Taka cząstkowa definicja prawdy dla \ref{sil-boch-T1} przybierze postać:
%T2'Zadanie \ref{sil-boch-T1} jest prawdziwe wtedy i~tylko wtedy, gdy śnieg jest biały.
\begin{flalign}
& \text{Zdanie \ref{sil-boch-T1} jest prawdziwe wtedy i~tylko wtedy, gdy śnieg jest biały.} &&\tag{T2'}\label{sil-boch-T2prim}
\end{flalign}
Równoważność ta należy do metajęzyka, mimo iż na jej końcu pojawiają się wyrażenia równokształtne z~\ref{sil-boch-T1} -- stanowią one tłumaczenie terminów i~zdań z~języka przedmiotowego na metajęzyk. Jeśli natomiast chcielibyśmy powiedzieć coś o~zdaniu \ref{sil-boch-T2}, musielibyśmy znów użyć innego języka i~orzec w~nim na przykład, że
%T3Zdanie \ref{sil-boch-T2} jest prawdziwe.
\begin{flalign}
& \text{Zdanie \ref{sil-boch-T2} jest prawdziwe.} &&\tag{T3}\label{sil-boch-T3}
\end{flalign}
Zatem \ref{sil-boch-T3} nie jest już elementem metajęzyka, lecz języka wyższego poziomu -- metametajęzyka. Jeśli chcemy wyrazić warunek prawdziwości zdania \ref{sil-boch-T2}, również musimy dokonać tego na poziomie metametajęzyka -- T-równoważność dla \ref{sil-boch-T2} przyjmuje formę:
%T3'Zdanie \ref{sil-boch-T2} jest prawdziwe wtedy i~tylko wtedy, gdy zadanie \ref{sil-boch-T1} jest prawdziwe.
\begin{flalign}
& \parbox[t]{.89\linewidth}{\strut Zdanie \ref{sil-boch-T2} jest prawdziwe wtedy i~tylko wtedy, gdy zdanie \ref{sil-boch-T1} jest prawdziwe.\strut} &&\tag{T3'}\label{sil-boch-T3prim}
\end{flalign}
Z~poziomu metametajęzyka, możemy mówić o~wszystkich językach niższego poziomu, a~więc nie tylko o~metajęzyku, lecz także o~języku przedmiotowym.
%T3''Zdanie \ref{sil-boch-T2} jest prawdziwe wtedy i~tylko wtedy, gdy zadanie \ref{sil-boch-T1} jest prawdziwe, to znaczy wtedy i~tylko wtedy, gdy śnieg jest biały.
\begin{flalign}
& \parbox[t]{.87\linewidth}{\strut Zdanie \ref{sil-boch-T2} jest prawdziwe wtedy i~tylko wtedy, gdy zdanie \ref{sil-boch-T1} jest prawdziwe, to znaczy wtedy i~tylko wtedy, gdy śnieg jest biały.\strut } &&\tag{T3''}\label{sil-boch-T3bis}
\end{flalign}
Słowem, nie możemy zdefiniować prawdy dla danego języka w~tym języku. Nie da się określić prawdziwości zdań w~języku, w~którym zdania te zostały sformułowane. By to zrobić, musimy wejść na poziom metajęzyka, na którym można mówić nie tylko o~wszystkim, o~czym mówiliśmy w~języku przedmiotowym, lecz również i~o tym języku. W~taki sposób tworzy się nieskończona hierarchia języków. Na każdym z~wyższych poziomów możemy wyrazić wszystko, co było możliwe do wyrażenia w~językach niższego poziomu, oraz sądy o~zdaniach języków niższego poziomu.

Sposób generowania nieskończonej hierarchii języków w~alternatywnym do propozycji Bocheńskiego\index[names]{Bocheński, Józef Maria} modelu teorii Niewysłowionego przedstawia Bartosz Brożek\index[names]{Brożek, Bartosz}\footnote{Zob. B. Brożek, \textit{Marzenie Leibniza. Rzecz o~języku religii}, Copernicus Center Press, Kraków 2016, rozdz.~4.}. W~ramach tego ujęcia teza o~niewysławialności Boga została zredukowana do metajęzykowego stwierdzenia, że wszystkie wypowiedzi o~Bogu są bezsensowne.

Rozważmy zdanie:
%D1Bóg jest intelektem.
\begin{flalign}
& \text{Bóg jest intelektem.} &&\tag{D1}\label{sil-boch-D1}
\end{flalign}
W~analizie Brożka\index[names]{Brożek, Bartosz}, gdy Dionizy\index[names]{Pseudo-Dionizy Areopagita} zaprzecza tego rodzaju zdaniom, chce jedynie stwierdzić, że nie można tak mówić, że z~takim zdaniami

\begin{quote}
jest coś nie tak, nie możemy w~ten sposób się wyrazić. Gdyby Dionizy\index[names]{Pseudo-Dionizy Areopagita} znał odróżnienie języka przedmiotowego i~metajęzyka, zamiast pisać ,,Bóg nie jest intelektem'' sformułowałby odpowiednią wypowiedź metajęzykową\footnote{Tamże.}.
\end{quote}
Zgodnie z~interpretacją Brożka\index[names]{Brożek, Bartosz}, taka metajęzykowa wypowiedź Dionizego\index[names]{Pseudo-Dionizy Areopagita} o~zdaniu \ref{sil-boch-D1} przybrałaby postać:
%D2Zdanie \ref{sil-boch-D1} jest bezsensowne.
\begin{flalign}
& \text{Zdanie \ref{sil-boch-D1} jest bezsensowne.} &&\tag{D2}\label{sil-boch-D2}
\end{flalign}
Zdanie uważa się za pozbawione sensu, gdy nawet hipotetycznie nie da się jednoznacznie określić, czy jest prawdziwe, czy fałszywe. Brożek\index[names]{Brożek, Bartosz} formułuje więc odpowiednią równoważność (analogiczną do T-równoważności Tarskiego\index[names]{Tarski, Alfred}):
%D2'Zdanie \ref{sil-boch-D1} jest bezsensowne wtedy i~tylko wtedy, gdy \ref{sil-boch-D1} nie jest prawdziwe i~\ref{sil-boch-D1} nie jest fałszywe.
\begin{flalign}
& \parbox[t]{.87\linewidth}{\strut Zdanie \ref{sil-boch-D1} jest bezsensowne wtedy i~tylko wtedy, gdy \ref{sil-boch-D1} nie jest prawdziwe i~\ref{sil-boch-D1} nie jest fałszywe.\strut} &&\tag{D2'}\label{sil-boch-D2prim}
\end{flalign}
Zakładając dwuwartościowość\footnote{Tzn. zakładając, że zdanie jest prawdziwe wtedy i~tylko wtedy, gdy nie jest fałszywe oraz że zdanie jest fałszywe wtedy i~tylko wtedy, gdy nie jest prawdziwe.} oraz przyjmując T-równoważność dla \ref{sil-boch-D1} i~jej odpowiednik dla zdań fałszywych (,,Zdanie \ref{sil-boch-D1} jest fałszywe wtedy i~tylko wtedy, gdy Bóg nie jest intelektem''), otrzymamy:
%D2''Zdanie \ref{sil-boch-D1} jest bezsensowne wtedy i~tylko wtedy, gdy nie jest tak, że Bóg jest intelektem i~nie jest tak, że Bóg nie jest intelektem.
\begin{flalign}
& \parbox[t]{.87\linewidth}{\strut Zdanie \ref{sil-boch-D1} jest bezsensowne wtedy i~tylko wtedy, gdy nie jest tak, że Bóg jest intelektem i~nie jest tak, że Bóg nie jest intelektem.\strut} &&\tag{D2''}\label{sil-boch-D2bis}
\end{flalign}
Powyższa równoważność należy do metajęzyka. Oczywiście, zawiera ona wyrażenie ,,Bóg jest intelektem'' równokształtne z~\ref{sil-boch-D1}, ale wyrażenie to stanowi tylko tłumaczenie \ref{sil-boch-D1} na metajęzyk. Trudno nie zauważyć, że równoważność ta prowadzi do sprzeczności. Warunkiem bezsensowności zdania \ref{sil-boch-D1} jest jednoczesna prawdziwość dwóch metajęzykowych zdań, z~których jedno jest negacją drugiego. Z~perspektywy przedstawianego tu argumentu istotniejsze jest jednak to, że w~wyniku analiz zdanie \ref{sil-boch-D2} okazało się kolejną (metajęzykową) wypowiedzią o~Bogu. A~zatem, z~punktu widzenia Dionizyjskiej\index[names]{Pseudo-Dionizy Areopagita} teologii milczenia musimy uznać, że jest bezsensowne.
%D3 Zdanie \ref{sil-boch-D2} jest bezsensowne.
\begin{flalign}
& \text{Zdanie \ref{sil-boch-D2} jest bezsensowne.} &&\tag{D3}\label{sil-boch-D3}
\end{flalign}
Otwiera to drogę do nieskończonej hierarchii języków, w~których kolejno stwierdzamy, że zdania o~Bogu z~niższego poziomu są bezsensowne. Zdaniem Brożka\index[names]{Brożek, Bartosz} taka intuicja stoi za tą interpretacją teologii apofatycznej, która każe twierdzić, że o~Bogu nie można mówić w~żadnym języku.

\begin{quote}
Niezależnie od tego, jak bogaty język skonstruujemy, nie będzie on wystarczającym narzędziem do opisu Transcendencji -- zawsze będziemy zmuszeni przenieść się ,,poziom wyżej'', gdzie przeżyjemy \textit{déjà vu}, znów okaże się, że język, którym dysponujemy, jest niewydolny\footnote{Tamże.}.
\end{quote}

Wydaje się, że Bocheński\index[names]{Bocheński, Józef Maria} próbował uniknąć tego progresu w~nieskończoność formułując swoją zasadę teologii milczenia w~postaci \ref{sil-boch-MNT}. Problem w~tym, że teorię zbudowaną na tej zasadzie bynajmniej nie można nazwać konsekwentnym apofatyzmem. Co prawda, w~oczach Bocheńskiego\index[names]{Bocheński, Józef Maria} jest ona bardziej konsekwentna od teologii negatywnej \textit{tout court} i~dostatecznie ekstremalna -- pozbawia dyskurs religijny jakiegokolwiek znaczenia i~generuje tzw. ,,zewnętrzne'' paradoksy. Jednakże teolog apofatyczny z~pewnością z~większym entuzjazmem przyjmie powyższą konstatację Brożka\index[names]{Brożek, Bartosz}, niż zadowoli się modelem Bocheńskiego\index[names]{Bocheński, Józef Maria}, który w~jego ocenie może mieć nieco zachowawczy charakter. Inaczej mówiąc, teolog apofatyczny doceni obecność dużego kwantyfikatora w~sformułowaniu \ref{sil-boch-MNT}, wolałby jednak, by ten kwantyfikator swoim zasięgiem obejmował \textit{wszystkie} języki, nie tylko języki przedmiotowe. Za takim postawieniem sprawy przemawiają także inne, filozoficzno-językowe argumenty związane ze strukturą teorii budowanych analogicznie do klasycznej koncepcji prawdy Tarskiego\index[names]{Tarski, Alfred}, o~których poniżej. Tymczasem spróbujmy jeszcze bliżej przyjrzeć się modelowi zaproponowanemu przez Bocheńskiego\index[names]{Bocheński, Józef Maria}.

W~koncepcji Bocheńskiego\index[names]{Bocheński, Józef Maria} \ref{sil-boch-MNT} jest metajęzykowym zdaniem mówiącym o~niewysławialności pewnego obiektu $g$ w~klasie języków przedmiotowych. Dla ułatwienia, załóżmy, że nie mamy do czynienia z~całą klasą języków przedmiotowych, lecz z~jednym takim językiem. Zresztą, nie jest to kontrowersyjne posunięcie, a~mówienie o~,,języku szachów'' czy innych tego typu językach, które z~pewną dozą dodatkowych założeń można nazwać językami przedmiotowymi, trochę zaciemnia obraz. Jeśli język przedmiotowy oznaczymy symbolem $\mathcal{L}_0$, metajęzyk możemy oznaczyć symbolem $\mathcal{L}_1$, metametajęzyk symbolem $\mathcal{L}_2$ itd. Zauważmy, że w~języku $\mathcal{L}_{n+1}$ mamy prawo mówić o~wszystkim, o~czym mówiliśmy w~języku $\mathcal{L}_{m\leq n}$, $\mathcal{L}_{n+1}$ zawiera jednak dodatkowe terminy, które pozwalają nam wypowiadać zdania dotyczące języków o~indeksach $m \leq n$. Na przykład, zgodnie z~koncepcją Tarskiego\index[names]{Tarski, Alfred}, język $\mathcal{L}_{n+1}$ będzie zawierał predykat ,,jest prawdziwy w~$\mathcal{L}_{n}$'', w~języku $\mathcal{L}_{n+2}$ dostępny jest predykat ,,jest prawdziwy w~języku $\mathcal{L}_{n+1}$'' itd. Konsekwencją tego jest fakt, że nie istnieje jedna definicja prawdy -- każda jest zrelatywizowana do danego języka i~może zostać wyrażona tylko w~języku wyższego poziomu\footnote{Stąd w~klasycznej teorii prawdy Tarskiego\index[names]{Tarski, Alfred} T-równoważności nazywa się ,,cząstkowymi'' definicjami.}. Podobnie zachowuje się predykat ,,jest bezsensowny w~$\mathcal{L}_{n}$'' w~modelu Brożka\index[names]{Brożek, Bartosz} -- nie mówi się tu o~bezsensowności zdania w~ogóle, lecz o~bezsensowności w~danym języku niższego poziomu. Wydaje się więc słusznym uznać, że predykat ,,jest niewysławialny'' nie jest predykatem dwuargumentowym, lecz predykatem jednoargumentowym zrelatywizowanym do konkretnego języka. Będziemy zatem mówić, że coś jest ,,niewysławialne w~$\mathcal{L}_{0}$'', ,,niewysławialne w~$\mathcal{L}_{1}$'', czy ,,niewysławialne w~$\mathcal{L}_{n}$'', co oznaczymy za pomocą litery predykatowej z~odpowiednim indeksem. Przy takim założeniu model teorii Niewysłowionego przybierze postać
%MNT2Nw\_\{L0\}(g). \label{sil-boch-MNT2}
\begin{flalign*}
&N_{w\mathcal{L}_{0}}(g).&\tag{MNT2}\label{sil-boch-MNT2}
\end{flalign*}
Zdanie to, oczywiście, pozostaje zdaniem metajęzykowym (czyli wypowiedzianym w~języku $\mathcal{L}_{1}$). Wydaje się więc, że model Bocheńskiego\index[names]{Bocheński, Józef Maria} nie narusza zasad wyznaczonych przez Tarskiego\index[names]{Tarski, Alfred} -- przyjmuje stratyfikację języka i~nie miesza jego poziomów. Dokładniej mówiąc, ogranicza pojęcie niewysławialności do niewysławialności w~danym języku, mówiąc o~tym w~języku wyższego poziomu. Wciąż jednak pozostawia wiele wątpliwości. Są one dwojakiego rodzaju. Po pierwsze, w~przeciwieństwie do koncepcji Tarskiego\index[names]{Tarski, Alfred} i~Brożka\index[names]{Brożek, Bartosz}, pojęcie, które u~Bocheńskiego wzbogaca każdy kolejny język -- ,,niewysławialny w~$\mathcal{L}_{n}$'', nie dotyczy zdań, lecz obiektów. Z~tego powodu analiza tego modelu staje się kłopotliwa z~logicznego punktu widzenia, bo to zdania, nie obiekty, są nośnikami sensu. Po drugie, z~tego samego powodu nie wiadomo, jak miałaby wyglądać cząstkowa definicja niewysławialności, czyli jakiś odpowiednik T-równoważności dla tej teorii.
%MNT2' Nw \mathcal{L}_{0}(g) wtw A. \label{sil-boch-MNT2bis}.
\begin{flalign*}
&N_{w\mathcal{L}_{0}}(g) \equiv A.&&\tag{MNT2'}\label{sil-boch-MNT2prim}
\end{flalign*}
Inaczej mówiąc, nie do końca wiadomo, jakie wyrażenie, formuła lub zdanie miałyby znaleźć się na poziomie języka przedmiotowego $\mathcal{L}_{0}$ i~motywować przejście na poziom metajęzyka $\mathcal{L}_{1}$, by na nim wypowiedzieć \ref{sil-boch-MNT2}\footnote{Ta i~poniższe uwagi dotyczą, rzecz jasna, także \ref{sil-boch-MNT} i~wszystkich innych wersji tej zasady.}. Być może adwokat modelu Bocheńskiego\index[names]{Bocheński, Józef Maria} z~radością przyjmie taką uwagę -- jest to pożądany efekt, w~końcu \ref{sil-boch-MNT2} mówi o~tym, że Bóg jest niewysławialny. W~rzeczywistości jednak nie ucieknie on od tego problemu i~będzie musiał się z~nim zmierzyć w~odpowiedzi na drugi zarzut. Mianowicie, nie jest jasne, jak ma wyglądać (cząstkowa) definicja niewysławialności. Przyjrzyjmy się jeszcze raz rozwiązaniom obecnym w~teoriach Tarskiego\index[names]{Tarski, Alfred} i~Brożka\index[names]{Brożek, Bartosz}. Częstkowa definicja prawdy mówi, że
%T2''Zadanie ,,Śnieg jest biały'' jest prawdziwe wtedy i~tylko wtedy, gdy śnieg jest biały.
\begin{flalign}
& \parbox[t]{.89\linewidth}{\strut Zdanie ,,Śnieg jest biały'' jest prawdziwe wtedy i~tylko wtedy, gdy śnieg jest biały.\strut} &&\tag{T2''}\label{sil-boch-T2bis}
\end{flalign}
Cząstkowa definicja bezsensowności ma postać
%D2'''Zdanie ,,Bóg jest intelektem'' jest bezsensowne wtedy i~tylko wtedy, gdy nie jest tak, że Bóg jest intelektem i~nie jest tak, że Bóg nie jest intelektem.
\begin{flalign}
& \parbox[t]{.88\linewidth}{\strut Zdanie ,,Bóg jest intelektem'' jest bezsensowne wtedy i~tylko wtedy, gdy nie jest tak, że Bóg jest intelektem i~nie jest tak, że Bóg nie jest intelektem.\strut} &&\tag{D2{'}{'}{'}}\label{sil-boch-D2ter}
\end{flalign}
Prawe człony tych równoważności zawierają tłumaczenia zdań z~języka przedmiotowego na metajęzyk lub metajęzykową eksplikację definiowanego pojęcia w~terminach tych zdań. Co więc powinniśmy podstawić za $A$~w \ref{sil-boch-MNT2prim}? Jedną z~propozycji na odpowiednik \mbox{T-równoważności} dla niewysławialności może być pewien rodzaj metajęzykowej tożsamości. W~takim wypadku za $A$~powinniśmy podstawić
%A'Nw\_\{\mathcal{L}_{0}\}(g).
\begin{flalign*}
&N_{w\mathcal{L}_{0}}(g).&&\tag{$A$'}\label{sil-boch-Aprim}
\end{flalign*}
Możemy też prześledzić, jak sam Bocheński\index[names]{Bocheński, Józef Maria} rozumie niewysławialność i~spróbować przedstawić jakąś definicję na bazie jego rozważań. A~twierdzi on często, że w~teorii Niewysławialnego Bóg pozbawiony jest wszelkich własności. Alternatywę dla prawego członu \ref{sil-boch-MNT2prim} mogłoby więc stanowić pewne zdanie wyrażone w~logice drugiego rzędu o~postaci
%$$A''\neg \exists Q \mathcal{L}_{0} Q \mathcal{L}_{0}(g).$$
\begin{flalign*}
&\neg \exists Q_{\mathcal{L}_{0}}\ Q_{\mathcal{L}_{0}}(g).&&\tag{$A$''}\label{sil-boch-Abis}
\end{flalign*}


Niezależnie od naszego wyboru należy uznać, że zdania te albo są przekładem z~języka przedmiotowego (co oznaczałoby, że posiadają swoje równokształtne odpowiedniki w~$\mathcal{L}_{0}$), albo stanowią \textit{definiens} pojęcia niewysławialności w~terminach zdań, formuł lub wyrażeń pochodzących z~$\mathcal{L}_{0}$. W~konsekwencji, w~zależności od tego, jaki kształt przybierze wypowiedź z~języka przedmiotowego tłumaczona w~prawym członie \ref{sil-boch-MNT2prim}\ na metajęzyk, albo ponownie mieszamy poziomy języka i~wracamy do paradoksu Niewyrażalnego na poziomie języka $\mathcal{L}_{0}$, albo dopuszczamy, by język ten zawierał wypowiedzi bezsensowne i~nieniosące znaczenia.

Do modelu Bocheńskiego\index[names]{Bocheński, Józef Maria} można wysuwać dodatkowe, pozalogiczne zastrzeżenia. Dotyczą one przede wszystkim zagadnienia niewysławialności i~postulowania (choćby potencjalnego) istnienia niewysławialnych obiektów. To, czy takie obiekty istnieją, jest przedmiotem badań filozoficznych, a~formuła \eqref{sil-boch-prenw} bynajmniej nie zostałaby jednomyślnie uznana za prawdziwą wśród metafizyków. Przeciw takiemu założeniu wystąpiliby choćby zwolennicy stosunkowo popularnej a~wyjątkowo zachowawczej w~kwestii nadmiarowego poszerzania ,,uniwersum dyskursu'' teorii zobowiązań ontologicznych Willarda Van Ormana Quine'a\index[names]{Quine, Willard V.O.}\footnote{Zob. W.V.O. Quine, \textit{O~tym, co istnieje}, [w:] tenże, \textit{Z~punktu widzenia logiki}, tłum. B. Stanosz, Aletheia, Warszawa 2000. Zob. także B.~Brożek, A. Olszewski, \textit{Kilka uwag o~kryterium Quine'a}, ,,Filozofia Nauki'', vol 18 (2010), nr 1, ss.~5-15 oraz K. Wójtowicz, \textit{O~pojęciu ,,zobowiązania ontologicznego}'', ,,Przegląd Filozoficzny — Nowa Seria'', r.~X (2001), nr 1(37), ss.~121-138.}. Można też sądzić, że postulowanie istnienia obiektów, o~których nie można nic powiedzieć, stoi na opak z~wielowiekową regułą zwaną brzytwą Ockhama\index[names]{Ockham, William}\footnote{Por. A. Baker, \textit{Simplicity}, [w:]~\textit{The Stanford Encyclopedia of Philosophy},~wyd. zima 2016, red. E.N. Zalta, {\textless}\url{https://plato.stanford.edu/archives/win2016/entries/simplicity/}{\textgreater}.}.

Zresztą przykład podawany przez Bocheńskiego\index[names]{Bocheński, Józef Maria} w~celu przekonania czytelnika co do prawdziwości \eqref{sil-boch-prenw}, z~dzisiejszego punktu widzenia wydaje się dosyć niefortunny. Bocheński zapewnia, że ,,rzecz całkiem oczywista, że krowa jest niewysłowiona w~języku szachów, tzn. że w~tym języku nie da się o~niej nic powiedzieć''. Jakkolwiek Bocheński\index[names]{Bocheński, Józef Maria} rozumie ,,język szachów'', motywacja stojąca za wykorzystaniem go w~przykładzie jest oczywista -- Bocheński\index[names]{Bocheński, Józef Maria} chce zestawić ze sobą dosyć prostą syntaktykę z~abstrakcyjnym, symbolicznym alfabetem i~ograniczonym zestawem reguł oraz losowy obiekt ze świata rzeczywistego i~przekonać nas do niemocy tego typu języka w~próbach opisu takich obiektów. Tymczasem najnowsza historia pokazuje, że mając do dyspozycji maszyny wyposażone w~prymitywny, abstrakcyjny alfabet składający się tylko z~dwóch symboli -- \{0, 1\} -- i~niewielki zestaw reguł manipulujących tymi symbolami, możemy wyrazić \textit{quicquidlibet} i~rozmawiać \textit{de omni re scibili}.

Nie do końca klarowne są natomiast motywacje, dla których Bocheński\index[names]{Bocheński, Józef Maria} swój model buduje wokół pojęcia niewysławialności, a~nie ontologicznie prostszej, ,,pozytywnej'' własności wysławialności (opatrzonej w~stosownych miejscach negacją). Nie jest to może najcięższy z~zarzutów przeciwko przedstawionej powyżej interpretacji Bocheńskiego\index[names]{Bocheński, Józef Maria} (o ile w~ogóle można to traktować jako zarzut). Jednakże uwaga ta jest warta odnotowania o~tyle, o~ile w~innych miejscach swojej pracy przykłada on wagę do takiego rozróżnienia i~opiera o~nie obronę teologii negatywnej przed paradoksem Niewyrażalnego.

Tak czy owak, to stratyfikacja i~tworzenie hierarchii języków wywołuje największe opory przed przyjęciem rozwiązań w~stylu klasycznej teorii prawdy. Język naturalny, w~którym formułowana jest teologia apofatyczna, nie zawiera jakiegoś eksplicytnego rozwarstwienia poziomów i~na próżno wyróżniać i~wskazywać w~nim język przedmiotowy, metajęzyk itd. Posługując się terminologią Tarskiego\index[names]{Tarski, Alfred}, język naturalny jest semantycznie zamknięty, czyli zawiera już wszystkie pojęcia semantyczne, które go opisują. To kolejny powód, dla którego teolog apofatyczny nie przychylałby się do ograniczenia mówienia o~Bogu wyłącznie do języków przedmiotowych. Jeśli język ma jakieś inne poziomy, nie ma żadnych powodów, by dopuścić do tego, by Bóg był na tych poziomach opisywalny. W~każdym razie, hierarchia poziomów języka nie jest częścią zwykłego dyskursu. Z~tego powodu wprowadzanie jej do formalnych ustaleń w~celu ominięcia paradoksów semantycznych uznawane jest często za zbyt drastyczne i~przesadne podejście oraz za rozwiązanie \textit{ad hoc}.

Poza tym, hierarchia języków wprowadza dodatkowe techniczne problemy, które nie występują, gdy mamy do czynienia z~językiem semantycznie zamkniętym. Nawet jeśli w~wyniku metajęzykowego zabiegu udaje nam się pozbyć paradoksu Niewyrażalnego, to razem z~nim usuwamy również wiele nieparadoksalnych przypadków samoodniesienia. Rozważmy na przykład następującą parę zdań\footnote{Poniższy argument jest trawestacją przykładu pochodzącego od Kripkego\index[names]{Kripke, Saul}. Por. S. Kripke, \textit{Outline of a~Theory of Truth}, ,,The Journal of Philosophy'', vol. 72 (1975), ss.~690-716.}:
%(R)Żaden obiekt nie jest wyrażalny w~języku dyskursu świeckiego.\label{sil-boch-relig}
%(Ś)Większość obiektów jest niewyrażalna w~języku dyskursu religijnego.\label{sil-boch-swiec}
\begin{flalign*}
&\text{Wszystkie zdania języka dyskursu świeckiego są fałszywe.}&\tag*{(R)}\label{sil-boch-relig}\\
&\text{Większość zdań języka dyskursu religijnego jest fałszywa.}&\tag*{(Ś)}\label{sil-boch-swiec}
\end{flalign*}

Załóżmy, że zdanie \ref{sil-boch-relig} zostało wypowiedziane w~języku dyskursu religijnego, a~zdanie \ref{sil-boch-swiec} w~języku dyskursu świeckiego. Które jest nich jest wyżej w~hierarchii języków? Przyjmując rozwiązanie Tarskiego\index[names]{Tarski, Alfred} należałoby uznać, że zdanie \ref{sil-boch-relig} jest na wyższym poziomie niż wszystkie wypowiedzi języka świeckiego i, odwrotnie, zdanie \ref{sil-boch-swiec} musi być na wyższym poziomie niż wszystkie wypowiedzi języka religijnego. Ponieważ \ref{sil-boch-relig} jest wypowiedzią w~języku religijnym, a~\ref{sil-boch-swiec} jest wypowiedzią w~języku świeckim, \ref{sil-boch-relig} musiałoby być na wyższym poziomie niż \ref{sil-boch-swiec}, a~\ref{sil-boch-swiec} na wyższym niż \ref{sil-boch-relig}. To oczywiście jest niemożliwe, więc w~teoriach wykorzystujących metajęzykowy zabieg w~stylu Tarskiego\index[names]{Tarski, Alfred} zdania te są uważane za błędnie skonstruowane. \ref{sil-boch-relig} i~\ref{sil-boch-swiec} są w~rzeczywistości pośrednio autoreferencyjne, ponieważ \ref{sil-boch-relig} odwołuje się do całości wypowiedzi w~języku świeckim,
%w~tym do \ref{sil-boch-swiec},
a~\ref{sil-boch-swiec} odwołuje się do większości zdań języka religijnego.
%, w~tym do \ref{sil-boch-relig}.
Niemniej jednak w~większości przypadków \ref{sil-boch-relig} i~\ref{sil-boch-swiec} są nieszkodliwe i~nie pociągają sprzeczności.

Z~powodu tych problemów we~współczesnej logice proponuje się inne, nowsze rozwiązania pomagające uniknąć paradoksów semantycznych (czy też, mówiąc bardziej ogólnie, paradoksów samozwrotności). Niemniej, przykłady modeli Bocheńskiego\index[names]{Bocheński, Józef Maria} i~Brożka\index[names]{Brożek, Bartosz} pokazują, że rozwiązania zaczerpnięte z~klasycznej teorii prawdy mogą być owocne na polu logicznej analizy teologii milczenia. To, który z~tych modeli zostanie wybrany za bardziej trafny, zależeć będzie od filozoficznego temperamentu i~apofatycznego zacięcia. Model Brożka\index[names]{Brożek, Bartosz} nie pomaga uniknąć paradoksów. Wręcz przeciwnie -- buduje nieskończoną hierarchię języków, w~której sprzeczność pojawia się na każdym z~językowych poziomów. Model Bocheńskiego\index[names]{Bocheński, Józef Maria} pozostawia niejasności co do rozumienia centralnego pojęcia tej teorii -- niewysławialności, prowadzając do podejrzenia, że sprzeczności nie zostały wcale usunięte, lecz wystarczająco dobrze zakamuflowane. Ponadto, z~teologicznego punktu widzenia, ,,negatywność'' tego rozwiązania pozostawia dużo do życzenia. Co prawda Bóg jest tutaj uznany za niewysławialnego, lecz tylko na podstawowym, przedmiotowym poziomie języka. Na każdym z~wyższych poziomów języka pozostaje on wysławialny. Mimo wszystko, przy dostatecznie dużej dozie dobrej woli można uznać, że próba obrony teologii milczenia przed sprzecznością, którą w~\textit{Logice religii} dokonuje Bocheński\index[names]{Bocheński, Józef Maria}, jest trafna lub chociaż logicznie akceptowalna, przynajmniej w~zakresie ,,wewnętrznego'' paradoksu Niewyrażalnego\footnote{Całą baterię argumentów przeciwko podejściu ,,w stylu Tarskiego\index[names]{Tarski, Alfred}'' do niewysławialności, częściowo pokrywających się z~przedstawionymi w~tym rozdziale, można znaleźć w: A. Kukla, \textit{Ineffability and Philosophy}, Routledge, London -- New York 1998, rozdz.~1.}.



%\part{Aspekt epistemiczny}
\part{Teologia apofatyczna jako teologiczny sceptycyzm}

%\part{Aspekt epistemiczny}

\chapter{Teologia apofatyczna jako teologiczny sceptycyzm}\label{scep}
%\section{Wprowadzenie}


%\chapter{Wprowadzenie}


Kolejna interpretacja teologii negatywnej przenosi apofatyczne akcenty z~mówienia na myślenie i~z języka na wiedzę. Jej centralnym twierdzeniem jest teza, że Bóg jest całkowicie niepoznawalny, niepojęty i~niemożliwy do skonceptualizowania czy zrozumienia (w przeciwieństwie do takich apofatycznych określeń znanych z~rozważań z~zakresu teologii milczenia, jak ,,niewyrażalny'', ,,nieopisywalny'' czy ,,niewysławialny''). W~związku z~tym interpretację tę nazywa się ,,teorią Niepoznawalnego''. W~literaturze można ją spotkać również pod nazwą ,,agnostycznej teologii negatywnej''\footnote{Takiej nazwy używa Paweł Rojek. Zob. P. Rojek, \textit{Logika teologii negatywnej}, ,,Pressje'', 29 (2012), ss.~216-230.}, ale ja wolę określać ją mianem ,,teologicznego sceptycyzmu''\footnote{W~nawiązaniu do klasycznego sceptycyzmu, którego przedstawiciele podważali możliwość prawdziwego poznawania lub nabywania wiedzy w~ogóle.}, ponieważ kwestionuje ona możliwość zdobycia jakiejkolwiek wiedzy o~Bogu. Oczywiście, przez sceptycyzm nie rozumiem żadnej formy niewiary czy wiarołomstwa. Teologowie negatywni, którzy podkreślali niepoznawalność Boga, byli -- jak się wydaje -- głęboko wierzącymi myślicielami religijnymi.

Wysunięcie propozycji takiej interpretacji każe zadawać zaangażowane filozoficznie pytania o~relacje między językiem a~wiedzą i~poznaniem. Podobne pytania otwierają także ciekawe wątki rozważań z~zakresu kognitywistyki i~nauk o~poznaniu. Zasadniczo, w~zależności od przyjętego stanowiska filozoficznego, konsekwencją tych ustaleń będzie wniosek, zgodnie z~którym teoria Niepojmowalnego miałaby wynikać z~teorii Niewysławialnego lub odwrotnie lub też teorie te miałyby być równospójne a~ich zasady równoważne. Jednakże niezależnie od tych ustaleń nie wypada nie zgodzić się z~tezą głoszącą, że teoria Niepojmowalnego stanowi pełnoprawną, odrębną i~dojrzałą interpretację teologii apofatycznej.

***

Także i~w przypadku tej interpretacji można wskazać przykłady podobnego sposobu myślenia wśród dwudziestowiecznych teologów i~filozofów religii nieutożsamianych wprost z~nurtem apofatycznym. Wątki mówiące o~tym, że Bóg przekracza wszystko, co możemy o~nim pomyśleć ponownie znajdziemy u~Rudolfa Oto, Karla Bartha czy Karla Rahnera i~innych wiodących współczesnych teologów. Wydaje się, że z~wymienionej trójki to u~Bartha są one najbardziej widoczne.

Barth w~monumentalnej \textit{Dogmatyce Kościelnej}\footnote{K. Barth, \textit{Church Dogmatics}, tom 2, cz. 1, tłum. T.F. Torrance, G.W. Bromiley, T\&T Clark, Edinburgh 1957, ss.~179-204.} przedstawia alternatywne rozumienie transcendencji, którą pojmuje już nie tylko jako ontologiczne oddzielenie i~rozróżnienie między stworzeniem a~Stwórcą, lecz także jako niezdolność ludzkich pojęć do uchwycenia istoty Boga, choć w~tym kontekście woli on mówić raczej o~,,tajemnicy'' albo ,,ukryciu'' Boga. Według Bartha człowiek siłą własnego rozumu nigdy nie utworzy właściwego pojęcia Boga -- nawet przez analogię do pojęć, które potrafi zrozumieć, takich jak ,,pan'', ,,stwórca'' czy ,,zbawca''. Konsekwencją zaprzeczenia możliwości poznania Boga za pomocą rozumu jest odrzucenie teologii naturalnej jako przedsięwzięcia pozbawionego jakichkolwiek szans na powodzenie\footnote{Zob. tamże.}. Z~punktu widzenia tej pracy interesujący jest fakt, że u~Bartha niepoznawalności Boga nie pociąga za sobą Jego niewysławialności. W~teologii Bartha nasze ludzkie pojęcia i~słowa mogą odnosić się do Boga, o~ile ugruntowane w~objawieniu\footnote{Oznacza to, że w~teologii Bartha możliwe jest jednak zdobycie \textit{jakiejś} wiedzy o~Bogu. Można Go poznać przez podobieństwo i~analogię, choć nie za pomocą naturalnych zdolności poznawczych -- dokonuje się to w~darmowym, niezasłużonym akcie łaski, jakim jest objawianie. Zob. K.E. Johnson, \textit{Divine Transcendence, Religious Pluralism and Barth's Doctrine of God}, ,,International Journal of Systematic Theology'', vol. 5 (2003), nr 2, ss.200-224.}.

***

Przyjęcie, że zasadnicza teza teologii negatywnej głosi, że Bóg przekracza wszystko, co możemy o~nim pomyśleć, stanowi popularne podejście w~interpretacji pism tych samych autorów, o~których wspominaliśmy już w~poprzedniej części pracy: Grzegorza z~Nyssy, Pseudo-Dionizego Areopagity, Tomasza z~Akwinu, Mikołaja z~Kuzy, czy Mistrza Eckharta. Jednakże zwolennicy teologicznego sceptycyzmu odwołują się najchętniej do pism Rambama, czyli Mojżesza Majmonidesa.

Mimo wszystkich różnic, podobieństwa między teologią milczenia i~teorią Niepojmowalnego są wciąż znaczne. Największym z~nich jest wynikający z~samozwrotności paradoksalny charakter, który dzielą obie teorie. Paradoks Niepoznawalnego tworzymy w~ten sam sposób i~na takich samych zasadach, co paradoks Niewysłowionego.


\section{Niewysławialność vs niepoznawalność. Uwaga o~relacjach między językiem a~umysłem w~kontekście teologii negatywnej}\label{scep-werapo}

Na pierwszy rzut oka sceptycyzm teologiczny wydaje się tożsamy z~teologią milczenia (lub bardzo do niej zbliżony). Istnieją pewne powody, aby sądzić, że przynajmniej jedna z~tych teorii pociąga za sobą drugą. To, w~którą stronę przebiegnie taka implikacja, zależeć może od filozoficznych przekonań dotyczących związków i~zależności między językiem a~umysłem, mową a~wiedzą czy myśleniem a~sposobem i~możliwością formułowania i~wyrażania sądów. Niemniej, obie teorie są do siebie -- przynajmniej \textit{prima facie} -- znacząco podobne. Wniosek ten może pojawić się także w~wyniku lektury tekstów źródłowych teologii negatywnej -- rzeczywiście wydaje się, że dla wielu autorów apofatyzm można wyrazić tak samo skutecznie i~trafnie tezą o~niewyrażalności, jak i~tezą o~niepojmowalności. Granica między niemożnością powiedzenia czegokolwiek o~Bogu a~brakiem możliwości posiadania o~nim jakiejkolwiek wiedzy zaciera się choćby u~bohatera poprzedniej części pracy -- Pseudo-Dionizego Areopagity, ale i~u wielu innych myślicieli apofatycznych.

Twierdzę jednak, że obie teorie i~związane z~nimi interpretacje teologii negatywnej nie mogą być równoważne. Po pierwsze z~tego powodu, że teologia milczenia czyni dyskurs religijny bezsensownym, podczas gdy teologiczny sceptycyzm (przynajmniej hipotetycznie) nie niesie takich konsekwencji. Przyjmując teorię Niewysłowionego nie możemy sensownie powiedzieć czegokolwiek o~Bogu. Nie wolno nam ani potwierdzić, ani zaprzeczyć jakiemukolwiek przymiotowi Boga. W~ramach teorii Niepoznawalnego dyskurs religijny może mieć znaczenie, choć ograniczone -- możemy przynajmniej stwierdzić, czego o~nim nie wiemy. Druga uwaga jest bardziej ogólna i~dotyczy wprost różnic i~związków między wiedzą i~poznaniem a~językiem i~mową\footnote{Mam świadomość, że z~jednej strony wiedza, umysł, poznanie i~myślenie, z~drugiej język, mowa, mówienie i~generowanie mowy niekoniecznie muszą być uznane za terminy tożsame i~równoważne. Jednakże w~przybliżeniu, jakie wymagane jest do przeprowadzenia rozważań przedstawianych w~tym rozdziale, są na tyle bliskoznaczne, że będą często stosowane zamiennie (o ile kontekst ich użycia nie wskaże, że jest inaczej).}. Szczegółowe zbadanie tych związków i~różnic nie jest celem niniejszej pracy. Niemniej jednak wydaje się uzasadnionym, by twierdzić, że z~jednej strony możemy myśleć i~mówić o~rzeczach, których nie znamy, a~z drugiej strony możemy rozumieć więcej, niż jesteśmy w~stanie wyrazić słowami. Spróbujmy przyjrzeć się bliżej tym kwestiom.

W~poprzedniej części pracy przedstawialiśmy dwa podejścia do źródeł niewysławialności i~niepojmowalności Boga\footnote{Por. rozdz.~\ref{sil-int-nazw}.}. Zasadniczo większość teologów negatywnych i~badaczy zajmujących się apofatyczną doktryną uważa, że niemożliwość zrozumienia i~poznania Boga wynika z~jego transcendentnego charakteru i~jest jego istotną własnością, elementem jego natury. Istnieją jednak myśliciele przekonani, że powodem niepoznawalności Boga są ograniczenia ludzkiego umysłu i~niewystarczające zdolności poznawcze człowieka, które sprawiają, że nie może on posiąść o~Bogu żadnej autentycznej wiedzy. Można pokusić się o~sformułowanie dwóch wersji takiego stanowiska -- słabszej i~silniejszej. Według pierwszej z~nich ograniczenia ludzkiego umysłu i~języka mogłyby zostać przezwyciężone, natomiast druga głosiłaby tezę przeciwną -- co prawda niewysławialność/niepoznawalność Boga wynika z~ograniczeń ludzkiego języka/umysłu, ale nigdy nie będzie on na tyle doskonały, by wyrazić/poznać naturę Boga. Ta druga, silna wersja niewiele się różni (przynajmniej, gdy idzie o~logiczne i~teologiczne konsekwencje) od podejścia, zgodnie z~którym niepoznawalność Boga należy do jego natury.

By zilustrować tę pierwszą, wyobraźmy sobie, że w~niedalekiej przyszłości rodzi się religijny geniusz, który potrafi poznać i~pojąć w~zupełności nieskończone bogactwo natury Boga. Czy taki religijny geniusz będzie w~stanie należycie sformułować, określić i~opisać tę naturę? Jeśli tak, czy będzie mógł ją odpowiednio zakomunikować innym ludziom? W~końcu, czy taki komunikat miałby szansę, by u~swoich odbiorców wywołać poprawny obraz Boga w~taki sposób, by i~oni poznali tę naturę w~pełni?

W~innym hipotetycznym scenariuszu wyobraźmy sobie, że w~dalekiej przyszłości ludzkość na drodze rozwoju pokonała swoje ograniczenia i~wykształciła umiejętności poznawcze do tego stopnia, że każdy człowiek -- wskutek długotrwałej medytacji oraz czytania mistycznych ksiąg -- jest w~stanie poznać i~pojąć w~całości bezkresną naturę Boga. Czy ludzie, którzy w~medytacji poznali tę naturę, mogą rozmawiać o~Bogu w~sposób sensowny i~znaczący? Czy, gdy dwaj przedstawiciele takiego społeczeństwa przyszłości będą wypowiadać słowo ,,Bóg'', w~ich umysłach aktywizują się tożsame reprezentacje mentalne?


\subsection{Problem niekomunikowalności języka religijnego}\label{scep-nkom}

Powyższe eksperymenty myślowe mają stanowić ilustracje różnych problemów dotyczących zależności między językiem a~umysłem w~kontekście teologii negatywnej. Nie zamierzam udzielać definitywnych odpowiedzi na pytania, jakie się pojawiają w~ich obrębie, choć temat ten zostanie jeszcze podjęty. Tymczasem zwróćmy uwagę na jeszcze jedno zagadnienie, na które wskazują te ilustracje -- problem (nie)komunikowalności znaczenia w~dyskursie religijnym. Generalnie jest to problem z~zakresu pragmatyki i~trzeba przyznać, że -- nawet w~jej obrębie, bez kontekstu religijnego -- nie należy on do najchętniej podejmowanych zagadnień. Jednym z~autorów, którzy wprost próbowali przedstawić i~zbadać je w~formalny sposób, jest Józef Maria Bocheński. Poświęca on temu problemowi krótki paragraf w~\textit{Logice religii}\footnote{J.M. Bocheński, \textit{Logika religii}, tłum. S. Magala, Instytut wydawniczy PAX, Warszawa 1990, §12.}.

W~rozważaniach Bocheńskiego teoria niekomunikowalności dyskursu religijnego to teoria, na gruncie której uważa się, że wyrażenia języka religijnego niosą jakieś obiektywne znaczenie, ale nie da się tego znaczenia przekazać -- nie można go zakomunikować innym użytkownikom dyskursu religijnego. By wskazać jakieś inne, pozareligijne warunki, w~których taka sytuacja może mieć miejsce, Bocheński podaje przykład więźnia, który w~ramach zachowania wspomnień, rozmawia sam ze sobą w~języku, którego nie zna żaden ze współwięźniów znajdujących się z~nim w~celi.

Główna zasada takiej teorii w~rekonstrukcji Bocheńskiego zawiera dwa trójargumentowe predykaty skonstruowane w~obrębie pragmatycznego układu odniesienia: $U(x,t,\phi)$ oznaczający ,,$x$ używa terminu $t$~w znaczeniu $\phi$'' oraz $M(t, \phi, x)$ wyrażający funkcję zdaniową ,,termin $t$~znaczy $\phi$ dla osobnika $x$''. Zasada ta przedstawiona jest zdaniem:
\begin{flalign}
		& \forall x \forall t \big(U(x,t,\phi) \to \forall y (M(t, \phi, y) \to y=x)\big). &\label{scep-nkom-form}
\end{flalign}

W~radykalnej interpretacji teoria niekomunikowalności Bocheńskiego głosi, że gdy dwie osoby, $a$ i~$b$, posługują się językiem religijnym, $a$ nie rozumie wypowiedzi $b$ i~\textit{vice versa} -- $b$ nie wie, co $a$ ma na myśli. Inaczej mówiąc, dyskurs religijny jest pozbawiony znaczenia dla odbiorcy, choć jednocześnie całkowicie znaczący dla nadawcy komunikatu. Bocheński odrzuca tę teorię z~przyczyn empirycznych -- to, co wiemy o~zachowaniu wiernych temu przeczy. Wierni zdają się rozumieć wypowiedzi religijne i~podobnie na nie reagować. Jednakże za jakąś jej wersją mogą przemawiać pewne podejścia, zgodnie z~którymi wiedza o~Bogu jest wiedzą osobistą i~wiedzą ,,o osobie'' i~jako taka pozbawiona jest twierdzeniowego (propozycjonalnego) charakteru\footnote{Próby badania ,,wiedzy o~osobie'', czyli właściwie rozwinięcia alternatywnej epistemologii, odmiennej od tradycyjnych studiów nad wiedzą o~obiektach lub zdaniach, pojawiają się od jakiegoś czasu w~literaturze. Por. M.A. Benton, \textit{Epistemology Personalized}, ,,The Philosophical Quarterly'' 67 (2017), nr 269, ss.~813–834; w~kontekście wiedzy o~Bogu: M.A. Benton, \textit{God and Interpersonal Knowledge}, ,,Res Philosophica'', vol. 95 (2018), nr 3, ss.~421-447 oraz L.J. Keller, \textit{Divine ineffability and franciscan knowledge}, ,,Res Philosophica'', vol. 95 (2018), nr 3, ss.~347-370.}. Tak czy owak, Bocheński odrzuca teorię niekomunikowalności powołując się na obserwacje zachowań użytkowników języka religijnego. Tak się składa, że omawiany tu problem najpełniejsze odzwierciedlenie w~kontekście filozoficznym (a przynajmniej pozareligijnym) znajduje właśnie na gruncie lingwistyki behawioralnej, a~konkretnie tezy o~niezdeterminowaniu przekładu Willarda Van Ormana Quine'a\footnote{Zob. W.V.O. Quine, \textit{Three Indeterminacies}, [w:] \textit{Perspectives on Quine}, red. R. Barret, R. Gibson, Blackwell, Basil 1990, ss 1-16 oraz Tenże, \textit{Indeterminacy of Translation Again}, ,,Journal of Philosophy'', vol. 84 (2012), nr 1, 5-10.}.

Punktem wyjścia tezy o~niezdeterminowaniu przekładu jest kolejny eksperyment myślowy\footnote{Quine nazywa ten eksperyment ,,przekładem radykalnym''.}. W~jego ramach wyobraźmy sobie lingwistę, który rozpoczął badania terenowe, by jako pierwszy opracować przekład nieznanego do tej pory języka używanego przez ludzi żyjących przed jego przybyciem w~izolacji. Nazwijmy ten język językiem źródłowym. Dla języka źródłowego nie istnieje jeszcze żaden słownik czy podręcznik -- słowem, nie ma do niego żadnego dostępu, nawet zapośredniczonego w~którymś ze znanych do tej pory języków. Jedyny materiał, jaki może zebrać lingwista tworzący pierwszy podręcznik języka źródłowego, to wypowiedzi jego użytkowników oraz obserwacja towarzyszących tym wypowiedziom okoliczności i~zachowań. Mając takie dane lingwista będzie próbował wyróżnić zbiór zdań obserwacyjnych\footnote{Zdania obserwacyjne to istotny element koncepcji Quine'a. Zob. A. Kubić, ,,\textit{Niezdeterminowanie'' granic poznania w~znaturalizowanym relatywizmie pojęciowym W.V.O. Quine'a}, ,,Annales Universitatis Mariae Curie-Sklodowska, sectio I~-- Philosophia-Sociologia'', vol. 43 (209), nr 2, ss.~73-92}, czyli wypowiedzi typu ,,Pada deszcz'' albo ,,To jest królik''. Na tej samej zasadzie będzie próbował zidentyfikować klasę nazw i~kryjących się za nimi znaczeń. By zweryfikować swoje przypuszczenia nasz lingwista będzie używał tych wypowiedzi w~podobnych okolicznościach, w~których je usłyszał, i~badał reakcje słyszących jego wypowiedzi tubylców. Wszelkie potwierdzenia i~zaprzeczenia a~także spodziewane lub nieoczekiwane reakcje użytkowników języka źródłowego będą utwierdzały naszego badacza w~jego przypuszczeniach albo kazały mu przesunąć znaczenia lub zmienić kontekst testowanych wypowiedzi. Dzięki takiej procedurze będzie on mógł stopniowo poszerzać swój zakres znajomości języka źródłowego o~coraz to nowe, bardziej złożone rodzaje wyrażeń. W~końcu, jego tworzony w~terenie słownik i~podręcznik (zbiór zasad przekładu) przybiorą formę kompletnych opracowań a~on sam będzie mógł całkiem efektywnie komunikować się z~pierwotnymi użytkownikami języka źródłowego.

Quine argumentuje, że taki projekt jest zawsze skazany na niepowodzenie i, choć do znaczenia mamy dostęp jedynie przez obserwację zachowań, stworzony w~taki sposób przekład nigdy nie odda tubylczego obrazu świata czy też -- co dość istotne w~systemie Quine'a -- stojącej za wypowiedzią języka źródłowego ontologii. Załóżmy, że nasz badacz obserwował, jak tubylcy wypowiadają ,,gavagai'' na widok przebiegającego nieopodal królika. Seria podobnych obserwacji, a~także wielokrotne użycie tego wyrażenia w~różnych kontekstach, utwierdziła badacza w~przekonaniu, że wypowiedź ta znaczy po prostu ,,królik''. Tymczasem nie ma żadnej pewności, czy rzeczywistym odniesieniem tego wyrażenia nie jest ,,futrzak'', ,,potencjalna kolacja dla niewielkiej liczny osób'' czy ,,fragmenty przestrzeni wypełnione niedużym zwierzęciem''. Załóżmy dodatkowo, że przed publikacją pierwszego podręcznika języka źródłowego do ludzi nim się posługujących przybył drugi lingwista i~wykonał tę samą pracę korzystając z~tych samych procedur. Nie ma przeszkód, by sądzić, że przekłady dokonane za pomocą tych dwóch różnych zbiorów zasad, choć dadzą poprawne przewidywania dotyczące zachowań użytkowników języka, będą diametralnie różne (a potencjalnie niespójne) jeśli idzie o~ich znaczenie w~języku przekładu. Możemy także założyć, że tekst przełożony na język źródłowy za pomocą pierwszego zbioru zasad, a~potem przetłumaczony z~powrotem na podstawie drugiego podręcznika, będzie diametralnie różny od oryginału\footnote{Wydaje się, że by dojść do takiego wniosku, nie potrzeba \textit{przekładu radykalnego}, wystarczy skorzystać ze współczesnych narzędzi automatycznych służących do tłumaczeń z~rzeczywistych języków.}.

Powyższe rozważania pozwalają uznać argument Bocheńskiego przeciwko teorii niekomunikowalności za bezzasadny. Co prawda w~przypadku teorii niekomunikowalności wypowiedzi dyskursu religijnego formułowane są w~obrębie jednego języka, ale jej konstrukcja sprawia, że sytuacja między dwoma uczestnikami dyskursu religijnego jest tożsama z~tą, w~jakiej się znalazł lingwista z~eksperymentu myślowego Quine'a. Wskazuje na to treść głównej zasady ujętej w~postaci wyrażenia \eqref{scep-nkom-form} -- znaczenie znane jest tylko autorowi wypowiadanego komunikatu, odbiorca nie do niego dostępu. Na pełną analogię z~przekładem radykalnym Quine'a wskazuje również przykład pozareligijnego kontekstu niekomunikowalności. Zatem, skoro mamy do czynienia z~tym samym zjawiskiem i~zgodzimy się z~przekazem Quine'a, do odrzucenia teorii niekomunikowalności nie wystarczy wyłącznie obserwacja zachowań wiernych. Mimo iż behawior wiernych zdaje się sugerować, że rozumieją oni wypowiedzi dyskursu religijnego a~obserwacje ich zachowań prowadzą do wniosków, że odpowiednio reagują na poszczególne wypowiedzi, niekoniecznie musi to świadczyć o~tym, że mają oni pełne rozumienie zdań języka religijnego oraz właściwe pojęcia terminów tego języka.

Oczywiście, argument Bocheńskiego można bronić (a zarazem teorię niekomunikowalności można odrzucać) na wiele różnych sposobów\footnote{Jednym z~wektorów obrony może zostać chociażby przywołanie konsekwencji kolejnego eksperymentu myślowego Hilarego Putnama, zwanego ,,bliźniaczą Ziemią''. Eksperyment ten ma przekonywać, że znaczenia ,,nie są w~głowach'', pokazując przy okazji, że do momentu obserwacji podważającej dotychczasowe rozumienie zdań czy odniesienia terminów, terminy znaczą to, co nam się wydaje, że znaczą. Por rozdz.~\ref{end}.} podobnie, jak na wiele sposobów można argumentować za tezą o~niezdeterminowaniu przekładu w~kontekście dyskursu religijnego. Jednakże prowadzenie takiej dyskusji nie jest to celem niniejszego opracowania. Nic jednak nie stoi na przeszkodzie, by zauważyć, że problem (nie)komunikowalności znaczenia w~języku religijnym, pojawiający się okazji rozważań nad teologią apofatyczną, obnaża kolejne ciekawe związki między językiem a~umysłem. Zgodnie z~zapowiedzią, związkom tym przyjrzymy się dokładniej w~następnej sekcji.

\subsection{Języka a~poznanie}

Znaleźliśmy się w~miejscu, w~którym mamy do dyspozycji dwie interpretacje teologii negatywnej, dwie tworzone w~ich ramach klasy teorii i~dwie towarzyszące im apofatyczne tezy wyrażające transcendencję Boga -- tezę o~niewysławialności (NW) i~tezę o~niepojmowalności (NP). W~ramach podsumowania doczasowych rozważań wyobraźmy sobie cztery teologiczne postawy względem tych tez: 1) solidarnie odrzucenie obu naraz; przyjęcie jednej z~nich z~jednoczesnym odrzuceniem drugiej, czyli 2) akceptację tezy o~niewysławialności i~odrzucenie tezy o~niepojmowalności lub odwrotnie 3) odrzucenie tezy o~niewysławialności i~przyjęcie tezy o~niepojmowalności; 4) przyjęcie obu tez uznając, że Bóg jest niewysławialny i~niepojmowalny.

\begin{enumerate}[label = \arabic*), itemindent=6mm, labelwidth=4mm, labelsep=2mm, itemsep=1em, leftmargin=0mm]
\item $\neg \text{NW} \land \neg \text{NP}$

Jednoczesne odrzucenie tezy o~niewysławialności i~tezy o~niepojmowalności to, oczywiście, podejście redukujące. Sprowadziłoby nas do pozycji teologa katafatycznego, według którego człowiek jest w~stanie pozytywnie wypowiadać się jaki Bóg jest i~rozumieć to, co wypowiada.

\item $\text{NW} \land \neg \text{NP}$

Możemy utrzymywać tezę o~niewysławialności jednocześnie odrzucając tezę o~niepojmowalności. Oznaczałoby to, że uznajemy, że Bóg jest pojmowalny, choć nie możemy go wyrazić i~opisać. Stanowisko to można sprowadzić do teorii niekomunikowalności, którą rozważaliśmy w~sekcji powyżej\footnote{Zob. rozdz.~\ref{scep-nkom}.}.

\item $\neg \text{NW} \land \text{NP}$

Sytuacja, w~której odrzucamy tezę o~niewysławialności a~akceptujemy tezę o~niepojmowalności, także jest logicznie do pomyślenia. Przedstawicielem tego podejścia jest na przykład Karl Barth\footnote{K. Barth, \textit{Church Dogmatics}, tom 2, cz. 1, tłum. T.F. Torrance, G.W. Bromiley, T\&T Clark, Edinburgh 1957, ss.~179-204.}, według którego nie możemy posiadać właściwego pojęcia Boga, ale (dzięki objawieniu) możemy jednak o~nim mówić. Można także wskazać takie wątki u~teologów negatywnych (na przykład u~Mojżesza Majmonidesa, Tomasza z~Akwinu czy Mikołaja z~Kuzy), które wskazywałyby, że rozwijają oni teorię Niepoznawalnego pozbawioną językowego komponentu. Zatem podobną postawę możemy odnaleźć zarówno na gruncie teologii negatywnej, jak u~myślicieli niezaliczanych wprost do nurtu apofatycznego.

\item $\text{NW} \land \text{NP}$

Przyjęcie tezy o~niewysławialności i~tezy o~niepojmowalności naraz prowadzi nas do pełnego apofatyzmu. Takie stanowisko jest najczęściej reprezentowane wśród teologów tego nurtu. Wszystko wskazuje na to, że różnicy między niepojmowalnością a~niewysławialnością nie dostrzegał choćby protagonista poprzedniej części niniejszej pracy, Pseudo-Dionizy Areopagita\footnote{Zob. rozdz.~\ref{sil-dionizy}.}. Wydaje się ono także stać w~zgodzie z~tym, jak teoretycznie i~zdroworozsądkowo traktujemy związki między językiem a~poznaniem.
\end{enumerate}

Dodatkowo, możemy wyróżnić (hipotetycznie) możliwe trzy (cztery) podejścia do inferencyjnych powiązań między NW a~NP: równoważność, wynikanie i~niezależność.

\begin{enumerate}[label = \arabic*), itemindent=6mm, labelwidth=4mm, labelsep=2mm, itemsep=1em, leftmargin=0mm]
\item $\text{NW} \equiv \text{NP}$

Według pierwszego podejścia tezy te są równoważne\footnote{Oczywiście, możemy mówić tylko o~równoważności zdań. W~przypadku teorii mówi się raczej o~równospójności. Dla uproszczenia wywodu pozostaniemy na razie na poziomie tez.}, co oznacza, że przyjęcie jednej z~nich pociąga za sobie natychmiastowo do przyjęcie drugiej i~odwrotnie (a także odrzucenie jednej z~nich musi prowadzić natychmiastowo do odrzucenia drugiej i~odwrotnie). Prawdopodobnie jest to ciche założenie większości teologów negatywnych, dla których różnica między wiedzą a~językiem była transparentna (o~ile można wysunąć takie przypuszczenie bez popadania w~anachronizm).

\item $\text{NW} \rightarrow \text{NP}$

Według drugiego podejścia teza o~niewysławialności pociąga za sobą tezę o~niepojmowalności. (jednocześnie odrzucenie niepojmowalności Boga musi prowadzić do odrzucenia jego niewysławialności). Teologowie apofatyczni przyjmując to stanowisko będą uznawali pojmowanie Boga za bardziej pierwotny fenomen niż zdolności do jego opisu. Zarazem będą twierdzić, że skoro nie można wyrazić Boga, nie ma żadnych możliwości, by posiąść o~nim jakąkolwiek wiedzę.

\item $\text{NW} \leftarrow \text{NP}$

Według trzeciego podejścia teza o~niepojmowalności jest silniejsza i~pociąga za sobą tezę o~niewysławialności (jednocześnie założenie, że Bóg jest wysławialny, prowadzi do uznania, że można Go zrozumieć i~pojąć). Teologowie negatywni sympatyzujący z~tym stanowiskiem będą twierdzić, że to mówienie o~Bogu jest bardziej pierwotnym fenomenem niż posiadanie o~Nim jakiejkolwiek wiedzy. Będą oni jednocześnie twierdzić, że skoro nie można pojąć natury Boga, nie można o~nim także nic powiedzieć.

\item $\neg (\text{NW} \rightarrow \text{NP}) \land \neg (\text{NP} \rightarrow \text{NW})$

Intuicyjnie można sobie wyobrazić sytuację, w~której teza o~niepojmowalności i~teza o~niewysławialności przyjęte są na podstawie odrębnych zbiorów przesłanek w~obrębie dwóch niezależnych, niemających części wspólnych teorii i~nie są one powiązane inferencyjnie, tj. żadna z~nich nie jest konsekwencją drugiej. Jednakże jest to założenie logicznie nie do przyjęcia (zdanie umieszczone powyżej jest kontrtautologią i~-- przynajmniej na gruncie logiki klasycznej -- będzie zawsze fałszywe) oraz bardzo mało prawdopodobne zarówno z~punktu widzenia samej teologii, jaki i~nauk o~poznaniu. Z~tego powodu takie stanowisko nie jest tutaj w~ogóle brane pod uwagę\footnote{Warto zauważyć, że wyróżnione związki inferencyjne mogą wynikać logicznie wprost z~przedstawionych powyżej postaw względem tez o~niewysławialności i~niepojmowalności. Wydaje się jednak, że wynikanie logicznie nie musi oznaczać, że teolog przyjmujący daną postawę koniecznie utrzymuje loginie równoważną z~nią zależność między tymi dwoma tezami. Z~jednej strony ma to związek z~charakterystyką działania umysłu, którą normatywne teorie myślenia i~rozumowania nie potrafią poprawnie uchwycić. Por. P. Urbańczyk, ``\textit{Internal'' Problems of Normative Theories of Thinking and Reasoning}, ,,Zagadnienia Filozoficzne w~Nauce'', 2016, nr 60, ss.~35-52. Z~drugiej strony, wiąże się to z~faktem, że mamy tu do czynienia raczej z~całymi teoriami, niż z~pojedynczymi tezami. Sprowadzenie dwóch apofatycznych doktryn do krótkich, konkretnych tez jest idealizacją mającą na celu ułatwienia przeprowadzenia rozważań.}.
\end{enumerate}

Wybór stanowiska określającego zależności między tezą o~niewysławialności i~tezą o~niepojmowalności z~pewnością nie będzie się dokonywał w~obrębie samej teologii. Musi zależeć przede wszystkim od bardziej ogólnych, filozoficznych upodobań -- objęcia pozycji teoretycznej wyznaczającej relację między językiem a~myśleniem a~także, co istotne ze współczesnego punktu widzenia, wyboru zestawu danych eksperymentalnych faworyzujących interpretacje na korzyść danego rodzaju relacji. W~szczególności (i analogicznie do rozważanych powyżej zależności między niewysławialnością i~niepojmowalnością) można wyróżnić trzy klasy interpretacji relacji między językiem a~poznaniem:

\begin{enumerate}[label = (\arabic*)]
\item  interpretacje, zgodnie z~którymi poznanie kształtuje język,

\item  interpretacje, według których język kształtuje poznanie oraz

\item  stanowiska, na gruncie których interpretacje (1) i~(2) nie są sprzeczne i~się nie wykluczają.
\end{enumerate}

Pierwszy sposób interpretacji to, jak się wydaje, przeważające i~zdroworozsądkowe założenie, zgodnie z~którym człowiek posiada jakiś system poznawczy (wewnętrzny system reprezentacji), który manifestuje się na wiele różnych sposobów, a~jednym z~nich jest język\footnote{Z. Kövecses, \textit{Język, umysł, kultura}, tłum. A. Kowlacze-Pawlik, M. Buchta, Wydawnictwo Universitas, Kraków 2011, s.~473.}. Druga interpretacja znana jest w~literaturze jako hipoteza relatywizmu językowego lub hipoteza Sapira-Whorfa, rzadziej jako ,,whorfianizm''\footnote{Zob. B.C. Scholz, F.J. Pelletier, G.K. Pullum, R. Nefdt, \textit{Philosophy of Linguistics}, [w:] \textit{The Stanford Encyclopedia of Philosophy}, wyd. wiosna 2022, red. E.N. Zalta, <\url{https://plato.stanford.edu/archives/spr2022/entries/linguistics/}>.}. Teorie trzeciej klasy można uzyskać wykazując, że język i~poznanie to przejawy tego samego zjawiska lub procesu, albo wypracowując kompromis\footnote{Współcześnie trzy klasy stanowisk w~obrębie filozofii języka i~w kontekście jego związków z~umysłem dzieli się raczej na stanowiska \textit{eksternalistyczne}, zgodnie z~którymi pierwotnym językowym fenomenem są rzeczywiste wypowiedzi użytkowników języka, \textit{emergentystyczne}, według których podstawowym fenomenem jest komunikacja i~poznanie społecznie oraz \textit{esencjalistyczne}, zgodnie z~którymi bazowy fenomen stanowi z~reguły wrodzona intuicja gramatyki i~znaczenia terminów. Zob. tamże. Jednakże na potrzeby naszych rozważań przedstawiony powyżej podział jest bardziej trafny i~w zupełności wystarczający.} mówiący, że zarówno poznanie kształtuje ekspresję języka, jak i~język wpływa na to, jak myślimy.

Jako przykład takiej koncyliacyjnej interpretacji należącej do trzeciej kategorii można wskazać dziś już trącącą myszką i~dość niszową koncepcję Johna Fodora zwaną hipotezą języka myśli. Zgodnie z~tą hiopotezą myślenie odbywa się w~języku mentalnym, zwanym czasem językiem \textit{myśleńskim}. Myśleński przypomina język mówiony w~wielu aspektach -- na przykład składa się ze słów i~zdań a~znaczenie zdań zależy od znaczenia jego komponentowych składowych. Warto dodać, że od lat siedemdziesiątych ubiegłego stulecia podejścia w~stylu Fodora wiązały się z~najczęściej z~przyjęciem obliczeniowego modelu umysłu. Szczegóły tej koncepcji nie są istotne dla naszych rozważań\footnote{Dobry wgląd w~koncepcję Fodora daje lektura M. Rescorla, \textit{The Language of Thought Hypothesis}, [w:] \textit{The Stanford Encyclopedia of Philosophy}, wyd. lato 2019, red. E.N. Zalta, <\url{https://plato.stanford.edu/archives/sum2019/entries/language-thought/}>.}. Zauważmy jedynie, że głosi ona, że umysł i~język są w~pewnym sensie tożsame -- myślenie, to w~gruncie rzeczy język a~język (myśleński) jest sposobem organizacji i~działania naszego umysłu.

W~rzeczywistości większość współczesnych i~historycznych teorii dotyczących tego, jak działa nasz umysł budowana jest w~zgodzie z~pierwszą klasą interpretacji. Teza mówiąca, że poznanie kształtuje język, stanowi albo milczące założenie takich teorii, albo wyraźnie wyartykułowaną hipotezę roboczą, zgodnie z~którą badanie i~próba opisu systemu poznawczego dokonuje się przez studiowanie języka\footnote{Choć, oczywiście, inne, pozajęzykowe źródła przesłanek do budowania takich teorii również brane są pod uwagę -- zwłaszcza we współczesnych naukach kognitywnych, które więcej niż chętnie sięgają po dane od stowarzyszonej z~nimi psychologii poznawczej.}. Według takiego paradygmatu powstawała chociażby popularna w~poprzedniej dekadzie i~do dzisiaj mająca liczne grono zwolenników hipoteza umysłu ucieleśnionego. Mówi ona, że nasze poznanie kształtowane jest na drodze interakcji, w~jakie wchodzimy z~otoczeniem, poruszając się w~nim i~używając zmysłów. Podstawowe pojęcia tworzone są na bazie naszej orientacji i~działania w~świecie, np. to, co lepsze, umieszczane jest ,,pojęciowo'' \textit{powyżej} albo \textit{na górze} -- to, co gorsze, natomiast \textit{na dole} lub \textit{poniżej}. Bardziej złożone i~abstrakcyjne pojęcia budowane są na bazie tych ,,cielesnych'' i~podstawowych. I~tak, na przykład, \textit{wspinamy się} po szczeblach kariery, ale \textit{popadamy} w~depresję itp. Jeden z~autorów tej idei, George Lakoff, podaje cały szereg tzw. metafor\footnote{Zob. np. G. Lakoff, M. Johnson, \textit{Metafory w~naszym życiu}, Wydawnictwo Aletheia, Warszawa 2010; G. Lakoff, R.E. Núñez, \textit{Where Mathematics Comes From. How the Embodied Mind Brings Mathematics into Being}, Basic Books, New York 2000.}, które w~tej koncepcji stanowią mechanizm poznawczy przenoszący znaczenie z~dziedzin wiążących się z~interakcjami naszych ciał ze środowiskiem na zupełnie nowe dziedziny, dopiero podlegające procesowi rozumienia. Znów -- szczegóły koncepcji ucieleśnienia są irrelewantne dla naszych rozważań. Zauważmy jedynie, że według tej teorii nasze myślenie (ukształtowane przez interakcje ze środowiskiem) wypływa na to, jak mówimy, a~badanie języka i~jego metafor ma nam dawać obraz tego, jak myślimy.

Zupełnie odmienne stanowisko reprezentują zwolennicy hipotezy relatywizmu językowego, zgodnie z~którą język kształtuje nie tylko to, jak opisujemy, lecz także jak postrzegamy i~pojmujemy świat. W~następstwie tego podejścia należy uznać, że umysły osób przynależących do odmiennych społeczności językowych będą się różnić w~takim zakresie, w~jakim różnią się języki tych społeczności. Pomysły te próbowano podeprzeć anegdotycznymi przykładami z~zakresu językoznawstwa, z~których chyba najbardziej popularny dotyczył niezwykle rozbudowanego słownika związanego z~rozpoznawaniem śniegu w~językach Inuitów\footnote{W~różnych pracach podawano różną liczbę terminów z~języków ,,eskimoskich'' mających oznaczać różne rodzaje śniegu -- od 3 do ok. 300. Zob. B. Kortmann, G.K. Pullum, \textit{The Great Eskimo Vocabulary Hoax and Other Irreverent Essays on The Study of Language}, University of Chicago Press, Chicago and London 1991, ss.~159-171.}. Mimo to hipoteza relatywizmu językowego, przynajmniej w~swojej radykalnej wersji, odniosła druzgocącą porażkę i~została odparta pod presją miażdżącej przewagi sprzecznych z~nią interpretacji wyników badań empirycznych. Należy jednak przyznać, że kontrowersje, jakie wokół niej narastały, stymulowały rozwój zarówno teoretycznych, jak i~empirycznych badań nad językiem. Pojawiały się też, zwłaszcza ostatnio, wyniki badań zdające się faworyzować jakąś słabszą wersję tej hipotezy.

Najprawdopodobniej pierwszym poważnym ciosem w~teorie głoszące, że język jest zjawiskiem bardziej pierwotnym i~fundamentalnym niż poznanie, są badania Brenta Berlina i~Paula Kaya\footnote{B. Berlin, P. Kay, \textit{Basic Color Terms: Their Universality and Evolution}, University of California Press, Berkeley~1991.} dotyczące percepcji kolorów. Wynika z~nich, że nazwy kolorów podstawowych mogą opierać się na uniwersalnych aspektach fizjologii widzenia barwnego, a~zatem to raczej postrzeganie determinuje używaną terminologię, a~nie odwrotnie. Inne dane zdają się wskazywać, że takie ,,antyrelatywistyczne'' efekty, które zostały zaobserwowane w~obrębie percepcji i~nazywania kolorów, mogą nie występować w~innych domenach. Znana seria eksperymentów przeprowadzona pod przewodnictwem Lery Boroditsky sugeruje, że język może jednak wpływać na (choć raczej nie determinować) nasze poznanie i~sposób myślenia o~rzeczywistości. Badania Boroditsky dotyczą m.in. różnych koncepcji czasu wśród użytkowników języka mandaryńskiego i~angielskiego\footnote{L. Boroditsky, \textit{Does Language Shape Thought? Mandarin and English Speakers' Conceptions of Time}, ,,Cognitive Psychology'', vol. 43 (2001), ss.~1-22.}, czy wpływu rodzaju gramatycznego rzeczowników na sposób mentalnych reprezentacji obiektów nieożywionych\footnote{L. Boroditsky, L.A. Schmidt, W. Phillips, \textit{Sex, Syntax, and Semantics}, [w:] \textit{Language in Mind: Advances in the Study of Language and Thought}, red. D. Gentner, S. Goldin-Meadow, The MIT Press, Cambridge -- London 2003.}. W~podobnym duchu, choć na gruncie teoretycznym, argumentuje Dan Slobin. Jego koncepcja ,,myślenia dla mówienia''\footnote{D.I. Slobin, \textit{From ``thought and language'' to ``thinking for speaking}'', [w:] \textit{Rethinking linguistic relativity}, red. J.J. Gumperz, S.C. Levinson, Cambridge University Press, Cambridge 1996, ss.~70-96.} przedstawia stosunkowo słaby wpływ języka na poznanie głosząc, że język subiektywizuje i~ukierunkowuje nasze doświadczenie i~kształtuje sposób, w~jaki myślimy, przynajmniej podczas generowania mowy. Za jeszcze bardziej ograniczonym wpływie języka na myślenie opowiada się Ronald Langacker\footnote{R. Langacker, \textit{Gramatyka kognitywna}. Wprowadzenie, Universitas, Kraków 2011.} zauważając, że w~większości języków stosuje się różne formy gramatyczne do wyrażenia tej samej treści poznawczej.

Oczywiście, powyższe krótkie wyliczenie kilku badań i~argumentów pojawiających się po obu stronach sporu o~relatywizm językowy nie rości pretensji do bycia dogłębnym i~rzetelnym opracowaniem. Ma ono na celu pokazanie, że za naszym teologicznym zagadnieniem rozróżnienia interpretacji apofatycznych doktryn może kryć się dojrzały problem filozoficzny mający poważne reperkusje w~empirycznych i~teoretycznych naukach o~poznaniu. W~niniejszej pracy nie chcę jednoznacznie określać prymatu tezy o~niewysławialności i~nad tezą o~niepojmowalności lub \textit{vice versa}. Tym bardziej nie zamierzam rozstrzygać kwestii powiązań między językiem a~umysłem. Mam jednak nadzieję, że mimo to wywód ten pozwala sądzić, że teologia milczenia i~sceptycyzm teologiczny mogą stanowić dwie odrębne interpretacje teologii negatywnej, mające rozróżnialne motywacje filozoficzne i~prowadzące do odmiennych konsekwencji. Z~punktu widzenia niniejszej pracy istotne jest również to, że obie interpretacje, w~celu dokonania ich formalnej rekonstrukcji, domagają się użycia odmiennych środków logicznych. Mimo iż dla wielu reprezentantów teologii negatywnej różnica między tymi doktrynami nie była wystarczająco wyraźna, można wskazać grupę apofatycznych myślicieli, którzy zadawali się świadomie rozwijać teorię Niepoznawalnego abstrahując od jej ewentualnego językowego odpowiednika. Najbardziej znamiennym ich reprezentantem jest Mojżesz ben Majmon, czyli Majmonides.

\section{Mojżesz Majmonides}

Mojżesz Majmonides (zwany także Mojżeszem ben Majmonem a~przez hebrajskojęzyczną część badaczy Rambamem), najprawdopodobniej najwybitniejszy filozof żydowski był niewątpliwie także jednym z~najbardziej radykalnych średniowiecznych przedstawicieli teologii negatywnej. Urodził się w~1138 roku w~znajdującej się wtedy pod islamskim panowaniem hiszpańskiej Kordobie. Jednakże, z~powodu zaostrzenia polityki przeciwko niemuzułmańskim mieszkańcom Hiszpanii, już w~młodzieńczym wieku zmuszony był uciekać do północnej Afryki. Tam powstała większość jego dzieł, z~których najbardziej znaczącym dla przedstawianych w~niniejszej pracy rozważań jest ukończony w~1190 roku \textit{Przewodnik błądzących}\footnote{Majmonides, \textit{The Guide of the Perplexed}, vol. 1-2, tłum. S. Pines, Chicago University Press, Chicago -- London 1963; Polskie tłumaczenie (pierwszej części): Tenże, \textit{Przewodnik błądzących}, dz. cyt. Z~punktu widzenia teologii negatywnej interesujące są zwłaszcza rozdziały 51-60 pierwszej części \textit{Przewodnika}.}, w~którym zawarł m.in. swoją apofatyczną doktrynę. Majmonides był dla judaizmu tym, kim Tomasz z~Akwinu dla chrześcijaństwa -- w~swoich pracach próbował wyjaśniać istotę judaizmu w~sposób zgodny z~myślą i~logiką neoplatoników i~Arystotelesa, których pisma tłumaczone z~greki na język arabski zaczęły przedostawać się do europejskiego kręgu kulturowego. Początkowo prace Rambama były uznawane za kontrowersyjne i~w pewnych ośrodkach rabinicznych zakazywano ich czytania. Szybko jednak zdobyły uznanie, by następnie stać się najbardziej wpływowymi w~historii myśli judaistycznej. Do dziś wśród badaczy myśli żydowskiej funkcjonuje powiedzenie mówiące, że ,,od Mojżesza do Mojżesza nie było nikogo jak Mojżesz''\footnote{K. Seeskin, \textit{Maimonides}, [w:] \textit{The Stanford Encyclopedia of Philosophy}, wyd. wiosna 2021, red. E.N. Zalta, <\url{https://plato.stanford.edu/archives/spr2021/entries/maimonides/}>.}.

Punktem wyjścia apofatycznej doktryny Majmonidesa jest uznanie, że Bóg jest bytem koniecznym, który nie posiada żadnej przyczyny -- ani dla swojego istnienia, ani dla swojej istoty. Jest bytem absolutnie prostym i~nieporównywalnym do czegokolwiek innego, co istnieje. Z~tego powodu nie możemy zrozumieć żadnego pojęcia, gdy jest ono użyte do jego opisu. Każde imię nadane mu w~Biblii, każdy przypisany mu predykat nie jest użyty ani dosłownie, ani metaforycznie, ani nawet analogicznie. Majmonides przekonuje, że użycie to jest jedynie wieloznaczne. Inaczej mówiąc, nawet w~Torze przymioty nadawane są Bogu nie inaczej, jak przez ekwiwokację. Ponieważ w~efekcie stanowią one homonimy, nie wiemy, co tak naprawdę oznaczają (gdy orzekane są o~Bogu). W~rzeczywistości musimy je wszystkie zanegować, aby uchronić nasz nieporadny i~nieświadomy umysł przed niebezpieczeństwem błędu bałwochwalstwa. Ale nawet negacja ,,żadną miarą nie daje poznania prawdziwej rzeczywistości rzeczy, o~której dany sąd jest negowany''\footnote{Majmonides, \textit{The Guide}…, dz. cyt., cz. 1, rozdz.~59.}. Dlatego Bóg pozostaje w~zupełności poza naszym pojmowaniem.

Nie możemy zrozumieć żadnego pojęcia, gdy jest ono użyte do opisania Boga. W~teologii Majmonidesa Bogu nie można przypisać żadnych własności. Jedyne atrybuty, jakie można o~nim ewentualnie orzec, dotyczą jego akcji i~ich konsekwencji (ale nie można twierdzić, że takie atrybuty wyrażają lub opisują jego naturę). Mówiąc bardziej szczegółowo, Majmonides wyróżnia pięć rodzajów predykatów -- dzieli je na takie, które oznaczają (i) definicje, (ii) definicje cząstkowe, (iii) własności (w tym ilość, położenie, dyspozycje itp.), (iv) relacje oraz (v) działania. Pierwsze cztery rodzaje predykatów opisują daną rzecz w~aspekcie jej istoty lub przypadłości. Predykaty opisujące działanie odnoszą się natomiast do wyników lub konsekwencji aktów danego agenta. Podają one opis tego, co agent zrobił, a~nie tego, czym jest. Jak przekonuje Majmonides, prawdziwie orzekane o~Bogu mogą być wyłącznie te predykaty wyrażające działanie (v). Nie dostarczają one jednak żadnej wiedzy o~istocie Boga. Innych rodzajów własności, a~mianowicie (i)-(iv), należy Bogu odmówić\footnote{Por. J.A. Buijs, \textit{The negative theology of Maimonides and Aquinas}, ``Review of metaphysics'', vol. 41 (1988), nr 164, s.~729.}. Orzekanie o~nim takich predykatów stałoby w~sprzeczności z~jego absolutnie prostą naturą.

W~konsekwencji, nie można posiąść żadnego pojęcia Boga ani żadnej o~nim wiedzy. Nie sposób przedstawić ani tzw. klasycznej równościowej definicji Boga (w której \textit{definiens} tworzy się podając rodzaj i~różnicę gatunkową), ani żadnej z~definicji cząstkowych. Możliwość skonstruowania klasycznej definicji Boga oznaczałaby, że istnieje jakiś wyższy od Boga rodzaj, pod który trzeba by było go podporządkować. Bóg, który nie ma żadnych przyczyn i~nad którym nie ma wyższego rodzaju, nie może zostać w~taki sposób zdefiniowany. W~podobny sposób nieuprawniona jest także żadna z~tzw. nieklasycznych równościowych definicji (w której \textit{definiens} to wyliczenie pojęć, które w~sumie składają się na \textit{definiendum}). W~przypadku Boga podanie jakiejkolwiek definicji cząstkowej musiałoby oznaczać, że ma on złożoną naturę i~nie jest absolutnie prosty. Z~tego samego powodu powstrzymujemy się przed przypisywaniem Bogu jakichkolwiek własności czy opisywaniem jego natury\footnote{Por. I. Franck, \textit{Maimonides and Aquinas on man's knowledge of God: A~twentieth century perspective}, ``Review of Metaphysics'', vol. 38 (1985), nr 3, s.~594.}. Wszelkie takie własności są niedoskonałymi ludzkimi pojęciami i~nie mogą reprezentować natury Boga.

Wyraźne inspiracje ideami Mojżesza ben Majmona widoczne są w~pracach najbardziej wpływowego średniowiecznego filozofa -- Tomasza z~Akwinu. To najprawdopodobniej z~tych inspiracji pochodzą najważniejsze oryginalne rozważania Tomasza, zawarte w~\textit{O~bycie i~istocie}\footnote{Tomasz z~Akwinu, \textit{De ente et essentia.} \textit{O~bycie i~istocie}, [wydanie dwujęzyczne] tłum. polskie M. Krąpiec, Redakcja Wydawnictw KUL, Lublin 1981.}
czy w~\textit{Sumie teologicznej}\footnote{Tomasz z~Akwinu, \textit{Suma teologiczna}, tłum. P. Bełch, tom 1-34, Katolicki Ośrodek Wydawniczy ,,Veritas'', Londyn 1962-1986.}
. Od Majmonidesa zdają się pochodzić choćby te Tomaszowe koncepcje, które każą sądzić, że -- choć możemy zrozumieć i~poznać, że Bóg istnieje -- natura czy też istota boska pozostaje zupełnie nieznana (\textit{penitus manet ignotum}) ludzkiemu umysłowi. Choć wydaje się, że apofatyzm Tomasza był słabszy niż Majmonidesa\footnote{Por. J.A. Buijs, \textit{Comments on Maimonides' negative theology}, ,,The New Scholasticism'', vol. 49 (1975), nr 1, ss.~87-93.}, często wymieniany jest on jednym tchem, obok Rambama i~Pseudo-Dionizego, wśród średniowiecznych teologów negatywnych.

Jednym z~powodów, dla którego apofatyzm Majmonidesa jest uważany za bardziej radykalny od Tomaszowego, jest podejście tego pierwszego do kwestii orzekania o~Bogu przez analogię. Majmonides uważa, że absolutne rozróżnienie między Bogiem a~stworzeniem logicznie wyklucza taką możliwość, natomiast Tomasz ją dopuszcza, choć nie bez zastrzeżeń. Odrzucenie orzekania przez analogię może posiadać jedną z~dwóch poniższych motywacji:
\begin{enumerate}[label = (\arabic*)]
\item Aby jakiś atrybut mógł zostać przypisany przez analogię do dwóch różnych bytów, np. do Boga i~człowieka, koniecznym jest, by atrybut ten przypisany tym bytom miał znaczenie tożsame, podobne lub aby jego znaczenia się pokrywały. Nie może być mowy o~analogii, jeśli takie podobieństwo lub nakładanie się na siebie znaczeń nie występuje. Jednakże, skoro nie możemy wiedzieć, co znaczy jakikolwiek predykat, gdy jest orzekany o~Bogu, nie mamy podstaw do przypisania go Bogu na podstawie orzekania przez analogię.

\item Doktryna o~absolutnej i~nieredukowalnej różnicy między Bogiem a~stworzeniem oznacza, że jakikolwiek atrybut orzekany o~Bogu -- gdyby takie przypisanie było możliwe -- byłby całkowicie i~zupełnie różny od predykatu wyrażonego tym samym terminem użytym w~odniesieniu do człowieka, a~zatem żadna analogia w~tym wypadku nie jest możliwa.
\end{enumerate}

Kolejną różnicą miedzy apofatyzmem Rambama i~Tomasza jest ich podejście do tez o~niewysławialności i~niepojmowalności oraz do zależności między nimi. Jak przekonuje Joseph A. Buijs\footnote{Zob. J.A. Buijs, \textit{The negative theology}…, dz. cyt., ss.~734-735.}, Tomasz obie tezy uważał za autonomiczne i~obie wywodził niezależnie ze swojej metafizyki. Jeśliby już szukać u~niego jakichś wskazówek determinujących wynikanie w~którąkolwiek ze stron, należałoby raczej przeprowadzić je, w~kontrapozycji, z~języka do myśli -- skoro pewne predykaty mogą (w sposób niedoskonały, ale prawdziwy) opisywać istotę Boga, możemy posiadać (niedoskonałą, choć prawdziwą) wiedzę o~Bogu. Dla Majmonidesa to epistemiczna forma apofatyzmu była bez wątpienia tą pierwotną i~bardziej podstawową. To ją wywodził ze swojej metafizyki, a~dopiero za jej konsekwencję uznawał semantyczną wersję apofatyzmu w~postaci teologii milczenia -- skoro istota Boga pozostaje dla nas niepozwalana, nie możemy żadną miarą też jej opisać w~sposób znaczący. Z~kolei Ehud Z. Benor idzie o~krok dalej w~interpretacji myśli Majmonidesa argumentując, że średniowieczny filozof odrzuca tezę o~niewyrażalności, rozwijając teorię Niepoznawalnego abstrahując od jej ewentualnych językowych konsekwencji. ,,Nie docenilibyśmy problemu, przed którym stanął Majmonides, gdybyśmy umiejscowili go w~niezdolności języka do wyrażenia trudnych idei metafizycznych''\footnote{E.Z. Benor, Meaning and reference In Maimonides' negative theology, ,,Harvard Theological Review'', vol. 88 (1995), nr 3, ss.~352-353.}. Hilary Putnam ujmuje tę myśl jeszcze dosadniej:

\begin{quote}
[...] nie chodzi tylko o~to, że ktoś czuje (jeśli jest religijny), że nie może właściwie wyrazić tego, co ma na myśli, używając do opisania Boga terminów, których dostarcza nasz język; chodzi o~to, że ktoś czuje, iż nie może mieć na myśli tego, co powinien mieć na myśli\footnote{H. Putnam, \textit{On negative theology}, ,,Faith and Philosophy'', vol. 14 (1997), nr 4, s.~410.}.
\end{quote}

Na koniec warto dodać, że Majmonides dopuszcza tezę, według której natura boska jest w~pewnym sensie poznawalna -- swoją istotę może zrozumieć i~pojąć wyłącznie Bóg. Poświęca on temu zagadnieniu spory fragment pierwszej części \textit{Przewodnika}\footnote{Zob. np. Majmonides, \textit{The Guide}…, dz. cyt., cz. 1, rozdz.~58.}
. Ben Majmon twierdzi, że wiedza boska i~ludzka są heterotypiczne -- są czymś całkowicie innym, mają zupełnie inne charaktery. Dla człowieka próba zrozumienia istoty Boga, byłaby ostatecznie próbą nabycia boskiej wiedzy i~zdolności rozumienia na boskim poziomie. Człowiek, który chciałby zrozumieć Boga, chciałby w~efekcie stać się takim, jak on. Pomijając te konsekwencje rozważań Majmonidesa możemy stwierdzić, że jego doktryna nie stanowi apofatyzmu kompletnego. Nie jest nim o~tyle, o~ile istnieje co najmniej jeden byt, który może pojąć i~zrozumieć boską istotę -- sam Bóg\footnote{Uwagę tę zamieszczam na marginesie zawartych tu rozważań. Wydała mi się interesująca, ponieważ potencjalnie i~od zupełnie innej strony wprowadza do doktryny Majmonidesa kolejne paradoksy samozwrotności.}.

\section{Paradoksalny charakter sceptycyzmu teologicznego}

Pomimo wszystkich różnic między teologią milczenia a~sceptycyzmem teologicznym, ciąży na nich ten sam problem samozwrotności. Obie interpretacje teologii negatywnej są na tyle podobne, by większość argumentów przeciw teorii Niewysławialnego bazujących na wskazaniu jej paradoksalnego charakteru można było przeformułować \textit{mutatis mutandis} przeciw teorii Niepojmowalnego. Zresztą cytaty ilustrujące sprzeczność i~samoodniesienie teologii milczenia przywoływane w~pierwszej części niniejszej pracy odnoszą się tak samo skutecznie, o~ile nawet nie trafniej, do epistemicznego aspektu apofatyzmu\footnote{Por. rozdz.~\ref{sil-int-par}.}. Możemy zatem przedstawić propozycję paradoksu teorii Niepojmowalnego sformułowanego w~sposób, który odpowiadałby paradoksom semantycznym z~poprzedniej części pracy.

\subsubsection{Paradoks Niepoznawalnego}

Zgodnie z~wykładnią teologicznego sceptycyzmu o~Bogu możemy wiedzieć jedynie to, że jest niepoznawalny.

Czy zatem -- zgodnie z~tą interpretacją teologii negatywnej -- Bóg jest czy nie jest niepoznawalny? Jeśli przyjmiemy, że nie jest on niepoznawalny, to nie ma poza tym żadnej innej wiedzy o~Bogu, a~zatem jest on niepoznawalny. Jeśli natomiast założymy, że jest on niepoznawalny, wiemy już coś o~Bogu -- mianowicie to, że jest niepoznawalny, co w~konsekwencji pociąga za sobą sprzeczność. Mianowicie wynika z~tego, że i~Bóg nie jest niepoznawalny wtedy i~tylko wtedy, gdy jest niepoznawalny.

W~ten sam sposób w~kontekście teorii Niepojmowalnego możemy budować także ,,zewnętrzne'' paradoksy teologii negatywnej z~uwzględnieniem wszystkich istniejących różnic między interpretacją zorientowaną na aspekty językowe a~tą, która podkreśla epistemiczny charakter apofatyzmu. Jeśli utrzymujemy, że o~Bogu nie możemy posiadać żadnej wiedzy, w~jaki sposób możemy upierać się, że jest, na przykład, jeden w~trzech osobach lub że jest sędzią sprawiedliwym? Podobnie, jak możemy zwracać się do czego lub przypisywać wartość czemuś, czego nie potrafimy pojąć?\footnote{Por. zewnętrzny twierdzeniowy oraz nietwierdzeniowy (performatywny) paradoks teologii milczenia tamże.} Te rodzaje paradoksów także zachowują ważność w~sceptycyzmie teologicznym.

\subsubsection{Posiadanie i~nabywanie wiedzy jako reprezentant procesów mentalnych w~kontekście teologicznego sceptycyzmu}

Zasadniczo paleta terminów w~apofatycznym słowniku teologicznego sceptycyzmu jest nieco szersza. W~obrębie tej interpretacji teologii negatywnej mówimy, że Bóg jest całkowicie niepoznawalny, niepojęty i~niemożliwy do skonceptualizowania czy zrozumienia, nie można posiąść o~nim żadnej wiedzy. Oczywiście, taki paradoks powinniśmy bez trudu móc skonstruować dla każdego z~wymienionych tu apofatycznych przymiotników\footnote{Nie wszystkie z~wymienionych tu apofatycznych terminów są gramatycznymi przymiotnikami, lecz źródłem tego faktu są \textit{notabene} ograniczenia języka polskiego. Na przykład w~języku angielskim każdą z~tych cech już da się wyrazić taką częścią mowy. Centralnym i~najczęściej stosowanym apofatycznym przymiotnikiem w~kontekście teologicznego sceptycyzmu jest \textit{unknowable}, ale często towarzyszą mu inne, takie jak \textit{inconceivable}, \textit{incomprehensible} czy \textit{inapprehensible}.}. Na przykład argument na rzecz sprzeczności teologicznego sceptycyzmu w~kontekście pojmowalności Boga wyrażony w~języku potocznym jest mniej-więcej następujący: teoria ta stwierdza, że o~Boga nie można pojąć; zrozumieliśmy więc coś o~Bogu -- mianowicie to, że jest niepojmowalny, co w~konsekwencji pociąga za sobą sprzeczność\footnote{W~pełnej wersji paradoks Niepojmowalnego przyjąłby następującą postać: Zgodnie z~wykładnią teologicznego sceptycyzmu jedynym, co możemy zrozumieć o~Bogu, jest idea, wedle której jest on niepojmowalny. Czy zatem -- zgodnie z~tą interpretacją teologii negatywnej -- Bóg jest czy nie jest niepojmowalny? Jeśli przyjmiemy, że nie jest on niepojmowalny, to nie ma niczego innego, co moglibyśmy o~nim pojąć, a~zatem jest niepojmowalny. Jeśli natomiast założymy, że jest on niepojmowalny, to rozumiemy przynajmniej to, że jest niepojmowalny, a~zatem nie jest niepojmowalny. W~konsekwencji Bóg nie jest niepojmowalny wtedy i~tylko wtedy, gdy jest niepojmowalny.}.

W~tym miejscu należy się kolejna uwaga o~różnorodności procesów umysłowych -- podobna do tej, którą przedstawialiśmy na marginesie omawiania zależności między językiem a~poznaniem\footnote{Zob. rozdz.~\ref{scep-werapo}.}. Przy tamtej okazji przekonywałem, że relacje między myśleniem, rozumieniem, nabywaniem i~utrzymywaniem przekonań, wiedzą oraz językiem i~jego użyciem są bardzo złożone. Tym razem zwracam uwagę, że samo myślenie, pojmowanie oraz nabywanie przekonań i~wiedzy mogą być i~są z~pewnością terminami technicznymi nauk kognitywnych, oznaczającymi nieco odmienne zjawiska i~procesy. W~obrębie niniejszej pracy są one jednak traktowane zbiorczo a~terminy oznaczające te zjawiska używane wymiennie. Jednakże za wyróżnione mentalne zjawisko i~proces w~kontekście teologicznego sceptycyzmu uważać będę posiadanie i~nabywanie wiedzy. Zatem zasadniczo, gdy mówimy, że Boga nie można poznać, pojąć, skoneceptualizować czy zrozumieć, mamy na myśli to, że nie da się posiąść o~nim żadnej wiedzy. Pozwalam sobie na taką idealizację z~co najmniej trzech powodów.
Po pierwsze, dla wygody -- pisanie za każdym razem o~niepojętym, niepoznawalnym, niemożliwym do zrozumienia i~skonceptualizowania Bogu z~pewnością ubogaca apofatyczny dyskurs, lecz na dłuższą metę jest niepraktyczne. Po drugie, z~punktu widzenia nauk kognitywnych teza, wedle której pojmowanie, myślenie czy rozumienie sprowadza się do nabywania lub posiadania wiedzy, być może nie jest pozbawiona kontrowersji, ale z~pewnością możliwa do obrony. Na potrzeby rozważań o~epistemicznym aspekcie teologii negatywnej przedstawianych w~niniejszej części pracy takie przybliżenie zjawisk mentalnych, jakie daje nam przyjęta idealizacja, jest w~zupełności wystarczające i~nie będzie wpływać na wyciągane w~ramach tych rozważań wnioski. Po trzecie i~być może najważniejsze, takie postawienie sprawy otwiera nam drogę do użycia dobrze ugruntowanych rachunków logicznych -- logiki epistemicznej i~(ewentualnie) doksastycznej.





\chapter{Epistemiczne paradoksy teologicznego sceptycyzmu}

W~świetle tego, co zostało powiedziane w~poprzednim rozdziale, w~celu formalnej rekonstrukcji sceptycyzmu teologicznego najrozsądniej byłoby skorzystać z~epistemicznej logiki modalnej. Język tego rachunku buduje się dodając do alfabetu symbol operatora $K$~mającego ujmować intuicje stojące za zjawiskiem wiedzy i~poznania. Epistemiczny operator $K$~możemy zatem czytać jako ,,jest znane (przez kogoś w~pewnym czasie), że'', ,,(ktoś) wie, że'', ,,wiadomo, że'' a~nawet, pod pewnymi warunkami, jako ,,jest poznawalne, że'' itp. Dla celów naszych analiz możemy przyjąć, że $K$~oznacza ,,jest znane przez kogokolwiek w~dowolnym czasie, że''. Do zbioru wyrażeń sensownych takiego języka musimy, rzecz jasna, dodać wyrażenia o~postaci $KA$ (o ile $A$~jest już wyrażeniem sensownym języka epistemicznej logiki modalnej).

Wydaje się, że rozsądnie jest przyjąć, iż według wykładni teologicznego sceptycyzmu wszystkie prawdziwe twierdzenia dotyczące Boga są nam nieznane. Niech ponownie $\mathcal{G}$~będzie zbiorem wszystkich prawdziwych sądów o~Bogu, niech $K$~będzie operatorem epistemicznym wyrażającym intuicje towarzyszące pojęciu wiedzy i~poznania, wówczas epistemiczna zasada teologii negatywnej przyjęłaby postać:
\begin{flalign*}
& \forall_{A \in \mathcal{G}}\ \neg K A. &\tag{KNT}\label{scep-par-preKNT}
\end{flalign*}
Zakładając, że $\mathcal{G}$ jest niepuste i $p \in \mathcal{G}$, otrzymujemy
%$p \land \neg K p$.\label{scep-par-postKNT}
\begin{flalign*}
& p \land \neg K p &\tag{KNT'}\label{scep-par-postKNT}
\end{flalign*}
W~sytuacji teologicznego sceptycyzmu wiemy, że nigdy nie poznamy żadnych prawd o~Bogu. Możemy więc jeszcze zapisać
%$K(A \land \neg K A)$.\label{scep-par-KNT-bis}
\begin{flalign*}
&K(A \land \neg K A)&\tag{KNT''}\label{scep-par-KNT-bis}
\end{flalign*}
W~ten sposób wkraczamy na teren dobrze znanych paradoksów epistemicznych.

Po pierwsze, zauważmy, że \ref{scep-par-postKNT} ma postać (odpowiednio zinterpretowanego) tzw. problemu Moore'a. Nie jest on paradoksem \textit{sensu stricto}, lecz raczej pewną łamigłówką epistemiczną. Wymaga narzucenia perspektywy pierwszoosobowej na interpretację operatora K, a~także potencjalnie wymusza wprowadzenie drugiego operatora (mającego wyrażać słabszą postawę propozycjonalną -- przekonanie) oraz użycie logiki multimodalnej. Problem ten w~kontekście teologicznego sceptycyzmu omówię pokrótce w~pierwszej sekcji niniejszego rozdziału.

Po drugie, \ref{scep-par-postKNT} i~\ref{scep-par-KNT-bis} wiążą się z~paradoksem poznawalności Churcha-Fitcha. Z~grubsza rzecz biorąc, paradoks ten mówi, że jeśli wiemy, że p~jest niepoznawalną prawdą, to niepoznawalne jest, że p~jest niepoznawalną prawdą (lub, przez kontrapozycję, że skoro wszystkie prawdy są co do zasady poznawalne, to wszystkie prawdy są znane). Podczas analizy tego paradoksu okaże się, że dosyć skromne założenia dotyczące działania epistemicznego operatora $K$~każą uznać \ref{scep-par-KNT-bis} za zdanie fałszywe.

Po trzecie, samozwrotną naturę powyższych wyrażeń możemy zredukować do tzw. paradoksu wiedzy (zwanego także -- przez analogię do paradoksu kłamcy -- paradoksem znawcy) i~związanego z~nim twierdzenia Montague'a, które uważane jest za generalizację twierdzenia Tarskiego (mówiącego, że każda teoria sformułowana w~semantycznie zamkniętym języku i~zwierająca T-równoważności jest sprzeczna). Zatem, mimo badań biorących za punkt wyjścia odrębne interpretacje teologii negatywnej oraz mimo stosowania odmiennych logicznych narzędzi do przedstawiania formalnej struktury tych interpretacji, w~ostateczności można uznać, że logiczne problemy teologii negatywnej mają tożsamą a~przynamniej podobną strukturę, zarówno, gdy rozpatrywane są w~jej aspekcie semantycznym, jak i~epistemicznym.

\section{Problem Moore'a}\label{scep-par-problemmoorea}

Choć opisywany w~tej sekcji problem ma swoje źródło w~filozofii (moralności), szybko znalazł zainteresowanie w~gronie bardziej analitycznie i~formalnie usposobionych badaczy. Jego nazwa została nadana przez Wittgensteina, który mówił o~\textit{paradoksie} Moore'a. Dziś, zagadnienie to nazywa się raczej \textit{problemem} niż paradoksem. Jego sednem jest przyznawanie prawdziwości czemuś, co do czego nie jest się przekonanym a~absurdalny charakter takich wyrażeń zależy silnie od kontekstu ich wypowiedzi. Przy odpowiedniej interpretacji sceptycyzm teologiczny można sprowadzić do tego epistemicznego zagadnienia.

Jego pierwsze znane sformułowanie pochodzi z~krótkiej wzmianki George'a Edwarda Moore'a zawartej w~pracy z~1942 roku z~poświęconego mu tomu z~serii \textit{Library of Living Philosophers}. Brzmi ono następująco:

\begin{quote}
,,W ostatni wtorek poszedłem do kina, ale w~to nie wierzę'' to całkowicie absurdalne stwierdzenie, chociaż sąd wyrażony tym zdaniem, jest w~zupełności możliwy logicznie\footnote{G.E. Moore, \textit{A~reply to my critics}, [w:] \textit{The Philosophy of G.E. Moore}, red. P.A. Schilpp, \textit{The Library of Living} \textit{Philosophers}, vol. 4, Northwestern University, Evanston 1942, ss.~543. Cytat za: H. Van Ditmarsch I~in., \textit{Everything is knowable - How to get to know whether a~proposition is True}, ,,Theoria'', vol. 78 (2012), nr 2, ss.~93-114.}.
\end{quote}

Moore sam nigdy nie opublikował pracy poświęconej pogłębionej dyskusji nad tym problem. Jednakże w~sporządzonych kilkanaście miesięcy później a~wydanych po jego śmierci notatkach -- w~odpowiedzi na komentarze Wittgensteina -- ponownie do tego wraca.

\begin{quote}
[…] zakładam, że absurdalne lub bezsensowne jest mówienie takich rzeczy jak ,,Nie wierzę, że pada deszcz, choć w~rzeczywistości pada'' […]. Chciałbym jednak zauważyć, że nie ma nic bezsensownego w~samym tym wyrażeniu. […] Nie ma nic bezsensownego w~powiedzeniu: ,,Jest całkiem możliwe, że choć nie wierzę, że pada deszcz, to w~rzeczywistości pada'' lub ,,Jeśli nie wierzę, że pada deszcz, choć w~rzeczywistości pada, to mylę się w~swoich przekonaniach''. We wszystkich tych przypadkach mamy do czynienia z~tym samym wyrażeniem, ale jest ono wypowiadane w~kontekście z~innymi słowami, w~taki sposób, że nie ma w~nich nic bezsensownego\footnote{G.E. Moore, \textit{Moore's Paradox}, [w:] \textit{G.E. Moore: Selected Works}, red. T. Baldwin, Routledge, London and New York 1993, s.~207.}.
\end{quote}

Już z~tych krótkich fragmentów możemy wyciągnąć kilka wniosków. Po pierwsze, ,,absurdalne'' czy ,,nonsensowne'' Moore'owi wydają się zdania o~postaci:
p, ale nie wierzę, że p.\label{scep-par-moore}
\begin{flalign}
& p \text{, ale nie wierzę, że } p. &\label{scep-par-moore}
\end{flalign}

Po drugie, zdania te nie stanowią par zdań sprzecznych. Używając jego słów, są ,,zupełnie możliwe logicznie''. Również, bez dodatkowych założeń, do sprzeczności nie będą prowadzić. Jest to jeden z~powodów, dla którego w~ich kontekście mówimy najczęściej o~\textit{problemie}, a~nie o~\textit{paradoksie}, Moore'a (choć należy uznać, że problem ten można nazwać paradoksem w~ogólnym, szerszym sensie\footnote{Por. rozdz.~\ref{sil-int-par}.}). Mimo tego, że wyrażenia o~postaci \eqref{scep-par-moore} nie stanowią ani bezpośrednio nie prowadzą do pary zdań sprzecznych, z~pewnością nazywanie ich ,,absurdalnymi'' czy ,,nonsensownymi'' nie jest pozbawione podstaw. Jest nich coś ,,logicznie osobliwego''\footnote{Por. H. Tennessen, \textit{Logical Oddities and Locutional Scarcities: Another Attack upon Methods of Revelation}, ,,Synthese'', vol. 11 (1959), nr 4, ss.~369-388.}.

Po trzecie, postawa propozycjonalna, o~której mowa w~\eqref{scep-par-moore} różni się od tej, z~którą mamy do czynienia w~\ref{scep-par-postKNT}. Logika epistemiczna, którą chcemy zaprzęgnąć do formalizacji teologicznego sceptycyzmu, próbuje uchwycić intuicje stojące za zjawiskiem wiedzy. W~przypadku \ref{scep-par-moore} należy mówić raczej o~przekonaniach i~użyć logiki przekonań, czyli logiki doksastycznej. W~takiej logice do alfabetu wprowadzamy modalny operator $B$~i dopuszczamy, by wyrażenia o~postaci $BA$ były wyrażeniami sensownymi (pod warunkiem, że A~jest wyrażeniem sensownym języka takiej logiki). Doksastyczny operator $B$, wraz rządzącymi nim aksjomatami, ma ujmować intuicje stojące za posiadaniem przekonań. $Bp$ Czytamy najczęściej jako ,,jest wiarygodne, że'' ,,jest zgodne z~(czyimiś) przekonaniami, że $p$'', ,,(ktoś) jest przekonany, że $p$'', ,,(ktoś) wierzy, że $p$''. Moore'owskie zdanie w~takiej logice zapiszemy:
%p~{\textbackslash}land {\textbackslash}neg B~p.\label{scep-par-moore-form}
\begin{flalign}
& p \land \neg B p. &\label{scep-par-moore-form}
\end{flalign}

W~formalnych rozważaniach nie unika się budowania rachunków multimodalnych, w~których występuje zarówno operator K, jak i~operator $B$. Z~reguły operatory te nie są wzajemnie definiowane (w taki sposób, w~jaki ma to miejsce na przykład w~przypadku operatorów możliwości i~konieczności w~aletycznej logice modalnej), choć zdarzają się takie podejścia. Najczęściej jednak w~multimodalnych rachunkach epistemicznych w~celu uchwycenia zależności miedzy wiedzą a~przekonaniem przedstawia się następujące aksjomaty\footnote{Por. J-J.C.Meyer, \textit{Modal Epistemic and Doxastic Logic}, [w:] \textit{Handbook of Philosophical Logic}, vol. 10, red. D.M. Gabbay, F. Guenthner, Springer, Dordrecht 2003, s.~6 oraz R. Rendsvig, J. Symons, \textit{Epistemic Logic}, [w:] \textit{The Stanford Encyclopedia of Philosophy}, wyd. lato 2021, red. E.N. Zalta, <\url{https://plato.stanford.edu/archives/sum2021/entries/logic-epistemic/}>.}:
%(K -{}-{}-+B) K{\textless}p -{}-{}-+ B{\textless}p oraz \label{scep-par-kdob}
%(KB) B{\textless}p -{}-{}-+ KB{\textless}p,
\begin{flalign}
\quad& Kp \to Bp &\tag{$K{\to}B$}\label{scep-par-kdob}\\
& Bp \to KBp &\tag{KB}
\end{flalign}

co zresztą jest zgodne z~tzw. klasyczną koncepcją wiedzy, wedle której jest ona prawdziwym i~uzasadnionym \textit{przekonaniem}\footnote{Por. E.L. Gettier, \textit{Is Justified True Belief Knowledge}?, ,,Analysis'', vol. 23 (1963), nr 6, ss.~121-123.}
. Oznacza to, że \ref{scep-par-postKNT} nie pociąga za sobą zdania w~stylu \ref{scep-par-moore-form}, a~przynajmniej nie wprost. Gdyby tak było, musielibyśmy uznać, że wiedza jest tożsama z~przekonaniem (albo odrzucić \ref{scep-par-kdob} i~uznać, że sytuacja, w~której coś wiemy jest ,,słabsza'' od tej, w~której jesteśmy co do tego przekonani). Istnieją jednak badacze, którzy problem Moore'a próbują analizować formalnie w~rachunku, który sytuację wiedzy i~przekonania wyraża jednym operatorem\footnote{Zob. H. Van Ditmarsch I~in., \textit{Everything is knowable}…, dz. cyt. Takie podejście uważam za błędne, a~przynajmniej mylące. Autorzy tej pracy redukują problem Moore'a do paradoksu Churcha-Fitcha. Taka redukcja prowadzi do spłycenia problemu I~jego filozoficznych konsekwencji. Por. C. de Almeida, \textit{What Moore's Paradox Is About}, ,,Philosophy and Phenomenological Research'', vol. 62 (2001), nr 1, s.~33-58.} lub rozważają analogon problemu Moore'a dla wiedzy, czyli zdania o~postaci
%p, ale nie wiem, czy p.\label{scep-par-moore-wiedz}
\begin{flalign}
& p \text{, ale nie wiem, czy } p. &\label{scep-par-moore-wiedz}
\end{flalign}

Do tych ostatnich należy Jaakko Hintikka\footnote{Zob. J. Hintikka, \textit{Knowledge and Belief}, Cornell University Press, Ithaka 1962, rozdz.~4.11.}, według którego -- podobnie, jak w~przypadku \eqref{scep-par-moore} -- wypowiadanie zdań o~postaci \eqref{scep-par-moore-wiedz} jest ,,cokolwiek niezręczne'', chociaż samo w~sobie spójne. Wydaje się więc, że mimo iż teologiczny sceptycyzm nie pociąga za sobą wprost problemu Moore'a, przedstawienie na jego przykładzie epistemicznego aspektu teologii apofatycznej jest jak najbardziej wskazane.

Czwartym i~ostatnim wnioskiem z~przytoczonych na początku tej sekcji fragmentów opisujących zdania Moore'a jest fakt, że ich ,,absurdalność'' i~,,bezsens'' są bardzo silnie uzależnione od kontekstu ich wypowiedzi. W~cytowanym fragmencie notatek wydanych \textit{post mortem} Moore zgadza się z~Wittgensteinem, że zdania o~postaci
%Jest całkiem możliwe, że choć nie wierzę, że p, to w~rzeczywistości p
\begin{flalign}
& \text{Jest całkiem możliwe, że choć nie wierzę, że } p \text{, to w~rzeczywistości } p. &
\end{flalign}
oraz
%Jeśli nie wierzę, że p, choć w~rzeczywistości p, to się mylę
\begin{flalign}
& \text{Jeśli nie wierzę, że } p \text{, choć w~rzeczywistości } p \text{, to się mylę.} &
\end{flalign}
są już pozbawione swojego absurdalnego charakteru. Podobnie, bezsensowność zdań Moore'a znika, gdy zrezygnujemy z~pierwszoosobowej perspektyw. Wydaje się, że nie ma nic paradoksalnego, ,,absurdalnego'', ,,dziwnego'' czy ,,osobliwego'' w~stwierdzeniu
%p, choć on nie wierzy, że p.
\begin{flalign}
& p \text{, choć on nie wierzy, że  } p. &
\end{flalign}

W~podobny sposób pozbawiamy się kłopotliwego charakteru zdań Moore'owskich, jeśli zmienimy czas gramatyczny wypowiadanych zdań na przeszły. Znów, wydaje się, że możemy bez żadnego ,,zgrzytu'' wypowiedzieć
%p, choć wtedy nie wierzyłem, że p
\begin{flalign}
& p \text{, choć wtedy nie wierzyłem, że } p. &
\end{flalign}
i~to niezależnie od tego, czy samo $p$~także wyrażone jest przy użyciu czasu przeszłego. Nie dziwi zatem, że w~formalnej rekonstrukcji tych zdań oraz intuicji związanych z~wiedzą i~przekonaniem nie dochodzimy do sprzeczności. Parę zdań sprzecznych możemy jednak otrzymać wprowadzając pewne (skromne) założenia wiążące operator $K$~oraz zakładając, że jesteśmy świadomi, że znaleźliśmy się w~sytuacji, o~której mówi \eqref{scep-par-moore-wiedz}. W~taki sposób dochodzimy do paradoksu Churcha-Fitcha.

\section{Paradoks Churcha-Fitcha}


Mówiąc w~skrócie, paradoks Churcha-Fitcha, zwany także paradoksem lub problemem poznawalności, zakłada, że istnienie nieznanych prawd jest niepoznawalne. Jeśli więc wszystkie prawdy są poznawalne, to w~rzeczywistości wszystkie prawdy są znane. Zagadnienie to posiada ciekawą historię i~choć przez niektórych badaczy zostało okrzyknięte ,,problemem na życzenie''\footnote{Taką etykietę paradoksowi Churcha-Fitcha przybił Piotr Łukowski argumentując, że wynika on z~niepoprawnego zapisu oraz wskazując na konstruktywistyczny przypadek zdań nierozstrzygniętych. Zob. P. Łukowski, \textit{Paradoksy}, Wydawnictwo Uniwersytetu Łódzkiego, Łódź 2006, s.~33. Argumentację Łukowskiego łatwo jest jednak odeprzeć lub odwrócić. Zob. B. Brogaard, J. Salerno, \textit{Fitch's Paradox of Knowability}, [w:] \textit{The Stanford Encyclopedia of Philosophy}, wyd. jesień 2019, red. E.N. Zalta, <\url{https://plato.stanford.edu/archives/fall2019/entries/fitch-paradox/}>. Zob.~także J.L Kvaanvig, \textit{The Knowability Paradox}, Oxford University Press, New York 2006.}, posiada ono zarówno niebagatelne motywacje filozoficzne, jak i~daleko idące konsekwencje -- nie tylko na gruncie teologii negatywnej, lecz przede wszystkim filozofii i~logiki.

Problem poznawalności został po raz pierwszy przedstawiony w~1963 roku w~krótkiej pracy poświęconej logicznymi analizom pojęcia wartości autorstwa Frederica Fitcha\footnote{F.B. Fitch, \textit{A~Logical Analysis of Some Value Concepts}, ,,The Journal of Symbolic Logic'', vol.~28, (1963), nr 2, ss.~135-142.}. Zawarte w~niej Twierdzenie 5., które było pierwszym opublikowanym sformułowaniem paradoksu, zostało dodane do manuskryptu w~celu uniknięcia pewnego rodzaju ,,błędu warunkowego'', który zagrażał definicji wartości opartej na świadomym pragnieniu\footnote{Z~grubsza rzecz biorąc, zgodnie z~analizami Fitcha, $x$~jest wartościowe dla $s$~wtedy i~tylko wtedy, gdy istnieje prawda $p$~taka, że gdyby $p$~było znane $s$, to $s$~pragnąłby $x$.}. W~oryginalnej postaci Twierdzenie 5. brzmi:
%(Fitch) Jeśli istnieje zdanie prawdziwe, którego prawdziwości nikt nie zna (lub nie znał lub nie będzie znał), to istnieje zdanie prawdziwe, którego prawdziwość jest niepoznawalna\footnote{``If there is some true proposition which nobody knows (or has known or will know) to be true, then there is a~true proposition which nobody can know to be true''. Tamże, s.~139.}.\label{scep-par-cf-fitch}
%\begin{flalign*}
%		& \parbox[t]{.87\linewidth}{ 
%		Jeśli istnieje zdanie prawdziwe, którego prawdziwości nikt nie zna (lub nie znał lub nie będzie znał), to istnieje zdanie prawdziwe, którego prawdziwość jest niepoznawalna\footnote{``If there is some true proposition which nobody knows (or has known or will know) to be true, then there is a~true proposition which nobody can know to be true''. Tamże, s.~139.}.} &\tag{Fitch}\label{scep-par-cf-fitch}
%\end{flalign*}
\begin{tw}[Fitch\footnote{``If there is some true proposition which nobody knows (or has known or will know) to be true, then there is a~true proposition which nobody can know to be true''. Tamże, s.~139.}]\label{scep-par-cf-fitch}
Jeśli istnieje zdanie prawdziwe, którego prawdziwości nikt nie zna (lub nie znał lub nie będzie znał), to istnieje zdanie prawdziwe, którego prawdziwość jest niepoznawalna.
\end{tw}
Twierdzenie to wraz z~prototypowym dowodem zostało zaproponowane i~przedstawione Fitchowi przez anonimowego recenzenta\footnote{Według informacji zawartej w~przypisie umieszczonym w~pracy Fitcha, anonimowy recenzent jest autorem poprzedzającego powyższe Twierdzenia 4. Badania Joe Salerno nie tylko ujawniły nazwisko tego recenzenta, lecz także wskazały jego wkład w~sformułowanie i~dowiedzenie Twierdzenia 5. Zob. B. Brogaard, J. Salerno, \textit{Fitch's Paradox of Knowability}, dz. cyt.} wczesnej wersji jego pracy, którą w~1945 roku (\textit{sic}!) zgłosił do tego samego czasopisma (choć wtedy, rzecz jasna, jej nie opublikował). W~2003 roku, dzięki badaniom archiwalnym Joe Salerno, okazało się recenzentem tym był Alonzo Church\footnote{Recenzja Churcha została opublikowana w~2009 roku. Zob. A. Church, \textit{Referee Reports on Fitch's ``A Definition of Value}'', [w:] \textit{New Essays on the Knowability Paradox}, red. J. Salerno, Oxford University Press, New York 2009, ss.~13-20.} a~inspiracją do sformułowania twierdzenia była dla Churcha najprawdopodobniej wspomniana w~poprzedniej sekcji praca Moore'a\footnote{G.E. Moore, \textit{A~reply to my critics}, dz. cyt. Zob. B. Brogaard, J. Salerno, dz. cyt. Por. rozdz.~\ref{scep-par-problemmoorea}. Fitch prawdopodobnie nie uznawał paradoksalności tego wyniku, natomiast Church podawał szereg sposobów na uniknięcie tej paradoksalności.}.

Mimo, iż paradoksu poznawalności po raz pierwszy pojawił się w~badaniach logicznych pod postacią twierdzenia \ref{scep-par-cf-fitch} (i jego dowodu), w~literaturze pod taką etykietą zdecydowanie dużo bardziej znana jest jego kontrapozycja.
%(Paradoks poznawalności) Jeśli wszystkie zdania prawdziwe są co do zasady poznawalne, to wszystkie zdania prawdziwe są w~rzeczywistości znane.\label{scep-par-cf-parpozn}
%\begin{flalign*}
%		& \parbox[t]{.65\linewidth}{ 
%		Jeśli wszystkie zdania prawdziwe są co do zasady poznawalne, to wszystkie zdania prawdziwe są w~rzeczywistości znane.} &\tag{Paradoks poznawalności}\label{scep-par-cf-parpozn}
%\end{flalign*}
\begin{tw}[Paradoks poznawalności]\label{scep-par-cf-parpozn}
Jeśli wszystkie zdania prawdziwe są co do zasady poznawalne, to wszystkie zdania prawdziwe są w~rzeczywistości znane.
\end{tw}

By udowodnić twierdzenie \ref{scep-par-cf-parpozn} należy zadać pewne (bardzo skromne logicznie) warunki na działanie modalnego operatora $K$. Musimy założyć rozdzielność $K$~względem koniunkcji:
%(M) K(A ${\wedge}$ B) {213} (KA ${\wedge}$ KA).
\begin{flalign}
& K(A \land B) \to (KA \land KA). &\tag{\textsf{M}}\label{ruleM}
\end{flalign}
Oznacza to, że ilekroć mamy wiedzę o~koniunkcji dwóch zdań, wiemy o~każdym z~jej członów. Musimy też przyjąć regułę \eqref{ruleT}, według której wiedza dotyczy zawsze zdań prawdziwych:
(T) KA -{\textgreater} A,
\begin{flalign}
& KA \to A, &\tag{\textsf{T}}\label{ruleT}
\end{flalign}
co znów odpowiada klasycznej koncepcji wiedzy, wedle której jest ona \textit{prawdziwym} i~uzasadnionym przekonaniem.

Wróćmy na moment do zdania \ref{scep-par-KNT-bis} i~sytuacji teologicznego sceptycyzmu. Oba powyższe aksjomaty wystarczają, by udowodnić, że zdanie to jest fałszywe a~teologiczny sceptycyzm jest teorią sprzeczną.
%Lemat {\textbackslash}neg K(A {\textbackslash}land {\textbackslash}neg K~A)\label{scep-par-cf-lemat}
%Dowód z~belgii lub art.
\begin{lem}\label{scep-par-cf-lemat}
$\neg K(A \land \neg K~A)$
\end{lem}
\begin{proof}
\begin{flalign}
& K (A \land \neg K A) & \text{(założenie, \ref{scep-par-KNT-bis})}\label{lematK1} \\
& K A \land K \neg K A & \text{(MP \ref{lematK1}, \ref{ruleM})}\label{lematK2} \\
& K A & (\land\text{ elim. \ref{lematK2})}\label{lematK3} \\
& K \neg K A & (- | | -)\label{lematK4} \\
& \neg K A & \text{(MP \ref{lematK4}, \ref{ruleT})}\label{lematK5} \\
& \neg K (A \land \neg K A) & \text{(\ref{lematK1}, \ref{lematK3}, \ref{lematK5}, reductio ad absurdum)}\qedhere
\end{flalign}
\end{proof}
Zatem, jak pokazuje powyższy lemat, nie możemy mieć wiedzy o~zdaniach w~stylu Moore'a. Stawiając tę sprawę w~kontekście teorii Niepoznawalnego: nawet jeśli istnieją jakieś nieznane nam prawdy (o Bogu), nie moglibyśmy o~tym wiedzieć. Staje się to jasne, gdy prześledzimy dalsze fragmenty dowodu twierdzenia \ref{scep-par-cf-parpozn}. Najpierw jednak przedstawmy je w~bardziej formalnym zapisie\footnote{Przy takiej notacji twierdzeniu \ref{scep-par-cf-fitch} odpowiada wyrażenie $p \land \neg K p \vdash p \land \neg \Diamond K p$. W~badaniach wokół paradoksu poznawalności najczęściej stosuje się kwantyfikację po zmiennych zdaniowych. W~takim zapisie twierdzenia \ref{scep-par-cf-fitch} oraz \ref{scep-par-cf-parpozn} przyjmują następującą postać: $\exists p (p \land \neg K p) \vdash \exists p (p \land \neg \Diamond K p)$ oraz $\forall p (p \to \Diamond K p) \vdash \forall p (p \to K p)$. Notacja ta wprowadza na raz kwantyfikację, operatory atletyczne oraz epistemiczne, więc -- by nie powodować dodatkowego zamieszania -- korzystam z~tego, że w~obrębie przedstawianych tu rozważań można pominąć kwantyfikatory bez utraty sensu twierdzenia oraz poprawności jego dowodu.}:
%Twierdzenie $A~\to \Diamond K~A \vdash A~\to K~A$.\label{scep-par-cf-unkno}
\begin{tw}\label{scep-par-cf-unkno}
$A \to \Diamond K A \vdash A \to K A$
\end{tw}
W~powyższej formule pojawia się dodatkowy symbol -- $\Diamond$. Stosowany jest tu w~standardowy, znany z~modalnej logiki aletycznej, sposób -- na oznaczenie operatora możliwości. Dualnym do operatora możliwości jest operator konieczności ($\square$). $\Diamond p$~oznacza ,,jest możliwe, że $p$'', $\square p$ czytamy jako ,,konieczne, że $p$''. Wyrażenie o~postaci $\Diamond K p$ odczytywane jako ,,jest możliwe, by wiedzieć, że $p$'' wydaje się dobrze oddawać intuicje towarzyszące stwierdzeniu, że sąd stojący za zdaniem $p$~jest \textit{poznawalny}. Zatem kolejny raz mamy tu do czynienia z~rachunkiem multimodalnym. Tym razem rachunkiem próbującym rekonstruować sytuacje, w~których mamy do czynienia z~wiedzą, oraz możliwością i~koniecznością lub, inaczej mówiąc, rachunek ten aspiruje do oddania głębszej struktury pojęcia poznawalności oraz zależności między zdaniami poznawalnymi a~zdaniami znanymi i~prawdziwymi. Wprowadzając dwa dodatkowe symbole ponownie musimy dodać wyrażenia o~postaci $\Diamond A$~oraz $\square A$~do zbioru wyrażeń sensownych naszego nowego, multimodalnego rachunku. By dowieść twierdzenia \ref{scep-par-cf-unkno} należy założyć, że zachowaniem tych dwóch operatorów rządzą znów niepozorne i~dobrze znane reguły logiki aletycznej:
%
%(Reguła wymuszania) reguła oraz
%
%$(\square \neg -\neg \Diamond) \square \neg A~\to \neg \Diamond A.$
\begin{flalign}
& \frac{A}{\square A} & \tag{RW}\label{necessitation-rule} \\[10pt]
\quad& \square \neg A~\to \neg \Diamond A. & \tag{$\square$ elim.}\label{nec-elim-rule}
\end{flalign}

Pierwsza z~wymienionych zasad każe uznać, że twierdzeniem jest wyrażenie $\square A$, o~ile $A$~jest już wcześniej udowodnionym twierdzeniem. Druga zasada mówi, że koniecznie fałszywe sądy są niemożliwe i~wynika wprost z~dualności operatorów $\Diamond$ i~$\square$. Mając te dwie zasady oraz lemat \ref{scep-par-cf-lemat} możemy podać dowód twierdzenia \ref{scep-par-cf-unkno}\footnote{Por. H. Wansing, \textit{Diamonds Are a~Philosopher's Best Friends: The Knowability Paradox and Modal Epistemic Relevance Logic}, ,,Journal of Philosophical Logic'', vol. 31 (2002), p. 593; W.H.~Holliday, \textit{Epistemic Logic and Epistemology}, [w:] \textit{Introduction to Formal Philosophy}, red. S.O.~Hansson, V.F. Hendricks, Springer, Cham 2018, p. 364.}.
%Dowód. Chyba cały na schematach.
\begin{proof}
\begin{flalign}
& A \to \Diamond K A 							& \text{(założenie)}\label{twwierF1} \\
& (A \land \neg K A) \to \Diamond K	(A \land \neg K A)		& \text{(\ref{twwierF1})}\label{twwierF2} \\
& \square \neg K(A \land \neg K~A)				& \text{(\ref{necessitation-rule}  \ref{scep-par-cf-lemat})}\label{twwierF3} \\
& \neg \Diamond  K(A \land \neg K~A)			&\text{(\ref{nec-elim-rule} \ref{twwierF3})}\label{twwierF4} \\
& \neg (A \land \neg K A)								& \text{(p. kontrapozycji \ref{twwierF2}, \ref{twwierF4})}\label{twwierF5}\\
& A \to K A  									& \text{(p. negacji implikacji \ref{twwierF5})}\qedhere
\end{flalign}
\end{proof}

Twierdzenie \ref{scep-par-cf-parpozn} i~(a, co za tym idzie, \ref{scep-par-cf-unkno}) uznawane jest za paradoksalne, ponieważ wychodząc od stosunkowo niepozornego założenia -- że wszystkie prawdy są poznawalne -- oraz wykorzystując naturalne i~logicznie niekontrowersyjne reguły rządzące operatorami $K$~oraz $\Diamond$ dochodzimy do nieoczekiwanie silnego i~trudnego do przyjęcia wniosku -- że wszystkie prawdy są znane\footnote{Podobnie paradoksalne wydaje się przejście od istnienia przypadkowej ignorancji do istnienia koniecznej niepoznawalności w~twierdzeniu \ref{scep-par-cf-fitch}.}. Inaczej mówiąc, twierdzenie to i~jego dowód wydają się paradoksalne dlatego, że zdają się sprowadzać stosunkowo słabe założenie o~poznawalności prawdy, do sytuacji wszechwiedzy.

Mimo, iż praca Fitcha została sporządzona w~celu analizy pojęcia wartości, paradoks poznawalności nigdy potem nie był używany w~podobnym kontekście. Od początku lat 80. wykorzystywany jest on najczęściej jako argument w~sporze realistów epistemicznych z~tzw. antyrealistami, którzy -- mimo iż przyjmują istnienie nieznanych sądów prawdziwych -- odrzucają ich jakąś zasadniczą niepoznawalność. Paradoks Churcha-Fitcha stosowany jest więc w~próbach obalenia tych teorii, które zakładają jakąś wersję zasady poznawalności (zgodnie z~którą wszystkie sądy prawdziwe są poznawalne). Wśród tych teorii wymienić należy choćby semantyczny realizm Michaela Dummetta, realizm wewnętrzny Hilarego Putnama, czy logiczny pozytywizm Koła Wiedeńskiego\footnote{Przegląd argumentów przeciw takim teoriom można znaleźć w: J. Salerno (red.), N\textit{ew Essays on the Knowability Paradox},-Oxford University Press, New York 2009, część II.}. Uogólniona postać paradoksu Churcha-Fitcha nadaje się także do uchwycenia innych samozwrotnych aporii, pozbawionych kontekstu poznawalności -- tworzonych także w~kontekście teologicznym\footnote{Zob. Tenże, \textit{Introduction}, [w:] Tamże, s.~4.}.

\section{Paradoks znawcy (knower's paradox)}

Genezą paradoksu znawcy są rozważania dotyczące paradoksu kata (zwanego także paradoksem niespodziewanego testu\footnote{To dwie najpopularniejsze polskojęzyczne nazwy tego problemu. W~literaturze angielskiej jest ich jeszcze więcej.}).

\subsubsection{Paradoks kata (niespodziewanego testu)}

Wyobraźmy sobie, że sędzia skazuje oskarżonego na karę śmierci, która ma zostać wykonana w~przyszłym tygodniu w~południe w~dniu, w~którym skazany nie będzie się tego spodziewał. Wydaje się, że taki wyrok nie może zostać wykonany. Skazany nie może zostać stracony w~niedzielę, bo jeśli przeżyje do sobotniego popołudnia, będzie musiał spodziewać się wykonania kary w~niedzielę. Nie może zostać stracony w~sobotę, bo jeśli przeżyje do piątkowego popołudnia a~niedzielę należy wykluczyć jako dzień wymierzenia kary, musiałby się spodziewać wykonania wyroku w~sobotę, co znów jest sprzeczne z~literą orzeczenia sądu. Z~tego samego powodu nie może zostać stracony w~piątek, czwartek, środę i~wtorek. W~końcu, wyroku nie można wykonać i~w poniedziałek, bo -- skoro skazany nie może zostać stracony we wszystkie pozostałe dni tygodnia -- musiałby się spodziewać, że karę wymierzą mu właśnie w~ten dzień.

\bigskip

Powyższa historia uważana jest za paradoksalną, ponieważ wniosek z~niej płynący przeczy naszym najogólniejszym intuicjom, które zdają się podpowiadać, że kat powinien móc zaskoczyć skazanego przeprowadzając niespodziewaną egzekucję (na przykład, dajmy na to, w~czwartek). Paradoks ten sam w~sobie jest interesującym zagadnieniem i~doczekał się rozmaitych odmian i~wariacji oraz bogatej literatury. W~przełomowej pracy David Kaplan i~Richard Montague\footnote{D. Kaplan, R. Montague, \textit{A~paradox regained}, ,,Notre Dame Journal of Formal Logic'', vol. 1 (1960), nr 3, ss.~79-90.} manipulują liczbą dni, w~których należy rozpatrywać możliwość wykonania wyroku. Okazuje się, że przy zredukowaniu tej liczby do zera, paradoks kata przeobraża się w~zdanie w~stylu paradoksu kłamcy angażujące już nie semantyczne pojęcie prawdy, lecz epistemiczne pojęcie wiedzy.
%({\textbackslash}lambda)Zdanie (lambda) jest nieznane.\label{scep-klamca}
\begin{flalign*}
		& \text{Zdanie $\lambda$ jest nieznane.} &\tag{$\lambda$}\label{scep-klamca}
\end{flalign*}

Aby pokazać autoreferencyjność tego zdania załóżmy najpierw, że \ref{scep-klamca} jest znane. Następnie -- zakładając, że obowiązuje aksjomat \eqref{ruleT} -- musimy uznać, że \ref{scep-klamca} jest prawdziwe, a~mówi ono, że \ref{scep-klamca} jest nieznane. Zatem \ref{scep-klamca} jest nieznane i~udowodniliśmy, że nasze pierwotne założenie jest fałszywe. Przyjmijmy więc, że \ref{scep-klamca} jest nieznane. Następnie -- zakładając, że o~udowodnionym zdaniu wiemy, że jest prawdziwe -- musimy uznać, że wiemy, że \ref{scep-klamca} jest nieznane, a~zatem \ref{scep-klamca} jest znane. W~ten sposób wracamy do naszego pierwotnego założenia i~domykamy pętlę błędnego koła. \ref{scep-klamca} jest znane wtedy i~tylko wtedy, gdy jest nieznane.

Już ta krótka demonstracja podpowiada, jakie reguły musimy przyjąć, by dowieść sprzeczności wywołanej przez paradoks znawcy. Podobnie, jak z~powodu zdań wyrażających paradoks kłamcy każda semantycznie zamknięta (tj. niewprowadzająca stratyfikacji języków) teoria (będąca rozszerzeniem arytmetyki pierwszego rzędu) zawierająca T-równoważności okazywała się niespójna\footnote{Por. rozdz.~\ref{sil-boch}.}, tak możemy udowodnić analogiczne twierdzenie w~kontekście epistemicznym.

\begin{tw}[Montague'a\footnote{R. Montague, \textit{Syntactical treatments of modality, with corollaries on reflexion principles and finite axiomatizability}, ,,Acta Philosophical Fennica'', vol. 16 (1963), ss.~153-167. Przedruk w: red. R.H. Thomason, \textit{Formal Philosophy: Selected Papers of Richard Montague}, Yale University Press, New Haven -- London 1974, ss.~286-302.}]\label{scep-znaw-montag}
Każda teoria (będąca rozszerzeniem arytmetyki pierwszego rzędu) zawierająca epistemiczną regułę wymuszania
\begin{flalign}
& \frac{A}{K A} & \tag{RW\textsubscript{K}}\label{Knecessitation-rule}
\end{flalign}
oraz aksjomat \eqref{ruleT}\footnote{W~rzeczywistości podobnych twierdzeń można dowodzić przy wykorzystaniu innych, także słabszych założeń. Dobry teorio-dowodowy przegląd paradoksu znawcy przedstawia: Z. Tworak, \textit{Paradoks znawcy (The Knower Paradox}), ,,Filozofia Nauki'', vol. 19 (2011), nr 3(75), ss.~29-47.}
jest sprzeczna.
\end{tw}


Dowód opiera się na tzw. lemacie o~diagonalizacji dopuszczającym możliwość sformułowania w~języku takiej teorii zdań w~stylu \ref{scep-klamca}\footnote{Dowód tego lematu naśladuje pewne fragmenty dowodu twierdzenia Gödla. Jego przedstawienie nie jest konieczne dla zrozumienia toku wywodu i~raczej zaciemniłoby przedstawiany tu obraz. Choć podobieństwo samozwrotności oraz twierdzeń Gödla jest samo w~sobie ciekawym zagadnieniem.} i~właściwie w~sposób formalny odtwarza wykazanie niespójności paradoksu znawcy\footnote{Por. J. Stern, \textit{Montague's Theorem and Modal Logic}, ,,Erkenntnis'', vol. 79 (2014), nr 3, s.~554.}.
\begin{proof}
\begin{flalign}
& \lambda \equiv \neg K \lambda 			& \text{(lemat o~diagonalizacji, paradoks znawcy)}\label{znafca1} \\
& K\lambda \to \lambda		& \text{(\ref{ruleT})}\label{znafca2} \\
& K \lambda \to \neg K \lambda				& \text{($\equiv$ elim., syl. hip., \ref{znafca1},\ref{znafca2})}\label{znafca3a} \\
& \neg K \lambda				& \text{(p. Claviusa, \ref{znafca3a})}\label{znafca3} \\
& \lambda			& \text{($\equiv$ elim., MP \ref{znafca1}, \ref{znafca3})}\label{znafca4} \\
& K \lambda								& \text{(\ref{Knecessitation-rule} \ref{znafca4})}\label{znafca5}\\
& \qquad \text{contr. \ref{znafca3}, \ref{znafca5}} 									& \nonumber\qedhere
\end{flalign}
%$\lambda \equiv \neg K~\lambda$ lemat o~diagonalizacji, paradoks znawcy
%
%$K\lambda \to \lambda$ (T)
%
%$\neg K (\lambda)\equiv$ elem., MP 1,2
%
%$\lambda\equiv$ elem., MP 1,3
%
%$K(\lambda)$RW
%
%Sprz. 3, 5
\end{proof}

Paradoks znawcy może posłużyć do podparcia wielu argumentów wytaczanych w~dyskusjach z~zakresu podstaw i~filozofii matematyki, epistemologii teorii dowodu czy filozofii umysłu\footnote{Zob. R. Sorensen, \textit{Epistemic Paradoxes}, [w:] \textit{The Stanford Encyclopedia of Philosophy}, wyd. wiosna 2022, red. E.N. Zalta, <\url{https://plato.stanford.edu/archives/spr2022/entries/epistemic-paradoxes/}>.}. Dla nas jednak jego najciekawszym zastosowaniem jest wykorzystanie go jako ilustracji teologicznego sceptycyzmu. Okazuje się, że samozwrotny, paradoksalny charakter teorii Niepojmowalnego można zredukować do prostego zdania w~tylu znawcy (\ref{scep-klamca}). W~końcu, gdy mówimy, że żadne prawdziwe zdania o~Bogu nie jest poznawalne, to zdanie wyrażające tę zasadę pośrednio mówi o~sobie, że samo nie powinno być znane.

Powyższa obserwacja wraz z~twierdzeniem \ref{scep-znaw-montag} zdają się sugerować, że do rozwikłania paradoksu Niepoznawalnego można próbować wykorzystać rozwiązania w~stylu Bocheńskiego przedstawione w~poprzedniej części pracy\footnote{Zob. rozdz.~\ref{sil-boch-nonsens}.}. Należy jednak pamiętać, że oprócz dodatkowych trudności wynikających z~faktu, że w~sytuacji epistemicznej próbujemy eksploatować rozwiązanie semantyczne, wszystkie obiekcje dyskutowane poprzednio\footnote{Zob. rozdz.~\ref{sil-boch-dyskusja}.} będą zasadne także dla takiego zastosowania.





%\section{Wprowadzenie}
Logiczna analiza dyskursu teologicznego nie jest może najpopularniejszą
spośród metod badawczych teorii teologicznych a prace jej poświęcone
stanowią niewielki odsetek zarówno w literaturze z dziedziny logiki i
filozofii języka, jak i teologii. Jednakże charakter badań
przedstawionych w niniejszej pracy nie jest przedsięwzięciem
odosobnionym w historii myśli. Logiką teologii w sposób systematyczny
zajmowali się już w latach trzydziestych ubiegłego stulecia
przedstawiciele tzw. Koła Krakowskiego -- Józef Bocheński, Jan
Łukasiewicz, Jan Salamucha, Jan Drewnowski oraz Bolesław
Sobociński\footnote{Zob. Z. Wolak, Naukowa filozofia koła
krakowskiego, „Zagadnienia Filozoficzne w Nauce”, nr.~36 (2005),
ss.~97-122. }. Pierwszy z nich zaznaczył się szczególnie w historii
tego typu rozważań swoją przełomową pracą \textit{The Logic of
Religion}\footnote{J.M. Bocheński, The Logic of Religion, New York
University Press, New York 1965. Pierwsze polskie wydanie: J.M.
Bocheński, Logika religii, tłum. S. Magala, Instytut wydawniczy PAX,
Warszawa 1990. W tej pracy posługuję się tym samym tłumaczeniem,
zamieszczonym w J.M. Bocheński, Logika i filozofia, Wydawnictwo Naukowe
PWN, Warszawa 1993. }. Nurt zapoczątkowany przez nich powrócił w
ostatnich latach do Krakowa jako program badawczy „logika w teologii”
prowadzony w ramach Centrum Kopernika Badań Interdyscyplinarnych.
Zaowocował on szeregiem spotkań, seminariów i konferencji, których
owocem jest wydana w 2013 roku praca zbiorowa \textit{Logic in
Theology}\footnote{B. Brożek, A. Olszewski, M. Hohol (red.), Logic in
Theology, Copernicus Center Press, Kraków 2013. }. W tej pracy będę
podążał śladami wymienionych wyżej autorów. Skupię się jednak na
szczególnej formie teologii zwanej teologią apofatyczną lub negatywną.
Jest to bardzo osobliwy rodzaj myślenia teologicznego.


Definicje:
\begin{defin}
Teologia apofatyczna -- znana także jako teologia negatywna, \textit{via
negativa} albo \textit{via negationis}, jest to teologia próbująca
opisać Boga, Boskie Dobro poprzez negację. Innymi słowy, o Bogu można
wyrażać się tylko w formach negatywnych - tzn. mówić jaki On nie jest,
a nigdy jaki jest.
\end{defin}

\begin{defin}
Teologia apofatyczna (gr. apofatikos -
,,przeczący'', nazywana też Teologią
negatywną) - nurt teologii oparty na założeniu, że jakiekolwiek
pozytywne poznanie natury Boga przekracza granice możliwości ludzkiego
rozumu. […] Teologia apofatyczna, podkreślając niewspółmierność
wszystkich poznawczych wysiłków zmierzających do opisania tajemnicy
Boga, odrzuca wszelkie symbole, obrazy i abstrakcyjne pojęcia jako
nieadekwatne do opisu natury Boga i próbuje przybliżyć Jego tajemnicę
za pomocą formuł przeczących, mówiąc, jaki Bóg nie jest. [wikipedia]
\end{defin}

\begin{defin}
Teologia apofatyczna to inna nazwa dla „teologii poprzez negację”, wedle
której Bóg jest poznawany poprzez negowanie pojęć, które mogłyby być mu
przypisane. Teologia apofatyczna podkreśla, że ludzkie pojęcia i język
są nieodpowiednim narzędziem do opisania Boga. [The Oxford Dictonary of
World Religions]
\end{defin}

\begin{defin}
Teologia Negatywna to nazwa nadana chrześcijańskiemu nurtowi\footnote{W
mojej opinii ograniczanie teologii apofatycznej do tradycji
chrześcijańskiej jest błędem. }, wedle którego Bóg jest absolutnie
transcendentny i na tyle przekracza nadawane mu nazwy i pojęcia, że w
rzeczywistości musimy je „zanegować”, by uwolnić Boga z tych ciasnych
kategorii. [Charles M. Stang]
\end{defin}

\begin{defin}
Teologia negatywna -- podobnie jak w teologii apofatycznej podejście do
Bożej Tajemnicy, które podkreśla, że raczej potrafimy powiedzieć, czym
Bóg nie jest, niż czym jest w rzeczywistości. Jest to sposób uprawiania
teologii, który kładzie większy nacisk na to, co się wyraża po łacinie
słowem sapientia (łac. „mądrość”) niż scientia (łac. „wiedza”). [Gerald
O'Collins SJ, Edward G. Farrugia SJ, Leksykon pojęć
teologicznych i kościelnych]
\end{defin}

\begin{defin}
Teologia apofatyczna -- (gr. „negatywny, przeczący”) Podstawowe pojęcie w
teologii Wschodu, często tłumaczone jako teologia negatywna. Kładzie
ono nacisk na niewspółmierność wszystkich naszych wysiłków
zmierzających do opisania absolutnej tajemnicy Boga. Każde nasze
twierdzenie o Bogu musi być ograniczone przez odpowiednie zaprzeczenie
i przez uznanie, że Bóg w sposób nieskończony przekracza wszelkie nasze
kategorie. [Gerald O'Collins SJ, Edward G. Farrugia
SJ, Leksykon pojęć teologicznych i kościelnych]
\end{defin}

Przymiotnik \textit{negativus}, który pojawia się w \textit{via
negativa} jest łacińską wersją greckiego przymiotnika
\textit{apofatikos}, pochodzącego od \textit{apofasis}, które
tłumaczone jest jako odwołanie (swoich słów), zaprzeczanie lub –
najczęściej -- jako negacja.

Mimo, iż teologia apofatyczna nigdy nie należała do głównego nurtu
teologicznego i filozoficznego myślenia o Bogu, w ciągu historii miała
całkiem pokaźne grono zwolenników.

\begin{itemize}
  \item W greckiej filozofii (prekursorzy): Platon, Plotyn, Proklos
  \item W tradycji chrześcijańskiej: Orygenes, Cyryl Jerozolimski, Ojcowie
Kapadoccy (Bazyli Wielki, Grzegorz z Nazjanzu, Jan Damasceńczyk, Maksym
Wyznawca, Jan Chryzostom, św. Jan od Krzyża, Marcin Luter.

Właściwi teologowie negatywni: Klemens Aleksandryjski, Grzegorz z Nyssy,
Pseoudo-Dionizy Areopagita, Mistrz Eckhart, Mikołaj z Kuzy, Tomasz z
Akwinu\footnote{Umieszczenie Tomasza z Akwinu wśród teologów
apofatycznych może wydać się kontrowersyjne, przemawiają jednak ku temu
pewne fragmenty jego pism. Por. P. Sikora, \textit{Logos
niepojęty.} }.

Tradycja apofatyczna jest także bardzo silna pośród wielu współczesnych
teologów prawosławnych: Vladimir Lossky, John Meyendorff, John S.
Romanides and Georges Florovsky.
  \item Pewne wątki teologii apofatycznej można odnaleźć także w tradycji
żydowskiej: Filon z Aleksandrii, Bahya ibn Paquda, Mojżesz Majmonides;
islamskiej: Ibn Arabi, Wasil ibn Ata, Avicebron; buddyjskiej i
hindusitycznej.
\end{itemize}

Teologia apofatyczna może być rozumiana jako przeciwieństwo teologii
katafatycznej (pozytywnej).


\begin{defin}
Teologia katafatyczna (gr. (\textgreek{Katafasic, katafatikoc}katafatika
- twierdzący, pozytywny, afirmacja) - , nazywana też Teologią
pozytywną. Nurt teologii oparty na założeniu, że człowiek jest w stanie
pozytywnie wypowiadać się {\textquotedbl}jaki Bóg jest{\textquotedbl}.
Teologia katafatyczna jest przeciwieństwem teologii apofatycznej.
[wikipedia]
\end{defin}


\begin{defin}
Teologia katafatyczna (gr. „twierdzący, pozytywny”) Pojęcie dopełniające
do teologii apofatycznej, czyli negatywnej, właśnie dlatego nazywane
czasami „teologią pozytywną”. Mimo że kategorie poznawcze, którymi się
kierujemy, są u samych swoich podstaw niewystarczające, możemy
wypowiadać wiele prawdziwych twierdzeń o Bogu, który się nam objawił w
sposób niezrównany w Jezusie Chrystusie i dał się nam poznać teraz
przez Ducha Świętego. Teologia apofatyczna jednak kładzie nacisk na to,
że nawet mimo faktu, iż Bóg sam siebie nam objawił i sam siebie nam
darował, pozostaje On dla nas zasadniczą tajemnicą. [Gerald
O'Collins SJ, Edward G. Farrugia SJ, Leksykon pojęć
teologicznych i kościelnych]
\end{defin}

Między tymi dwoma nurtami myślenia teologicznego istnieje wiele
odmienności. Po pierwsze, teologia negatywna dużo częściej (choć nie
zawsze) posiada charakter mistyczny, ponieważ jej negacje mają służyć
głównie do umożliwienia spotkania z transcendentnym Bogiem. Po drugie,
neguje ona nie tylko wszystkie twierdzenia dane na gruncie teologii
pozytywnej, lecz także -- krótko mówiąc -- neguje swoje własne negacje.
Tym samym teologia negatywna wydaje się być na pierwszy rzut oka
sprzeczna. By ukazać, jak wymagającym zadaniem jest zajmowane się tym
rodzajem teologii z logicznej perspektywy, posłużmy się następującym
cytatem z Pseudo-Dionizego Areopagity:




\begin{quote}
    Bóg nie jest ani duszą, ani intelektem, ani wyobrażeniem, ani
mniemaniem, ani pojmowaniem, ani słowem i pojmowaniem; […] nie jest bez
ruchu ani w ruchu, ani nie odpoczywa. […] Nie jest […] ani Boskością,
ani dobrocią […]. Nie jest też niczym z niebytu ani czymś z bytu. […]
Nie istnieje ani słowo, ani imię, ani wiedza o Nim. […] Jest ponad
wszelkim twierdzeniem i ponad wszelkim zaprzeczeniem.\footnote{
Pseudo-Dionizy Areopagita, Teologia Mistyczna, Rozdział V [w:]  Pisma
teologiczne, tłum. M. Dzielska, Wydawnictwo Znak, Kraków2005.}
\end{quote}



Jak zauważa Paweł Rojek\footnote{P. Rojek, Logika teologii negatywnej,
Pressje, nr 29 (2012), s. 216. }, wydaje się, ze Dionizy nie tylko
narusza pewne fundamentalne zasady teologii katafatycznej, lecz także
podstawowe prawa logiczne. Twierdząc, że Bóg nie jest ani niebytem, ani
bytem, Dionizy wydaje się ignorować prawa wyłączonego środka.
Oczywiście istnieją ugruntowane w literaturze i dobrze uzasadnione
filozoficznie rachunki logiczne, które odrzucają zasadę wyłączonego
środka\footnote{Do najważniejszych z nich należy logika
intuicjonistyczna. }, jednak lista naruszanych przez tego
myśliciela praw na tym się nie kończy. Twierdzenie, że Bóg nie jest
Boskością każe sądzić, że Dionizy podważa prawo tożsamości. Ponadto, w
sposób oczywisty zaprzecza samemu sobie utrzymując, że nie ma żadnego
mówienia o Bogu, podczas gdy sam się tego zadania podejmuje.

Rzeczywiście, nie ma wielkiej przesady w twierdzeniu, że teologia
negatywna jest najbardziej mistyczną, nielogiczną i niejasną teorią
Boga, jaka kiedykolwiek powstała\footnote{Tamże. }. Z tego powodu
niektórzy filozofowie, filozofowie logiki, lub teologowie mogliby
utrzymywać, że ten rodzaj teologicznego myślenia nie posiada żadnej
teoretycznej wartości a zdania takiej teologii powinny być traktowane
raczej jako modlitwa, hymn uwielbienia, wyraz czci lub środek
prowadzący do zjednoczenia z Bogiem\footnote{Takie stanowisko jest
popularne wśród myślicieli prawosławnych takich, jak Władimir Łosski.
Utrzymują je także Andrew Louth i Paul Rorem. Zob. Tamże, s. 217.
}. Mogliby oni stwierdzić, że logika formalna nie będzie w ogóle
użyteczna w wyjaśnianiu i precyzowaniu takiej teologii a każda próba
odkrycia jej logicznej struktury i formalizacji z góry skazana jest na
niepowodzenie. Na przykład Jan Woleński wprost pisze, że




\begin{quote}
    Owa teologia (negatywna) utrzymuje, że nie możemy stwierdzić niczego
pozytywnego o Bogu i jego własnościach. Powinniśmy powstrzymać się od
stwierdzeń pozytywnych i ograniczyć się do takich zdań, jak: „Nie wiem,
jaki Bóg jest, ani jaki nie jest…”. Według tego rodzaju teologii, luka
poznawcza wyłaniająca się z takich stwierdzeń jest w wystarczającym
stopniu wypełniona poprzez przekonanie rozumiane jako wiara religijna.
Jeśli wierzymy, nie musimy przejmować się oczywistymi sprzecznościami w
zbiorze zdań teologicznych. [..] Niewątpliwie, logika nie spełnia
zadniej istotnej roli teologii negatywnej, która nie jest szczególnie
zainteresowana argumentami.\footnote{J. Woleński, Theology and Logic,
[w:] Logic in Theology, red. B. Brożek et. al, Copernicus Center Press,
Kraków 2013, ss. 11-12.}
\end{quote}




W tej pracy przyjmuję odmienne stanowisko. Uważam, że także teologia
negatywna może być badana za pomocą narzędzi formalnych a wynik tego
badania może być owocny i ciekawy filozoficznie. Mimo, iż zadanie
formalizacji tego typu rozważań na pierwszy rzut oka nie wygląda zbyt
obiecująco, istnieją autorzy, którzy uznają, że teologia negatywna
posiada pewną teoretyczną (a nie tylko duchową) wartość a w literaturze
można spotkać co najmniej kilka prób zachowania spójności tej teorii.
Niektóre z tych prób wykorzystują narzędzia formalne. Wedle mojej
wiedzy, pierwszym logikiem, który próbował zmierzyć się z doktryną
teologii apofatycznej był o. Józef Maria Bocheński. Ostatecznie jednak
nie podołał on sprzecznościom, w które jest ona uwikłana i konkludował,
że należy ją odrzucić\footnote{J.M. Bocheński, Logika religii. tłum. S.
Magala, [w:]: J. M. Bocheński. Logika i filozofia. Wybór pism, PWN,
Warszawa 1993, s. 325--468. }. Niektórzy autorzy -- tacy jak np.
John J. Jones -- próbowali obronić spójność teologii apofatycznej nie
używając do tego środków formalnych\footnote{Zob. J.J. Jones, Sculpting
God: The Logic of Dionysian Negative Theology. „Harvard Theological
Review” nr 89 (1996), ss. 355–371. }.W mojej opinii, najbardziej
udaną i najciekawszą próbą logicznej analizy tego typu myślenia
teologicznego jest artykuł Pawła Rojka, „Logika teologii
negatywnej”\footnote{P. Rojek, dz. cyt. }. Przedstawia on w nim
typologię różnych interpretacji teologii Pseudo-Dionizego Areopagity i
próbuję dopasować do nich modele formalne. Praca ma charakter szkicowy
i przygotowawczy. Może ona jednak stanowić dobry punkt wyjścia dla
dalszych rozważań.


\clearpage
\section{Bocheńskiego analiza teologii negatywnej}
\subsection{Teoria tego, co niewysłowione}
Bocheński swoje rozważania na temat teologii negatywnej zawarł w dziele
\textit{Logika religii}\footnote{J.M. Bocheński, dz. cyt., s.~416-418.
 }. Przede wszystkim, odróżnia on teologię negatywną od teorii
tego, co niewysłowione. Ta ostatnia głosi, że przedmiotu religii nie da
się wyrazić, a zatem cały dyskurs religijny jest pozbawiony znaczenia.
Według wielu komentatorów teoria ta jest wewnętrznie sprzeczna. Z
reguły argumentacja za taką tezą polega na wskazaniu, że twierdząc, że
nie da się niczego powiedzieć o Bogu, teoria ta sama coś o Nim mówi, a
zatem jest sprzeczna i należy ją odrzucić. Bocheński występuje
przeciwko takiemu przedstawianiu teorii tego, co niewysłowione.
Twierdzi on, że da się ją uratować od sprzeczności, lecz nawet mimo
tego, nie odpowiada ona potrzebom dyskursu religijnego\footnote{Zob.
J.M. Bocheński, dz. cyt., ss. 353-356. }.

Bocheński uważa, że jeśli przestrzega się pewnych obowiązujących w
logice konwencji, zarzut sprzeczności stawiany teorii niewysłowionego
przestanie obowiązywać. Należałoby wpierw dowieść, że danym układzie
odniesienia teoria ta prowadzi do sprzeczności, tymczasem nikt takiego
dowodu nie przedstawił. Według Bocheńskiego jest zupełnie przeciwnie –
nietrudno wykazać, że teoria tego, co niewysłowione jest spójna.
Poniżej przedstawię jego argumentację.

Załóżmy, że dwuargumentowy predykat  $Nw(x,l)$ oznacza „$x$ jest
niewyrażalne w języku $l$”.

Zapiszmy teraz formułę zawierającą ten predykat

\begin{equation}
\exists x \exists l Nw(x, l)
\end{equation}

Wydaje się, że nie tylko można ją wypowiedzieć nie popadając w
sprzeczność, lecz także jest ona prawdziwa, nietrudno znaleźć taki
obiekt $x$ i taki język $l$, które spełniałyby zapisany wyżej warunek.
(Bocheński podaje przykład krowy i języka szachów: nie da się opisać
krowy w języku szachów).

Możemy powyższy przykład uogólnić i sformułować metajęzykową definicję
Boga o następującej postaci:

\begin{equation}\label{boch}
\forall l Nw(a, l).
\end{equation}

Na pierwszy rzut oka, wydaje się, ze ta formuła jest bardziej
problematyczna -- twierdzenie, że $x$ jest niewysłowione w żadnym języku
zdaje się prowadzić do sprzeczności. Można jednak uniknąć tego
problemu, stosując zwykłe konwencje wykorzystywane do pozbywania się
antynomii semantycznych. Należy założyć, że żadne zdanie traktujące o
pewnej klasie języków, nie jest formułowane w żadnym z tych języków.
Aby było pozbawione sprzeczności musi zostać sformułowane w innym
języku, czyli odpowiednim metajęzyku. Możemy więc założyć, że klasa
języków wspominana w \ref{boch} jest klasą języków przedmiotowych. W takim
wypadku \ref{boch} jest zdaniem metajęzyka pierwszego stopnia. Po takim
zabiegu, sformułowana definicja jest znacząca i pozbawiona
sprzeczności. Nie ma bowiem niespójności w twierdzeniu, że coś nie daje
się wysłowić w jakimś języku, lub nawet w klasie języków, o ile
twierdzenie to jest w języku nienależącym do tej klasy. Według
Bocheńskiego, przy takim założeniu, standardowym z punktu widzenia
logiki ogólnej, teoria tego, co niewysłowione pozostaje znacząca i
spójna a zarzut sprzeczności zostaje oddalony.

Bocheński odrzuca jednak teorię niewysłowionego z co najmniej dwóch
powodów. Po pierwsze, na mocy \ref{boch} nie można przypisać Bogu
jakiekolwiek własności językowo przedmiotowej. Jedyną własnością, jaką
możemy mu przypisać, jest metajęzykowa własność bycia niewysłowionym w
żadnym z języków przedmiotowych. W takim wypadku wierny, nie mógłby
akceptować żadnego zdania dyskursu religijnego, które przypisałoby Bogu
jakąkolwiek własność-przedmiotowo-językową. Wydaje się to niespójne z
faktycznym dyskursem religijnym. Po drugie, niemożliwe byłoby oddawanie
czci obiektowi, o którym wiemy tylko i wyłącznie, że nie można o nim
nic powiedzieć. Jeśli wierny miałby czcić obiekt pozbawiony własności
przedmiotowo-językowych, równie dobrze tym obiektem mógłby nie być Bóg
a szatan\footnote{Por. J.M. Bocheński, dz. cyt., ss. 354-356. }.


\subsection{Logika teologii negatywnej}
Według teologii negatywnej nie jest tak, że dyskurs religijny nic nie
znaczy, jednakże jakiekolwiek ma on znaczenie, ma je na drodze czystej
negacji. Jak zauważa Bocheński, żaden ze zwolenników teologii
negatywnej nie starał się jej sformułować dostatecznie precyzyjnie a
przez swoją niejasną postać prowokuje ona do krytyki. Większość uwag
krytycznych przedstawionych przez Bocheńskiego wskazuje na paradoksalny
charakter tej teorii.

Po pierwsze, mając wyrażenie „$x$ jest niebiałe” i stosując je w postaci
negacji do przedmiotu religii otrzymamy negację niebiałości. Będzie to
oznaczać, że przedmiot religii jest biały, a zatem otrzymamy własność
całkowicie pozytywną. Po drugie, jeśli dopuścimy do tego, by o
przedmiocie religii orzekać wszystkie negacje, popadniemy w
sprzeczność. Możemy bowiem mu przypisać własność bycia niebiałym (czyli
negację własności bycia białym) oraz własność bycia nie-niebiałym
(czyli negację bycia niebiałym), a zatem także własność bycia biały, o
ile utrzymujemy silne prawo podwójnej negacji.

Według Bocheńskiego można uniknąć tych sprzeczności, gdy ograniczy się
zakres teorii do klasy własności pozytywnych. Wpierw jednak należałoby
takie własności zdefiniować, jednakże brak jest dostatecznie
precyzyjnych definicji tego rodzaju. By uchronić teologię negatywną od
wymienionych sprzeczności a zarazem nadać jej trochę precyzji,
Bocheński proponuje następującą, indukcyjną definicję własności
pozytywnych:


\begin{enumerate}
\item Własność postrzegana bezpośrednio jest własnością pozytywną.
\item Własność definiowana za pomocą formuły zawierającej wyłącznie
symbole własności pozytywnych i terminów logiki pozytywnej jest
własnością pozytywną.\footnote{Tamże, s.~416. }
\end{enumerate}
Od razu jednak dodaje, że niniejsza definicja nie jest zadawalająca z
dwóch powodów. Po pierwsze, jest ona za szeroka -- klasa własności
pozytywnych ograniczona jest dużo surowiej, niż wymagaliby tego
zwolennicy teologii apofatycznej. Po drugie, pojęcie własności
postrzeganej bezpośrednio jest bardzo nieścisłe. Posługując się
przykładem Bocheńskiego -- dlaczego nie można by postrzegać
bezpośrednio, że krowa nie jest niebieska? Dodaje on, że przedstawione
powyżej problemy nie są najpoważniejszymi z tych które nękają teologię
negatywną. Dlatego, na potrzeby dalszych rozważań zakłada on, że
pojęcie własności pozytywnej zostało poprawnie zdefiniowane.

Bocheński próbuje podać znaczenie tej teorii w sposób formalny w taki
sposób, który zadowoliłby jej zwolenników. Niech $t$ będzie terminem
dyskursu świeckiego. Możemy wówczas utrzymywać, że znaczenie $t$ nie
stosuje się do przedmiotu religijnego. Zapis $M(t, \pi, \phi)$
oznacza „$t$ jest terminem występującym w dyskursie
świeckim w znaczeniu $\phi$”.
$\alpha$ natomiast oznacza klasę własności świeckich. Treść zdań wiary
przybierze zatem następującą postać:

\begin{equation}\label{boch2}
   M(t, \pi, \phi) \land
\phi \in \alpha
\to \neg \phi (PR),
\end{equation}
gdzie $PR$ oznacza przedmiot religii, czyli Boga. Wówczas dyskurs
religijny zawiera tylko twierdzenia, które orzekają o Bogu negację
własności pozytywnych wyrażone w terminach świeckich. Dzięki
ograniczeniu teorii do klasy własności pozytywnych, tak sformułowana
teologia negatywna nie zawiera sprzeczności. W przeciwieństwie do
formalizmu użytego w teorii tego, co niewysłowione nie jest to własność
metajęzykowa, lecz własność przedmiotowo-językowa drugiego stopnia.
Jednakże, podobnie jak w teorii tego, co niewysłowione, nie możemy
przypisać Bogu żadnej własności przedmiotowo-językowej pierwszego
stopnia. Z tego powodu dzieli ona te same te same wady, co teoria
niewysławialnego. Nawet, jeśli jest niesprzeczna, nie jest zgodna z
dyskursem religijnym i praktyką religijną. Nie można czcić czegoś,
czemu nie można przypisać żadnej własności pozytywnej. Z tego powodu
powinna zostać odrzucona\footnote{Zob. tamże s.~417-418. }.

\clearpage
\section{Logika teologii negatywnej według Gellmana}

Jerome I. Gellman\footnote{Zob. J.I. Gellman, The Meta-Philosophy of Religious Language,
"Nous", nr 11 (1971), ss.~151--161.}
 ocenia adekwatność różnych analiz znaczenia języka religijnego.
Uważa on, że analizy te powinny być uporządkowane nie ze względu na to,
jak wiele z języka religijnego mogą objąć, lecz ze względu na to,
których jego elementów dotyczą. Zdania języka religijnego nie mają
równego statusu. Jedne z nich są bardziej istotne od drugich. Np. w
przypadku tradycji chrześcijańskiej, najważniejszym fragmentem języka
religijnego są teksty biblijne i pisma Ojców Kościoła. Na tej podstawie
Gellman wyróżnia on dwa rodzaje problemów teologicznych: zewnętrzne i
wewnętrzne. Przykładem pierwszego jest niezgodność twierdzenia o
wszechwiedzy Boga (w szczególności wiedzy dotyczącej przyszłości) z
tezą o wolnej woli. Zagadnienie to nie pojawia się w podstawowych
tekstach chrześcijańskich czy judaistycznych -- zostało wypracowane
dopiero przez filozofów scholastycznych, którzy te teksty komentowali.
Przykładem religijnego problemu wewnętrznego jest twierdzenie o
wszechmocy, wszechwiedza i dobroci Boga w obliczu tego, że na świecie
istnieje niesprawiedliwość. Według Gellmana każda adekwatna analiza
języka religijnego nie powinna usuwać wewnętrznych problemów danej
religii.

Gellman rozważa krótko kazus teologii negatywnej. Kojarzy ją jednak
wyłącznie z teologami żydowskimi i arabskimi\footnote{Wśród podanych
przez niego przykładów teologów negatywnych pojawia się Mojżesz
Majmonides, Abraham Ibn Daud oraz Bahya Ibn Pakuda. }. W jego
rozważaniach teologia negatywna jest teorią, w której jakikolwiek
predykat P języka „skończonych bytów” nie może być przedziwie orzekany
o Bogu. Na potrzeby tych rozważań Gellman definiuje zakres gatunkowy
predykatu P jako dziedzinę obiektów, o których można orzec P albo jego
dopełnienie. W świetle tej definicji, główną tezą teologii negatywnej
jest, że Bóg nie należy do zakresu żadnego z predykatów naszego języka.
Jeśli mówimy na przykład, że Bóg nie jest mądry, mamy na myśli raczej
negacje \textit{wykluczającej}, niż negację \textit{wyboru}. Naszym
zamiarem stwierdzenie tylko, że to nieprawda, że predykat P przysługuje
Bogu, niekoniecznie sugerując jednocześnie, że można o nim orzec
dopełnienie tego predykatu. Bóg jest poza jego gatunkowym zakresem, to
znaczy, że nie należy do zbioru obiektów, o których można orzec $P$ lub
nie-$P$. Gellman dodaje, że jeśli traktujemy istnienie jako kwantyfikator
a nie jako predykat, stwierdzenie istnienia Boga polega na stwierdzeniu
istnienia obiektu poza rodzajowym zakresem wszystkich naszych
predykatów. Zdanie to można zapisać następująco:

\begin{equation}\label{gellman}
    \exists x \forall P \neg
(P(x) \lor \neg P(x))
\end{equation}


Według Gellmana, zdanie „Bóg jest potężny” teolog negatywny zrozumie
jako negację dopełnienia predykatu „jest potężny”. W innym kontekście
oznaczałoby to przypisanie obiektowi, o którym mowa, tej właśnie
własności. Jednakże w przypadku Boga negowanie dopełnienia predykatu P
nie oznacza przypisywania mu P, ponieważ dopełnienie dane jest innym
rodzajem negacji -- negacją wykluczającą. Mówiąc ogólnie, zdania języka
religijnego negują dopełnienia wszystkich wymienianych przez nie
własności Boga, który jest poza zakresem wszystkich naszych predykatów.
One, z kolei -- będąc predykatami języka skończonych bytów -- z
konieczności muszą oznaczać niedoskonałe własności.

W końcu, Gellman odrzuca teologię negatywną, jako teorię nieadekwatną.
Po pierwsze, ze względu na wynikającą z niej niepoznawalność Boga. Po
drugie, dlatego, że w ramach tej teorii nie można postawić wewnętrznych
problemów teologicznych. Na przykład mówienie o wszechmocy,
wszechwiedzy i dobroci Boga -- w interpretacji Gellmana -- oznacza
jedynie, iż nie przypisujemy mu takich własności, jak słabość, głupota
czy moralna niedoskonałość. Nie twierdzimy przy tym, że można o nim
orzekać takie predykaty, jak moc wiedza czy dobroć. W takim wypadku nie
ma mowy o problemie wynikającym z obecnej w świecie niesprawiedliwości.

\clearpage
\section{Teologia apofatyczna w rozważaniach Jonesa}

Jones, podobnie jak Rojek, analizuje bezpośrednio teksty
Pseudo-Dionizego Areopagity. Zadaniem, jakie sobie stawia, jest
stawienie czoła sprzecznościom uwikłanym w jego doktrynę. Analizuje on
fragmenty różnych prac Dionizego po to, by właściwie zinterpretować
niejasne wyrażenia zawarte w \textit{Teologii mistycznej} i odsłonić
logiczną strukturę jego apofatycznej wykładni. Sugeruje on, że głównym
celem Dionizego było zaprzeczenie, że Bóg należy do kategorii bytów.
Próbuje też wskazać, że wbrew powszechnemu mniemaniu, teologia
negatywna nie jest teorią logicznie spójną.

Według Jonesa, teologia dionizyjska jest w dużej mierze teologią
krytyczną. Polemizuje ona z błędnym sposobem mówienia o Bogu -- takim,
który traktuje Go jak inne byty, czyli rzeczy lub pojęcia. W
\textit{Teologii mistycznej} Dionizy pisze:

\begin{quote}
    that is to say, to those caught up with the things of the world, who
imagine that there is nothing beyond instances of individual being and
who think that by their own intellectual resources they can have a
direct knowledge of him who has made the shadows his hiding place. And
if initiation into the divine is beyond such people, what is to be said
of those others, still more uninformed, who describe the transcendent
Cause of all things in terms derived from the lowest orders of being,
and who claim that it is in no way superior to the godless, multiformed
shapes they themselves have made?\footnote{Pseudo-Dionizy Areopagita,
Teologia Mistyczna, rozdział I.}
\end{quote}


Według Areopagity, bałwochwalcy mylą Boga z przedmiotami, zaś inni
„niedoinformowani”, prawdopodobnie środkowi platonicy, z pojęciami. W
innym tekście próbuje przedstawić, jak ci ostatni mogliby krytykować
wykorzystywanie materialnych obrazów do przestawienia Boga, preferując
raczej łączenie Boga z pojęciami:


\begin{quote}
    It could be arguedt hat if the [scripturew riters wantedt o
givecorporeal form to what is purely incorporeal, they. . . should have
begun with what we would hold to be noblest, immaterial and
transcendent beings [for instance, Word and Mind].\footnote{Tenże,
Hierarchia niebiańskia, rozdział II.}

Now these sacreds hapesc ertainlys how morer everencea nd seem vastly
superior to the making of images drawn from the world. Yet they are
actually no less defective than this latter, for the Deity is far
beyond every manifestationo f being and of life. . . every reason or
intelligence falls short of similarity to [the Deity].\footnote{Tamże.}
\end{quote}

Dionizy zgadza się z filozoficznym podejściem, wedle którego materialne
obrazy nie mogą przedstawiać boskiej istoty. Jednakże, odrzuca on takie
rozwiązanie, wedle którego lepszym sposobem przedstawiania Boga są
pojęcia. Jak wskazuje Jones, zarówno przedmioty, jak i pojęcia nie
wystarczają do opisania dionizyjskiego Boga z tego samego powodu -- jest
On ponad wszelkim bytem.

Fakt, że Bóg przekracza wszelki byt, nadaje strukturę językowi dyskursu
teologicznego. Nadawanie Bogu jakichkolwiek przymiotów przysługujących
bytom jest bowiem z gruntu błędne. W języku naturalnym, gdy ktoś powie
„$x$ jest białe”, odbiorca tej wiadomości zrozumie także, że $x$ nie jest
czerwone. Przypisywanie jakiemuś obiektowi danych własności jest
jednocześnie zaprzeczeniem, że posiada on pewne inne własności.
Podobnie, w drugą stronę, gdy ktoś powie, że $x$ nie jest czerwone,
odbiorca może zakładać, że $x$ posiada inne własności -- jest białe,
przeźroczyste, lub niewidzialne, lecz np. słyszalne. Założy zatem, że
istnieje pewna charakterystyka tego obiektu, można mu przypisać pewne
własności mimo, że nie będzie wiedział, jakie własności mu rzeczywiście
przysługują. Każdy przedmiot posiada jakąś -- taką a nie inną –
charakterystykę. Zwykle, gdy mówimy o rzeczach, twierdzenia i
przeczenia sprzeciwiają się sobie. Według Dionizego nie dzieje się tak
w przypadku Boga. Bóg nie jest jednym z bytów, zatem język służący do
opisu bytów nie jest dla Niego właściwy.

W \textit{Imionach Boskich} Dionizy pisze:


\begin{quote}
    He is all things since he is the Cause of all things. But he is also
superior to them all because he precedes them and is transcendentally
above them. Therefore every attribute may be predicated of him, and yet
he is not any thing.\footnote{Tenże, Imiona boskie, rozdział V.}
\end{quote}



W języku teologicznym Arepagity twierdzenia i zaprzeczenia należą do
odmiennych grup, tworząc odmienne sposoby mówienia o Bogu. Ponieważ
funkcjonują one w odmienny sposób, nie należy ich ze sobą mieszać. Te
pierwsze przedstawiają Boga jako przyczynę wszystkiego, te drugie
wyrażają jego transcendencję. Oba sposoby mówienia można stosować naraz
zarówno do opisu Boga, jak i opisu przedmiotów, jednakże w ten sposób
nie zdołamy wyrazić unikalności Boga -- tego, że jest czymś odrębnym od
wszystkich bytów.


\subsection{Twierdzenia}

W celu uniknięcia stosowania języka, który nie odzwierciedla
wyjątkowości Boga, można próbować każde twierdzenie o Bogu
interpretować w taki sposób, by nie wynikało z niego żadne
zaprzeczenie. Na przykład, można zestawić kilka twierdzeń, które w
języku naturalnym nie mogą służyć do opisania żadnego z bytów. Ten
sposób mówienia charakteryzują różnorodne imiona nadawane Bogu w
\textit{Imionach Boskich}, takie jak „moc sama w sobie” czy „prawda”.
Skoro połączenie tych określeń w sposób oczywisty nie może odnosić się
do żadnego z bytów, nadaje się on do wyróżnienia Boga spośród bytów.
Dionizy nazywa to teologią pozytywną. Według Jonesa, w orzeczeniach
tego typu twierdzeń -- na przykład w twierdzeniu „Bóg jest prawdą” –
słowo „ jest” występuje w sensie metaforycznym. Bóg jednocześnie
posiad, jaki i nie posiada przypisywane mu w orzeczniku własności, w
zależności od kontekstu, którym to zdanie jest użyte. Jak wskazuje
Jones, ten podwójny sens -- tożsamość i odmienność -- wynika z roli, jaką
twierdzenia odgrywają w wyrażaniu boskiej przyczynowości. Dla
Dionizego, tak samo jak dla grackich neoplatoników, przyczyna jest
jednocześnie immanentna, jak i odrębna względem swojego skutku. Z tego
powodu twierdził on, że twierdzenia zawarte w teologii pozytywnej są
metaforycznym sposobem wyrażania odmienności Boga od wszelkich bytów.


\subsection{Zaprzeczenia Jednostkowe oraz Zaprzeczenia Wszystkich Bytów}

Według Jonesa, zaprzeczenia w doktrynie Dionizyjskiej z logicznego
punktu widzenia są bardziej wymagające od twierdzeń.


\begin{quote}
    Gdy ktoś stwierdzi, że Bóg jest mocą i prawdą, unika w ten sposób
pomylenia go z przedmiotami i pojęciami, ponieważ żadne z pojęć i
przedmiotów nie jest jednocześnie mocą i prawdą. Gdy jednak ktoś
oznajmi, że Bóg nie jest ani mocą ani prawdą, tak naprawdę nie wykluczy
w ten sposób wiele: Bóg wciąż może być „lwem” albo „pijakiem lub też
wieloma innymi rzeczami.\footnote{J.J. Jones, Sculpting God: The Logic
of Dionysian Negative Theology, „Harvard Theological Review”, 89
(1996), s. 360. Imiona “lew” oraz “pijak” są przykładami zaczerpniętymi
od Pseudo-Dionizego, zob. Pseudo-Dionizy Arepagita, Imiona Boskie,
Rozdział V oraz VII.}
\end{quote}



Zdaniem Jonesa, strategią, jaką przyjął Dionizy by odróżnić Boga od
bytów, jest używanie wzajemnie sprzecznych zaprzeczeń o Bogu -- takich,
które nie mogą być jednocześnie prawdziwe, gdy orzekamy je o
jakimkolwiek bycie. Z tego powodu w ostatnim rozdziale \textit{Teologii
mistycznej} Areopagita pisze, że Bóg ani nie jest żywy, ani nie
pozostaje bez życia itp. To właśnie taki sposób mówienia wydaje się
nietypowy, nadje on teologii Dionizego jej paradoksalny charakter i w
sposób oczywisty łamie prawo wyłączonego środka.

Dionizy często powtarza, że Bóg jest zaprzeczeniem wszelkich bytów.
Dlaczego zatem pod koniec \textit{Teologii mistycznej} stwierdza, że
jest On także ponad wszelkim zaprzeczeniem? %[TBC]


\clearpage

\section{Logika teologii apofatycznej według Rojka}

\subsection{Cztery tezy apofatycznej doktryny Pseudo-Dionizego Areopagity}

Przedmiotem analiz Rojka są pisma Pseudo-Dionizego Areopagity, głównie
krótki traktat -- \textit{Teologia Mistyczna}, uważany przez wielu
komentatorów za jedno z najważniejszych źródeł teologii negatywnej.
Pierwszy etap analiz polega na wydobyciu z tego tekstu podstawowych tez
doktryny Dionizego. Oto pierwszy fragment z początku traktatu:

\begin{quote}
    Należy Jej, jako przyczynie wszystkiego, (T1) przypisać i o Niej
stwierdzić to wszystko, co się mówi o bytach; a bardziej dokładnie,
(T2) winno się zaprzeczyć tym wszystkim twierdzeniom, jako że Ona w
swej nadsubstancjalności jest ponad wszystkim. Nie należy jednak
sądzić, że w Jej przypadku zaprzeczenia i twierdzenia sprzeciwiają się
sobie.\footnote{Pseudo-DionizyAreopagita., dz. cyt, rozdział I, 2.
Oznaczenia w nawiasach pochodzą od P. Rojka.}
\end{quote}





Fragment ten zawiera dwie podstawowe tezy teologii apofatycznej w ujęciu
Areopagity. Według pierwszej z nich, o Bogu można w sposób uprawniony
twierdzić wszystko, tzn. jest on podmiotem wszelkich twierdzeń. Druga –
przeciwnie -- głosi, że można o Nim wszystko przeczyć, tzn. jest On
podmiotem wszelkich przeczeń.

Według Rojka twierdzenie oznacza orzekanie o czymś pewnych pozytywnych
własności\footnote{P. Rojek, dz. cyt., s. 218. }. Jego
interpretacja powoduje jednak poważny problem. W takim wypadku
należałoby bowiem wyraźnie zdefiniować pojęcie własności pozytywnej.
Rojek, mimo iż dwukrotnie powraca do tego zagadnienia\footnote{Tamże,
ss. 221, 226. }, nie podaje żadnego kryterium wyróżniania zbioru
takich własności. Jest to dość kontrowersyjny i wyraźnie najsłabszy
punkt jego rozważań. Więcej miejsca temu problemowi poświęcimy w
dyskusji (rozdział [reference]), na razie pierwszą tezę sformułujmy za
autorem:


\bigskip

\noindent (T1) Bóg ma wszystkie własności pozytywne\footnote{Wszystkie tezy
podaję w dosłownym sformułowaniu. }.

\bigskip

Druga teza zdaje się być jej prostym przeciwieństwem:


\bigskip

\noindent (T2) Bóg ma negację wszystkich własności pozytywnych.


\bigskip

Obie tezy są dobrze ugruntowane w doktrynie Areopagity. Podobne
sformułowania można odnaleźć w innych jego pismach. Co ciekawe,
zazwyczaj obie występują w swoim towarzystwie, jedna po drugiej.  Np. w
\textit{Imionach Boskich} pisze, że

\begin{quote}
    [Bóg] jest wszystkim jako przyczyna wszystkiego […] (T1) wszystko na raz
można o Nim twierdzić, (T2) choć On nie jest żadną rzeczą z tego
wszystkiego, co jest: (T1) posiada każdy kształt i każda formę i (T2)
jest bezpostaciowy i pozbawiony pięknej formy\footnote{Pseudo-Dionizy
Areopagita, Imiona Boskie, Rozdział V, 8 [w:]  Pisma teologiczne, tłum.
M. Dzielska, Wydawnictwo Znak, Kraków 2005. Oznaczenia w nawiasach
pochodzą ode mnie.}.
\end{quote}


Według Dionizego, zgodnie z (T2) nasze poznanie Boga polega na „negacji
wszystkiego, co istnieje”\footnote{Tamże, Rozdział 1, 5. }.
Dionizy wychodzi jednak ponad podobne sformułowania. W ostatnim
rozdziale \textit{Teologii mistycznej} stwierdza, że:

\begin{quote}
    [Bóg] (T2)\footnote{Rojek oznaczył ten fragment jako (T1). } nie
jest ani duszą, ani intelektem, ani wyobrażeniem, ani mniemaniem, ani
rozumem, ani pojmowaniem, ani słowem […]; (T4) nie może być nazwany ani
pojęty; […] (T3) nie jest bez ruchu ani w ruchu, ani nie odpoczywa, […]
(T4) nie można Go objąć ani intelektem, ani wiedzą; nie jest ani prawdą
[…], ani jednym, ani jednością, (T2) ani Boskością, ani dobrocią, […]
(T3) nie jest też niczym z niebytu ani czymś z bytu; […] (T4) nie
istnieje ani słowo, ani imię, ani wiedza o Nim; (T3) nie jest ani
ciemnością, ani światłością, ani błędem, ani prawdą; nie można o Nim
(T3) niczego zaprzeczać ani (T1)\footnote{Oznaczenie tego fragmentu
jako (T1) również wydaje się niewłaściwe. Należałoby } nic pewnego
twierdzić, bo twierdząc o Nim lub zaprzeczając rzeczy niższego rzędu,
nic o Nim ani nie stwierdzamy, ani nie zaprzeczamy. Ta najdoskonalsza i
jedyna Przyczyna wszystkiego jest bowiem (T2) ponad wszelkim
twierdzeniem i (T3) ponad wszelkim zaprzeczeniem: wyższa nad to
wszystko, całkowicie niezależna od tego wszystkiego i przekraczająca
wszystko.\footnote{Pseudo-Dionizy Areopagita, Teologia Mistyczna, dz.
cyt., Rozdział V. }
\end{quote}




W powyższym cytacie można odnaleźć wystąpienia wyróżnionych wcześniej
tez. Dionizy, zgodnie z (T2), przeczy bowiem, że Bóg posiada
przypisywane mu zwykle własności -- nie jest ani duszą, ani intelektem,
ani wyobrażeniem. Stwierdza też, że Bóg nie posiada własności bycia
bogiem (Bóg nie jest boskością), co -- jak zostało wcześniej zauważone –
wydaje się być twierdzeniem wewnętrznie sprzecznym. Ratunku dla takiego
sformułowania można szukać w uznaniu, że słowo „Bóg” występuje w innym
znaczeniu w podmiocie i orzeczniku tego zdania lub -- co później
wykorzystuje Rojek -- negacja użyta w tym zdaniu nie jest negacją
klasyczną. W końcu Areopagita konstatuje, że nie możemy i Bogu nic
twierdzić, ponieważ jest On ponad wszelkim twierdzeniem.

Jak twierdzi Rojek, Dionizy jednak nie poprzestaje na tych tezach i
twierdząc na przykład, że Bóg „nie jest bez ruchu”, „nie jest niczym z
niebytu” i jest „ponad wszelkim zaprzeczeniem” idzie o krok dalej. W
przypadku Boga, zaprzeczenia nie dotyczą tylko pozytywnych własności,
lecz także odnoszą się do samych siebie. Dionizy nie tylko przeczy, że
Bóg posiada pozytywne własności, zaprzecza także, że ma On
własnościnegatywne. Według Rojka podobne sformułowania pozwalają
sądzić, że kolejną tezą doktryny Areopagity jest:


\bigskip

\noindent (T3) Bóg ma negacje wszystkich negacji pozytywnych własności.


\bigskip

Podobnie, jak pierwsze dwie tezy, także powyższa nie jest odosobnionym
wtrąceniem i można ją wydobyć także z innych fragmentów traktatu.

Ostatnią proponowaną przez Rojka tezą teologii apofatycznej jest:


\bigskip

\noindent (T4) Bóg jest niepoznawalny.


\bigskip

Dionizy w wielu innych swoich dziełach powtarza, że Bóg nie może być
poznany, nazwany, pojęty, wyrażony słowem oraz jest pozbawiony
imienia\footnote{Por. P. Rojek, dz. cyt., s. 220. }.

Wydobyte przez Rojka tezy dionizyjskiej doktryny zdają się być cokolwiek
problematyczne. Tezy (T1) i (T2) na pierwszy rzut oka wydają się być
wzajemnie sprzeczne.Sprzeczności, w które uwikłana jest teologia
negatywna są głównym powodem, dla którego wielu filozofów i teologów
odmawia jej jakiekolwiek wartości teoretycznej (przyznając jednocześnie
pewną wartość duchową, mistyczną). Jednakże niektórzy z nich przechodzą
nad tymi sprzecznościami do porządku dziennego twierdząc, że są one
niezbywalną własnością istoty boskiej. Orędownikiem takiego sposobu
myślenia był Mikołaj z Kuzy. Jest on autorem idei Boga jako
\textit{coincindentiaoppositorum} -  zbieżności przeciwieństw. Pomysł
ten wydaje się konsekwentnym rozwinięciem takiej doktryny teologicznej,
która dopuszcza sprzeczności w Bogu.

Można jednak bronić teologii apofatycznej jako teorii Boga w inny
sposób. Strategią wielu popularnych interpretacji tego rodzaju myślenia
teologicznego jest przyjęcie (T2) jako głównej tezy i odrzucenie tezy
(T1). Często mówi się o teologii apofatycznej jako o doktrynie wedle
której o Bogu można mówić tylko w kategoriach negatywnych -- jaki nie
jest. Sam Dionizy uznawał (T2) jako właściwszy sposób mówienia o bogu.
W takim wypadku ruch polegający na zignorowaniu (T1) wydaje się być
dopuszczalnym i wygodnym sposobem na ominięcie sprzeczności. O wiele
bardziej przekonywająca byłaby interpretacja obejmująca wszystkie
zaproponowane tezy.

Innym wyjęciem z tego problemu byłoby takie zinterpretowanie tych tez,
by teoria, które je zawiera zachowała spójność. Podobne rozwiązanie
sugeruje sam tekst Dionizego -  pisze on, że:

\begin{quote}
    Nie należy jednak sądzić, że w Jej [tj. przyczyny wszystkiego -- P.U]
przypadku zaprzeczenia i twierdzenia sprzeciwiają się
sobie.\footnote{Pseudo-Dionizy Areopagita., Teologia mistyczna,
rozdział I, 2. }
\end{quote}




Pozwala to sądzić, że rozumiał on negację w inny, nieklasyczny sposób.
Trop ten jest główną przesłanką w interpretacji proponowanej przez
Rojka.

Podobny problem dotyczy (T3) -- jest ona sprzeczna z (T2) i, zgodnie z
silnym prawem podwójnej negacji, sprowadza się do (T1).Ponadto, jak
zauważa Rojek, pewne sformułowania tej tezy, takie jak „[Bóg] nie jest
bez ruchu ani w ruchu”, „nie jest też niczym z niebytu ani czymś z
bytu”, sugerują, że Dionizy odrzucał także prawo wyłączonego środka.
Rozwiązaniem także tego problemu ma być odmienna, nieklasyczna
interpretacja spójnika negacji, tak, by klasyczne prawo podwójnej
negacji przestało obowiązywać.

Teza (T4) mówi o niepoznawalności Boga. Można ją jednak rozumieć słabiej
– jako tezę o niewyrażalności boskiej istoty. Dionizy nie rozróżniał
między niepoznawalnością i niewyrażalnością. Poza tym, umieszczanie
takiego sformułowania w traktacie o bogu wydaje się cokolwiek dziwne.
Można traktować je jako emfatyczny dodatek. Rojek podaje jednak
interpretacje teologii negatywnej, które zawierają także (T4).




\subsection{Typologia interpretacji teologii apofatycznej}

Według Rojka, do zbioru zdań teologii negatywnej należą cztery
następujące tezy wyabstrahowane z \textit{Teologii mistycznej}
Pseudo-Dionizego Areopagity:


\bigskip

\noindent (T1) Bóg ma wszystkie własności pozytywne,

\noindent (T2) Bóg ma negację wszystkich własności pozytywnych,

\noindent (T3) Bóg ma negacje wszystkich negacji pozytywnych własności,

\noindent (T4) Bóg jest niepoznawalny.


\bigskip

Mimo problemów, jakie nękają teorię sformułowaną za pomocą powyższych
tez, próbuje on podać kilka jej spójnych modeli. Strategie, jakie
przedstawia polegają na przyjęciu jednej z tez jako podstawowej i
odrzuceniu bądź reinterpretacji pozostałych.

Pierwsza z proponowanych przez Rojka interpretacji uznaje (T4) za
podstawową tezę teologii negatywnej. Według niej, podstawową ideą tego
rodzaju myślenia teologicznego jest niepoznawalność i niewyrażalność
Boga. Tę interpretację Rojek nazywa \textit{teologią agnostyczną},
ponieważ w jej świetle, nie możemy posiąść żadnej wiedzy o Bogu
–zarówno pozytywnej, jak i negatywnej.Jest ona nazwana w ten sposób w
odróżnieniu od \textit{teologii gnostycznej} obejmującej pozostałe
interpretacje doktryny dionizyjskiej, wedle których możemy posiąść
jakąś wiedzę o Bogu, nawet jeśli posiada ona wyłącznie negatywny
charakter. Interpretacje te traktują (T4) jako odmienną formę wyrażenia
boskiej transcendencji.

Teologie gnostyczne Rojek dzieli na \textit{pozytywne} i
\textit{negatywne}. Te pierwsze przyjmują (T1) jako tezę podstawową i
odrzucają tezy (T2) - (T4), jako przesadne sformułowania mówiące o
boskiej transcendencji. Rozumiana w ten sposób teologia pozytywna nie
może być uważana za zadowalającą interpretację teologii apofatycznej,
ponieważ traci ona jej negatywny charakter. W tych drugich bazową tezą
jest (T2), natomiast pozostałe tezy są odrzucane lub reinterpretowane.

W końcu, teologię negatywną Rojek dzieli względem tego, jak
poszczególnych interpretacjach rozumiana jest negacja. Według niego,
można ją rozumieć w zwykły „negatywny” sposób. Zwykłe rozumienie
negacji obecne jest w interpretacjach nazwanych \textit{negatywną
teologią negatywną}. Rojek jednak proponuje pewne odmienne szczególne
rozumienie negacji, które nazywa „pozytywnym”. Zastosowanie go do
doktryny Pseudo-Dionizego skutkuje powstaniem interpretacji nazwanej
\textit{pozytywną teologią negatywną}\footnote{Stosuję tu nazewnictwo
podane przez Rojka. On sam zdaje sobie sprawę z niezgrabności nadanych
tym interpretacjom etykiet. Zob. P. Rojek, dz. cyt., s. 220. }.
Oczywiście, można również stwierdzić, że teologia apofatyczna jest
zwyczajnie niespójna. Taka interpretacja nie leży jednak w kręgu
zainteresowań Rojka i nie formalizuje on jej w omawianym w niniejszym
rozdziale tekście.


\begin{figure}[h]
{\centering
\includegraphics[width=1\linewidth]{typologia.jpg}
\caption{Proponowana przez Rojka
typologia interpretacji teologii apofatycznej.}
}
\end{figure}

Problemem, któremu Rojek poświęca nieco uwagi, lecz mimo wszystko
postanawia go nie rozstrzygać, jest problem własności pozytywnych. Jest
to problem poważny, ponieważ w świetle przedstawionej powyżej typologii
różnica między teologią pozytywną i negatywną polega na orzekaniu o
najwyższej istocie pozytywnych własności lub ich negacji. Jak
zdefiniować zbiór takich własności? Rojek wymienia kryterium
syntaktyczne, które miałoby polegać na „obecności negacji w
predykacie”\footnote{Tamże, s. 221. }. Nie do końca wiadomo, jak
rozumieć takie kryterium. Przypuszczam, że formalnie można zapisać tę
propozycję w logice predykatów II-rzędu w następujący sposób:

\begin{equation}
    P(Q) {=}_{df} \forall x \neg \exists N (Q(x) \equiv \neg N(x)),
\end{equation}
gdzie $P(Q)$ oznacza „własność $Q$ jest pozytywna”. Jednakże Rojek
natychmiast dodaje, że kryterium syntaktyczne nie może być uważane za
wystarczające. Powołuje się tu na klasyczny przykład predykatu „jest
ślepy” -- nie zawiera on negacji, lecz odnosi się do braku i z tego
powodu wydaje się być pozytywny. Trafniejszym argumentem wydaje się być
powołanie na św. Tomasza, wedle którego własności tradycyjnie
przypisywane bytowi absolutnemu, takie jak prostota, doskonałość czy
jedność, są w istocie własnościami negatywnymi\footnote{św. Tomasz z
Akwinu Teologiczna: I, 3, Summa contra Gentiles 11. }.

Problem własności pozytywnych Rojek pozostawia nierozwiązany tłumacząc,
że skupia się na formie teologii negatywnej, nie na jej treści. Dodaje
jednak, że istnieje jeszcze jeden bardzo szczególny sposób ich
rozumienia. Można je traktować, jako najwyższy sposób istnienia danej
własności, czyli tzw. perfekcje. Najczęściej mówi się o takich
perfekcjach, jak wszechwiedza -- najwyższy sposób wiedzy, czy też
wszechmoc -- najwyższy stopień mocy. Są one istotnym elementem w
ontologicznych dowodach istnienia Boga, które w historii były
proponowane m.in. przez św. Anzelma, Kartezjusza, Leibniza, Gödla i
Perzanowskiego. W tych argumentach Boga można uznać za podmiot
wszelkich własności pozytywnych rozumianych jako perfekcje:

\begin{equation}
    G(x) \equiv \forall Q (P(Q) \to Q(x)).
\end{equation}


Definicję tę Rojek zaczerpnął od Perzanowskiego\footnote{J.
Perzanowski, Ontological Arguments II: Cartesian and Leibnizian, [w:]
red. H. Burkhardt, B. Smith, Handbook of Metaphysics and Ontology, t.
2, PhilosophiaVerlag, München 1991,ss.~625–633. } i stanowi ona
wzór kolejnych definicji Boga w proponowanych przez niego następnie
interpretacjach teologii negatywnej. Warto wiec poświęcić jej nieco
uwagi. Po pierwsze, jest ona zapisana w rachunku predykatów drugiego
rzędu, gdyż zawiera warunek, że własności przypisywane Bogu muszą być
pozytywne. Po drugie, Bóg formalizowany jest jako predykat („$x$ jest
Bogiem”, „$x$ jest bogopodobny”), nie jako stała logiczna\footnote{Por.
choćby M. Durrant, The „Meaning of ‘God’-I, Royal Institute of
Philosophy Supplement,vol. 31 (1992), ss. 71-84. }, lub deskrypcja
określona, co proponował Bocheński\footnote{J.M. Bocheński, dz. cyt.,
s. 381. }. Po trzecie, powyższa definicja powstała na użytek
formalizacji dowodów ontologicznych, które należą raczej do jakiejś
formy teologii pozytywnej. Zawarte w niej $P(Q)$ czytamy jako „$Q$ jest
własnością pozytywną” w pewnym szczególnym sensie opisanym powyżej -- „$Q$
jest perfekcją”. Rojek nie sugeruje jednak, że w teologii apofatycznej
Pseudo-Dionizego zaprzecza się jedynie tego typu własnościom
pozytywnym. Przeciwnie, pisze wprost, że zakres negowanych własności
jest szerszy\footnote{P. Rojek, dz. cyt., s. 222. }. Po raz
kolejny jednak odżegnuje się od określenia, o jakie konkretnie
własności chodzi. Jest to, moim zdaniem, jeden z najsłabszych punktów
jego propozycji. Wrócę do tego ponownie w dyskusji.


\subsection{Agnostyczna teologia negatywna}

Strategią pierwszej przedstawianej przez Rojka interpretacji teologii
negatywnej jest przyjęcie (T4) jako tezy podstawowej i odrzucenie lub
modyfikacja pozostałych tez, za cenę zachowania spójności teorii.
Według tej interpretacji, celem teologii negatywnej jest wskazanie
boskiej transcendencji -- głosi ona, że Bóg jest zasadniczo
niepojmowalny i niewyrażalny.

Takie rozumienie teologii negatywnej jest popularne wśród wielu
komentatorów -- także tych, którzy rozważają logiczno-językową strukturę
tej teorii. Wśród nich Rojek wymienia Michaela Gellmana, Johna J.
Jonesa oraz Paula Rorema.

W rozważaniach Gellmana teologia negatywna jest teorią, w której
jakikolwiek predykat P języka „skończonych bytów” nie może być
przedziwie orzekany o Bogu. Jej główną tezą jest, że Bóg nie należy do
zakresu żadnego z predykatów naszego języka. Jeśli mówimy na przykład,
że Bóg nie jest mądry, mamy na myśli raczej negacją
\textit{wykluczającą}, niż negację \textit{wyboru}. Naszym zamiarem
jest wyłącznie stwierdzenie, że to nieprawda, że predykat P przysługuje
Bogu, niekoniecznie sugerując, że można o nim orzec dopełnienie tego
predykatu, czyli nie-P\footnote{J.I. Gellman, The Meta-Philosophy of
Religious Language. „No\^us”, nr 11 (1971), s. 158. }. Według
Gellmana, zdanie „Bóg jest potężny” teolog negatywny zrozumie jako
negację dopełnienia predykatu „jest potężny”. W innym kontekście
oznaczałoby to przypisanie obiektowi, o którym mowa, tej właśnie
własności. Jednakże w przypadku Boga negowanie dopełnienia predykatu P
nie oznacza przypisywania mu P. Mówiąc ogólnie, zdania języka
religijnego negują dopełnienia wszystkich wymienianych przez nie
własności Boga, który jest poza zakresem wszystkich naszych predykatów.
One, z kolei -- będąc predykatami języka skończonych bytów -- z
konieczności muszą oznaczać niedoskonałe własności\footnote{Zob.
Tamże. }.

Według Jonesa, teologia negatywna Pseudo-Dionizego Areopagity jest w
dużej mierze teologią krytyczną. Polemizuje ona z błędnym sposobem
mówienia o Bogu -- takim, który traktuje Go jak byty, czyli rzeczy lub
pojęcia\footnote{J.J. Jones, dz. cyt., s. 357. }. Fakt, że Bóg
przekracza wszelki byt, nadaje również strukturę językowi dyskursu
teologicznego. Nie idzie tylko o to, że przypisywanie Bogu
jakichkolwiek przymiotów przysługujących bytom jest z gruntu błędne.
Zwykle, gdy mówimy o rzeczach, twierdzenia i przeczenia sprzeciwiają
się sobie. W wykładni Dionizego nie dzieje się tak w przypadku Boga.
Bóg nie jest jednym z bytów, zatem język służący do opisu bytów nie
jest dla Niego właściwy. W wypracowanym przez niego języku teologicznym
twierdzenia i zaprzeczenia należą do odmiennych grup, tworząc odmienne
sposoby mówienia o Bogu. Ponieważ funkcjonują one w odmienny sposób,
nie należy ich ze sobą mieszać. Te pierwsze przedstawiają Boga jako
przyczynę wszystkiego, te drugie wyrażają jego transcendencję. Oba
sposoby mówienia można stosować naraz zarówno do opisu Boga, jak i
opisu przedmiotów, jednakże w ten sposób nie zdołamy wyrazić
unikalności Boga -- tego, że jest czymś odrębnym od wszystkich
bytów\footnote{Zob. Tamże, s. 360. }. Możemy tego dokonać
wyłącznie przez negację, która według Jonesa jest kluczowym punktem
myśli Dionizego. Istotnym jest, że Jones wyraźnie odróżnia negację od
zaprzeczenia.


\begin{quote}
        W przeciwieństwie do zaprzeczenia, negacja odnosi się do (nie)możliwości
poznania i powiedzenia czegokolwiek o Bogu. Jest to, jeśli można tak
powiedzieć, reguła drugiego rzędu posługiwania się nazwami pierwszego
rzędu.\footnote{Tamże, s. 381. Większą część tego cytatu podaję za
Rojkiem, dz. cyt., s. 222. }
\end{quote}






Najbardziej agnostyczną interpretację dionizyjskiej teologii negatywnej
zaproponował Paul Rorem. Podkreśla on podobieństwo teologii
Pseudo-Dionizego z późną filozofią neoplatońską. W obu tych doktrynach
byty uporządkowane są względem pewnej hierarchii i w celu dotarcia do
bytu absolutnego należy „wspiąć się” po tej „drabinie” bytów. By
spotkać Boga należy wpierw zanegować nasze wrażenia i wyobrażenia i
przekroczyć je, by dojść do ich pojęciowych znaczeń. Następnie
zanegowane zostać powinny także owe znaczenia oraz wszelkie inne
pojęcia umysłu, ponieważ przekroczenie naszej wiedzy prowadzi do
niepoznawalnego, do cichego zjednoczenia z Bogiem. Innymi słowy, Rorem
zwraca uwagę, że u Areopagity drogą do Boga jest zaprzeczenie
wszystkich bytów. Dionizy jednak wielokrotnie stwierdza, że Bóg jest
także ponad wszelkim zaprzeczeniem. Ostatecznie więc, należy zanegować
także wszelkie negacje, nie pozostawając już żadnym pojęciem
Boga\footnote{Por. P. Rorem, Pseudo-Dionysius. A Commentary on the
Texts and an Introduction to Their Influence, Oxford University Press,
Oxford -- New York 1993, ss. 210-211. }. Jak zauważa Rorem, Dionizy
„zaprzecza i wykracza poza wszystkie nasze pojęcia lub «pojęciowe»
atrybuty Boga i kończy na odrzuceniu wszelkiego mówienia i myślenia,
nawet negatywnego”\footnote{Tenże, przypis do Pseudo-Dionizy
Areopagita, The Complete Work, tłum. C. Luibheid,. Paulist Press, New
York 1987, s. 99. Cytuję za P. Rojek, dz. cyt. }.

Przedstawione powyżej prace pozwalają sądzić, że wynikiem agnostycznej
interpretacji teologii negatywnej nie jest zatem teza, że Bóg posiada
własności negatywne, lecz twierdzenie, że jest On niepoznawalny i
niewyrażalny. Zgodnie ze wskazaną w ten sposób dwuznacznością tezy (T4)
Rojek wyróżnia dwie wersje tej teorii. Wedle pierwszej z nich, Bóg jest
niewyrażalny, nie sposób Go wysłowić. Konsekwentnym rozwinięciem tej
teorii jest stwierdzenie, że dyskurs religijny jest pozbawiony
jakiegokolwiek znaczenia. Wedle drugiej wersji, Bóg jest tylko
niepoznawalny, nie można posiąść o nim wiedzy. I choć dyskurs religijny
posiada w tej teorii jakieś znaczenie, może on opisać wyłącznie to,
czego o Bogu nie wiemy. Rojek omawia obie wersje tej interpretacji.



\subsubsection{Teoria Niewysławialnego}

Według wielu komentatorów ta wersja agnostycznej teologii negatywnej
jest wewnętrznie sprzeczna. Z reguły argumentacja polega na wskazaniu,
że teoria ta twierdząc, że nie da się niczego powiedzieć o Bogu, sama
coś o Nim mówi, a zatem jest sprzeczna. W takim wypadku należałoby ją
odrzucić. Jednym krytyków teorii Niewysławialnego\footnote{Rojek używa terminu „Niewyrażalny”. W
niniejszej pracy pozostaje przy terminologii polskiego przekładu pracy
Bocheńskiego, Logika i teologia, dz. cyt. } jest Michael Durrant.
Pisze on, że

\begin{quote}
    w tej teorii, mówiąc, że natura Boga jest zasadniczo niewyrażalna,
opisujemy właśnie naturę Boga -- jest mianowicie zasadniczo
niewyrażalna. Innymi słowy, ci, którzy bronią tego stanowiska, nie mogą
tego robić nie przecząc sobie.\footnote{M. Durrant, The Meaning of
‘God’ (I). [w:] Religion and Philosophy, red.  M. Warner, Cambridge
University Press, Cambridge 1992, s. 74. Cytuję za P. Rojek, dz. cyt.,
s. 222-223. }
\end{quote}


Podobnie argumentuje John Hick, który uważa, że nie ma sensu


\begin{quote}
mówić o X, że żadne nasze pojęcie się do niego nie stosuje. Jest bowiem
w oczywisty sposób niemożliwe odnosić się do czegoś, co nie posiada
nawet własności 'bycia możliwym przedmiotem
odniesienia.\footnote{J. Hick, An Interpretation of Religion. Human
Responses to the Transcendent, Yale University Press, New Haven –
Londyn 1989, s. 239. Cytuję za P. Sikora, Logos Niepojęty, Wydawnictwo
Universitas, Kraków 2010, s. 118. }
\end{quote}


Dodaje on także, że określenie

\begin{quote}
,,taki, że nasze pojęcia się do niego nie stosują'' nie może,
jeśli chcemy uniknąć paradoksu, odnosić się do własności, którą
opisuje.\footnote{Tamże. }
\end{quote}

Przeciwko takiemu przedstawianiu teorii Niewysławialnego występuje Józef
Maria Bocheński. Twierdzi on, że da się ją uratować od sprzeczności,
lecz nawet mimo tego, nie odpowiada ona potrzebom dyskursu
religijnego\footnote{Zob. J.M. Bocheński, dz. cyt., ss. 353-356.
}. Bocheński uważa, że jeśli przestrzega się pewnych obowiązujących w
logice konwencji, zarzut sprzeczności stawiany teorii Niewysławialnego
przestanie obowiązywać. Należałoby wpierw dowieść, że danym układzie
odniesienia teoria ta prowadzi do sprzeczności, tymczasem nikt takiego
dowodu nie przedstawił. Według Bocheńskiego sytuacja przedstawia się
zupełnie przeciwnie -- nietrudno wykazać, że teoria Niewysławialnego
jest spójna. Poniżej przedstawię jego argumentację.

Załóżmy, że dwuargumentowy predykat  $Nw(x,l)$ oznacza „$x$ jest
niewyrażalne w języku $l$”.

Zapiszmy teraz formułę zawierającą ten predykat

\begin{equation}
    \exists x \exists l Nw(x, l)
\end{equation}


Wydaje się, że nie tylko można ją wypowiedzieć nie popadając w
sprzeczność, lecz także jest ona prawdziwa, nietrudno znaleźć taki
obiekt $x$ i taki język $l$, które spełniałyby zapisany wyżej warunek.
(Bocheński podaje przykład krowy i języka szachów: nie da się opisać
krowy w języku szachów).

Możemy powyższy przykład uogólnić i sformułować metajęzykową definicję
Boga o następującej postaci:

\begin{equation}\tag{G2}
    G(x) \equiv \forall l Nw(x, l)\footnote{Definicję podaję za
Rojkiem, różni się ona od zapisu Bocheńskiego tym, że u tego ostatniego
„Bóg” zapisany został jako stała logiczna: $\forall l Nw(a, l)$.}
\end{equation}



Na pierwszy rzut oka, wydaje się, ze ta formuła jest bardziej
problematyczna -- twierdzenie, że $x$ jest niewysłowione w żadnym języku
zdaje się prowadzić do sprzeczności. Można jednak uniknąć tego
problemu, stosując zwykłe konwencje wykorzystywane do pozbywania się
antynomii semantycznych. Należy założyć, że żadne zdanie traktujące o
pewnej klasie języków, nie jest formułowane w żadnym z tych języków.
Aby było pozbawione sprzeczności musi zostać sformułowane w innym
języku, czyli odpowiednim metajęzyku. Możemy więc założyć, że klasa
języków wspominana w (G2) jest klasą języków przedmiotowych. W takim
wypadku (G2) jest zdaniem metajęzyka pierwszego stopnia. Po takim
zabiegu, sformułowana definicja jest znacząca i pozbawiona
sprzeczności. Nie ma bowiem niespójności w twierdzeniu, że coś nie daje
się wysłowić w jakimś języku, lub nawet w klasie języków, o ile
twierdzenie to jest w języku nienależącym do tej klasy. Według
Bocheńskiego, przy takim założeniu, standardowym z punktu widzenia
logiki ogólnej, teoria Niewysławialnego pozostaje znacząca i spójna a
zarzut sprzeczności zostaje oddalony.

Bocheński odrzuca jednak teorię Niewysławialnego z co najmniej dwóch
powodów. Po pierwsze, na mocy (G2) nie można przypisać Bogu
jakiekolwiek własności językowo przedmiotowej. Jedyną własnością, jaką
możemy mu przypisać, jest metajęzykowa własność bycia niewysłowionym w
żadnym z języków przedmiotowych. W takim wypadku wierny, nie mógłby
akceptować żadnego zdania dyskursu religijnego, które przypisałoby Bogu
jakąkolwiek własność-przedmiotowo-językową. Wydaje się to niespójne z
faktycznym dyskursem religijnym. Po drugie, niemożliwe byłoby oddawanie
czci obiektowi, o którym wiemy tylko i wyłącznie, że nie można o nim
nic powiedzieć. Jeśli wierny miałby czcić obiekt pozbawiony własności
przedmiotowo-językowych, równie dobrze tym obiektem mógłby nie być Bóg
a szatan\footnote{Por. J.M. Bocheński, dz. cyt., ss. 354-356. }.

Rojek dodaje do tego podobny argument, jednakże umieszczony w kontekście
teologii negatywnej. Mianowicie, jeśli teoria Niewysławialnego ma być
właściwą interpretacją teologii negatywnej, można poddać w wątpliwość
zasadność jednoczesnego utrzymywania tez (T1)-(T3). Jeśli Bóg jest
niewyrażalny, cały język religijny jest pozbawiony znaczenia, nie
możemy więc ani sensownie potwierdzić, ani zaprzeczyć żadnym z Jego
własności. Dla Rojka jest to wskazówka, by (T4) nie interpretować
semantycznie, lecz epistemologicznie lub nawet ontologicznie\footnote{
Zob. P. Rojek, dz. cyt., s. 223. }. W ten sposób przechodzi się od
teorii Niewysławialnego do teorii Niepoznawalnego.


\subsubsection{Teoria Niepoznawalnego}

Teoria Niepoznawalnego jest lepszym modelem teologii apofatycznej,
ponieważ nie tylko przyjmuje (T4) za twierdzenie podstawowe, lecz także
umożliwia pewną interpretację tez (T2) oraz (T3).By to pokazać, Rojek
wprowadza pojęcie nieokreśloności oraz definiuje nowe, odmienne od
klasycznego pojęcie negacji. W tym celu wykorzystuje on logiczną teorię
nieokreśloności, którą na potrzeby modelowania filozofii nauki rozwinął
rosyjski logik, Aleksander Zinowjew\footnote{A. Zinowjew, Foundations
of the Logical Theory of Scientific Knowledge (Complex Logic), D.
Reidel Publishing Company, Dordrecht 1967. Zob. Także A. Zinowjew,
Logika nauki, tłum. Z. Simbierowicz, PWN, Warszawa 1976. }.

Zinowjew rozważa pewne nieklasyczne przypadki, do których nie można
zastosować prawa wyłączonego środka. Polegają one na przyjęciu
możliwości istnienia obiektów, co do których nie da się ustalić, czy
posiadają one jakąś własność $Q$, czy też posiadają własność
$\neg Q$. Takimi obiektami mogą być na przykład rozważana
w mechanice kwantowej cząstka elementarna, której parametry -- zgodnie z
zasadą nieoznaczoności -- nie mogą zostać ustalone, twierdzenie, które
na gruncie danego rachunku logicznego nie może zostać ani dowiedzione,
ani obalone, albo po prostu obiekt zmieniający się w czasie. Na
potrzeby tych przypadków wprowadźmy nowy funktor nieokreśloności i
oznaczmy go $?$\footnote{Dla utrzymania spójności zapisu, formuły teorii
nieokreśloności Zinowjewa przedstawiam w notacji zaproponowanej przez
Rojka, dz. cyt., s. 223.  }. Niech zapis $?Q(x)$ oznacza „nie można
ustalić, czy $Q(x)$, czy $\neg Q(x)$”, to znaczy „$x$ ma w sposób
nieokreślony $Q$”. Ponieważ w teorii Niepoznawalnego Bóg uznawany jest za
obiekt, którego nie da się poznać, za pomocą powyższego funktora możemy
podać nową definicję Boga, jako obiektu, który wszystkie swoje
własności posiada w sposób nieokreślony.

\begin{equation}\tag{G3}
   G(x) \equiv \forall Q (P(Q) \to ?Q(x)).
\end{equation}



Formule tej należy poświęcić nieco uwagi. Po pierwsze, w odróżnieniu od
(G2) powyższa definicja nie jest wyrażona w metajęzyku, lecz w języku
przedmiotowym. Po drugie, co istotniejsze, by uniknąć kłopotów z
formalizowaniem Boga przy użyciu predykatu $G(x)$, należy ją ograniczyć.
Do tej definicji trzeba dołożyć dodatkowe założenie, o postaci:

\begin{equation}
    \neg P(G)
\end{equation}


Innymi słowy, należy założyć, że predykat „jest Bogiem” (lub „jest
bogopodobny”) nie należy do zbioru predykatów pozytywnych. W
przedziwnym bowiem razie, wpadlibyśmy w błędne koło i o niczym nie
można byłoby określić, że jest Bogiem.

Jak zauważa Rojek, wykorzystując ten formalizm do modelowania teorii
Niepoznawalnego, można -- w pewnym szczególnym sensie -- zaprzeczyć, że
Bóg posiada wszystkie własności pozytywne. Oczywiście, nie idzie tutaj
o stwierdzenie, że o Bogu można na orzec jakiekolwiek
$\neg Q$, przy założeniu, że $Q$ jest własnością pozytywną.
Taki przypadek jest niezgodny definicją (G3). Idzie o to, że po
wprowadzeniu funktora nieokreśloności $?$, zdanie „nieprawda, że $x$ jest
$Q$”\footnote{Rojek mówi o wyrażeniu „$x$ jest nie-$Q$”, moim zdaniem
błędnie. Por. tamże. } staje się dwuznaczne. Może ono przyjąć jedno
z dwóch znaczeń: albo „$x$ jest nie-$Q$”, albo „nie można stwierdzić, czy $x$
jest $Q$”. W nomenklaturze Rojka, w pierwszym przypadku z zdaniu pojawia
się negacja w znaczeniu \textit{de re}, o drugim znaczeniu natomiast
można mówić, że zawiera ono negację \textit{de dicto}. Nieco inne
nazewnictwo wprowadził Zinowjew, który pierwszy rodzaj negacji nazwał
negacją wewnętrzną -- dotyczy ona bowiem wyłącznie predykatu, drugi
negacją zewnętrzną -- odnosi się ona bowiem do całego negowanego w ten
specyficzny sposób zdania.

Przyjmijmy teraz dwa różne symbole, w celu odróżnienia odmiennych pojęć
negacji. Niech $\neg$ oznacza negację \textit{de re},
natomiast symbol $\sim$ negację \textit{de dicto}. Przy
takiej notacji, formułę $\neg Q(x)$ czytamy jako „$x$ jest
nie-$Q$”, lub  inaczej: „$x$ ma własność nie-$Q$”\footnote{U Rojka: „$x$ ma
własność $\neg Q$”. }, natomiast formuła
$\sim\! Q(x)$\footnote{Rojek formułę, negowaną za pomocą
funktora $\sim$ opatruje w dodatkowe nawiasy po to, by
podkreślić odmienny, „zewnętrzny” charakter tego rodzaju negacji.
Uznaję ten zabieg za niekonieczny. Notacja Rojka jest odmienna od
stosowanej oryginalnie przez Zinowjewa. } oznacza „nie można
stwierdzić, czy $x$ jest $Q$”, lub  innymi słowy „nie jest twierdzi się, że
$x$ ma własność $Q$”. Poniższe aksjomaty definiują relacje, jakie zachodzą
pomiędzy funktorem nieokreśloności a funktorami negacji \textit{de re}
oraz negacji \textit{de dicto}:

\begin{equation}\tag{Z1}
\sim\! Q(x) \equiv ?Q(x) \lor \neg Q(x),
\end{equation}
\begin{equation}\tag{Z2}
\sim\!\neg Q(x) \equiv Q(x)\ \lor\ ?Q(x),
\end{equation}
\begin{equation}\tag{Z3}
\sim ?Q(x) \equiv Q(x) \lor \neg Q(x).
\end{equation}
Na ich podstawie można dowieść następujących tez:

\begin{equation}
   \vdash \quad ?Q(x)\ \equiv\ \sim\! Q(x)\ \land\ \sim\!\neg Q(x).
\end{equation}
 



Gdybyśmy chcieli dokonać kolapsu z powrotem do logiki klasycznej,
musielibyśmy przyjąć założenie, że negacje \textit{de re} i \textit{de
dicto} są wzajemnie definiowalne, to znaczy, że
$\neg Q(x) \equiv \sim\!Q(x)$. Jednakże, jeśli
chcemy opisać wspomniane wyżej przypadki nieklasyczne, w których
posiadanie lub nie posiadanie danej własności przez konkretny obiekt
nie może zostać określone, musimy zgodzić się na posiadanie dwóch
różnych, nierównoważnych pojęć negacji w systemie. Z tego powodu
poniższa formuła nie jest tezą prezentowanego rachunku:


\begin{equation}
    \nvdash \quad \sim\! Q(x) \to  \neg Q(x).
\end{equation}
Dzieje się tak, ponieważ $\sim\!Q(x)$ pociąga za sobą $\neg Q(x)$ lub
$?Q(x)$. Analogicznie, w prezentowanym tu rachunku nie zachodzi
następująca wersja prawa podwójnej negacji



\begin{equation}
   \nvdash \quad  \sim\! (\neg Q(\textit{x})) \to  Q(\textit{x}),
\end{equation}
ponieważ $\sim\!\neg Q(x)$ implikuje $Q(x)$ lub $?Q(x)$. Prawdziwe są
natomiast następujące formuły:



\begin{equation}
    \vdash \quad \neg Q(x) \to  \sim\! Q(x),
\end{equation}
\begin{equation}
    \vdash \quad Q(x) \to  \sim\! \neg Q(x).
\end{equation}






Należy zauważyć, że w przypadka z nieokreślonością nie można także mówić
o klasycznym prawie wyłączonego środka


\begin{equation}
     \nvdash \quad Q(x) \lor \neg Q(x),
\end{equation}
ponieważ istnieje dodatkowa, trzecia możliwość - $?Q(x)$. Do grona tez tego
rachunku należy zatem następująca formuła:

\begin{equation}
    \vdash \quad Q(x) \lor  \neg Q(x) \lor  ?Q(x)
\end{equation}


Co ciekawe, prawo wyłączonego środka obowiązuje dla negacji \textit{de
dicto}:

\begin{equation}
    \vdash \quad Q(x) \lor  \sim\! Q(x),
\end{equation}
\begin{equation}
    \vdash \quad \neg Q(x) \lor  \sim\! \neg Q(x),
\end{equation}
\begin{equation}
\vdash \quad ?Q(x) \lor  \sim\! ?Q(x).
\end{equation}

Jak zauważył Rojek, rachunek Zinowjewa stworzony pierwotnie na potrzeby
formalizacji teorii naukowych jest przydatny także do modelowania
omawianej w tym paragrafie interpretacji teologii apofatycznej. Dzięki
jego własnościom, można przedstawić w nim w sposób niesprzeczny aż trzy
tezy wypreparowane z pism Pseudo-Dionizego Areopagity. Formalną wersją
tezy (T4) jest definicja Boga podana przez (G3). Negację, o której mowa
w (T2) można traktować, jako negację \textit{de dicto}, ponieważ
zgodnie z teorią niepoznawalności Boga, nie możemy stwierdzić, czy
posiada On jakieś własności. W takim podejściu, (T2) stwierdza, że o
Bogu można orzec zewnętrzne negacje wszystkich własności pozytywnych. A
zatem, w języku tego rachunku należałoby to zapisać w następujący sposób:



\begin{equation}
    G(x) \to  (\forall Q (P(Q) \to  \sim\!Q(x)).
\end{equation}



Co więcej, powyższa formuła jest bezpośrednią konsekwencją definicji
(G3) i (1), które jest jednym z praw tego rachunku. Ponieważ negacja
\textit{de dicto}, jest różna od negacji \textit{de re}, nie możemy
jednocześnie twierdzić, że Bóg posiada (wewnętrzną) negację wszystkich
pozytywnych własności, czyli, że można o nim orzekać każde nie-$Q$, o ile
$Q$ jest pozytywne. Dzieje się tak, ponieważ z obu tych formuł wynika
także


\begin{equation}
    G(x) \to  (\forall Q (P(Q) \to
\sim\!\neg Q(x)),
\end{equation}


którą Rojek uważa za formalną postać tezy (T3). Formuła ta zawiera
podwójne przeczenie, jednakże pierwsza negacja ma charakter \textit{de
dicto}, natomiast druga jest negacją \textit{de re}. Brzmieniem tej
formuły jest „Nie można stwierdzić, czy Bóg posiada wszystkie negacje
pozytywnych własności”.

W takim wypadku, wszystkie stwierdzenia Dionizego o tym, że Bóg nie jest
ani $Q$, ani nie-$Q$, na przykład że nie jest On „ani wielkością, ani
małością, ani równością, ani nierównością, ani podobieństwem, ani
niepodobieństwem”, „nie jest też niczym z niebytu ani czymś z
bytu”\footnote{Pseudo-Dionizy Areopagita, Teologia mistyczna, dz.
cyt., rozdział V. } itp.,  można z łatwością interpretować w
świetle powyższej formalizacji. Podobnie sformułowania Areopagity o
tym, że Bóg jest „ponad wszelkim twierdzeniem i ponad wszelkim
zaprzeczeniem”\footnote{Tamże. } można ująć w ramach
prezentowanego rachunku, jako koniunkcję warunków zawartych już w
formułach (8) i (9).


\begin{equation}
    G(x) \to  (\forall Q (P(Q) \to  \sim\!(Q(x)) \land
\sim\!\neg Q(x)).
\end{equation}



Jak zauważa Rojek, istnieje wyjątek od konsekwentnego stosowania tej
interpretacji. Nie można w ten sposób formalizować  twierdzeń, że Bóg
nie jest ani poznawalny ani niepoznawalny, ani określony, ani
nieokreślony. Taka treść predykatu $Q$ naraziłaby bowiem tę interpretację
na sprzeczność. Jednakże wydaje się, że żadne z dzieł Dionizego nie
zawiera podobnych sformułowań\footnote{Także Jones zauważa, że teza o
niepoznawalności Boga ma wyjątkowy charakter w pismach Dionizego i
nigdy nie występuje w podobnych parach. Zob. J.J. Jones, dz. cyt., s.
358. }. Jedynie (T1) nie znajduje swojego miejsca w powyższej
interpretacji, bowiem -- skoro Bóg jest niepoznawalny -- nie można o nim
nic sensownie stwierdzić.

Powyższe analizy pozwalają sądzić, że wykorzystując odpowiednie
narzędzia formalne można obronić przed sprzecznościami obie formy
agnostycznej teologii negatywnej -- zarówno teorię Niewysławialnego, jak
i teorię Niepoznawalnego. Ta druga wydaje się o tyle lepszą
interpretacją teologii Pseudo-Dionizego, o ile na jej gruncie można
przyjąć aż trzy z wypreparowanych z tekstu Areopagity tez. Teoria
Niewysławialnego ustępuje w tym zestawieniu  i obejmuje tylko jedną
tezę dionizyjskiej doktryny. Obie jednak zawodzą w interpretacji (T1) i
jak zauważył Bocheński -- są zasadniczo niespójne z faktycznym dyskursem
i praktyką religijną. Zgodnie z ich duchem, wierny nie mógłby
akceptować żadnych zdań dyskursu religijnego, które przypisują Bogu
jakieś własności. Tymczasem, każdy dyskurs religijny zawiera
przynajmniej kilka takich zdań. Poza tym, gdyby o Bogu nie można było
orzekać żadnych własności, nie mógłby On być przedmiotem czci\footnote{
Zob. J.M. Bocheński, dz. cyt., ss. 355-356.}.




\subsection{Negatywna teologia negatywna}

Wad tych pozbawiona jest druga interpretacja teologii negatywnej, którą
Rojek nazywa negatywną teologią negatywną. Jej punktem wyjścia jest
teza (T2). Na gruncie tej interpretacji nie twierdzi się ani, że
dyskurs religijny pozbawiony jest znaczenia, ani że Bóg jest
niepoznawalny. Uważa się natomiast, że o przedmiocie religijnym można
orzekać wyłącznie negatywne własności. Nie jest więc tak, że nic nie
wiemy o Bogu -- wiemy, że nie posiada on pozytywnych własności. Należy
przyznać, że jest to dość popularna interpretacja teologii
apofatycznej. Na przykład, do właśnie w taki sposób rozumianej teologii
negatywnej odnosił się Józef Maria Bocheński w dziele \textit{Logika
religii}.

W analizie Rojka, negację obecną w (T2) na gruncie tej interpretacji
należy traktować jako zwykłą, klasyczną negację. Jeśli przyjmiemy
logikę klasyczną, zachowujemy zasadę niesprzeczności. W konsekwencji,
tezy (T1) oraz (T3) muszą zostać odrzucone, jako niespójne z tezą
przyjętą tu za podstawową.

Wedle tej wersji teologii negatywnej, wszystko, co możemy orzec o Bogu,
posiada ściśle negatywny charakter. Na tym polega boska transcendencja.
Choć Bóg zatraca tutaj swój sprzeczny charakter (obecny w jakimś
stopniu w agnostycznej teologii negatywnej), wciąż pozostaje w jakimś
sensie niepoznawalny. Jedyne, co o Nim wiemy to jaki nie jest. Można
więc pokusić się także na próbę uwzględnienia tezy (T4) w obrębie
niniejszych analiz.

Rojek twierdzi, że w tym modelu teologii negatywnej właściwą definicją
Boga jest formalizacja tezy (T2) w obrębie klasycznego rachunku
predykatów II-rzędu. Bóg -- analogicznie do poprzednich definicji –
opisany jest jako obiekt, o którym można orzekać negacje wszystkich
pozytywnych własności:


\begin{equation}\label{G4}\tag{G4}
    G(x) \equiv  \forall Q (P(Q) \to
\neg Q(x)).
\end{equation}





Podobnie, jak poprzednie definicje, także formuła (G4), wymaga
wprowadzenia szeregu ograniczeń po to, by w proponowany model nie
wkradła się żadna sprzeczność. Pierwsze ograniczenia dotyczą zbioru
pozytywnych własności. Jak wspominałem wcześniej, Rojek nie podaje
żadnej definicji, ani kryterium wyróżniania własności pozytywnych ze
zbioru wszystkich własności. Czyni jednak na ten temat kilka uwag,
które w większym bądź mniejszym stopniu nawiązują do rozważań
Bocheńskiego

Po pierwsze, według Rojka, zbiór pozytywnych własności nie może zawierać
własności negatywnych. Szczerze powiedziawszy, nie jest do końca jasne
także to, jak Rojek rozumie pojęcie własności negatywnej. W tym wypadku
jednak mamy dość tradycyjną, syntaktyczną definicję tego pojęcia.
Własność negatywna, zwana czasem także własnością dopełniającą, jest
jedną z własności złożonych i stanowi po prostu negację własności. W
języku naturalnym pojęcie negacji własności wrażamy prze użycie
przedrostka \textit{nie}- (w języku łacińskim
\textit{non}-)\footnote{Zob. J. Paśniczek, Predykacja, Copernicus
Center Press, Kraków 2014. }. Wydaje się, że Rojek pojęcie
własności negatywnej rozumie w podobny sposób, twierdzi bowiem, że nie
stosując tego ograniczenia, doprowadzimy do orzekania o Bogu także
własności pozytywnych (co z kolei sugeruje, że jest on także bliski
stosowaniu syntaktycznej definicji własności pozytywnej -- jako
niezawierającej spójnika negacji). Według Rojka, jeśli nie wprowadzimy
tego ograniczenia, nasz model umożliwi pozytywne twierdzenia o Bogu,
bowiem w przyjętej tu logice klasycznej $\neg \neg Q$
implikuje $Q$. Ponadto uważa on, że dopuszczenie do
zaprzeczania własności negatywnych wprowadzi do systemu sprzeczność,
bowiem umożliwi orzekanie o Bogu zarówno $\neg Q$, jak i
$\neg \neg Q$\footnote{Myślę, że
konsekwencji tej można by uniknąć wprowadzając porządną definicję
własności pozytywnych. Niestety, jak już zostało wspomniane, nie
została ona podana. }. Jednakże dla Rojka syntaktyczne kryterium
wyróżniania własności pozytywnych nie jest satysfakcjonujące.
Alternatywą dla niego byłoby kryterium epistemologiczne, zaproponowane
przez Bocheńskiego\footnote{J.M. Bocheński, dz. cyt., s. 416. }.
Wedle tego kryterium własności pozytywne definiuje się indukcyjnie,
jako własności postrzegane bezpośrednio lub zapisywane za pomocą
formuł, zawierających wyłącznie symbole własności pozytywnych i terminy
logiki pozytywnej. Rojek zgadza się z Bocheńskim, że także takie
kryterium nie jest wystarczająco ścisłe. Nie zamierza jednak
rozwiązywać tego problemu i podawać odpowiedniej definicji\footnote{
Por. P. Rojek, dz. cyt., s. 226. }.

Po drugie, po raz kolejny należy wykluczyć predykat „jest Bogiem” ze
zbioru predykatów oznaczających własności pozytywne. Uznanie własności
bycia Bogiem za własność pozytywną doprowadziłoby do sprzeczności:

\begin{equation}
    G(x) \equiv \neg G(x).
\end{equation}


Wedle Rojka, twierdzenie „Bóg nie jest boski” należy przeinterpretować w
taki sposób, by w orzeczniku tego zdania nie znalazł się predykat $G(x)$,
który użyty jest w (G4), lecz wiązka pozytywnych własności zwykle
przypisywanych Bogu. W mojej opinii problem ten mógłby zostać
rozwiązany poprzez formalizowanie Boga przy pomocy stałej logicznej,
zamiast predykatu „$x$ jest Bogiem”.

Jeśli przyjmiemy wszystkie te ograniczenia na proponowaną powyżej
definicję Boga, możemy doprowadzić ten model do ciekawych filozoficznie
konsekwencji. Zgodnie z omawianą wcześniej interpretacją Johna J.
Jonesa, głównym celem Dionizego było wykazanie, że Bóg jest ponad
wszelkim bytem, w szczególności nie należy on do kategorii przedmiotów.
Według popularnego rozumienia przedmiotu, można traktować go jako
podmiot własności. Taką definicją posługiwał się na przykład Stanisław
Leśniewki, według którego coś jest przedmiotem, o ile istnieje jakaś
własność, którą można o nim orzec\footnote{Rojek powołuje się tu na
dwa źródła: J. Słupecki, Stanisław Leśniewski’s Calculus of Names,
„Studia Logica”, nr 3 (1955), ss. 7--76 oraz J. Perzanowski, The Way
of Truth, [w:] red. R. Poli, P. Simons, Formal Ontology, Kluwer
Academic Publishers, Dordrecht 1996, ss. 61--130. }:

\begin{equation}
    Ob(x) \equiv  \exists Q  (Q(x)).
\end{equation}


Jeśli uzupełnimy tę definicję o warunek, że każdy przedmiot powinien
posiadać nie jakąkolwiek, ale pozytywną wartość, otrzymamy następującą
modyfikację\footnote{W świetle stawianych na definicję (G4) ograniczeń
konieczność wprowadzania w tę definicję dodatkowego zastrzeżenia wydaje
się wątpliwa. }:

\begin{equation}\label{D4}\tag{D4}
    Ob(x) \equiv  \exists Q P(Q) \land  Q(x).
\end{equation}


Zgodnie z tak zmodyfikowaną definicję, Bóg nie należy do zbioru
przedmiotów, ponieważ nie może on posiadać żadnych pozytywnych
własności.

\begin{equation}
    \forall x (G(x) \to  \neg Ob(x))\footnote{U Rojka zmienna x nie jest związana
    żadnym kwantyfikatorem}.
\end{equation}


Rojek konkluduje, ze gdy przyjmiemy powyższe rozważania za dobrą monetę,
Bóg istnieje w jakiś inny sposób, odmienny od tego, jak istnieją inne
byty. Według niego można tu mówić o istnieniu w sensie Quine’a (w
przeciwieństwie do istnienia w sensie przedmiotów Leśniewskiego).
Konkluzja ta zdaje się być w zgodzie z  rozważaniami tych zwolenników
teologii negatywnej, którzy podkreślają transcendencję Boga -- to, że
przekracza on wszystkie byty. W przedstawianym powyżej modelu nie można
Go bowiem traktować jako przedmiot w sensie uchwyconym w definicji
(D4).

Czy Bóg zdefiniowany za pomocą (G4) jest poznawalny? W jakimś sensie
możemy wyrazić jego naturę -- możemy mówić, jaki nie jest. Jednakże,
uczciwie rzecz ujmując, na gruncie tego modelu  nie możemy posiąść
żadnej pozytywnej wiedzy o Bogu. Z tego powodu w tej teorii znajdzie
się również miejsce na pewną interpretację tezy (T4). Rojek ujmuje ją w
następujący sposób:



\begin{quote}
    W normalnym wypadku rozumiemy natury rzeczy przez porównanie ich z tym,
czym one nie są. Jak mawiał Spinoza, „określenie jest negacją”. Bez
opozycji znaczeniowych i różnic nie moglibyśmy używać języka w sposób,
w jaki go faktycznie używamy. Bóg jako negacją wszelkich własności
znajduje się jednak poza całym systemem opozycji i różnic. Pełna
negacja prowadzi do całkowitego nieokreślenia.\footnote{P. Rojek, dz.
cyt., s. 226. }
\end{quote}




A zatem w pewnym sensie także na gruncie tej teorii możemy mówić o
niepoznawalności czy nawet niewyrażalności Boga.

Przedstawiony powyżej model teologii negatywnej -- o ile zgodzimy się na
wszystkie jego ograniczenia -- można uznać za logicznie spójny. Choć nie
uwzględnia on wszystkich tez wyabstrahowanych z dzieła Areopagity, w
atrakcyjny sposób rozwija on tezę o wyłącznie negatywnym charakterze
wiedzy o Bogu. Ponadto, nie wymaga on stosowania wyrafinowanych
narzędzi formalnych, pozostając przy logice klasycznej. Jego dodatkowym
atutem jest pewna interesująca filozoficznie własność -- Bóg rozumiany
jest tu jako istotnie odmienny od całej dziedziny bytów rozumianych
jako przedmioty. Jednakże w prezentowanym modelu nie ma miejsca na
interpretację tez (T1) oraz (T3). Ponadto, trudno go pogodzić z
dyskursem i praktyką religijną -- trudno jest bowiem oddawać cześć
czemuś, o czym wiemy tylko, czym nie jest\footnote{Jest to zarzut
stawiany przez Bocheńskiego, dz. cyt., s. 418. }.



\subsection{Pozytywna teologia negatywna}

Ostatnia interpretacja teologii negatywnej jest autorskim pomysłem
Rojka\footnote{Por. P. Rojek, dz. cyt., ss. 227-230. }. Również na
jej gruncie, podstawą tezą teologii negatywnej jest (T2). A zatem o
Bogu można orzec wyłącznie własności negatywne. Ponieważ model
stosowany do formalizacji tej interpretacji używa pewnego
specyficznego, nieklasycznego pojęcia negacji (innego niż wprowadzona
we wcześniejszych rozdziałach negacja \textit{de divto}), może on
również zinterpretować w odpowiedni sposób także tezy (T2) oraz (T3).
Model ten może również zawrzeć satysfakcjonującą interpretację tezy
(T4).

Punktem wyjścia interpretacji Rojka jest spostrzeżenie, że negacje w
pojawiające się tezxach teologii negatywnej Pseudo-Dionizego pełnią
specyficzną funkcję, odmienną od funkcji negacji klasycznej. Dionizy
zdawał sobie sprawę, że negacje, które stosuje nie należy rozumieć w
sensie braku (\textit{privatio}). Twierdził też, że w przypadku
wypowiedzi o Bogu „nie należy sądzić, że zaprzeczenia i twierdzenia
sprzeciwiają się sobie”\footnote{Pseudo-Dionizy Areopagita, dz. cyt.,
rozdział I, 2. }. Uznawał on więc możliwość jednoczesnego przyjęcia
zarazem twierdzenia o Bogu, jak i zaprzeczenie tego twierdzenia. Jak
zauważa Rojek, Dionizy wielokrotnie powtarza, że Bóg jest „powyżej”,
„poza” i „ponad” bytami lub że je „obejmuje”. Poniższy cytat jest
przykładem tego, jak Dionizy używał negacji:



\begin{quote}
    [Bóg] jest wszystkim jako przyczyna wszystkiego […]. I jest ponad
wszystkim, istniejąc ponadsubstancjalnie wcześniej niż wszystko, co
jest. Dlatego też wszystko na raz można o Nim twierdzić, choć On nie
jest żadną rzeczą.\footnote{Tamże, rozdział V, 8. }
\end{quote}





W obliczu takich sformułowań Rojek stwierdza, że dionizyjska negacja
oznacza jednoczesne zawieranie czegoś oraz byciu poza czy też ponad to
coś. W takim rozumieniu, negacja używana przez Areopagitę nie wyklucza
afirmacji. Rojek dodaje, że nawet jeśli tak rozumiane przeczenie nie
powinno się nazywać negacją, to zarzut stawiany Dionizemu nie powinien
polegać na oskarżaniu go o sprzeczność, tylko o niewłaściwe użycie
słów.

Rojek próbuje wykazać, że podobne rozumienie negacji można spotkać także
w języku potocznym. W tym celu posługuje się następującym przykładem:

\begin{quote}
    Załóżmy, że na stole leży 100 tysięcy zł w gotówce (nie jest to łatwe w
trakcie kryzysu). Załóżmy dalej, że ktoś pyta, czy na stole jest 10
groszy. Odpowiedź twierdząca byłaby oczywiście słuszna, lecz w pewien
sposób myląca. Wydaje się, że odpowiedź przecząca byłaby dopuszczalna,
a nawet bardziej wskazana. „Nie, ponieważ na stole leży o wiele, wiele
więcej niż 10 groszy”. Słowo „nie” nie oznacza w tej odpowiedzi
klasycznej negacji, lecz wyraża nieadekwatność supozycji pytania do
zachodzącego stanu rzeczy. Odpowiedź „tak” na to pytanie sugerowałaby,
że suma na stole jest w jakiś sposób porównywalna z 10 groszami. Taka
sama sytuacja zachodzi w wypadku takich pytań jak: „Czy Jan jest
zwierzęciem” („Nie! Jest człowiekiem!”), „Czy Romeo lubi Julię?” („Nie!
On ją kocha!”) itd.\footnote{P. Rojek, dz. cyt., s. 227. }
\end{quote}






Można mnożyć analogiczne przykłady. W podobnym duchu można stwierdzić,
że na obiedzie u teściowej na pytanie „Czy zupa była dobra?” dużo
lepiej (i bezpieczniej) jest odpowiedzieć „Nie, była pyszna!”.
Przykłady te wskazują, że w pewnych wypowiedziach  języka naturalnego
używamy zaprzeczeń, które polegają na klasycznej negacji logicznej. Jak
zauważa Rojek, przeczenie to ma również pozytywny, nie tylko negatywny
charakter i wyraża nie brak, ale nadmiar. Ponadto, uważa on, że takie
użycie przeczenie nie narusza reguł konwersacyjnych Grice’a, w
szczególności reguły ilości, która zabrania przekazywania większej
ilości informacji niż jest to konieczne. Informacja o tym, że dany
obiekt jest czymś większym, niż sądzi rozmówca, w pewnych kontekstach
wydaje się niezbędna. Nawet gdyby reguła ilości została naruszona,
zarzut ten jest dużo słabszy od zarzutu popadania w
sprzeczność\footnote{Zob. tamże. }.

W celu uzasadnienia stosowności używania tego rodzaju „pozytywnej”
negacji Rojek podaje też jego przykłady z dyskursu filozoficznego.
Podobne rozumienie negacji można spotkać u Stróżewskiego, który
rozróżnia dwa rodzaje negacji: przekreślający oraz różnicujący. Ta
pierwsza usuwa negowaną rzecz, ta druga jedynie podkreśla różnicę i jej
rezultatem nie jest brak, lecz w zasadzie jakieś pozytywne
stwierdzenie\footnote{Por. W. Stróżewski, Z problematyki negacji, [w:]
Tenże, Istnienie i sens, Wydawnictw Znak, Kraków 1994,
s.~373--395. }. Podobnego sensu negacji Rojek doszukuje się także
w heglowskim terminie \textit{Aufhebung}, które zwykle tłumaczy się
jako negację lub zniesienie. Powołuje się on na cytat z Hegla, który
wskazuje na dwuznaczny charakter tego słowa:

\begin{quote}
    Należy pamiętać o podwójnym sensie niemieckiego słowa aufheben
(odkładać, ściągać). Aufheben znaczy, po pierwsze, usuwać, anulować,
stąd mówi się, że prawo czy instytucja zostały anulowane. Po drugie,
aufheben znaczy także zachować, i w tym sensie mówimy, że coś zostało
zachowane. Tego podwójnego użycia języka, który nadaje temu samemu
słowu znaczenie pozytywne i negatywne, nie należy traktować jako czegoś
przypadkowego, ani tym bardziej krytykować język za wywoływanie
zamieszania. Powinniśmy raczej dostrzec w tym spekulatywnego ducha
naszego języka, przekraczającego podziały nagiego, rozsądkowego
albo-albo.\footnote{G.W.F. Hegel, Logic. Being Part One of the
Encyclopaedia of the Philosophical Sciences, tłum. W. Wallace. Oxford
University Press, Oxford 1975, §96. Cytuję za P. Rojek, dz. cyt.,
s.~228. }
\end{quote}






Rojek podkreśla, że u Hegla to, co zostało zniesione
(\textit{aufgehoben}) nie znika, lecz zostaje zachowane w doskonalszej
postaci.

Przedstawione powyżej wywody pozwalają sądzić, że zarówno w języku
potocznym, jak i filozoficznym można stosować negację w ten
specyficzny, „pozytywny” sposób. Takie rozumienie negacji nie może być
tożsame z negacją używaną w klasycznych rachunkach logicznych. Według
Rojka, właśnie w takim sensie Dionizy stosował negację w swoich
dziełach. Areopagita nie chciał twierdzić, że Bóg nie posiada żadnych
własności, lecz że posiada je w wyższy, pełniejszy sposób.

Na potrzeby analizy swojej interpretacji teologii negatywnej, Rojek
wprowadza zarys syntaktyki rachunku logicznego, który operowałby na tym
specyficznym rozumieniu negacji w sposób formalny. Stwierdzenie, że $x$
pozytywnie nie ma $Q$ oznacza w nim, że $x$ ma $Q$ w pewien szczególny,
wyższy sposób. Niech sformułowanie $!Q(x)$ oznacza „$x$ pozytywnie nie ma
$Q$”, lub innymi słowy „$x$ ma pozytywną negację $Q$”. Znaczenie negacji
pozytywnej Rojek próbuje (częściowo) ustalić za pomocą następujących
aksjomatów\footnote{Zob. P. Rojek, dz. cyt., s.~228. }:


\begin{equation}\label{A1}\tag{A1}
    !Q(x) \to  Q(x),
\end{equation}
\begin{equation}\label{A2}\tag{A2}
    \neg (Q(x) \to !Q(x)),
\end{equation}
\begin{equation}\label{A3}\tag{A3}
    !!Q(x) \to  !Q(x).
\end{equation}


Z tych aksjomatów natomiast wynikają następujące tezy:


\begin{equation}
    \vdash \quad \neg Q(x) \to  \neg !Q(x),
\end{equation}
\begin{equation}
    \vdash \quad \neg (!Q(x) \to  \neg Q(x
\end{equation}



Rojek zdaje sobie sprawę, że przedstawiony przez niego rachunek ma
charakter szkicowy, jednakże zaznacza, że powyższe aksjomaty i tezy
wystarczają do zaproponowanej przez niego analizy dionizyjskiej
doktryny. A zatem, w tej interpretacji Bóg posiada wszystkie negacje
pozytywnych własności, lecz negacje rozumiane są właśnie w ten
specyficzny, „pozytywny” sposób:




\begin{equation}\label{G5}\tag{G5}
 G(x) \equiv  \forall Q (P(Q) \to  !Q(x)).
\end{equation}



Powyższa formuła w niniejszym modelu stanowi formalizację tezy (T2). Z
niej oraz z aksjomatu (A1) wynika




\begin{equation}
G(x) \to  \forall Q (P(Q) \to  Q(x)).
\end{equation}




Oznacza to, że Bóg posiada wszystkie pozytywne własności nie tylko w
wyższy, pełniejszy sposób, lecz także w zwykły sposób. Innymi słowy,
Bóg jest nie jest mądry (w pozytywnym sensie) i jest mądry,
(pozytywnie) nie jest dobry i zarazem jest dobry, itd. Jest to pierwszy
z przedstawianych modeli teologii negatywnej, który zawiera zarówno
interpretację tezy (T2), jak i (T1).

Dzięki wprowadzanej w aksjomacie redukcji negacji pozytywnych, można w
prezentowanym systemie bezsprzecznie orzec o Bogu pozytywne negacje
pozytywnych negacji wszystkich pozytywnych własności. Na gruncie tego
modelu możemy zatem sformułować tezę (T3) w następujący sposób:




\begin{equation}
G(x) \to  \forall Q (P(Q) \to  !!Q(x)).
\end{equation}




W końcu, w interpretacji Rojka znajdzie się także miejsce dla tezy (T4).
Teza o niepoznawalności Boga zawiera się bowiem we wprowadzonym
„pozytywnym” pojęciu negacji. Gdy orzeka się o Bogu pozytywną negację
jakiejś pozytywnej własności, twierdzi się, że nie tylko posiada On tę
własność, lecz także, że posiada On ją w pewien wyższy, pełniejszy
sposób. Istota niepoznawalności Boga polega na tym, że nie i wiemy co
to znaczy posiadać jakąś własność w taki sposób, w jaki posiada ją Bóg.



\begin{quote}
    Choć wiemy, że Bóg jest mądry, nie wiemy, na czym polega bycie mądrym w
wypadku Boga. Wiemy tylko, że jego mądrość jest czymś więcej niż ludzka
mądrość. Nasza niewiedza nie dotyczy jednak tego, że Bóg jest mądry,
lecz ogranicza się tylko do sposobu, w jaki Bóg posiada mądrość. Wiemy,
że Bóg jest Q, wiemy, że jest !Q, ale nie wiemy, co to dokładnie
znaczy.\footnote{Tamże, s.~229. }
\end{quote}





Rojek zwraca uwagę, że przedstawiona prze niego interpretacja teologii
apofatycznej jest zbliżona do rozwiniętej prze św. Tomasza z Akwinu
teorii analogii. Jest o tyle interesująca uwaga, o ile niektórzy
komentatorzy próbują doszukać się apofatycznych wątków także w teologii
Akwinaty\footnote{Zob. P. Sikora, dz. cyt., ss.~79-87, a także B.
Davies, Aquinas on What God is Not, [w:] red. B. Davies,. Thomas
Aquinas. Contemporary Philosophical Perspectives, Oxford University
Press Oxford 2002, ss.~227–242; J. Wissink, Two Forms of Negative
Theology Explained Using Thomas Aquinas, [w:] red. I. N. Bulhof, L.
Kate, Flight of the Gods. Philosophical Perspectives on Negative
Theology, Fordham University Press, Fordham 2000, ss.~100--120; F.
O'Rourke, Pseudo-Dionysius and the Metaphysics of
Aquinas, EJ. Brill, Leiden -- New York 1992. }. Teoria analogii
głosi, że predykaty orzekane o Bogu nie są ani jednoznaczne, ani
wieloznaczne, lecz analogiczne. Predykaty jednoznaczne mają to samo
znaczenie, predykaty wieloznaczne, tak jak homonimy, mają całkowicie
różne znaczenia. Wciąż istnieją spory, co do właściwej interpretacji
znaczeń terminów analogicznych, jednakże z grubsza  rzecz biorąc, przy
ich pomocy nie tylko wyrażamy to, co one oznaczają, lecz także
wskazujemy ponad to, co przez nie rozumiemy. W ten sposób możemy
orzekać o Bogu pewne własności, jednocześnie nie wiedząc do końca, w
jaki sposób te własności Mu przysługują. Według Rojka, sens teologii
negatywnej Areopagity jest identyczny -- Dionizego i Tomasza różni tylko
sposób wypowiadania się. Gdy Tomasz mówi, że Bogu dana własność
przysługuje w pewien wyższy sposób, Dionizy orzeka o Bogu (pozytywną)
negację tej własności. Zasadniczo jednak wyrażają oni tę samą
myśl\footnote{Por. P. Rojek, dz. cyt., s.~229-230. }.



\clearpage
\section{Uwagi krytyczne}

Rojek przedstawia trzy interpretacje teologii negatywnej i próbuje podać
ich spójne formalizacje. Wszystkie trzy opierają się na tezach
wypreparowanych z \textit{Teologii mistycznej} Pseudo-Dionizego
Areopagity, różnią się jednak zasobem tez, które potrafią niesprzecznie
zinterpretować.

Pierwsza interpretacja, nazwana agnostyczną, podana jest w dwóch
wersjach: jednej kładącej nacisk na niewyrażalność Boga, drugiej
podkreślającej Jego niepoznawalność. Teoria Niewysławialnego, mimo iż
została obroniona przed ciążącym na niej zarzutem sprzeczności, okazuje
się niezdolna do uchwycenia tez Dionizego z wyjątkiem tezy (T4).
Formalizacja teorii Niepoznawalnego zasadniczo wykorzystuje logikę
nieokreśloności -- rachunek stworzony przez Aleksandra Zinowjewa na
potrzeby ścisłego badania pewnych filozoficznonaukowych koncepcji.
Manewrując dwoma różnymi rodzajami negacji -- \textit{de re} oraz
\textit{de dicto} -- obejmuje ona swoim zasięgiem wszystkie dionizyjskie
tezy teologii negatywnej, z wyjątkiem (T1).

Druga, „negatywna” interpretacja teologii negatywnej zdołała wcielić
tezy (T2) oraz (T4), zasadniczo odrzucając jednak (T1) oraz (T3). Mimo,
iż jej formalizacja wykorzystuje jedynie logikę klasyczną, posiada ona
pewną interesującą własność. Na jej gruncie można stwierdzić, że Bóg
nie należy do kategorii przedmiotów, co wyraża pożądaną przez wielu
teologów negatywnych transcendencję Boga. Wydaje się jednak, że zarówno
teorie agnostyczne, jak i negatywna teologia negatywna nie są zgodne z
dyskursem religijnym i praktyką religijną. Poza tym, nie obejmując
swoim zasięgiem wszystkich tez Dionizego, nie mogą służyć za dobre
interpretacje rozwijanej przez niego teologii.

Wad tych pozbawiona jest ostatnia zaproponowana przez Rojka
interpretacja, którą nazywa pozytywną teologią negatywną. Jej szkicowa
formalizacja stanowi spójny model dla wszystkich czterech tez teologii
Psudo-Dionizego. Ponadto, jej treść jest zbliżona do ogólnie
przyjmowanej teorii analogii św. Tomasza z Akwinu. Według Rojka, za jej
przyjęciem przemawia dodatkowy argument -- pasuje ona do modelu
chrześcijańskiego, którym niepoznawalność Boga i jego transcendencja
wynika nie ze spekulacji, lecz z Objawienia, w którym Bóg sam siebie
określa jako ukryty.

Modele podane przez Rojka wzbudzają jednak wiele zastrzeżeń. Po
pierwsze, zachowują one spójność teologii negatywnej, jednak za cenę
wielu restrykcji i ograniczeń. Zostały one przeze mnie wymienione przy
omawianiu definicji przedmiotu religijnego w każdej z proponowanych
interpretacji.

Po drugie, Rojek (świadomie!) nie podaje żadnego satysfakcjonującego
kryterium pozytywności własności. Według Rojka, różnica między teologią
pozytywną a teologią apofatyczną polegać ma na orzekaniu o Bogu
pozytywnych bądź negatywnych własności. W takim razie, jaka jest między
nimi różnica? Które własności są pozytywne a które negatywne? Jak
stwierdza Rojek, kryterium syntaktyczne jest niewystarczające. Podaje
tu klasyczny przykład ślepoty -- predykat „jest ślepy” nie zawiera
negacji, jednakże wciąż odnosi się do braku czegoś, no jest naturalne i
czego się spodziewamy. Z tego powodu wydaje się on być własnością
negatywną. Klasyczni filozofowie taki przykład własności negatywnej
nazywali negacją w sensie braku (\textit{privatio}). Rojek dodaje, że
większość własności, które zwyczajowo przypisuje się bytowi absolutnemu
na gruncie filozofii Boga, posiada charakter negatywny. Wykazał to św.
Tomasz\footnote{Zob. np. św. Tomasz z Akwinu Teologiczna: I, 3. }.
Można próbować, tak jak robił to Bocheński próbować podać
epistemologiczne kryterium pozytywności własności. Na przykład, możemy
podać indukcyjną definicję takiej własności:

\begin{enumerate}
\item Bezpośrednio postrzegana własność jest własnością pozytywną.
\item Własność zdefiniowana przez formułę zawierającą włącznie symbole
własności pozytywnych i terminu logiki pozytywnej jest własnością
pozytywną.
\end{enumerate}
Jednakże, zarówno Bocheński, jak i Rojek odrzucają i takie kryterium,
jako niewystarczająco ścisłe. W ostateczności, zbiór własności
pozytywnych można podać przez wyliczenie. Jednakże w takim wypadku
wszystkie proponowane przez Rojka teorie będą miały mocno ograniczony
zakres. W każdym razie, jeśli ktoś próbuje ograniczyć zasięg danej
teorii do klasy własności pozytywnych, takie własności muszą zostać
zdefiniowane. Uważam to za najsłabszy punkt jego rozważań.

Problem ten dodatkowo pogłębia fakt, że Dionizy w żadnym miejscu wprost
nie stwierdza, że własności orzekane lub zaprzeczane o Bogu mają
pozytywny bądź negatywny charakter. Jest to terminologia wprowadzona w
tezy doktryny Areopagity przez samego Rojka. W obliczu tego faktu tym
bardziej dziwi, że celowo odżegnuje się on od podania stosownej
definicji wprowadzonych przez niego pojęć.

Po trzecie, w żadnej z przedstawionych interpretacji predykat „jest
Bogiem” (lub „jest bogopodobny”) nie może być własnością pozytywną. W
modelu teorii Niepoznawalnego prowadzi to do zaskakującej konsekwencji
– mianowicie jeśli coś jest Bogiem, to nie można ustalić, czy to jest
Bogiem, czy nie jest Bogiem. W modelu negatywnej teologii negatywnej z
uznania własności bycia Bogiem za pozytywną wynika zwykła sprzeczność.
Natomiast w modelu pozytywnej teologii negatywnej pozytywna własność
bycia Bogiem oznaczałaby, że jeśli coś jest Bogiem, to pozytywnie nim
nie jest, czyli posiada tę własność w wyższy, pełniejszy sposób.
Wszystkie te niedogodności spowodowane są sposobem, w jaki Rojek
formalizuje przedmiot religijny -- czynie to nie za pomocą stałej
logicznej (wtedy słowo „Bóg” traktowane byłby jak nazwa), lecz za
pomocą predykatu (w takim wypadku „Bóg” jest własnością).

Po czwarte, system formalny zaproponowany dla sformalizowania pozytywnej
teologii negatywnej, ma bardzo szkicowy i wstępny charakter. Ciężko w
jego przypadku mówić o rachunku logicznym, nie zaproponowano dla niego
żadnej semantyki, tym bardziej nie udowodniono jego trafności i
pełności. Można powiedzieć, że praca potrzebna do formalizacji tej
teorii a zatem także wykazanie jej niesprzeczności, została wykonana
jedynie częściowo, w zarysie. I jeśli chcemy mówić o spójności
pozytywnej teologii negatywnej, należałoby ją dokończyć.

Po wtóre, można powiedzieć, że teologia negatywna interpretowana przy
użyciu pojęcia „pozytywnej” negacji traci swój negatywny charakter i
staje się formą pozytywnej, afirmatywnej teologii\footnote{Por. S.
Ruczaj, Analogia i apofatyczny pazur, „Pressje”, nr 30/31 (2012),
ss.~280-282.}.

Wszystkie te zarzuty nie oznaczają jednak, że logika formalna jest
bezużyteczna w badaniu teorii i interpretacji teologii negatywnej.
Przeciwnie, uważam, że rekonstrukcja różnych wersji teologii
apofatycznej w terminach formalnych systemów logicznych może być
obustronnie korzystna. Z jednej strony, jak pokazał Rojek, może ona
podać spójne modele interpretacji, które obejmują wszystkie tezy
teologii negatywnej i tym samym pomóc w wyborze najlepszej z nich. Z
drugiej strony, badanie teologii negatywnej przy użyciu narzędzi
formalnych może pomóc w dostarczeniu pewnych formalnych kryteriów
podziału dla odmiennych pojęć negacji i zbadaniu relacji zachodzących
między nimi, co może okazać się korzystne dla logiki i jej filozofii.


%\clearpage
\begin{thebibliography}{99}
\addtocounter{section}{1}
\addcontentsline{toc}{section}{\arabic{section}\quad  {Literatura cytowana}}
\interliniatexowa{1.6}

\bibitem{} J.M. Bocheński, \textit{Logika religii}, tłum. S. Magala, [w:]: Tenże, \textit{Logika i
filozofia}. Wybór pism, Wydawnictwo Naukowe PWN, Warszawa 1993, s.
325$-$468.

\bibitem{} J.M. Bocheński, \textit{The Logic of Religion}, New York University
Press, New York 1965.

\bibitem{} J.M. Bocheński, \textit{Logika i filozofia. Wybór pism}, Wydawnictwo
Naukowe PWN, Warszawa 1993.

\bibitem{} J.Bowker (red.), \textit{The Oxford Dictonary of World Religions},
Oxford University Press, Oxford 1997.

\bibitem{} B. Brożek, A. Olszewski, M. Hohol (red.), \textit{Logic in Theology},
Copernicus Center Press, Kraków 2013.

\bibitem{} I. N. Bulhof, L. Kate (red.), \textit{Flight of the Gods. Philosophical
Perspectives on Negative Theology}, Fordham University Press, Fordham
2000.

\bibitem{} H. Burkhardt, B. Smith (red.), \textit{Handbook of Metaphysics and
Ontology}, t. 2, PhilosophiaVerlag, München 1991.

\bibitem{} B. Davies, \textit{Aquinas on What God is Not}, [w:] red. B. Davies,
\textit{Thomas Aquinas. Contemporary Philosophical Perspectives},
Oxford University Press Oxford 2002, ss. 227–242.

\bibitem{} B. Davies\textit{, Thomas Aquinas. Contemporary Philosophical
Perspectives}, Oxford University Press Oxford 2002.

\bibitem{} M. Durrant,\textit{ The Meaning of ‘God’ (I)}, [w:] \textit{Religion and
Philosophy}, red.  M. Warner, Cambridge University Press, Cambridge
1992, ss.~71$-$84.

\bibitem{} J.I. Gellman, \textit{The Meta-Philosophy of Religious Language},
„No\^us”, nr 11 (1971), ss.~151-161.

\bibitem{} G.W.F. Hegel, \textit{Logic. Being Part One of the Encyclopaedia of the
Philosophical Sciences}, tłum. W. Wallace. Oxford University Press,
Oxford 1975.

\bibitem{} J. Hick,\textit{ An Interpretation of Religion. Human Responses to the
Transcendent}, Yale University Press, New Haven – Londyn 1989.

\bibitem{} J. Wissink, \textit{Two Forms of Negative Theology Explained Using
Thomas Aquinas}, [w:] red. I. N. Bulhof, L. Kate, \textit{Flight of the
Gods. Philosophical Perspectives on Negative Theology}, Fordham
University Press, Fordham 2000, ss. 100$-$120.

\bibitem{} J.J. Jones, \textit{Sculpting God: The Logic of Dionysian Negative
Theology}, „Harvard Theological Review” nr 89 (1996), ss. 355–371.

\bibitem{} J.A. Lamm (red.), \textit{The Wiley-Blackwell Companion to Christian
Mysticism}, Wiley-Blackwell, Malden 2013.

\bibitem{} G. O'Collins, E.G. Farrugia, \textit{Leksykon pojęć
teologicznych i kościelnych}, tłum. J. Ożóg, B. Żak, Wydawnictwo WAM,
Kraków 2002.

\bibitem{} F. O'Rourke, \textit{Pseudo-Dionysius and the
Metaphysics of Aquinas}, EJ. Brill, Leiden – New York 1992.

\bibitem{} J. Paśniczek, \textit{Predykacja}, Copernicus Center Press, Kraków 2014.

\bibitem{} J. Perzanowski\textit{, Ontological Arguments II: Cartesian and
Leibnizian}, [w:] red. H. Burkhardt, B. Smith, \textit{Handbook of
Metaphysics and Ontology}, t. 2, PhilosophiaVerlag, München 1991,ss.
625–633.

\bibitem{} J. Perzanowski, \textit{The Way of Truth}, [w:] red. R. Poli, P.
Simons\textit{, Formal Ontology}, Kluwer Academic Publishers, Dordrecht
1996, ss. 61$-$130.

\bibitem{} R. Poli, P. Simons (red.), \textit{Formal Ontology}, Kluwer Academic
Publishers, Dordrecht 1996.

\bibitem{} Pseudo-Dionizy Areopagita, \textit{Pisma teologiczne}, tłum. M.
Dzielska, Wydawnictwo Znak, Kraków 1997.

\bibitem{} Pseudo-Dionizy Areopagita, \textit{Teologia Mistyczna}, [w:]
\textit{Pisma teologiczne}, tłum. M. Dzielska, Wydawnictwo Znak, Kraków
1997.

\bibitem{} Pseudo-Dionizy Areopagita, \textit{The Complete Work}, tłum. C.
Luibheid,. Paulist Press, New York 1987.

\bibitem{} P. Rojek, \textit{Logika teologii negatywnej}, „Pressje”, nr 29 (2012),
ss.~216-230.

\bibitem{} P. Rorem, Komentarze [w:] Pseudo-Dionizy Areopagita, \textit{The
Complete Work}, tłum. C. Luibheid,. Paulist Press, New York 1987.

\bibitem{} P. Rorem, \textit{Pseudo-Dionysius. A Commentary on the Texts and an
Introduction to Their Influence}, Oxford University Press, Oxford – New
York 1993.

\bibitem{} S. Ruczaj, \textit{Analogia i apofatyczny pazur}, „Pressje”, nr 30/31
(2012), ss. 280-282.

\bibitem{} P. Sikora, \textit{Logos Niepojęty}, Wydawnictwo Universitas, Kraków
2010.

\bibitem{} J. Słupecki, \textit{Stanisław Leśniewski’s Calculus of Names}, „Studia
Logica”, nr 3 (1955), ss. 7$-$76.

\bibitem{} C.M. Stang, \textit{Negative Theology from Gregory of Nyssa to Dionysius
the Areopagite}, [w:] \textit{The Wiley-Blackwell Companion to
Christian Mysticism}, J.A. Lamm (red.), Wiley-Blackwell, Malden 2013,
ss.~161-176.

\bibitem{} W. Stróżewski, \textit{Istnienie i sens}, Wydawnictw Znak, Kraków 1994.

\bibitem{} W. Stróżewski, \textit{Z problematyki negacji}, [w:] Tenże,
\textit{Istnienie i sens}, Wydawnictw Znak, Kraków 1994, ss.~373$-$395.

\bibitem{} Tomasz z Akwinu, św., \textit{Traktat o Bogu. Summa teologii}, tłum. G.
Kurylewicz, Z. Nerczuk, M. Olszewski, Wydawnictwo Znak, Kraków 2001.

\bibitem{} M. Warner (red.), \textit{Religion and Philosophy}, Cambridge University
Press, Cambridge 1992.

\bibitem{} Z. Wolak, \textit{Naukowa filozofia koła krakowskiego}, „Zagadnienia
Filozoficzne w Nauce”, nr.~36 (2005), ss.~97-122.

\bibitem{} J. Woleński, \textit{Theology and Logic}, [w:] \textit{Logic in
Theology}, red. B. Brożek et. al, Copernicus Center Press, Kraków 2013,
ss.~11-38.

\bibitem{} A. Zinowjew\textit{, Foundations of the Logical Theory of Scientific
Knowledge (Complex Logic)}, D. Reidel Publishing Company, Dordrecht
1967.

\bibitem{} A. Zinowjew, \textit{Logika nauki}, tłum. Z. Simbierowicz, PWN, Warszawa
1976.

\end{thebibliography}


\end{document}
