\begin{thebibliography}{100}
  \addcontentsline{toc}{chapter}{Bibliografia} 

\providecommand{\url}[1]{\texttt{#1}}
\providecommand{\urlprefix}{URL }
\providecommand{\selectlanguage}[1]{\relax}
%%
%% This is file `merlin.tex',
%% generated with the docstrip utility.
%%
%% The original source files were:
%%
%% merlin.mbs  (with options: `bblbst')
%% 
%% IMPORTANT NOTICE:
%% 
%% For the copyright see the source file.
%% 
%% Any modified versions of this file must be renamed
%% with new filenames distinct from merlin.tex.
%% 
%% For distribution of the original source see the terms
%% for copying and modification in the file merlin.mbs.
%% 
%% This generated file may be distributed as long as the
%% original source files, as listed above, are part of the
%% same distribution. (The sources need not necessarily be
%% in the same archive or directory.)
%% Copyright 1994-2011 Patrick W Daly
 % ===============================================================
 % IMPORTANT NOTICE:
 % This bibliographic style (bst) file has been generated from one or
 % more master bibliographic style (mbs) files, listed above.
 %
 % This generated file can be redistributed and/or modified under the terms
 % of the LaTeX Project Public License Distributed from CTAN
 % archives in directory macros/latex/base/lppl.txt; either
 % version 1 of the License, or any later version.
 % ===============================================================
 % Name and version information of the main mbs file:
 % \ProvidesFile{merlin.mbs}[2011/11/18 4.33 (PWD, AO, DPC)]
 % This is babelbst.tex for English.
 % It should serve as a model for other languages.
 % Alternatively, store it under a different name (e.g. englbst.tex)
 % and then \input it with a command in babelbst.tex.
\def\bbland{, }                \def\bbletal{i~in.}
\def\bbleditors{editors}        \def\bbleds{red.}
\def\bbleditor{editor}          \def\bbled{red.}
\def\bbledby{edited by}
\def\bbledition{edition}        \def\bbledn{wyd.}
\def\bblvolume{volume}          \def\bblvol{vol.}
\def\bblof{of}
\def\bblnumber{number}          \def\bblno{nr}
\def\bblin{[w:]}
\def\bblpages{pages}            \def\bblpp{ss.}
\def\bblpage{page}              \def\bblp{s.}
\def\bbleidpp{pages}
\def\bblchapter{chapter}        \def\bblchap{rozdz.}
\def\bbltechreport{Technical Report}
\def\bbltechrep{Tech. Rep.}
\def\bblmthesis{Master's thesis}
\def\bblphdthesis{Ph.D. thesis}
\def\bblfirst{First}            \def\bblfirsto{1st}
\def\bblsecond{Second}          \def\bblsecondo{2nd}
\def\bblthird{Third}            \def\bblthirdo{3rd}
\def\bblfourth{Fourth}          \def\bblfourtho{4th}
\def\bblfifth{Fifth}            \def\bblfiftho{5th}
\def\bblst{st}  \def\bblnd{nd}  \def\bblrd{rd}
\def\bblth{th}
\def\bbljan{January}  \def\bblfeb{February}  \def\bblmar{March}
\def\bblapr{April}    \def\bblmay{May}       \def\bbljun{June}
\def\bbljul{July}     \def\bblaug{August}    \def\bblsep{September}
\def\bbloct{October}  \def\bblnov{November}  \def\bbldec{December}
\endinput
%%
%% End of file `merlin.tex'.
\newcommand{\Capitalize}[1]{\uppercase{#1}}
\newcommand{\capitalize}[1]{\expandafter\Capitalize#1}
\providecommand{\bibAnnoteFile}[1]{%
  \IfFileExists{#1}{\begin{quotation}\noindent\textsc{Key:} #1\\
  \textsc{Annotation:}\ \input{#1}\end{quotation}}{}}
\providecommand{\bibAnnote}[2]{%
  \begin{quotation}\noindent\textsc{Key:} #1\\
  \textsc{Annotation:}\ #2\end{quotation}}

% \bibitem{Adams1993}
% E.~Adams.
% \newblock \emph{{Formalizing the logic of positive, comparative, and
%   superlative}}.
% \newblock Notre Dame Journal of Formal Logic, \bblvol{}~34 (1993), \bblno{}~1,
%   \bblpp{} 90--99.
% \bibAnnoteFile{Adams1993}

\bibitem{Alston1956}
W.P. Alston, \textit{Ineffability}, ,,The Philosophical Review'', vol 65, nr 4 (1956), ss.~506-522.

\bibitem{Altas2021}
E.~Altaş, A.~Asadov, \textit{The Power and Limits of Reason: Al-Razı on the Possibility of General and Particular Metaphysical Knowledge},
,,Nazariyat İslam Felsefe ve Bilim Tarihi Araştırmaları Dergisi (Journal for the History of Islamic Philosophy and Sciences)'', vol. 7 (2021), nr 2, s.~121-155.

% \bibitem{publ59931}
% J.~Antas.
% \newblock \emph{{O mechanizmach negowania: wybrane semantyczne i pragmatyczne
%   aspekty negacji}}.
% \newblock Towarzystwo Autor{\'{o}}w i Wydawc{\'{o}}w Prac Naukowych
%   Universitas, Krak{\'{o}}w 1991.
% \bibAnnoteFile{publ59931}

% \bibitem{Atlas1977}
% J.~D. Atlas.
% \newblock \emph{{Negation, Ambiguity, and Presupposition}}.
% \newblock Linguistics and Philosophy, \bblvol{}~1 (1977), \bblno{}~3, \bblpp{}
%   321--336.
% \bibAnnoteFile{Atlas1977}

\bibitem{sep-simplicity}
A. Baker, \textit{Simplicity}, [w:]~\textit{The Stanford Encyclopedia of Philosophy},
wyd. zima 2016, red. E.N. Zalta, {\textless}\url{https://plato.stanford.edu/archives/win2016/entries/simplicity/}{\textgreater}.

\bibitem{Barnes1994}
L.P. Barnes, \textit{Rudolf Otto and the Limits of Religious Description}, ,,Religious Studies'', vol. 30 (1994), nr 2, ss.~219-230.

% \bibitem{Barrett1992}
% C.~Barrett.
% \newblock \emph{{The logic of mysticism–II}}.
% \newblock \capitalize\bblin{} M.~Warner (\bbled{}), \emph{Religion and
%   Philosophy. Royal Institute of Philosophy Supplement}, \bblvol{}~31,
%   Cambridge University Press, Cambridge 1992.
% \newblock \bblpp{} 61--69.
% \bibAnnoteFile{Barrett1992}

\bibitem{Barth1957}
K. Barth, \textit{Church Dogmatics}, tom 2, cz. 1, tłum. T.F. Torrance, G.W. Bromiley, T\&T Clark, Edinburgh 1957.

\bibitem{sep-liar-paradox}
J.C. Beall, M. Glanzberg, D. Ripley, \textit{Liar Paradox}, [w:] \textit{The Stanford Encyclopedia of Philosophy},
wyd. jesień 2020, red. E.N. Zalta, {\textless}\url{https://plato.stanford.edu/archives/fall2020/entries/liar-paradox/}{\textgreater}.

% \bibitem{Ben-Yami2014}
% H.~Ben-Yami.
% \newblock \emph{{The quantified argument calculus}}.
% \newblock Review of Symbolic Logic, \bblvol{}~7 (2014), \bblno{}~1, \bblpp{}
%   120--146.
% \bibAnnoteFile{Ben-Yami2014}

\bibitem{Benor1995}
E.Z. Benor, \textit{Meaning and reference in Maimonides' negative theology}, ,,Harvard Theological Review'', vol. 88 (1995), nr 3, ss.~339-360.

\bibitem{Benton2017}
M.A. Benton, \textit{Epistemology Personalized}, ,,The Philosophical Quarterly'' 67 (2017), nr 269, ss.~813–834.

\bibitem{Benton2018}
M.A. Benton, \textit{God and Interpersonal Knowledge}, ,,Res Philosophica'', vol. 95 (2018), nr 3, ss.~421-447.

% \bibitem{Bergmann2006}
% M.~Bergmann\bbland{}J.~E. Brower.
% \newblock \emph{{A Theistic Argument Against Platonism (and in Support of
%   Truthmakers and Divine Simplicity)}}.
% \newblock Oxford Studies in Metaphysics, \bblvol{}~2 (2006), \bblpp{} 357--386.
% \bibAnnoteFile{Bergmann2006}

\bibitem{Berlin1991}
B. Berlin, P. Kay, \textit{Basic Color Terms: Their Universality and Evolution}, University of California Press, Berkeley 1991.

\bibitem{Bezhanishvili2019}
N. Bezhanishvili, A. Colacito, D. de Jongh, \textit{A~Study of Subminimal Logics of Negation and Their Modal Companions},
[w:] \textit{Language, Logic, and Computation}, red.~A.~Silva, S.~Staton, P. Sutton, C. Umbach, Springer, Berlin -- Heidelberg 2019, ss.~21-41.

\bibitem{Bliss2020}
R. Bliss, \textit{Fundamentality}, [w:] \textit{The Routledge Handbook of Metametaphysics}, red.~tenże, J.T.M. Miller, Routledge, Abingdon -- New York 2020, ss.~211-221.

\bibitem{Bliss2018}
R. Bliss, G. Priest (red.), \textit{Reality and its Structure: Essays in Fundamentality}, Oxford University Press, Oxford 2018.

\bibitem{Bochenski1965}
J.M. Bocheński, \textit{The Logic of Religion}, New York University Press, New York 1965.

\bibitem{Bochenski1990}
J.M.~Bocheński, \textit{Logika religii}, tłum. S. Magala, Instytut Wydawniczy PAX, Warszawa 1990.

\bibitem{Bochenski1993}
J.M. Bocheński, \textit{Logika i~filozofia}, Wydawnictwo Naukowe PWN, Warszawa 1993.

% \bibitem{Boesel2010}
% C.~Boesel\bbland{}C.~Keller.
% \newblock \emph{{Apophatic Bodies. Negative Theology, Incarnation and
%   Relationality}}.
% \newblock Fordham University Press, New York 2010.
% \bibAnnoteFile{Boesel2010}

\bibitem{sep-self-reference}
T. Bolander, \textit{Self-Reference}, [w:] \textit{The Stanford Encyclopedia of Philosophy},
wyd. jesień 2017, red. E.N. Zalta, {\textless}\url{https://plato.stanford.edu/archives/fall2017/entries/self-reference/}{\textgreater}.

\bibitem{Boroditsky2001}
L. Boroditsky, \textit{Does Language Shape Thought? Mandarin and English Speakers' Conceptions of Time}, ,,Cognitive Psychology'', vol. 43 (2001), ss.~1-22.

\bibitem{Boroditsky2003}
L. Boroditsky, L.A. Schmidt, W. Phillips, \textit{Sex, Syntax, and Semantics},
[w:] \textit{Language in Mind: Advances in the Study of Language and Thought}, red. D. Gentner, S.~Goldin-Meadow, The MIT Press, Cambridge -- London 2003.

\bibitem{Bowker2000}
J. Bowker, \textit{Apophatic theology}, [w:] \textit{The Concise Oxford Dictionary of World Religions}, red. tenże, Oxford University Press, Oxford 2000.

% \bibitem{Broadie1987}
% A.~Broadie.
% \newblock \emph{{Maimonides and Aquinas on the names of god}}.
% \newblock Religious Studies, \bblvol{}~23 (1987), \bblno{}~2, \bblpp{}
%   157--170.
% \bibAnnoteFile{Broadie1987}

\bibitem{sep-fitch-paradox}
B. Brogaard, J. Salerno, \textit{Fitch's Paradox of Knowability}, [w:] \textit{The Stanford Encyclopedia of Philosophy},
wyd. jesień 2019, red. E.N. Zalta, <\url{https://plato.stanford.edu/archives/fall2019/entries/fitch-paradox/}>.

% \bibitem{Brower2008}
% J.~E. Brower.
% \newblock \emph{{Making sense of divine simplicity}}.
% \newblock Faith and Philosophy, \bblvol{}~25 (2008), \bblno{}~1, \bblpp{}
%   3--30.
% \bibAnnoteFile{Brower2008}

\bibitem{Brozek2002}
B. Brożek, \textit{Rola paradoksu kłamcy w~konstrukcji logicznych teorii prawdy}, ,,Zagadnienia Filozoficzne w~Nauce'', nr 30 (2002), ss.~48-88.

\bibitem{Brozek2010a}
B. Brożek, \textit{The Double Truth Controversy: An Analytical Essay}, Copernicus Center Press, Kraków 2010.

\bibitem{Brozek2016}
B. Brożek, \textit{Marzenie Leibniza. Rzecz o~języku religii}, Copernicus Center Press, Kraków 2016.

\bibitem{Brozek2010}
B.~Brożek, A. Olszewski, \textit{Kilka uwag o~kryterium Quine'a}, ,,Filozofia Nauki'', vol 18 (2010), nr 1, ss.~5-15.

\bibitem{Brozek2013}
B. Brożek, A. Olszewski, M. Hohol (red.), \textit{Logic in Theology}, Copernicus Center Press, Kraków 2013.

\bibitem{Brylla2017}
D. Brylla, \textit{Rozważania o~apofatycznej kategorii ,,negacja negacji}'', ,,Seminare'', vol.~38 (2017), nr 1, ss.~65-76.

% \bibitem{Bucur2007}
% B.~G. Bucur.
% \newblock \emph{{The Theological Reception of Dionysian Apophatism in the
%   Christian East and West: Thomas Aquinas and Gregory Palamas}}.
% \newblock The Downside Review, \bblvol{} 125 (2007), \bblno{} 439, \bblpp{}
%   131--146.
% \bibAnnoteFile{Bucur2007}

\bibitem{Buijs1975}
J.A. Buijs, \textit{Comments on Maimonides' negative theology}, ,,The New Scholasticism'', vol. 49 (1975), nr 1, ss.~87-93.

\bibitem{Buijs1988}
J.A. Buijs, \textit{The negative theology of Maimonides and Aquinas}, ,,Review of Metaphysics'', vol. 41 (1988), nr 164, s.~723-738.

% \bibitem{Bulhof2000}
% I.~N. Bulhof\bbland{}L.~ten {Kate (Ed.)}.
% \newblock \emph{{Flight of the Gods}}.
% \newblock New York 2000.
% \bibAnnoteFile{Bulhof2000}

\bibitem{Bunnin2009}
N. Bunnin, J. Yu, \textit{Negative theology}, [w:] ciż, \textit{The Blackwell Dictionary of Western Philosophy}, Blackwell Publishing, Malden -- Oxford 2009, ss.~465-466.


\bibitem{Cameron2008}
R.P. Cameron, \textit{Truthmakers and ontological commitment: or how to deal with complex objects and mathematical ontology without getting into trouble},
,,Philosophical Studies: An International Journal for Philosophy in the Analytic Tradition'', vol.~140 (2008), nr 1, Selected Papers from the 2007 Bellingham Summer Philosophy Conference, ss.~1-18

\bibitem{sep-paradoxes-contemporary-logic}
A. Cantini. R. Bruni, \textit{Paradoxes and Contemporary Logic}, [w:] \textit{The Stanford Encyclopedia of Philosophy},
wyd. jesień 2021, red. E.N. Zalta, {\textless}\url{https://plato.stanford.edu/archives/fall2021/entries/paradoxes-contemporary-logic/}{\textgreater}.

\bibitem{Carabine1995}
D. Carabine, \textit{The Unknown God. Negative Theology in the Platonic Tradition: Plato to Eriugena}, \textit{Louvain Theological and Pastoral Monographs}, Peeters Press, Louvain 1995.

% \bibitem{Carrasquillo2013}
% F.~J. Carrasquillo.
% \newblock \emph{{The intertwining of multiplicity and unity in Dionysius'
%   metaphysical mysticism}}.
% \newblock Topicos (Mexico),  (2013), \bblno{}~44, \bblpp{} 207--236.
% \bibAnnoteFile{Carrasquillo2013}

% \bibitem{Chalmers2009}
% D.~J. Chalmers, D.~Manley\bbland{}R.~Wasserman.
% \newblock \emph{{Metametaphysics: new essays on the foundations of ontology}}
%   2009.
% \bibAnnoteFile{Chalmers2009}

\bibitem{Church1956}
A. Church, \textit{Introduction to Mathematical Logic}, Princeton University Press, Princeton 1956.

\bibitem{Church2010}
A. Church, \textit{Referee Reports on Fitch's ``A Definition of Value}'', [w:] \textit{New Essays on the Knowability Paradox},
red. J. Salerno, Oxford University Press, New York 2009, ss.~13-20.

\bibitem{ClaudiodeAlmeida2001}
C. de Almeida, \textit{What Moore's Paradox is About}, ,,Philosophy and Phenomenological Research'', vol. 62 (2001), nr 1, s.~33-58.

\bibitem{Cohoe2017}
C.M. Cohoe, \textit{Why the One cannot have parts: Plotinus on divine simplicity, ontological independence, and perfect being theology}, ,,Philosophical Quarterly'', vol. 67 (2017), nr 269, ss.~751-771.

% \bibitem{Colacito2016a}
% A.~Colacito.
% \newblock \emph{{Minimal and Subminimal Logic of Negation}} 2016.
% \bibAnnoteFile{Colacito2016a}

\bibitem{Colacito2016}
A.~Colacito, D.~de~Jongh\bbland{}A.~L. Vargas, \textit{Subminimal negation}, ,,Soft Computing'', vol. 21 (2016), ss.~165-174

% \bibitem{Constas2003}
% N.~Constas \bbletal{}.
% \newblock \emph{{Dionysius the Areopagite : Select Bibliography}}.
% \newblock \bblvol{}~5 (2003), \bblpp{} 0--3.
% \bibAnnoteFile{Constas2003}

\bibitem{Corrigan2019}
K. Corrigan, M.L. Harrington, \textit{Pseudo-Dionysius the Areopagite}, [w:] \textit{The Stanford Encyclopedia of Philosophy},
wyd. zima 2019, red. E.N. Zalta, {\textless}\url{https://plato.stanford.edu/archives/spr2015/entries/pseudo-dionysius-areopagite/}{\textgreater}

\bibitem{Czernecka2020}
B. Czernecka-Rej, \textit{On Four Types of Argumentation For Classical Logic}, ,,Roczniki Filozoficzne'', vol. 68, nr 4 (2020), ss.~271-289.

\bibitem{Czelakowski2022}
J.~Czelakowski, A.~Olszewski, \textit{Logics of Order and Related Notions}, ,,Studia Logica'' 2022.

\bibitem{Dadaczynski2013}
J. Dadaczyński, \textit{What kind of logic does contemporary theology need}?, [w:] \textit{Logic in Theology}, red. B. Brożek, A. Olszewski, M. Hohol, Copernicus Center Press, Kraków 2013, ss.~39-60.

\bibitem{Dadaczynski2014}
J. Dadaczyński, \textit{Kilka uwag o logice teologii}, ,,Zagadnienia Filozoficzne w~Nauce'', nr~57 (2014), ss.~33-58.

\bibitem{Damascius2010}
Damascius, \textit{Problems \& Solutions Concerning First Principles}, tłum. S. Ahbel-Rappe, Oxford University Press, Oxford 2010.

\bibitem{Davies2002}
B.~Davies, \textit{Aquinas on What God is Not}, [w:]  \textit{Thomas Aquinas. Contemporary Philosophical Perspectives}, red. tenże, Oxford University Press, Oxford 2002, ss.~227-242.

% \bibitem{Derrida1992}
% J.~Derrida.
% \newblock \emph{{How to Avoid Speaking: Denials}}.
% \newblock \capitalize\bblin{} \emph{Derrida and Negative Theology}, State
%   University of New York Press 1992.
% \newblock \bblpp{} 73--142.
% \newblock \urlprefix\url{https://books.google.pl/books?id=xv9JLVBNHSYC}.
% \bibAnnoteFile{Derrida1992}

\bibitem{Dodds1928}
E. Dodds, \textit{The Parmenides of Plato and the Origin of the Neoplatonic ‘One'}, ,,The Classical Quarterly'', vol. 22 (1928), nr 3-4, ss.~129-142.

% \bibitem{Dosen1987}
% K.~Dosen.
% \newblock \emph{{Negation and impossibility}}.
% \newblock \capitalize\bblin{} J.~Perzanowski (\bbled{}), \emph{Negation and
%   impossibility}, Jagiellonian University Press, Krak{\'{o}}w 1987.
% \newblock \bblpp{} 85--91.
% \bibAnnoteFile{Dosen1987}

% \bibitem{Drob2016}
% S.~L. Drob.
% \newblock \emph{{The Doctrine of Coincidentia Oppositorum in Jewish
%   Mysticism}}.
% \newblock Kabbalah and Postmodernism,  (2016), \bblpp{} 1--20.
% \bibAnnoteFile{Drob2016}

\bibitem{durrant1973logical}
M. Durrant, \textit{The Logical Status of ‘God' and the Function of Theological Sentences}, Macmillan, Edinburgh 1973.

\bibitem{Durrant1992}
M. Durrant, \textit{The Meaning of ‘God'– I}, [w:] \textit{Religion and Philosophy}, red. M. Warner, Cambridge University Press, Cambridge 1992, ss.~71-84.

\bibitem{Dworak2012}
P.~Dwořák, \textit{Teologia negatywna i~analogia}, ,,Pressje'', nr 30/31 (2012), ss.~278-280.

\bibitem{Dzidek2001}
T. Dzidek, \textit{Teologia apofatyczna. Uznana bezradność rozumu}, [w:] tenże, \textit{Granice rozumu w~teologicznym poznaniu Boga}, Wydawnictwo M, Kraków 2001, ss.~275-314.

\bibitem{Englebretsen1973}
G. Englebretsen, \textit{The Logic of Negative Theology}, ,,New Scholasticism'', vol. 47 (1973), ss.~228--232.

% \bibitem{Englebretsen1975}
% G.~Englebretsen.
% \newblock \emph{{Sommers' theory and natural theology}}.
% \newblock International Journal for Philosophy of Religion, \bblvol{}~6 (1975),
%   \bblno{}~2, \bblpp{} 111--116.
% \bibAnnoteFile{Englebretsen1975}

% \bibitem{Fagenblat2017}
% M.~Fagenblat.
% \newblock {\selectlanguage{English}\emph{{Negative Theology As Jewish
%   Modernity}}}.
% \newblock New Jewish Philosophy and Thought, Indiana University Press,
%   Bloomington and Indianapolis 2017.
% \bibAnnoteFile{Fagenblat2017}

\bibitem{Fakhri2021}
O. Fakhri, \textit{The ineffability of God}, ,,International Journal for Philosophy of Religion'', vol. 89 (2021), ss.~25-41.

\bibitem{Fine2009}
K. Fine, \textit{The Question of Ontology}, [w:] \textit{Metametaphysics. New Essays on the Foundations of Ontology},
red. D. Chalmers i~in., Oxford University Press, Oxford -- New York 2009, ss.~157-177.

% \bibitem{Fisher2001}
% J.~Fisher.
% \newblock \emph{{The theology of dis/similarity: Negation in
%   pseudo-dionysius}}.
% \newblock Journal of Religion, \bblvol{}~81 (2001), \bblno{}~4, \bblpp{}
%   529--547.
% \bibAnnoteFile{Fisher2001}

\bibitem{Fitch1963}
F.B. Fitch, \textit{A~Logical Analysis of Some Value Concepts}, ,,The Journal of Symbolic Logic'', vol.~28, (1963), nr 2, ss.~135-142.

\bibitem{Franck1985}
I. Franck, \textit{Maimonides and Aquinas on Man's Knowledge of God: A~Twentieth Century Perspective}, ,,Review of Metaphysics'', vol. 38 (1985), nr 3, ss.~591-615.

% \bibitem{Friedman1992}
% R.~Z. Friedman.
% \newblock \emph{{Maimonides and Kant on Metaphysics and Piety}}.
% \newblock The Review of Metaphysics, \bblvol{}~45 (1992), \bblno{}~4, \bblpp{}
%   773--801.
% \newblock \urlprefix\url{http://www.jstor.org/stable/20129254}.
% \bibAnnoteFile{Friedman1992}

\bibitem{Gab2020}
S. Gäb, \textit{Languages of ineffability: the rediscovery of apophaticism in contemporary analytic philosophy of religion},
[w:] \textit{Negative Knowledge}, red. S. Hüsch i~in., Narr, Tübingen 2020, ss.~191-206.

\bibitem{sep-logic-modal}
J. Garson, \textit{Modal Logic}, [w:] \textit{The Stanford Encyclopedia of Philosophy},
wyd. lato 2021, red. E.N. Zalta, {\textless}\url{https://plato.stanford.edu/archives/sum2021/entries/logic-modal/}{\textgreater}.

% \bibitem{Geach1992}
% P.~Geach.
% \newblock \emph{{The Meaning of ' God '— II}}.
% \newblock \capitalize\bblin{} M.~Warner (\bbled{}), \emph{Religion and
%   Philosophy. Royal Institute of Philosophy Supplement}, Cambridge University
%   Press, Cambridge 1992.
% \newblock \bblpp{} 85--90.
% \bibAnnoteFile{Geach1992}

\bibitem{Gellman1977}
J.I. Gellman, \textit{The Meta-Philosophy of Religious Language}, ,,Nous'', vol. 11 (1971), ss.~151-161.

\bibitem{Gerson2018}
L. Gerson, \textit{Plotinus}, [w:] \textit{The Stanford Encyclopedia of Philosophy},
wyd. jesień 2018, red. E.N. Zalta, {\textless}\url{https://plato.stanford.edu/archives/fall2018/entries/plotinus/}{\textgreater}.

\bibitem{GerritMannoury1947}
G. Mannoury, \textit{Les fondements psycho-linguistiques des mathématiques}, Éditions du Griffon, Neuchâtel 1947.

\bibitem{Gettier1963}
E.L. Gettier, \textit{Is Justified True Belief Knowledge}?, ,,Analysis'', vol. 23 (1963), nr 6, ss.~121-123.

% \bibitem{Giora2006}
% R.~Giora.
% \newblock \emph{{Anything negatives can do affirmatives can do just as well,
%   except for some metaphors}}.
% \newblock Journal of Pragmatics, \bblvol{}~38 (2006), \bblno{}~7, \bblpp{}
%   981--1014.
% \bibAnnoteFile{Giora2006}

\bibitem{Glock2001}
H.-J. Glock, \textit{Słownik Wittgensteinowski}, tłum. M. Hernik, M. Szczubiałka, Wydawnictwo Spacja, Warszawa 2001.

% \bibitem{Grabarczyk2008}
% P.~Grabarczyk.
% \newblock \emph{{Od wra{\.{z}}enia do wyra{\.{z}}enia , czyli o
%   niezdeterminowaniu przek{\l}adu i wzgl{\c{e}}dno{\'{s}}ci odniesienia}}.
% \newblock Przegl{\c{a}}d Filozoficzny — Nowa Seria, \bblvol{}~17 (2008),
%   \bblno{}~4, \bblpp{} 223--232.
% \bibAnnoteFile{Grabarczyk2008}

\bibitem{Grelling1936}
K. Grelling, \textit{The Logical Paradoxes}, ,,Mind'', vol. 45 (1936), nr 180, ss.~481-486.

\bibitem{Gruszczynski2010}
R. Gruszczyński, R. Pietruszczak, \textit{How to define a~mereological (collective) set}, ,,Logic and Logical Philosophy'', vol. 19 (2010), nr 4, ss.~309-328.

% \bibitem{Gruszczynski2015}
% R.~Gruszczynski\bbland{}A.~C. Varzi.
% \newblock \emph{{Mereology then and now}}.
% \newblock Logic and Logical Philosophy, \bblvol{}~24 (2015), \bblno{}~4,
%   \bblpp{} 409--427.
% \bibAnnoteFile{Gruszczynski2015}

% \bibitem{Hasson2006}
% U.~Hasson\bbland{}S.~Glucksberg.
% \newblock \emph{{Does understanding negation entail affirmation? An examination
%   of negated metaphors}}.
% \newblock Journal of Pragmatics, \bblvol{}~38 (2006), \bblno{}~7, \bblpp{}
%   1015--1032.
% \bibAnnoteFile{Hasson2006}

\bibitem{Hegel1975}
G.W.F. Hegel, \textit{Logic. Being. Part One of the Encyclopaedia of the Philosophical Sciences}, tłum. W. Wallace. Oxford University Press, Oxford 1975.

% \bibitem{Heller2016}
% M.~Heller.
% \newblock \emph{{Teologia dzisiaj - detronizowanie kr{\'{o}}lowej}}.
% \newblock \capitalize\bblin{} M.~Wiertek (\bbled{}), \emph{Promotio Doctoris
%   Honoris Causa Pontificiae Universitatis Cracoviensis Ioannis Pauli II
%   Reverendissimus Professor Michael Heller}, Uniwersytet Papieski Jana Paw{\l}a
%   II w Krakowie, Krak{\'{o}}w 2016.
% \newblock \bblpp{} 53--62.
% \bibAnnoteFile{Heller2016}

\bibitem{Hick1989}
J. Hick, \textit{An Interpretation of Religion. Human Responses to the Transcendent}, Yale University Press, New Haven -- London 1989.

\bibitem{Hick1995}
J. Hick, \textit{A~Christian Theology of Religions: The Rainbow of Faiths}, Westminster John Knox Press, Louisville 1995.

\bibitem{Hick2000}
J. Hick, \textit{Ineffability}, ,,Religious Studies'', vol. 36 (2000), ss.~35-46.

\bibitem{Hintikka1962}
J. Hintikka, \textit{Knowledge and Belief}, Cornell University Press, Ithaka 1962.

\bibitem{Holliday2018}
W.H.~Holliday, \textit{Epistemic Logic and Epistemology}, [w:] \textit{Introduction to Formal Philosophy}, red. S.O.~Hansson, V.F. Hendricks, Springer, Cham 2018, ss.~351-369.

\bibitem{Horn2001}
L.R. Horn, \textit{A~Natural History of Negation}, CSLI Publications, Stanford 2001.

\bibitem{Hryniewicz1989}
W. Hryniewicz, \textit{Apofatyczna teologia}, [w:] \textit{Encyklopedia katolicka}, t.~1, red. F. Gryglewicz i~in., TN KUL, Lublin 1989, kol. 745-748.

\bibitem{Hryniewicz2000}
W. Hryniewicz, \textit{Katafatyczna teologia}, [w:] \textit{Encyklopedia katolicka}, t.~8, red. F. Gryglewicz i~in., TN KUL, Lublin 2000, kol. 976-978.

\bibitem{Insole2000}
C.J. Insole, \textit{Why John Hick cannot, and should not, stay out of the jam pot}, ,,Religious Studies'', vol. 36 (2000), ss.~25-33.

\bibitem{Insole2002}
C.J. Insole, \textit{Metaphor and the Impossibility of Failing to Speak about God}, ,,International Journal for Philosophy of Religion'', vol. 52 (2002), ss.~35-43.

\bibitem{J.MichaelDunn1996}
J.M. Dunn, \textit{Generalized Ortho Negation}, [w:] \textit{Negation: A~Notion in Focus}, red.~H.~Wansing, Walter de Gruyter, Berlin -- New York 1996, ss.~3-26.

\bibitem{Jacobs2015-JACTII-2}
J.D. Jacobs, \textit{The Ineffable, Inconceivable, and Incomprehensible God: Fundamentality and Apophatic Theology},
[w:] \textit{Oxford Studies in Philosophy of Religion VI}, red. R. Audi i~in., Oxford University Press, New York 2015, ss.~158-176.

\bibitem{Jespersen2017}
B. Jespersen, M. Carrara, M. Duží, \textit{Iterated privation and positive predication}, ,,Journal of Applied Logic'', vol. 25 (2017), ss.~S48-S71.

% \bibitem{Joe2002}
% J.~Joe\bbland{}C.~Lee.
% \newblock \emph{{A ‘Removal' Type of Negative Predicates}}.
% \newblock \capitalize\bblin{} N.~Akatsuka, S.~Strauss\bbland{}B.~Comrie
%   (\bbleds{}), \emph{Japanese-Korean Linguistics: Volume 10}, Center for the
%   study of language and information 2002.
% \newblock \bblpp{} 559--572.
% \newblock \urlprefix\url{https://books.google.pl/books?id=Es2_xQEACAAJ}.
% \bibAnnoteFile{Joe2002}

\bibitem{Johnson2003}
K.E. Johnson, \textit{Divine Transcendence, Religious Pluralism and Barth's Doctrine of God}, ,,International Journal of Systematic Theology'', vol. 5 (2003), nr 2, ss. 200-224.

\bibitem{Jones1996}
J.N. Jones, \textit{Sculpting God: The Logic of Dionysian Negative Theology}, ,,Harvard Theological Review'', vol. 89 (1996), ss.~355–371.

% \bibitem{Jugrin2018}
% D.~Jugrin.
% \newblock \emph{{Negative theology in contemporary interpretations}}.
% \newblock European Journal for Philosophy of Religion, \bblvol{}~10 (2018),
%   \bblno{}~2, \bblpp{} 149--170.
% \bibAnnoteFile{Jugrin2018}

% \bibitem{Kabzinski2003}
% J.~K. Kabzi{\'{n}}ski.
% \newblock \emph{{Absurd okre{\'{s}}laj{\c{a}}cy negacj{\c{e}}}} 2003.
% \bibAnnoteFile{Kabzinski2003}

\bibitem{Kahn1992}
C.H. Kahn, \textit{Byt u~Parmenidesa i~Platona}, ,,Przegląd Filozoficzny -- Nowa Seria'', vol.~1 (1992), nr 4, ss.~95-116.

% \bibitem{Kalamaras1997}
% G.~Kalamaras.
% \newblock \emph{{The center and circumference of silence: Yoga,
%   poststructuralism, and the rhetoric of paradox}}.
% \newblock International Journal of Hindu Studies, \bblvol{}~1 (1997),
%   \bblno{}~1, \bblpp{} 3--18.
% \bibAnnoteFile{Kalamaras1997}

\bibitem{Kaplan1960}
D. Kaplan, R. Montague, \textit{A~paradox regained}, ,,Notre Dame Journal of Formal Logic'', vol. 1 (1960), nr 3, ss.~79-90.

\bibitem{Kars2018}
A. Kars, \textit{What is ``negative theology''? Lessons from the encounter of two sufis}, ,,Journal of the American Academy of Religion'', vol. 86 (2016), nr 1, ss.~181-211.

% \bibitem{Kasher1994}
% H.~Kasher.
% \newblock \emph{{Self-Cognizing Intellect and Negative Attributes in
%   Maimonides' Theology}}.
% \newblock Harvard Theological Review, \bblvol{}~87 (1994), \bblno{}~4, \bblpp{}
%   461--472.
% \bibAnnoteFile{Kasher1994}

% \bibitem{Kaup2006}
% B.~Kaup, J.~L{\"{u}}dtke\bbland{}R.~A. Zwaan.
% \newblock \emph{{Processing negated sentences with contradictory predicates: Is
%   a door that is not open mentally closed?}}
% \newblock Journal of Pragmatics, \bblvol{}~38 (2006), \bblno{}~7, \bblpp{}
%   1033--1050.
% \bibAnnoteFile{Kaup2006}

\bibitem{Keller2018}
L.J. Keller, \textit{Divine ineffability and franciscan knowledge}, ,,Res Philosophica'', vol. 95 (2018), nr 3, ss.~347-370.

% \bibitem{Kenney1993}
% J.~P. Kenney.
% \newblock \emph{{The Critical Value of Negative Theology}}.
% \newblock Harvard Theological Review, \bblvol{}~86 (1993), \bblno{}~4, \bblpp{}
%   439--453.
% \bibAnnoteFile{Kenney1993}

\bibitem{Klonowski2019}
M. Klonowski, K. Krawczyk, \textit{Problem wszechwiedzy logicznej. Krytyka światów nienormalnych i~propozycja nowego rozwiązania}, ,,Filozofia Nauki'', vol. 27 (2019), nr~1, ss.~27-48.

% \bibitem{Knepper2009}
% T.~D. Knepper.
% \newblock \emph{{Three misuses of Dionysius for comparative theology}}.
% \newblock Religious Studies, \bblvol{}~45 (2009), \bblno{}~2, \bblpp{}
%   205--221.
% \bibAnnoteFile{Knepper2009}

\bibitem{Knepper2017}
T.D.~Knepper, L.E.~Kalmanson (red.), \textit{Ineffability: An Exercise in Comparative Philosophy of Religion}, Springer, Cham 2017.

\bibitem{Kortmann1993}
B. Kortmann, G.K. Pullum, \textit{The Great Eskimo Vocabulary Hoax and Other Irreverent Essays on The Study of Language}, University of Chicago Press, Chicago -- London 1991.

\bibitem{Kovecses2011}
Z. Kövecses, \textit{Język, umysł, kultura}, tłum. A. Kowlacze-Pawlik, M. Buchta, Wydawnictwo Universitas, Kraków 2011

% \bibitem{Kraft2012}
% J.~Kraft.
% \newblock \emph{{Skepticism between Beginner's and Lottery Luck}}.
% \newblock \capitalize\bblin{} \emph{The Epistemology of Religious Disagreement}
%   2012.
% \newblock \bblpp{} 59--68.
% \bibAnnoteFile{Kraft2012}

\bibitem{Kripke1975}
S. Kripke, \textit{Outline of a~Theory of Truth}, ,,The Journal of Philosophy'', vol. 72 (1975), ss.~690–716.

\bibitem{Krol2013}
Z. Król, \textit{The Implicit Logic of Plato's Parmenides}, ,,Filozofia Nauki'', vol. 21 (2013), nr 1, ss.~121-135.

\bibitem{Kubic2019}
A. Kubić, ,,\textit{Niezdeterminowanie'' granic poznania w~znaturalizowanym relatywizmie pojęciowym W.V.O. Quine'a},
,,Annales Universitatis Mariae Curie-Sklodowska, sectio I~-- Philosophia-Sociologia'', vol. 43 (209), nr 2, ss.~73-92.

% \bibitem{Kugler2003}
% P.~K{\"{u}}gler.
% \newblock \emph{{The logic and language of Nirvāna: A contemporary
%   interpretation}}.
% \newblock International Journal for Philosophy of Religion, \bblvol{}~53
%   (2003), \bblno{}~2, \bblpp{} 93--110.
% \bibAnnoteFile{Kugler2003}

\bibitem{Kugler2005}
P. Kügler, \textit{The meaning of mystical ‘darkness'}, ,,Religious Studies'', vol. 41 (2005),  ss.~95-105.

\bibitem{Kukla1998}
A. Kukla, \textit{Ineffability and Philosophy}, Routledge, London -- New York 1998.

\bibitem{Kvanvig2006}
J.L Kvanvig, \textit{The Knowability Paradox}, Oxford University Press, New York 2006.

\bibitem{LakoffJohnson}
G. Lakoff, M. Johnson, \textit{Metafory w~naszym życiu}, tłum. T.P. Krzeszowski, Wydawnictwo Aletheia, Warszawa 2010.

\bibitem{LakoffNunez}
G. Lakoff, R.E. Núñez, \textit{Where Mathematics Comes From. How the Embodied Mind Brings Mathematics into Being}, Basic Books, New York 2000.

\bibitem{Langacker2011}
R. Langacker, \textit{Gramatyka kognitywna. Wprowadzenie}, Universitas, Kraków 2011.

\bibitem{Lebens2017}
S.R.~Lebens, \textit{Negative Theology as Illuminating and/or Therapeutic Falsehood}, [w:] \textit{Negative Theology as Jewish Modernity},
red. M. Fagenblat, Indiana University Press, Bloomington 2017, ss.~85-108.

\bibitem{Lebens2014}
S.R. Lebens, \textit{Why so negative about negative theology? The search for a~Plantinga-proof apophaticism},
,,International Journal for Philosophy of Religion'', vol. 76 (2014), nr 3, ss.~259-275.

% \bibitem{Leftow2011}
% B.~Leftow.
% \newblock \emph{{Why perfect being theology?}}
% \newblock International Journal for Philosophy of Religion, \bblvol{}~69
%   (2011), \bblno{}~2, \bblpp{} 103--118.
% \bibAnnoteFile{Leftow2011}

% \bibitem{Leszczynski2014}
% R.~M. Leszczy{\'{n}}ski.
% \newblock \emph{{Koncepcja Henady w ontologii Plotyna}}.
% \newblock Studia Religiologica. Zeszyty Naukowe Uniwersytetu
%   Jagiello{\'{n}}skiego, \bblvol{}~47 (2014), \bblno{}~2, \bblpp{} 89--104.
% \bibAnnoteFile{Leszczynski2014}

% \bibitem{Light1998}
% A.~Light.
% \newblock \emph{{Sculpting god: An exchange (2)}}.
% \newblock Harvard Theological Review, \bblvol{}~91 (1998), \bblno{}~2, \bblpp{}
%   205--206.
% \bibAnnoteFile{Light1998}

\bibitem{sep-pm-notation}
B. Linsky, \textit{The Notation in Principia Mathematica}, [w:] \textit{The Stanford Encyclopedia of Philosophy},
wyd. zima 2021, E.N. Zalta, <\url{https://plato.stanford.edu/archives/win2021/entries/pm-notation/}>.

% \bibitem{Lourie2013}
% B.~Louri{\'{e}}.
% \newblock \emph{{Philosophy of Dionysius the Areopagite: Modal Ontology}}.
% \newblock \capitalize\bblin{} \emph{Logic in Orthodox Christian Thinking},
%   \bblvol{}~X 2013.
% \newblock \bblpp{} 230--257.
% \bibAnnoteFile{Lourie2013}

\bibitem{Lukowski1997}
P. Łukowski, \textit{An approach to the liar paradox}, ,,RIMS Kokyuroku'', vol. 1010 (1997), ss.~68-80.

\bibitem{ukowski2006}
P. Łukowski, \textit{Paradoksy}, Wydawnictwo Uniwersytetu Łódzkiego, Łódź 2006.

\bibitem{Maimonides1963}
M. Maimonides, \textit{The Guide of the Perplexed}, vol. 1-2, tłum. S. Pines, Chicago University Press, Chicago -- London 1963

% \bibitem{Maitzen1993}
% S.~Maitzen\bbland{}W.~P. Alston.
% \newblock \emph{{Perceiving God: The Epistemology of Religious Experience.}},
%   \bblvol{} 102 1993.
% \bibAnnoteFile{Maitzen1993}

\bibitem{Majmonides2008}
Majmonides, \textit{Przewodnik błądzących}, cz.~1, tłum. U.~Krawczyk, H.~Halkowski, Wydawnictwo Pardes, Kraków 2008.

% \bibitem{Marcos2005}
% J.~Marcos.
% \newblock \emph{{On negation: Pure local rules}}.
% \newblock Journal of Applied Logic, \bblvol{}~3 (2005), \bblno{}~1, \bblpp{}
%   185--219.
% \bibAnnoteFile{Marcos2005}

% \bibitem{Martin1974}
% R.~L. Martin.
% \newblock \emph{{Sortal ranges for complex predicates}}.
% \newblock Journal of Philosophical Logic, \bblvol{}~3 (1974), \bblno{} 1-2,
%   \bblpp{} 159--167.
% \bibAnnoteFile{Martin1974}

\bibitem{Martin1975}
R.L. Martin, P.W. Woodruff, \textit{On representing ‘true-in-L' in L}, ,,Philosophia'', vol. 5 (1975), nr 3, ss.~213-217.

% \bibitem{McCabe1992}
% H.~McCabe.
% \newblock \emph{{The Logic of Mysticism—I}}.
% \newblock \capitalize\bblin{} M.~Warner (\bbled{}), \emph{Religion and
%   Philosophy. Royal Institute of Philosophy Supplement}, \bblvol{}~31,
%   Cambridge University Press, Cambridge 1992.
% \newblock \bblpp{} 45--59.
% \bibAnnoteFile{McCabe1992}

\bibitem{Meixner1997}
U. Meixner, \textit{Axiomatic formal ontology}, ser. \textit{Synthese Library}, red. J. Hintikka i~in., vol. 264, Springer, Dordrecht 1997.

\bibitem{Meixner1998}
U. Meixner, \textit{Negative Theology, Coincidentia Oppositorum, and Boolean Algebra}, ,,History of Philosophy \& Logical Analysis'', vol. 1 (1998), nr 1, ss.~75-89.

% \bibitem{Meixner2009}
% U.~Meixner.
% \newblock \emph{{From Plato to Frege: Paradigms of Predication in the History
%   of Ideas}}.
% \newblock Metaphysica, \bblvol{}~10 (2009), \bblno{}~2, \bblpp{} 199--214.
% \bibAnnoteFile{Meixner2009}

\bibitem{Meyer2003}
J-J.C. Meyer, \textit{Modal Epistemic and Doxastic Logic}, [w:] \textit{Handbook of Philosophical Logic}, vol. 10, red. D.M. Gabbay, F. Guenthner, Springer, Dordrecht 2003, ss.~1-38.

% \bibitem{Miernowski1993}
% J.~Miernowski.
% \newblock \emph{{The Law of Non-Contradiction and French Renaissance
%   Literature: Skepticism and Negative Theology}}.
% \newblock South Central Review, \bblvol{}~10 (1993), \bblno{}~2, \bblpp{}
%   49--66.
% \bibAnnoteFile{Miernowski1993}

% \bibitem{Min2007}
% A.~K. Min.
% \newblock \emph{{Naming the Unnameable God: Levinas, Derrida, and Marion}}.
% \newblock Self and Other: Essays in Continental Philosophy of Religion,
%   \bblvol{}~60 (2007), \bblno{}~1, \bblpp{} 99--116.
% \bibAnnoteFile{Min2007}

\bibitem{Eckhart1986}
Mistrz Eckhart, \textit{Kazania}, tłum. W. Szymona, W~drodze, Poznań 1986.

\bibitem{Montague1963}
R. Montague, \textit{Syntactical treatments of modality, with corollaries on reflexion principles and finite axiomatizability}, ,,Acta Philosophical Fennica'', vol. 16 (1963), ss.~153-167.
Przedruk w: R.H. Thomason (red.), \textit{Formal Philosophy: Selected Papers of Richard Montague}, Yale University Press, New Haven -- London 1974, ss.~286-302.

\bibitem{Moore1942}
G.E. Moore, \textit{A~reply to my critics}, [w:] \textit{The Philosophy of G.E. Moore}, red. P.A.~Schilpp,
\textit{The Library of Living Philosophers}, vol. 4, Northwestern University, Evanston 1942, ss.~535-677.

\bibitem{Moore1993}
G.E. Moore, \textit{Moore's Paradox}, [w:] \textit{G.E. Moore: Selected Works}, red. T. Baldwin, Routledge, London and New York 1993, ss.~207-2012.

\bibitem{Mortley1975}
R. Mortley, \textit{Negative Theology and Abstraction in Plotinus}, ,,The American Journal of Philology'', vol. 96 (1975), nr 4, ss.~363-377

% \bibitem{Mortley1982}
% R.~Mortley.
% \newblock \emph{{The Fundamentals of the Via Negativa}}.
% \newblock The American Journal of Philology, \bblvol{} 103 (1982), \bblno{}~4,
%   \bblp{} 429.
% \bibAnnoteFile{Mortley1982}

% \bibitem{Murphy1984}
% F.~X. Murphy.
% \newblock \emph{{The Patristic Origins of Orthodox Mysticism}}.
% \newblock Mystics Quarterly, \bblvol{}~10 (1984), \bblno{}~2, \bblpp{} 59--63.
% \bibAnnoteFile{Murphy1984}

\bibitem{Obolevitch2006}
T.~Obolevitch, \textit{Katafatyczny wymiar wschodniochrześcijańskiej teologii apofatycznej. Wokół propozycji S. Bułgakowa}, ,,Logos i~Ethos'', vol. 21 (2006) nr 2, ss.~56-63.

\bibitem{Obolevitch2008}
T. Obolevitch, \textit{Teologia negatywna a~nauka w~ujęciu Siemiona Franka}, ,,Zagadnienia Filozoficzne w~Nauce'', nr 42 (2008), ss.~68-77.

\bibitem{Obolevitch2014}
T.~Obolevitch, \textit{Filozofia rosyjskiego renesansu patrystycznego}, Copernicus Center Press, Kraków 2014.

\bibitem{OCollins2002}
G. O'Collins, E.G. Farrugia, \textit{Leksykon pojęć teologicznych i~kościelnych z~indeksem angielsko-polskim}, tłum. J. Ożóg, B. Żak, Wydawnictwo WAM, Kraków 2002.

\bibitem{Odintsov2005}
S.P. Odintsov, \textit{On the structure of paraconsistent extensions of Johansson's logic}, ,,Journal of Applied Logic'', vol. 3 (2005), ss. 43-65.

\bibitem{Odintsov2008}
S.P. Odintsov, \textit{Constructive Negations and Paraconsistency}, Trends in Logic, vol.~26, Springer-Verlag, New York 2008.

\bibitem{Olszewski2014}
A. Olszewski, \textit{Pewna krytyka teologii naturalnej}, ,,Analecta Cracoviensia'', vol. 46 (2014), ss.~207-220.

\bibitem{Olszewski2016}
A. Olszewski, \textit{Negation in the language of theology -- some issues}, ,,Zagadnienia Filozoficzne w~Nauce'', nr 65 (2018), ss.~87-107.

\bibitem{ORourke1992}
F. O'Rourke, \textit{Pseudo-Dionysius and the Metaphysics of Aquinas}, EJ.~Brill, Leiden -- New York 1992.

\bibitem{Otto1923}
R.~Otto, \textit{The Idea of the Holy. An Inquiry into the Non-Rational Factor in the Idea of the Divine and Its Relation to the Rational}, tłum. JW. Harvey, Oxford University Press, London 1923.

% \bibitem{Palczewski1995}
% R.~Palczewski.
% \newblock \emph{{Reprezentacja logiczna wiedzy i przekonania. Podstawowe
%   problemy logiki epistemicznej}} 1995.
% \bibAnnoteFile{Palczewski1995}

% \bibitem{Paradis2006}
% C.~Paradis\bbland{}C.~Willners.
% \newblock \emph{{Antonymy and negation-The boundedness hypothesis}}.
% \newblock Journal of Pragmatics, \bblvol{}~38 (2006), \bblno{}~7, \bblpp{}
%   1051--1080.
% \bibAnnoteFile{Paradis2006}

\bibitem{Pasniczek2014}
J. Paśniczek, \textit{Predykacja. Elementy ontologii formalnej przedmiotów, własności i~sytuacji}, Copernicus Center Press, Kraków 2014.

\bibitem{Perzanowski1991}
J. Perzanowski, \textit{Ontological Arguments II: Cartesian and Leibnizian}, [w:] \textit{Handbook of Metaphysics and Ontology},
t.~2, red. H. Burkhardt, B. Smith,  PhilosophiaVerlag, München 1991, ss.~625-633

\bibitem{Perzanowski1996}
J.~Perzanowski, \textit{The Way of Truth}, [w:] \textit{Formal Ontology}, red. R. Poli, P. Simons, Kluwer Academic Publishers, Dordrecht 1996, ss. 61-130.

% \bibitem{Piechowicz2011}
% R.~Piechowicz.
% \newblock \emph{{J{\c{e}}zyk religii a zadanie logika}}.
% \newblock \capitalize\bblin{} \emph{Logika i teologia}, Ksi{\c{e}}garnia
%   {\'{s}}w. Jacka, Katowice 2011.
% \newblock \bblpp{} 133--139.
% \bibAnnoteFile{Piechowicz2011}

\bibitem{Pietruszczak2005}
A. Pietruszczak, \textit{Pieces of mereology}, ,,Logic and Logical Philosophy'', vol. 14 (2005), nr 2, ss.~211-234.

% \bibitem{Plantinga1980}
% A.~Plantinga.
% \newblock \emph{{Does God Have a Nature?}}
% \newblock 2, Marquette University Press 1980.
% \bibAnnoteFile{Plantinga1980}

\bibitem{Plantinga2000}
A. Plantinga, \textit{Warranted Christian Belief}, Oxford University Press, New York 2000.

\bibitem{Platon2005}
Platon, \textit{Fajdros}, [w:] tenże, \textit{Dialogi}, Wydawnictwo Antyk, Kęty 2005.

\bibitem{Platon2021}
Platon, \textit{Parmenides}, Wydawnictwo Marek Derewiecki, Kęty 2021.

\bibitem{Plotinus1984}
Plotinus, \textit{Ennead}, VI. 2, red. i~tłum. A.H. Armstrong, Loeb Classical Library, Harvard University Press, Cambridge 1984.

% \bibitem{Poczobut1999}
% R.~Poczobut.
% \newblock \emph{{Sprzeczno{\'{s}}ci doksastyczne a zagadnienie
%   racjonalno{\'{s}}ci przekona{\'{n}}}}.
% \newblock Filozofia Nauki, \bblvol{}~7 (1999), \bblno{} 3-4, \bblpp{} 61--84.
% \bibAnnoteFile{Poczobut1999}

\bibitem{Poczobut2013}
R. Poczobut, \textit{Paradoksy w~wyjaśnianiu świadomości}, ,,Ethos. Kwartalnik Instytutu Jana Pawła II KUL'', vol. 26 (2013), nr 1(101), ss.~62-80.

\bibitem{Pontow2006}
C. Pontow, R. Schubert, \textit{A~mathematical analysis of theories of parthood}, ,,Data and Knowledge Engineering'', vol. 59 (2006), nr 1, ss.~107-138.

% \bibitem{Post1973}
% J.~F. Post.
% \newblock \emph{{Shades of the liar}}.
% \newblock Journal of Philosophical Logic, \bblvol{}~2 (1973), \bblno{}~3,
%   \bblpp{} 370--386.
% \bibAnnoteFile{Post1973}

\bibitem{Pouivet2013}
R. Pouivet, \textit{Bocheński on divine ineffability}, ,,Studies in East European Thought'', vol. 65 (2013), nr 1-2, ss.~43-51.

% \bibitem{Power1976}
% W.~L. Power.
% \newblock \emph{{Musings on the mystery of God}}.
% \newblock International Journal for Philosophy of Religion, \bblvol{}~7 (1976),
%   \bblno{}~1, \bblpp{} 300--310.
% \bibAnnoteFile{Power1976}

\bibitem{Priest1994}
G. Priest, \textit{The Structure of the Paradoxes of Self-Reference}, ,,Mind'', vol. 103 (1994), nr 409, ss.~25-34.

\bibitem{Priest1998}
G. Priest, \textit{What so bad about contradictions}?, ,,The Journal of Philosophy'', vol. 95 (1998), nr 8, ss.~410-426.

\bibitem{Prus2021}
J. Pruś, \textit{Semantyczna teoria prawdy a~antynomie semantyczne}, ,,Rocznik Filozoficzny Ignatianum'', vol 27 (2021), nr 1, ss.~341-363.

\bibitem{Areopagita1997}
Pseudo-Dionizy Areopagita, \textit{Pisma teologiczne}, tłum. M. Dzielska, Wydawnictwo Znak, Kraków 1997.

\bibitem{Areopagita1999}
Pseudo-Dionizy Areopagita, \textit{Pisma teologiczne II}, tłum. M. Dzielska, Wydawnictwo Znak, Kraków 1999.

% \bibitem{Pseudo-Dionysius1897}
% Pseudo-Dionysius\bbland{}J.~Parker.
% \newblock \emph{{The Works of Dionysius the Areopagite}}.
% \newblock James Parker and Co., Oxford 1897.
% \bibAnnoteFile{Pseudo-Dionysius1897}

\bibitem{Putnam1997}
H. Putnam, \textit{On negative theology}, ,,Faith and Philosophy'', vol. 14 (1997), nr 4, ss.~407-422.

\bibitem{Quine1990}
W.V.O. Quine, \textit{Three Indeterminacies}, [w:] \textit{Perspectives on Quine}, red. R. Barret, R. Gibson, Blackwell, Cambridge 1990, ss 1-16.

\bibitem{Quine2010}
W.V.O. Quine, \textit{O~tym, co istnieje}, [w:] tenże, \textit{Z~punktu widzenia logiki}, tłum. B.~Stanosz, Aletheia, Warszawa 2000, ss. 7-48.

\bibitem{Quine2012}
W.V.O. Quine, \textit{Indeterminacy of Translation Again}, ,,Journal of Philosophy'', vol.~84 (2012), nr 1, ss.~5-10.

\bibitem{Ramsey1926}
F.P. Ramsey, \textit{The Foundations of Mathematics}, ,,Proceedings of the London Mathematical Society'', vol. s2-25 (1926), nr 1, ss. 338-384.

\bibitem{Ramsey1990}
F.P. Ramsey, \textit{Universals}, [w:] Tenże, \textit{Philosophical Papers}, red. D.H. Mellor, Cambridge University Press, Cambridge -- New York -- Melbourne 1990, ss.~8-32.

\bibitem{Rappaport1999}
R.A. Rappaport, \textit{Ritual and Religion in the Making of Humanity}, Cambridge University Press, Cambridge 1999.

\bibitem{Rea2021}
M.C. Rea, \textit{Essays in Analytic Theology}, vol.~1, ser. \textit{Oxford Studies in Analytic Theology},
Oxford University Press,  Oxford 2021.

% \bibitem{Rea2020}
% M.~C. Rea.
% \newblock \emph{{God beyond Being}}.
% \newblock \capitalize\bblin{} \emph{Essays in Analytic Theology}, \bblvol{}~I,
%   Oxford University Press 2020.
% \newblock \bblpp{} 120--138.
% \bibAnnoteFile{Rea2020}

% \bibitem{Redmond1990}
% W.~Redmond.
% \newblock \emph{{A logic of faith}}.
% \newblock International Journal for Philosophy of Religion, \bblvol{}~27
%   (1990), \bblno{}~3, \bblpp{} 165--180.
% \bibAnnoteFile{Redmond1990}



\bibitem{sep-logic-epistemic}
R. Rendsvig, J. Symons, \textit{Epistemic Logic}, [w:] \textit{The Stanford Encyclopedia of Philosophy},
wyd. lato 2021, red. E.N. Zalta, <\url{https://plato.stanford.edu/archives/sum2021/entries/logic-epistemic/}>.

\bibitem{sep-language-thought}
M. Rescorla, \textit{The Language of Thought Hypothesis}, [w:] \textit{The Stanford Encyclopedia of Philosophy},
wyd. lato 2019, red. E.N. Zalta, <\url{https://plato.stanford.edu/archives/sum2019/entries/language-thought/}>.

\bibitem{Richard1967}
J. Richard, \textit{The principles of mathematics and the problem of sets},
[w:] \textit{From Frege to Gödel. A~source book in mathematical logic 1879–1931}, red. J. van Heijenoort, Harvard University Press, Cambridge 1967, ss.~142-144.

\bibitem{Rocca2004}
G. Rocca, \textit{Speaking the Incomprehensible God. Thomas Aquinas on the Interplay of Positive and Negative Theology}, The Catholic University of America Press, Washington 2004.

\bibitem{Rojek2012}
P. Rojek, \textit{Logika teologii negatywnej}, ,,Pressje'', nr 29 (2012), ss.~216-231.

% \bibitem{Rojek2013}
% P.~Rojek.
% \newblock \emph{{The Logic of Palamism}}.
% \newblock \capitalize\bblin{} A.~Schumann (\bbled{}), \emph{Logic in Orthodox
%   Christian Thinking}, De Gruyter, 2012 2013.
% \newblock \bblpp{} 38--81.
% \bibAnnoteFile{Rojek2013}

\bibitem{Rojek2013a}
P. Rojek, \textit{Towards a~Logic of Negative Theology}, [w:] \textit{Logic in Religious Discourse}, red. A. Schumann, De Gruyter, Berlin -- Boston 2013, ss.~192-215.

\bibitem{Rorem1993}
P. Rorem, \textit{Pseudo-Dionysius. A~Commentary on the Texts and an Introduction to Their Influence}, Oxford University Press, New York -- Oxford 1993.

\bibitem{Rorem2008}
P. Rorem, \textit{Negative Theologies and the Cross}, ,,Harvard Theological Review'', vol.~101 (2008), ss.~451-464.

\bibitem{RoremLuibheid}
P. Rorem, C. Luibheid, \textit{Dionysius the Areopagite. The Complete Work}, Paulist Press, New York 1987

% \bibitem{Roszak2015}
% P.~Roszak.
% \newblock \emph{{Dwie pr{\c{e}}dko{\'{s}}ci teologii? o celu, metodzie i
%   perspektywach teologii analitycznej}}.
% \newblock Teologia w Polsce, \bblvol{}~9 (2015), \bblno{}~2, \bblpp{} 75--93.
% \bibAnnoteFile{Roszak2015}

\bibitem{Rowe1999}
W.L. Rowe, \textit{Religious pluralism}, ,,Religious Studies'', vol. 35 (1999), ss.~139-150.

\bibitem{Ruczaj2012}
S. Ruczaj, \textit{Analogia i apofatyczny pazur}, ,,Pressje'', nr 30/31 (2012), ss.~280-282.

\bibitem{Russell1903}
B. Russell, \textit{The Principles of Mathematics}, vol. I, Cambridge University Press, Cambridge 1903.

% \bibitem{Russell1936}
% B.~Russell.
% \newblock \emph{{The Limits of Empiricism}}.
% \newblock Proceedings of the Aristotelian Society, \bblvol{}~36 (1936),
%   \bblpp{} 131--150.
% \bibAnnoteFile{Russell1936}

\bibitem{Rybarkiewicz2020}
D. Rybarkiewicz, \textit{Dialeteizm punktem spotkania Boga religii z~Bogiem filozofów? Uwag kilka: więcej pytań niż odpowiedzi},
[w:] \textit{Analiza, racjonalność, filozofia religii. Księga jubileuszowa dedykowana Profesorowi Ryszardowi Kleszczowi}, Wydawnictwo Uniwersytetu Łódzkiego, Łódź 2020, ss.~375-385.

\bibitem{Salerno2009}
J. Salerno (red.), \textit{New Essays on the Knowability Paradox}, Oxford University Press, New York 2009.

\bibitem{Salerno2009b}
J. Salerno, \textit{Introduction}, [w:] \textit{New Essays on the Knowability Paradox}, red. tenże, Oxford University Press, New York 2009.

\bibitem{Schlamm1992}
L. Schlamm, \textit{Numinous Experience and Religious Language}, ,,Religious Studies'', vol.~28 (1992), nr 4, ss.~533-551.

\bibitem{sep-linguistics}
B.C. Scholz, F.J. Pelletier, G.K. Pullum, R. Nefdt, \textit{Philosophy of Linguistics}, [w:] \textit{The Stanford Encyclopedia of Philosophy},
wyd. wiosna 2022, red. E.N. Zalta, <\url{https://plato.stanford.edu/archives/spr2022/entries/linguistics/}>.

% \bibitem{Schumann2012}
% A.~Schumann (\bbled{}).
% \newblock \emph{{Logic in Orthodox Christian Thinking}} 2012.
% \bibAnnoteFile{Schumann2012}

\bibitem{Scott2016}
M. Scott, G. Citron, \textit{What is apophaticism? Ways of talking about an ineffable God}, ,,European Journal for Philosophy of Religion'', vol. 8 (2016), nr 4, ss.~23-49.

\bibitem{Searl1993}
J. Searl, \textit{Metaphor}, [w:] \textit{Metaphor and Thought}, red. A. Ortony, Cambridge University Press, Cambridge 1993, ss.~83-111.

\bibitem{Sedlar2019}
I. Sedlár, K. Ebela, \textit{Term Negation in First-Order Logic}, ,,Logique et Analyse'', vol.~247 (2019), ss.~265-284.

\bibitem{sep-maimonides}
K. Seeskin, \textit{Maimonides}, [w:] \textit{The Stanford Encyclopedia of Philosophy},
wyd. wiosna 2021, red. E.N. Zalta, <\url{https://plato.stanford.edu/archives/spr2021/entries/maimonides/}>.

% \bibitem{Sells1994}
% M.~A. Sells.
% \newblock \emph{{Mystical Languages of Unsaying}}, \bblvol{} 115.
% \newblock The University of Chicago Press, London, Chicago 1994.
% \bibAnnoteFile{Sells1994}

\bibitem{Shaw2013}
J. Shaw, \textit{Truth, Paradox, and Ineffable Propositions}, ,,Philosophy and Phenomenological Research'', vol. 86 (2013), nr 1, ss.~64-104.

\bibitem{Shramko2005}
Y.~Shramko, \textit{Dual Intuitionistic Logic and a~Variety of Negations: The Logic of Scientific Research}, ,,Studia Logica: An International Journal for Symbolic Logic'', vol. 80 (2005), ss.~347-367.

\bibitem{Sider2011}
T. Sider, \textit{Writing the Book of the World}, Oxford University Press, Oxford 2012.

\bibitem{Sikora2010}
P. Sikora, \textit{Logos niepojęty}, Wydawnictwo Universitas, Kraków 2010.

% \bibitem{Skarga1998}
% B.~Skarga.
% \newblock \emph{{Teologia negatywna a cz{\l}owiek}}.
% \newblock Kwartalnik Filozoficzny, \bblvol{}~26 (1998), \bblno{}~4, \bblpp{}
%   5--20.
% \bibAnnoteFile{Skarga1998}

\bibitem{Skowron2021}
B. Skowron, \textit{Część i~całość. W~stronę topoontologii}, Oficyna Wydawnicza Politechniki Warszawskiej, Warszawa 2021.

\bibitem{Slobin1996}
D.I. Slobin, \textit{From ``thought and language'' to ``thinking for speaking}'', [w:] \textit{Rethinking linguistic relativity},
red. J.J. Gumperz, S.C. Levinson, Cambridge University Press, Cambridge 1996, ss.~70-96.

\bibitem{Slupecki1955}
J.~Słupecki, \textit{Stanisław Leśniewski’s Calculus of Names}, ,,Studia Logica'', nr 3 (1955), ss. 7-76

% \bibitem{Small2002ReflectionsOG}
% C.~G. Small.
% \newblock \emph{{Reflections on G{\"{o}}del's Ontological Argument}}.
% \newblock \capitalize\bblin{} \emph{Klarheit in Religionsdingen. Aktuelle
%   Beitr{\"{a}}ge zur Religionsphilosophie, Band III of Grundlagenprobleme
%   unserer Zeit}, Leipziger Universit{\"{a}}tsverlag 2003.
% \newblock \bblpp{} 109--144.
% \bibAnnoteFile{Small2002ReflectionsOG}

\bibitem{Sobel1987}
J.H. Sobel, \textit{Gödel's ontological proof}, [w:] \textit{On Being and Saying. Essays for R.L.~Cartwright}, red. J.J. Thompson, MIT Press, Cambridge 1987, ss.~241-261.

\bibitem{Sommers1965-SOMP-2}
F. Sommers, \textit{Predicability}, [w:] \textit{Philosophy in America}, red. M. Black, Routledge, London 2002, ss.~262-281.

% \bibitem{Sommers1971}
% F.~Sommers.
% \newblock \emph{{Structural ontology}}.
% \newblock Philosophia, \bblvol{}~1 (1971), \bblno{} 1-2, \bblpp{} 21--42.
% \bibAnnoteFile{Sommers1971}

\bibitem{sep-epistemic-paradoxes}
R. Sorensen, \textit{Epistemic Paradoxes}, [w:] \textit{The Stanford Encyclopedia of Philosophy},
wyd. wiosna 2022, red. E.N. Zalta, <\url{https://plato.stanford.edu/archives/spr2022/entries/epistemic-paradoxes/}>.

\bibitem{Soskice1985}
J.M. Soskice, \textit{Metaphor and Religious Language}, Clarendon Press, Oxford 1985.

\bibitem{Speranza2010}
J.L. Speranza, L.R. Horn, \textit{A brief history of negation}, ,,Journal of Applied Logic'', vol.~8 (2010), nr~3, ss.~277-301.

% \bibitem{Spychalska2011}
% M.~Spychalska.
% \newblock \emph{{Processing of sentences with predicate negation: The role of
%   opposite predicates}}.
% \newblock  (2011).
% \newblock
%   \urlprefix\url{http://web.stanford.edu/$\sim$danlass/esslli2011stus/spychalska.pdf}.
% \bibAnnoteFile{Spychalska2011}

\bibitem{Stace1952}
W.T. Stace, \textit{Time and Eternity: An Essay in the Philosophy of Religion}, Princeton University Press, Princeton 1952.

\bibitem{Stace1961}
W.T. Stace, \textit{Mysticism and Philosophy}, Macmillan \& Co Ltd, London 1961.

\bibitem{Stang2013}
C.M. Stang, \textit{Negative Theology from Gregory of Nyssa to Dionysius the Areopagite}, [w:] \textit{The Wiley-Blackwell Companion to Christian Mysticism},
red. J.A. Lamm, Blackwell Publishing, Malden -- Oxford 2013, ss.~161-176.


% \bibitem{Stepien1994}
% T.~St{\c{e}}pie{\'{n}}.
% \newblock \emph{{Teologia negatywna w pismach ,,Corpus Areopagiticum''}}.
% \newblock Warszawskie Studia Teologiczne, \bblvol{}~7 (1994), \bblpp{}
%   231--255.
% \bibAnnoteFile{Stepien1994}

\bibitem{Stepien1997}
T. Stępień, \textit{Przedmowa}, [w:] Pesudo-Dionizy Areopagita, \textit{Pisma teologiczne}, Wydawnictwo Znak, Kraków 1997,  ss.~9-36.

\bibitem{Stepien2013}
T. Stępień, \textit{Znany -- nieznany Bóg. Uwagi na temat rozwoju doktryny niepoznawalności Boga u~chrześcijańskich autorów od II do VI wieku}, ,,Internetowy Magazyn Filozoficzny Hybris'', vol. 20 (2013), ss.~83-108.

\bibitem{Stern2014}
J. Stern, \textit{Montague's Theorem and Modal Logic}, ,,Erkenntnis'', vol. 79 (2014), nr 3, s.~551-570.

\bibitem{Stone1936}
M.H. Stone, \textit{The Theory of Representation for Boolean Algebras}, ,,Transactions of the American Mathematical Society'', vol. 40 (1963), nr 1, ss.~1-37.

\bibitem{Strozewski1967}
W. Stróżewski, \textit{Z historii problematyki negacji, cz. 1: Ontologiczna problematyka negacji w ,,De quatuor oppositis''}, ,,Studia Mediewistyczne'', vol. 8 (1967), ss.~183-246.

\bibitem{Strozewski1994}
W. Stróżewski, \textit{Z problematyki negacji}, [w:] tenże, \textit{Istnienie i sens}, Wydawnictw Znak, Kraków 1994, s.~373-395.

% \bibitem{Swinburne2014}
% R.~Swinburne.
% \newblock \emph{{Gregory Palamas and our Knowledge of God}}.
% \newblock Studia Humana, \bblvol{}~3 (2014), \bblno{}~1, \bblpp{} 3--12.
% \bibAnnoteFile{Swinburne2014}

\bibitem{Synowiecki1994}
A. Synowiecki, \textit{Od mitu o~nauce do powagi naukowej, cz. I}, ,,Studia Philosophiae Christianae'', vol. 30 (1994), nr 2, ss.~245-271.

\bibitem{Tahko2018}
T.E. Tahko, \textit{Fundamentality}, [w:] \textit{The Stanford Encyclopedia of Philosophy},
wyd. jesień 2018, red. E.N. Zalta, {\textless}\url{https://plato.stanford.edu/archives/fall2018/entries/fundamentality/}{\textgreater}.

\bibitem{Tarski1933}
A. Tarski, \textit{Pojęcie prawdy w~językach nauk dedukcyjnych}, Prace Towarzystwa Naukowego Warszawskiego, Wydział III Nauk Matematyczno-Fizycznych, vol. 34, Warszawa 1933.

\bibitem{Tarski1935}
A. Tarski, \textit{Zur Grundlegung der Boole'schen Algebra I}, ,,Fundamenta Mathematicae'', vol. 24 (1935), nr 1, ss.~177-198.

\bibitem{Tarski2001}
A. Tarski, \textit{Podstawowe pojęcia metodologii nauk dedukcyjnych}, [w:] tenże, \textit{Pisma logiczno-filozoficzne, t.~2: Metalogika},
tłum. J. Zygmunt, Wydawnictwo Naukowe PWN, Warszawa 2001, ss.~31-92.


% \bibitem{Teahan1978}
% J.~F. Teahan.
% \newblock \emph{{A Dark and Empty Way: Thomas Merton and the Apophatic
%   Tradition}}.
% \newblock The Journal of Religion, \bblvol{}~58 (1978), \bblno{}~3, \bblpp{}
%   263--287.
% \bibAnnoteFile{Teahan1978}

\bibitem{Tennessen1959}
H. Tennessen, \textit{Logical Oddities and Locutional Scarcities: Another Attack upon Methods of Revelation}, ,,Synthese'', vol. 11 (1959), nr 4, ss.~369-388.

\bibitem{Thijssen2018}
H. Thijssen, \textit{Condemnation of 1277}, [w:] \textit{The Stanford Encyclopedia of Philosophy},
wyd. zima 2018, red. E.N. Zalta, {\textless}\url{https://plato.stanford.edu/archives/win2018/entries/condemnation/}{\textgreater}.

\bibitem{Thomason1972}
R.H. Thomason, \textit{A~semantic theory of sortal incorrectness}, ,,Journal of Philosophical Logic'', vol. 1 (1972), ss.~209-258.

% \bibitem{Thomason1974}
% R.~H. Thomason\bbland{}R.~Montague.
% \newblock \emph{{Formal Philosophy: Selected Papers of Richard Montague}},
%   \bblvol{}~53.
% \newblock Yale University Press, New Haven and London 1974.
% \bibAnnoteFile{Thomason1974}

\bibitem{Tichy1978}
P. Tichy, \textit{Questions, Answers, and Logic}, ,,American Philosophical Quarterly'', vol.~15 (1978), nr 4, ss.~275-284.

% \bibitem{Tipton2012}
% I.~Tipton.
% \newblock \emph{{Locke: Knowledge and its limits}}.
% \newblock \capitalize\bblin{} \emph{British Philosophy and the Age of
%   Enlightenment} 2012.
% \newblock \bblpp{} 69--95.
% \bibAnnoteFile{Tipton2012}

% \bibitem{464620}
% M.~Tkaczyk.
% \newblock \emph{{Grahama Priesta metoda uzasadniania tezy dialeteizmu}}.
% \newblock Warszawskie Studia Teologiczne, \bblvol{}~27 (2014), \bblno{}~2,
%   \bblpp{} 23--35.
% \bibAnnoteFile{464620}

\bibitem{Akwinu}
Tomasz z Akwinu, \textit{Suma teologiczna}, tłum. P. Bełch, tom 1-34, Katolicki Ośrodek Wydawniczy ,,Veritas'', Londyn 1962-1986.

\bibitem{Akwinu1981}
Tomasz z~Akwinu, \textit{De ente et essentia.} \textit{O~bycie i~istocie}, [wydanie dwujęzyczne] tłum. M.~Krąpiec, Redakcja Wydawnictw KUL, Lublin 1981.

% \bibitem{Triebitz2016}
% M.~Triebitz.
% \newblock \emph{{Rambam's Theory of Negative Theology: Divine Creation and
%   Human Interpretation}}.
% \newblock  (2016), \bblno{} July, \bblpp{} 1--23.
% \bibAnnoteFile{Triebitz2016}

\bibitem{Tromans2020}
O. Tromans, \textit{Similarity Within (Ultimate) Dissimilarity: Burrell and Milbank on the Interplay of Positive and Negative Theology},
,,Heythrop Journal -- Quarterly Review of Philosophy and Theology'', vol. 61 (2020), nr 5, ss.~749-762.

\bibitem{Turner1995}
D. Turner, \textit{The Darkness of God: Negativity in Christian Mysticism}, Cambridge University Press, Cambridge 1995.

% \bibitem{Turon2007}
% A.~Turo{\'{n}}.
% \newblock \emph{{O negacji}}.
% \newblock Czasopismo Filozoficzne,  (2007), \bblno{}~2, \bblpp{} 75--85.
% \bibAnnoteFile{Turon2007}

\bibitem{Tworak2004}
Z.~Tworak, \textit{Kłamstwo kłamcy i~zbiór zbiorów. O~problemie antynomii}, ser. \textit{Filozofia i logika}, nr~89, Wydawnictwo Naukowe UAM, Poznań 2004.

\bibitem{Tworak2011}
Z. Tworak, \textit{Paradoks znawcy (The Knower Paradox}), ,,Filozofia Nauki'', vol. 19 (2011), nr 3(75), ss.~29-47.

\bibitem{Urbanczyk2013}
P. Urbańczyk, \textit{Logika i~teologia}, ,,Zagadnienia Filozoficzne w~Nauce'', nr 57 (2013), ss.~143-151.

\bibitem{Urbanczyk2014}
P. Urbańczyk, \textit{Geneza intuicjonistycznego rachunku zdań i~Twierdzenie Gliwienki}, ,,Zagadnienia Filozoficzne w~Nauce'', nr 56 (2014), ss.~33-56.

\bibitem{Urbanczyk2016}
P. Urbańczyk, ``\textit{Internal'' Problems of Normative Theories of Thinking and Reasoning}, ,,Zagadnienia Filozoficzne w~Nauce'', 2016, nr 60, ss.~35-52.

\bibitem{Urbanczyk2018}
P. Urbańczyk, \textit{The logical challenge of negative theology}, ,,Studies in Logic, Grammar and Rhetoric'', vol. 54 (2018), nr 1, ss.~149-174.


\bibitem{VanDitmarsch2012}
H. Van Ditmarsch, W.~Van Der Hoek, P.~Iliev, \textit{Everything is knowable -- How to get to know whether a~proposition is True}, ,,Theoria'', vol. 78 (2012), nr 2, ss.~93-114.

\bibitem{Wang1987}
H. Wang, \textit{Reflections on Kurt Gödel}, MIT Press, Cambridge 1987.

% \bibitem{Wansing1996}
% H.~Wansing (\bbled{}).
% \newblock \emph{{Negation: A Notion in Focus}}, \bblvol{}~7.
% \newblock Walter de Gruyter, berlin and \bbledn{} 1996.
% \bibAnnoteFile{Wansing1996}

% \bibitem{Wansing2001}
% H.~Wansing.
% \newblock \emph{{Negation}}.
% \newblock \capitalize\bblin{} L.~Goble (\bbled{}), \emph{The Blackwell Guide to
%   Philosophical Logic}, Blackwell Publishing, Malden – Oxford 2001.
% \newblock \bblpp{} 415--436.
% \bibAnnoteFile{Wansing2001}

\bibitem{Wansing2002}
H. Wansing, \textit{Diamonds Are a~Philosopher's Best Friends: The Knowability Paradox and Modal Epistemic Relevance Logic}, ,,Journal of Philosophical Logic'', vol. 31 (2002), ss.~591-612.

\bibitem{Whitehead1910}
A.N. Whitehead, B. Russell, \textit{Principia Mathematica}, vol. 1, Cambridge University Press, Cambridge 1910.

\bibitem{Wissink2000}
J. Wissink, \textit{Two Forms of Negative Theology Explained Using Thomas Aquinas}, [w:]
 \textit{Flight of the Gods. Philosophical Perspectives on Negative Theology}, red. I.N.~Bulhof, L. Kate, Fordham University Press, Fordham 2000, ss.~100-120

\bibitem{Wittgenstein2011}
L. Wittgenstein, \textit{Tractatus logico-philosophicus}, tłum. B. Wolniewicz, Wydawnictwo Naukowe PWN, Warszawa 2022.

\bibitem{Witwicki2021}
W. Witwicki, \textit{Przedmowa}, [w:] Platon, \textit{Parmenides}, Wydawnictwo Marek Derewiecki, Kęty 2021.

\bibitem{Wojtowicz2001}
K. Wójtowicz, \textit{O~pojęciu ,,zobowiązania ontologicznego''}, ,,Przegląd Filozoficzny — Nowa Seria'', r.~X (2001), nr 1(37), ss.~121-138.

\bibitem{Wolak2005}
Z. Wolak, \textit{Naukowa filozofia koła krakowskiego}, ,,Zagadnienia Filozoficzne w~Nauce'', nr~36 (2005), ss.~97-122.

\bibitem{Wolenski1993}
J. Woleński, \textit{Samozwrotność i~odrzucanie}, ,,Filozofia Nauki'', vol. 1 (1993), nr 1, ss.~89-102.

\bibitem{Wolenski2000}
J. Woleński, \textit{Czy Leśniewski był filozofem?}, ,,Filozofia Nauki'', vol. 8 (2000), nr 3-4, ss.57-68.

\bibitem{Wolenski2005}
J. Woleński, \textit{Metateoretyczne problemy epistemologii}, ,,Diametros'', vol. 6 (2005), ss.~70-93.

\bibitem{Wolenski2013}
J. Woleński, \textit{Theology and Logic}, [w:] \textit{Logic in Theology}, red. B.~Brożek, A.~Olszewski, M.~Hohol,
Copernicus Center Press, Kraków 2013, ss.~11-38.

% \bibitem{10.2307/1451300}
% H.~A. Wolfson.
% \newblock \emph{{Crescas on the Problem of Divine Attributes}}.
% \newblock The Jewish Quarterly Review, \bblvol{}~7 (1916), \bblno{}~2, \bblpp{}
%   175--221.
% \newblock \urlprefix\url{http://www.jstor.org/stable/1451300}.
% \bibAnnoteFile{10.2307/1451300}

% \bibitem{Wolfson1952}
% H.~A. Wolfson.
% \newblock \emph{{Albinus and Plotinus on divine attributes}}.
% \newblock Harvard Theological Review, \bblvol{}~45 (1952), \bblno{}~2, \bblpp{}
%   115--130.
% \bibAnnoteFile{Wolfson1952}

% \bibitem{WoodhullJennifer}
% G.~{Woodhull, Jennifer}.
% \newblock \emph{{Decoding Mystical Rhetoric: Scholars, Mystics And Silence}}.
% \bibAnnoteFile{WoodhullJennifer}

% \bibitem{Wright1959}
% G.~H. Wright.
% \newblock \emph{{On the Logic of Negation, by G.H. Von Wright}}.
% \newblock Societas Scientiarum Fennica. Commentationes physico-mathematicae,
%   XXII, 4, Helsinki 1959.
% \newblock \urlprefix\url{https://books.google.pl/books?id=JkAGtAEACAAJ}.
% \bibAnnoteFile{Wright1959}

% \bibitem{Yadav2016}
% S.~Yadav.
% \newblock \emph{{Mystical Experience and the Apophatic Attitude Sameer}}.
% \newblock Journal of Analytic Theology, \bblvol{}~4 (2016), \bblno{}~1,
%   \bblpp{} 17--43.
% \newblock
%   \urlprefix\url{https://journals.tdl.org/jat/index.php/jat/article/view/jat.2016-4.180017240021a/279}.
% \bibAnnoteFile{Yadav2016}

\bibitem{zinov1973foundations}
A.A. Zinov'ev, \textit{Foundations of the Logical Theory of Scientific Knowledge (Complex Logic)}, D.~Reidel Publishing Company, Dordrecht 1973.

\bibitem{Zinowjew1976}
A.~Zinowjew, \textit{Logika nauki}, tłum. Z.~Simbierowicz, Wydawnictwo Naukowe PWN, Warszawa 1976.

\end{thebibliography}
